\setcounter{page}{1}
\section{实数和数列极限}
\exercise

1. 设$a$为有理数,$b$为无理数.求证:$a+b$与$a-b$都是无理数;当$a\neq 0$时,$ab$与$b/a$也是无理数.
\begin{proof}
采用反证法.设$a+b$是有理数,那么就存在互质的整数$p,q$,使得
\begin{equation}
    a+b = \frac{p}{q}
\end{equation}
由于$a$都是有理数,所以存在$p_1,q_1 \in \mathbb{Z}$使得$a = \displaystyle \frac{p_1}{q_1}$,其中$p_1,q_1$互质.于是
\begin{equation}
    b = a+b - a = \frac{p}{q} - \frac{p_1}{q_1} = \frac{p q_1 - q p_1}{q q_1}
\end{equation}
这说明$b$是有理数,矛盾.于是$a+b$是无理数.

再证$a-b$是无理数,采用反证法,先假设$a-b$是有理数,即,存在互质的整数$p_2,q_2$使得$\displaystyle a-b=\frac{p_2}{q_2}$,那么
\begin{equation}
    b = a - (a-b) = \frac{p_1}{q_1} - \frac{p_2}{q_2} = \frac{p_1 q_2 - q_1 p_2}{q_1 q_2}
\end{equation}
这说明$b$是有理数,矛盾.于是$a-b$是无理数.
\end{proof}

2. 证明:两个不同的有理数之间有无限多个有理数,也有无限多个无理数.
\begin{proof}
设$\displaystyle a_1 = \frac{p_1}{q_1}, b_1 = \frac{p_2}{q_2}$是有理数,且$a_1 \neq b_1$,令
\begin{equation}
    c_1 = \frac{a_1+b_1}{2} = \frac{1}{2} \left( \frac{p_1}{q_1} + \frac{p_2}{q_2} \right) = \frac{p_1 q_2 + p_2 q_1}{2 q_1 q_2}
\end{equation}
则显然$c_1$是无理数,且$a_1 < c_1 < b_1$,对任意的$n\geq 2, n \in \mathbb{N}$,我们令
\begin{equation}
    c_{n-1} = \frac{a_{n-1}+b_{n-1}}{2}, \; a_n = c_{n-1}, \; b_n = b_{n-1}
\end{equation}
由于有理数集关于加法运算封闭,所以按此方法可计算出无穷多个有理数$c_n$满足$a_1 < c_1 < c_2 < \cdots < b$,这就证明了两个不同的有理数之间有无穷多个有理数.

设$t_1$是一个无理数,那么我们可以找到两个整数$n_1, m_1$,满足$n_1 < m_1$,并且使得
\begin{equation}
    n_1 < t_1 < m_1
    \label{ieq:n1t1m1}
\end{equation}
在$0$到$b_1 - a_1$之间必定存在$N_1 \in \mathbb{N}^\star$,使得
\begin{equation}
    0 < \frac{m_1}{N_1} - \frac{n_1}{N_1} = \frac{m_1-n_1}{N_1} < b_1 - a_1
\end{equation}
再结合不等式(\ref{ieq:n1t1m1})也就是
\begin{equation}
    a_1 < \frac{t_1}{N_1} - \frac{n_1}{N_1} + a_1 < b_1
\end{equation}
令
\begin{equation}
    s_1 = \frac{t_1}{N_1} - \frac{n_1}{N_1} + a_1
\end{equation}
显然,由于$a_1$是有理数,而$\displaystyle \frac{t_1}{N_1} - \frac{n_1}{N_1}$是无理数,所以$s_1$是无理数并且介于$a_1$到$b_1$之间.对任意$n \in \mathbb{N}^\star$,令
\begin{equation}
    s_{n+1} = \frac{s_n+b_1}{2}, 
\end{equation}
则可构造出无穷多个无理数$s_{n}$,并且满足$a_1 < s_1 < s_2 < \cdots < b_1$,这就证明了任意两个有理数之间有无限多个无理数.
\end{proof}

16. 设$n=2,3,\cdots, \, x > -1$且$x\neq 0$.求证:$(1+x)^n > 1+nx$.
\begin{proof}
当$n=2$时,
\begin{equation}
    (1+x)^2 = 1 + 2x + x^2 > 1 + 2x + 0 = 1 + 2x
\end{equation}
于是对于$n=2$命题成立.假设对某个$k \in \mathbb{N}^\star, \, k \geq 2$,当$n=k$时命题成立,则
\begin{align}
    (1+x)^{k+1} &= (1+x)^k (1+x) > (1+kx)(1+x) \\
    &= 1 + (k+1)x + kx^2 > 1+(k+1)x
\end{align}
这说明如果对$n=k$命题成立,那么对$n=k+1$命题也成立.根据数学归纳法原理,命题对一切$n\in\mathbb{N}^\star$都成立.
\end{proof}

\exercise

5. 用精确语言表达``数列$\{a_n\}$不以$a$为极限''这一陈述.

答:存在$\epsilon_0 > 0$,使得对任意$N \in \nat$,都存在$n > N$并且$|a_n - a| \geq \epsilon_0$.

顺便一提函数极限的否定定义:存在$\epsilon_0 > 0$,使得对任意$\delta > 0$,都存在$|x - x_0| < \delta$并且$|f(x_0) - y_0| \geq \epsilon_0$.

7. 设$a,b,c$是三个给定的实数,令$a_0 = a, b_0 = b, c_0 = c$,并归纳地定义
\begin{equation}
    \begin{cases}
        a_n = \displaystyle\frac{b_{n-1}+c_{n-1}}{2} \\
        b_n = \displaystyle\frac{a_{n-1}+c_{n-1}}{2} \\
        c_n = \displaystyle\frac{a_{n-1}+b_{n-1}}{2}
    \end{cases}, \quad n = 1,2,3,\cdots
\end{equation}
求证:
\begin{equation}
    \lim_{n\to\infty} a_n = \lim_{n\to\infty} b_n = \lim_{n\to\infty} c_n = \frac{1}{3} \left( a_0+b_0+c_0 \right).
\end{equation}

\begin{proof}
    对每一个$n \in \nat$,将$r_n,s_n,t_n$分别定义为$a_n,b_n,c_n$中最小的、次小的和最大的数.$a_n,b_n,c_n$收敛到同一个数等价于它们之中的最小数和最大数都收敛到同一个数,也就是说,我们要去证明$r_n,t_n$都收敛到同一个数.

由
\begin{align}
    &\mathrel{\phantom{\implies}} s_n \leq t_n \\
    &\implies r_n + s_n \leq r_n + t_n \\
    &\implies \frac{r_n+s_n}{2} \leq \frac{r_n+t_n}{2} 
\end{align}
以及
\begin{align}
    &\mathrel{\phantom{\implies}} r_n \leq s_n \\
    &\implies r_n + t_n \leq s_n + t_n \\
    &\implies \frac{r_n+t_n}{2} \leq \frac{s_n+t_n}{2}
\end{align}
得到
\begin{equation}
    \frac{r_n+s_n}{2} \leq \frac{r_n+t_n}{2} \leq \frac{s_n+t_n}{2}
\end{equation}
因此
\begin{equation}
    r_{n+1} = \frac{r_n+s_n}{2}, \; s_{n+1} = \frac{r_n+t_n}{2}, \; t_{n+1}=\frac{s_n+t_n}{2}
\end{equation}
于是
\begin{equation}
    t_{n+1} - r_{n+1} = \frac{s_n+t_n}{2} - \frac{r_n+s_n}{2} = \frac{t_n - r_n}{2}
\end{equation}
并且由此得
\begin{equation}
t_n - r_n = \frac{t_{n-1}-r_{n-1}}{2} = \frac{t_{n-2}-r_{n-2}}{4} = \cdots = \frac{1}{2^{n}}\left(t_0 - s_0\right)
\end{equation}
于是对任意$\epsilon > 0$,我们可以取$N = \lceil \log_2 \left(t_0-s_0\right) - \log_2 \epsilon \rceil$,于是当$n > N$的时候就有
\begin{equation}
    |t_n-r_n|=\frac{1}{2^n}(t_0-s_0) < \frac{1}{2^N}(t_0-s_0) \leq \epsilon
\end{equation}
为了把$r_n$也就是$t_n$的共同极限找出来,我们要寻找一个始终位于$r_n$和$t_n$中间的常数,为此,我们要去证明$\displaystyle\frac{r_0+s_0+t_0}{3}$就是这个数.首先,根据
\begin{equation}
    r_{n+1} = \frac{r_n+s_n}{2}, \; s_{n+1} = \frac{r_n+t_n}{2}, \; t_{n+1}=\frac{s_n+t_n}{2} \; (\forall n \in \integer)
\end{equation}
我们得
\begin{equation}
    r_{n+1}+s_{n+1}+t_{n+1} = \frac{2r_n+2s_n+2t_n}{2} = r_n + s_n + t_n \; (\forall n \in \integer)
\end{equation}
这说明
\begin{equation}
    r_0+s_0+t_0=r_n+s_n+t_n, \; (\forall n \in \integer)
\end{equation}
于是放缩得
\begin{equation}
    r_n \leq \frac{r_0+s_0+t_0}{3} \leq t_n, \; (\forall n \in \integer)
\end{equation}
这说明,在数轴上,不管$n$取哪一个整数,$\displaystyle\frac{r_0+s_0+t_0}{3}$这个数,总是插在$r_n$与$t_n$之间,又根据$r_n,s_n,t_n$的定义,有$r_n \leq \min \{a_n,b_n,c_n\}$以及$t_n \geq \max \{a_n, b_n, c_n \}$,于是,依几何事实,有
\begin{align}
    |a_n - \frac{r_0+s_0+t_0}{3}| < |r_n - t_n| < \epsilon, \\
    |b_n - \frac{r_0+s_0+t_0}{3}| < |r_n - t_n| < \epsilon, \\
    |c_n - \frac{r_0+s_0+t_0}{3}| < |r_n - t_n| < \epsilon
\end{align}
又由于$\displaystyle\frac{r_0+s_0+t_0}{3}=\frac{a_0+b_0+c_0}{3}$,所以
\begin{equation}
    \lim_{n\to\infty} a_n = \lim_{n\to\infty} b_n = \lim_{n\to\infty} c_n = \frac{a_0+b_0+c_0}{3}
\end{equation}
于是命题得证.
\end{proof}

\exercise
\exercise
\exercise

5. 令$\displaystyle a_n = (n!)^\frac{1}{n}$,求证:$\{ a_n \}$是一个递增数列.
\begin{proof}
猜测$\{ a_n \}$是递增的,为此,我们尝试去证$\displaystyle \frac{a_{n+1}}{a_n} > 1$,由题设:
\begin{equation}
    \frac{a_{n+1}}{a_n} = \frac{(1 \times 2 \times \cdots \times n+1)^\frac{1}{n+1}}{(1 \times 2 \times \cdots \times n)^\frac{1}{n}}
    \label{eq:a1}
\end{equation}
对式(\ref{eq:a1})左右两端同时自乘$n+1$次,得:
\begin{equation}
    \left( \frac{a_{n+1}}{a_n} \right)^{n+1} = \frac{1 \times 2 \times \cdots \times n+1}{1 \times 2 \times \cdots \times n \times (1 \times 2 \times \cdots \times n)^\frac{1}{n}} = \frac{n+1}{(1 \times 2 \times \cdots \times n)^\frac{1}{n}}
    \label{eq:a2}
\end{equation}
对式(\ref{eq:a2})左右两端同时自乘$n$次,得:
\begin{equation}
    \left(\left(\frac{a_{n+1}}{a_n} \right)^{n+1}\right)^{n} = \frac{(n+1)^n}{1 \times 2 \times \cdots \times n} = \frac{\prod_{i=1}^{n} \left( n+1 \right) }{\prod_{i=1}^{n} i} = \prod_{i=1}^n \frac{n+1}{i} > \prod_{i=1}^{n} 1 = 1
\end{equation}
也就是
\begin{equation}
   \left( \frac{a_{n+1}}{a_n} \right)^{n (n+1)} > 1
\end{equation}
从而$\displaystyle \frac{a_{n+a}}{a_n} >1$,又根据$a_n > 0$,于是数列$\{ a_n \}$是递增的.
\end{proof}

\exercise

1. 求下列极限
\begin{table}[H]
    \centering
    \begin{tabularx}{0.8\textwidth} {  >{\raggedright\arraybackslash}X >{\raggedright\arraybackslash}X  }
       (1)$\displaystyle \lim_{n\to\infty} \left(1+\frac{1}{n-2}\right)^n$; &  (2)$\displaystyle \lim_{n \to \infty} \left(1-\frac{1}{n+3}\right)^n$; \\ [1.5em]
       (3)$\displaystyle \lim_{n \to \infty} \left(\frac{1+n}{2+n}\right)^n$; & (4) $\displaystyle \lim_{n \to \infty} \left(1+\frac{3}{n}\right)^n$; \\ [1.5em]
       (5)$\displaystyle \lim_{n \to \infty} \left(1+\frac{1}{2n^2}\right)^{4n^2}$.
      \end{tabularx}
\end{table}
(1). 解:令$m = n-2$,则
\begin{align}
    \lim_{n \to \infty} \left(1+\frac{1}{n-2}\right)^n &= \lim_{m \to \infty} \left(1+ \frac{1}{m}\right)^{m+2} \\
    &= \lim_{m \to \infty} \left(1+\frac{1}{m}\right)^m \left(1+\frac{1}{m}\right)^2 \\
    &= \lim_{m \to \infty} \left(1+\frac{1}{m}\right)^m \lim_{m \to \infty} \left(1+\frac{1}{m}\right)^2 \\
    &= \mathrm{e} \cdot 1 \\
    &= \mathrm{e}
\end{align}
(2). 解:令$m=n+3$,则
\begin{align}
    \lim_{n \to \infty} \left(1-\frac{1}{n+3}\right)^n &= \lim_{m \to \infty} \left(1-\frac{1}{m}\right)^{m+2} \\
    &= \lim_{m \to \infty} \left(1-\frac{1}{m}\right)^m \lim_{m \to \infty} \left(1-\frac{1}{m}\right)^2 \\
    &= \lim_{m \to \infty} \left(1-\frac{1}{m}\right)^m \\
    &= \mathrm{e}^{-1}
\end{align}
(3). 解:
\begin{align}
    \lim_{n \to \infty} \left(\frac{1+n}{2+n}\right)^n &= \lim_{n \to \infty} \left(1 - \frac{1}{2+n}\right)^n
\end{align}
令$m = 2+n$,那么
\begin{align}
    \text{原式} &= \lim_{m \to \infty} \left(1-\frac{1}{m}\right)^{m-2} \\
    &= \lim_{m\to\infty} \frac{\left(1-\frac{1}{m}\right)^m}{\left(1-\frac{1}{m}\right)^2} \\
    &= \frac{\displaystyle\lim_{n \to \infty}\left(1-\frac{1}{m}\right)^m}{\displaystyle\lim_{m\to\infty} \left(1-\frac{1}{m}\right)^2} \\
    &= \frac{\mathrm{e}^{-1}}{1} \\
    &= \mathrm{e}^{-1}
\end{align}
(4). 解:令$m=3n$,则
\begin{align}
    \lim_{n\to\infty}\left(1+\frac{3}{n}\right)^n &= \lim_{m\to\infty}\left(1+\frac{3}{3m}\right)^{3m} \\
    &= \lim_{m\to\infty}\left(1+\frac{1}{m}\right)^{3m} \\
    &= \lim_{m\to\infty}\left(1+\frac{1}{m}\right)^{m}\left(1+\frac{1}{m}\right)^{m}\left(1+\frac{1}{m}\right)^{m} \\
    &= \lim_{m\to\infty}\left(1+\frac{1}{m}\right)^{m} \lim_{m\to\infty}\left(1+\frac{1}{m}\right)^{m}\lim_{m\to\infty}\left(1+\frac{1}{m}\right)^{m} \\
    &= \mathrm{e}\cdot\mathrm{e}\cdot\mathrm{e}\\
    &=\mathrm{e}^3
\end{align}
(5). 解:令$m=2n^2$,则
\begin{align}
    \lim_{n\to\infty} \left(1+\frac{1}{2n^2}\right)^{4n^2} &= \lim_{m\to\infty}\left(1+\frac{1}{m}\right)^{2m} \\
    &= \lim_{m\to\infty}\left(1+\frac{1}{m}\right)^{m} \left(1+\frac{1}{m}\right)^{m} \\
    &= \lim_{m\to\infty}\left(1+\frac{1}{m}\right)^{m} \lim_{m\to\infty}\left(1+\frac{1}{m}\right)^{m} \\
    &=\mathrm{e} \cdot \mathrm{e} \\
    &=\mathrm{e}^2
\end{align}

2. 设$k \in \mathbb{N}^\star$,求证:$\displaystyle \lim_{n \to \infty} (1+\frac{k}{n})^n = e^k$.
\begin{proof}
采用数学归纳法.当$k=1$时:
\begin{equation}
    \lim_{n\to \infty}(1+\frac{k}{n})^n = \lim_{n \to \infty}(1+\frac{1}{n})^n = e = e^1 = e^k
\end{equation}
因此当$k=1$时命题成立.现在假设当$k = m-1, \, (m-1\in \mathbb{N}^\star)$时命题成立.那么有
\begin{align}
    &\phantom{=} \lim_{n \to \infty} (1+\frac{m}{n})^n \\
    (\text{令} \; n = m t) \quad & = \lim_{m \to \infty} (1 + \frac{1}{t})^{m t} \\
    &= \lim_{t \to \infty} (1+\frac{1}{t})^{(m-1) t} (1+\frac{1}{t})^t \\
    &= \left( \lim_{t \to \infty} (1+\frac{1}{t})^{(m-1) t} \right) \cdot \left( \lim_{t \to \infty} (1+\frac{1}{t})^t \right)\\ 
    (\text{令} \; t = \frac{s}{m-1}) \quad &= \left( \lim_{s \to \infty} (1 + \frac{m-1}{s})^s \right) \cdot e \\
    (\text{由归纳假设} ) \quad &= \mathrm{e}^{m-1} \cdot \mathrm{e} \\
    &= \mathrm{e}^m
\end{align}
这里证明了对于$k = m$命题成立.从而根据数学归纳法原理,对任意$k \in \mathbb{N}^\star$命题都成立.
\end{proof}

3. 求证:$\displaystyle \{ (1+\frac{1}{n})^n \}$是严格递增数列.
\begin{proof}
由算式---几何平均不等式:
\begin{equation}
(1+\frac{1}{n})^n = (\frac{n+1}{n})^n = 1 \cdot \prod_{i=1}^{n} \frac{n+1}{n} < \left( \frac{1 + n(\frac{n+1}{n})}{n+1} \right)^{n+1} = \left( 1 + \frac{1}{n+1} \right)^{n+1}
\end{equation}
从而$\displaystyle \{ (1+\frac{1}{n})^n\}$严格递增.
\end{proof}
4. 求证:$\displaystyle \{ (1+\frac{1}{n})^{n+1}\}$是严格递减数列.
\begin{proof}
令$\displaystyle q_n = (1 + \frac{1}{n})^{n+1}$,原命题等价于$\{ \displaystyle \frac{1}{q_n}\}$是严格递增数列.由算式---几何平均不等式
\begin{equation}
    \frac{1}{q_n} = \left( \frac{n}{n+1} \right)^{n+1} = 1 \cdot \prod_{i = 1}^{n+1} \frac{n}{n+1} < \left( \frac{1 + (n+1) \frac{n}{n+1}}{n+2} \right)^{n+2} = \left( \frac{1+n}{n+2}\right)^{n+2} = \frac{1}{q_{n+1}}
\end{equation}
从而数列$\{ \displaystyle \frac{1}{q_n} \}$严格单调递增,因此数列$\{ q_n \}$也就是$\{ \displaystyle (1 + \frac{1}{n})^{n+1}\}$严格单调递减.
\end{proof}
5. 证明不等式:
\begin{equation}
    \left( 1+\frac{1}{n} \right)^n < \mathrm{e} < \left( 1 + \frac{1}{n} \right)^{n+1}.
\end{equation}
\begin{proof}
令$\displaystyle p_n = \left( 1+\frac{1}{n} \right)^n, \, q_n = \left( 1+\frac{1}{n} \right)^{n+1}$.先证明不等式的左半部分.

采用数学归纳法,当$n=1$时:
\begin{equation}
    p_1 = (1+\frac{1}{1})^1 = 2 < e
\end{equation}
显然成立.假设当$n=m-1, \, (m-1 \in \mathbb{N}^\star)$时命题成立,我们去证明当$p_{m-1} < \mathrm{e}$时有$p_m < \mathrm{e}$,为此,采用反证法,假设对某个$m$,当$p_{m-1} < \mathrm{e}$时有$p_m > \mathrm{e}$,记$\epsilon_0 = \min \{ |p_{m-1} - \mathrm{e}|, |p_m - \mathrm{e}| \}$,显然$\epsilon_0 > 0$,由数列$\{ p_n \}$的严格单调性,对任意$n \in \mathbb{N}^\star$都有$|p_n - \mathrm{e}| \geq \epsilon_0 > 0$,但是由于数列$\{ p_n \}$的极限是$\mathrm{e}$,所以由极限的定义,对任意$\epsilon > 0$,存在$N_0 \in \mathbb{N}^\star$,使得当$n > N_0$时有$|p_n - e| < \epsilon$,由此推出矛盾,故对于$p_m > e$的假设是错误的.又根据数列$\{ p_n \}$的严格单调性,$p_m \neq e$,因此只能是$p_m < e$. 从而当$p_{m-1} < e$成立时$p_m < e$也成立,根据数学归纳法原理,对任意$n \in \mathbb{N}^\star$,$p_n < e$都成立.

利用数列$\{ q_n \}$的严格单调性质以及$\{ q_n \}$的极限也是$\mathrm{e}$的事实可类似地证明$e < q_n, \, (n \in \mathbb{N}^\star)$.
\end{proof}
6.利用对数函数$\ln \, x$的严格递增性质,证明:
\begin{equation}
    \frac{1}{n+1} < \ln \, \left( 1+\frac{1}{n} \right) < \frac{1}{n}
\end{equation}
对一切$n \in \mathbb{N}^\star$成立.
\begin{proof}
先证不等式的左半部分.要证$\displaystyle \frac{1}{n+1} < \ln \, \left( 1+\frac{1}{n} \right)$等价于去证$\displaystyle 1 < (n+1) \ln \, \left( 1+\frac{1}{n} \right)$,利用对数函数$\ln \, x$的严格单调递增性质,也就是等价于$\ln \, \mathrm{e} < \ln \, \left(\left( 1 + \frac{1}{n} \right)^{n+1} \right)$,由题5得到的结论:$\displaystyle \mathrm{e} < \left( 1+\frac{1}{n} \right)^{n+1}$(对任意$n \in \mathbb{N}^\star$),以及对数函数$\ln \, x$的严格单调性质,可知$\ln \, \mathrm{e} < \ln \, \left( \left( 1 + \frac{1}{n} \right)^{n+1} \right)$成立,因此$\displaystyle 1 < (n+1) \ln \, \left( 1+\frac{1}{n} \right)$成立,因此$\displaystyle \frac{1}{n+1} < \ln \, \left( 1 + \frac{1}{n} \right)$成立,即,要证的不等式的左半部分成立.

再用类似的方法来证明不等式的右半部分.由题5得到的结论:$\displaystyle \left( 1 + \frac{1}{n} \right)^n < \mathrm{e}$(对任意$n \in \mathbb{N}^\star$),对这个不等式两端取对数并且利用对数函数$\ln \, x$的单调递增性质,得到$\displaystyle n \ln \, \left( 1 + \frac{1}{n} \right) < \ln \, \mathrm{e} = 1$,再对这个不等式两端除以$n$即得$\displaystyle \ln \, \left( 1 + \frac{1}{n}\right) < \frac{1}{n}$.于是命题得证.
\end{proof}
7. 设$n \in \mathbb{N}^\star$且$k = 1,2,\cdots$. 证明不等式:
\begin{equation}
    \frac{k}{n+k} < \ln \, \left( 1+\frac{k}{n} \right) < \frac{k}{n}
\end{equation}
思路:回顾题4、题5以及题6的证明过程,我们受到启发:要证$\displaystyle \frac{k}{n+k} < \ln \, \left( 1 + \frac{k}{n}\right)$成立就要去证$\displaystyle \ln \, \left( \mathrm{e}^k \right) < (n+k) \ln \, \left( 1 + \frac{k}{n}\right)$成立,相应地就要去证$\displaystyle \mathrm{e}^k < \left(1+\frac{k}{n}\right)^{n+k}$并且要证数列$\displaystyle \{ \left(1+\frac{k}{n}\right)^{n+k} \}$严格单调递减,而对于数列$\displaystyle \{ \left( 1 + \frac{k}{n} \right)^{n+k}\}$的单调递减性的证明可以参考和借鉴题4的方法.
\begin{proof}
先证明数列$\displaystyle \{ \left(1+\frac{k}{n}\right)^{n+k}\}$是严格单调递减的.令$\displaystyle p_n =  \left(1+\frac{k}{n}\right)^{n+k}$,这等价于证明数列$\displaystyle \{ \frac{1}{p_n} \}$是严格单调递减的.利用算式---几何均值不等式,我们有:
\begin{equation}
    \frac{1}{p_n} =  \left(\frac{n}{n+k}\right)^{n+k} = 1 \cdot \prod_{i=1}^{n+k} \left(\frac{n}{n+k}\right) <(\frac{1+(n+k) \frac{n}{n+k}}{n+k+1})^{n+k+1} = \frac{1}{p_{n+1}}
\end{equation}
这就证明了数列$\displaystyle \{ \frac{1}{p_n}\}$是严格单调递减的,从而数列$\{p_n \}$也就是$\{ \displaystyle \left(1+\frac{k}{n}\right)^{n+k}\}$是严格单调递增的.又根据
\begin{equation}
    \lim_{n \to \infty} \left(1+\frac{k}{n}\right)^{n+k} = \mathrm{e}^k
\end{equation}
以及数列$\{ \displaystyle \left( 1 + \frac{k}{n} \right)^{n+k}\}$的严格单调递减性,以及$\mathrm{e}^k < p_1 = \displaystyle \left(1+\frac{k}{n}\right)^{n+k}$的事实,可证明
\begin{equation}
    \mathrm{e}^k < \left( 1 + \frac{k}{n}\right)^{n+k}
    \label{eq:ek}
\end{equation}
对每一个$k, n\in \mathrm{N}^{\star}$都成立.对式(\ref{eq:ek})两边取对数,再根据对数函数$\ln \, x$的严格单调递增性,得到
\begin{equation}
    k < (n+k) \ln \, \left(1+\frac{k}{n}\right)
\end{equation}
也就是
\begin{equation}
    \frac{k}{n+k} < \ln \, \left(1+\frac{k}{n}\right)
\end{equation}
从而要证明的不等式的左半部分成立.采用类似的方法可类似地证明不等式的右半部分成立.
\end{proof}
8. 对$n \in \mathbb{N}^\star$,求证:
\begin{equation}
    \frac{1}{2} + \frac{1}{3} + \cdots + \frac{1}{n} < \ln \, \left( n+1 \right) < 1 + \frac{1}{2} + \cdots + \frac{1}{n}.
\end{equation}
\begin{proof}
利用题6的结论:$\displaystyle \frac{1}{n+1} < \ln \, \left( 1 + \frac{1}{n} \right) = \ln \, (n+1) - \ln \, n$,我们有
\begin{equation}
\sum_{i = 1}^{n} \frac{1}{i + 1} < \sum_{i=1}^n \ln \, \left( i+1 \right) - \ln \, i = \sum_{i=1}^n \ln \, \left( i + 1\right) - \sum_{i=1}^n \ln \, i = \sum_{i=1}^{n+1} \ln \, i - \sum_{i=1}^{n} \ln \, i = \ln \, \left( n+1 \right)
\end{equation}
于是左半部分得证.再利用题6的结论:$\displaystyle \ln \, \left( n+1 \right) - \ln \, n = \ln \, \left( 1 + \frac{1}{n} \right) < \frac{1}{n}$,可以得到
\begin{equation}
\sum_{i=1}^n \ln \, \left( i+1 \right) - \ln \, i = \sum_{i=1}^n \ln \, \left( i+1 \right) - \sum_{i=1}^n \ln \, i = \sum_{i=1}^{n+1} \ln \, i - \sum_{i=1}^n \ln \, i = \ln \, \left(n+1 \right) < \sum_{i=1}^n \frac{1}{i}
\end{equation}
于是右半部分得证.
\end{proof}
9. 令
\begin{equation}
    x_n = 1 + \frac{1}{2} + \cdots + \frac{1}{n} - \ln \, \left( n+1 \right) \quad (n \in \mathbb{N}^\star).
\end{equation}
证明:$\displaystyle \lim_{n \to \infty} x_n $存在,此极限记为$\gamma$,叫做Euler(欧拉,1707\textasciitilde 1783)常数.
\begin{proof}
先证单调性.
\begin{align}
    x_{n+1} &= 1 + \frac{1}{2} + \cdots + \frac{1}{n} + \frac{1}{n+1} - \ln \, \left(n+2\right) \\
    &< 1 + \frac{1}{2} + \cdots + \frac{1}{n} + \ln \, \left( 1 + \frac{1}{n} \right) - \ln \, \left( n+2\right) \\
    &= 1 + \frac{1}{2} + \cdots + \frac{1}{n} + \ln \, \left( n+1 \right) - \ln \, n - \ln \, \left( n+2 \right) \\
    &< 1 + \frac{1}{2} + \cdots + \frac{1}{n} + \ln \, \left( n+1 \right) \\
    &= x_n \quad (n \in \mathbb{N}^\star)
\end{align}
再证有界性.利用题8的结论,有
\begin{equation}
    0 < 1 + \frac{1}{2} + \cdots + \frac{1}{n} - \ln \left( n + 1\right) 
\end{equation}
从而数列$\{ x_n \}$既单调又有界,所以数列$\{ x_n \}$的极限存在.
\end{proof}
10. 利用第9题,证明:
\begin{equation}
    1+\frac{1}{2} + \frac{1}{3} + \cdots + \frac{1}{n} = \ln \, n + \gamma + \epsilon_n,
\end{equation}
其中$\displaystyle \lim_{n \to \infty} \epsilon_n = 0$.
\begin{proof}
令$\displaystyle p_n = 1 + \frac{1}{2} + \cdots + \frac{1}{n} - \ln \, n$,我们先来证明数列$\{ p_n \}$的极限存在.先证单调性:
\begin{align}
    p_{n+1} &= 1 + \frac{1}{2} + \cdots + \frac{1}{n} + \frac{1}{n+1} - \ln \, \left( n + 1 \right) \\
    &< 1 + \frac{1}{2} + \cdots + \frac{1}{n} + \ln \, \left( 1 + \frac{1}{n} \right) - \ln \, \left(n+1\right) \\
    &= 1 + \frac{1}{2} + \cdots + \frac{1}{n} - \ln \, n \\
    &= p_n \quad (n \in \mathbb{N}^\star)
\end{align}
再证有界性.利用题8的结论:$\displaystyle \ln \, n < \ln \, \left( n+1 \right) < 1 + \frac{1}{2} + \cdots +\frac{1}{n}$,于是
\begin{equation}
    p_n = 1 + \frac{1}{2} +\cdots + \frac{1}{n} - \ln \, n > 0
\end{equation}
从而数列$\{ p_n \}$有界,因此数列$\{ p_n \}$的极限存在,即$ \displaystyle \lim_{n \to \infty} p_n$存在.令$\displaystyle x_n = 1 + \frac{1}{2} + \cdots + \frac{1}{n} - \ln \, \left( n + 1\right)$,我们来证$\displaystyle \lim_{n \to \infty} p_n$的确等于$\displaystyle \lim_{n \to \infty} x_n$:
\begin{align}
    \lim_{n \to \infty} p_n &= \lim_{n \to \infty} \left( x_n + \ln \, \left( n+1 \right) - \ln \, n \right) \\
    &= \lim_{n \to \infty} x_n + \lim_{n \to \infty} \left( \ln \, \left(n+1\right) - \ln \, n \right) \\
    &= \lim_{n \to \infty} x_n + 0 \\
    &= \lim_{n \to \infty} x_n
\end{align}
其中$\displaystyle \lim_{n \to \infty} \left( \ln \, \left(n+1\right) - \ln \, n\right) = 0$的条件是从题6的结论:
\begin{equation}
\frac{1}{n+1} < \ln \, \left(1 + \frac{1}{n}\right) = \ln \, \left(n+1\right) - \ln \, n < \frac{1}{n}
\end{equation}
得到的.令$\displaystyle \epsilon_n = 1 + \frac{1}{2} +\cdots + \frac{1}{n} - \ln \, n - \gamma$,令$\displaystyle \gamma = \lim_{n \to \infty} x_n$,我们要证$\displaystyle \lim_{n \to \infty} \epsilon_n = 0$:
\begin{align}
    \lim_{n \to \infty} \epsilon_n &= \lim_{n \to \infty} \left(1 + \frac{1}{2} + \cdots + \frac{1}{n} - \ln \, n - \gamma \right) \\
    &= \lim_{n \to \infty} \left( p_n - \gamma \right) \\
    &= \lim_{n\to \infty} p_n - \lim_{n\to \infty} \gamma \\
    &= \left( \lim_{n\to \infty} p_n \right) - \gamma \\
    &= \lim_{n\to \infty} p_n - \lim_{n \to \infty} x_n \\ 
    &= 0
\end{align}
命题得证.
\end{proof}
11. 证明不等式:
\begin{equation}
    \left(\frac{n+1}{\mathrm{e}}\right)^n < n! < \mathrm{e} \left(\frac{n+1}{\mathrm{e}}\right)^{n+1}
\end{equation}
\begin{proof}
先证不等式左半部分.先将要证的不等式改写成便于证明的等价形式:
\begin{align}
    \left( \frac{n+1}{\mathrm{e}}\right)^n < n! &\iff n \left(\ln \, \left(n+1\right) - 1\right) < \sum_{i=1}^n \ln \, i \\
    &\iff \sum_{i=1}^n \left(\ln \, \left(n+1\right) - \ln \, i\right) < n \label{ieq:1}
\end{align}
我们现采用数学归纳法证明不等式(\ref{ieq:1})成立,显然当$n=1$时不等式(\ref{ieq:1})成立,假设当$n = m-1$时($m-1 \in \mathbb{N}^\star$)不等式(\ref{ieq:1})成立.那么当$n=m$时:
\begin{align}
    \sum_{i=1}^m \left(\ln\,\left(m+1\right) - \ln \, i\right) &= \sum_{i=1}^{m-1} \left(\ln \, m - \ln \, i\right) + \sum_{i=1}^m \left(\ln\,\left(m+1\right) - \ln \, i\right) - \sum_{i=1}^{m-1} \left(\ln \, m - \ln \, i\right) \\
    &= \left(\sum_{i=1}^{m-1} \ln \, m - \ln \, i \right) + m \left( \ln \, \left(m+1\right) - \ln \, m \right) \\
    &< m-1 + m \cdot \frac{1}{m} \\
    &= 1
\end{align}
由此可见当$n=m$时不等式(\ref{ieq:1})也成立,根据数学归纳法原理,不等式(\ref{ieq:1})对每一个$n\in\mathbb{N}^\star$都成立,从而不等式(\ref{ieq:1})的等价形式
\begin{equation}
\left(\frac{n+1}{\mathrm{e}}\right)^n < n!
\end{equation}
对每一个$n\in\mathbb{N}^\star$都成立.

现在再去证明不等式右半部分,先将它改写成便于证明的等价形式:
\begin{align}
    n! < \mathrm{e}\left(\frac{n+1}{\mathrm{e}}\right)^{n+1} &\iff \sum_{i=1}^n \ln \, i < (n+1)\ln \, \left(n+1\right) - n \\
    &\iff n < \left( \sum_{i=1}^n \ln \, \left(n+1\right) - \ln \, i \right) + \ln \, \left(n+1\right)
\end{align}
令
\begin{equation}
    p_n = \left( \sum_{i=1}^n \ln \, \left(n+1\right) - \ln \, i \right) + \ln \, \left(n+1\right)
\end{equation}
我们现在采用数学归纳法证明$n < p_n$对每一个$n\in\mathbb{N}^\star$成立.当$n=1$时:
\begin{align}
    p_n = p_1 = 2 \ln \, 2 = 2 \left( \ln \, 2 - \ln \, 1\right) = 2 \ln \, \left( 1 + \frac{1}{1} \right) > 2 \cdot \frac{1}{1+1} = 1 = n
\end{align}
从而当$n=1$时$n < p_n$成立,现在假设:当$n=m-1, \, (m-1 \in \mathbb{N}^\star)$时,$n < p_n$成立.于是
\begin{align}
    p_m &= p_{m-1} + p_{m} - p_{m-1} \\
    &= p_{m-1} + (m+1) \left( \ln \, \left(m+1\right) - \ln \, m \right) \\
    &= p_{m-1} + (m+1) \ln \, \left(1 + \frac{1}{m}\right) \\
    &> m-1 + (m+1) \frac{1}{m+1} \\
    &= m
\end{align}
从而对于$n=m$,$n < p_n$也成立,根据数学归纳法原理,$n < p_n$对每一个$n \in \mathbb{N}^\star$都成立,也就是
\begin{equation}
    n < \left( \sum_{i=1}^n \ln \, \left(n+1\right) - \ln \, i \right) + \ln \, \left(n+1\right)
\end{equation}
以及它的等价形式
\begin{equation}
    n! < \mathrm{e}\left(\frac{n+1}{\mathrm{e}}\right)^{n+1}
\end{equation}
对每一个$n \in \mathbb{N}^\star$都成立.
\end{proof}
12. 证明:利用题11的结论:
\begin{align}
    &\phantom{\implies} \left(\frac{n+1}{\mathrm{e}}\right)^n < n! < \left(\frac{n+1}{\mathrm{e}}\right)^{n+1} \\
    &\implies \frac{1}{\mathrm{e}^n} < \frac{n!}{(n+1)^n} < \frac{n+1}{\mathrm{e}^n} \\
    &\implies \frac{1}{\mathrm{e}} < \frac{(n!)^\frac{1}{n}}{n+1} < \frac{(n+1)^\frac{1}{n}}{\mathrm{e}} \\
    &\implies \lim_{n\to\infty} \frac{1}{\mathrm{e}} < \lim_{n \to \infty} \frac{(n!)^\frac{1}{n}}{n+1} < \lim_{n\to\infty} \frac{(n+1)^\frac{1}{n}}{\mathrm{e}} \\
    &\implies \lim_{n\to\infty} \frac{(n!)^\frac{1}{n}}{n+1} = \frac{1}{\mathrm{e}}
\end{align}
再次利用题11的结论:
\begin{align}
    &\phantom{\implies} \left(\frac{n+1}{\mathrm{e}}\right)^n < n! < \left(\frac{n+1}{\mathrm{e}}\right)^{n+1} \\
    &\implies \frac{1}{\mathrm{e}^n} < \frac{n!}{(n+1)^n} < \frac{n+1}{\mathrm{e}^n} \\
    &\implies \frac{1}{\mathrm{e}} < \frac{(n!)^\frac{1}{n}}{n+1} < \frac{(n+1)^\frac{1}{n}}{\mathrm{e}} \\
    &\implies \frac{1}{n \mathrm{e}} < \frac{(n!)^\frac{1}{n}}{n(n+1)} < \frac{(n+1)^\frac{1}{n}}{\mathrm{e}} \\
    &\implies \lim_{n\to\infty} \frac{1}{n \mathrm{e}} = \lim_{n\to\infty} \frac{(n!)^\frac{1}{n}}{n(n+1)} = \lim_{n\to\infty} \frac{(n+1)^\frac{1}{n}}{\mathrm{e}} = 0 \\
    &\implies \lim_{n\to\infty} \left(\frac{(n!)^\frac{1}{n}}{n} - \frac{(n!)^\frac{1}{n}}{n+1}\right) = 0 \\
    &\implies \lim_{n\to\infty} \frac{(n!)^\frac{1}{n}}{n} = \lim_{n\to\infty} \frac{(n!)^\frac{1}{n}}{n+1} + \lim_{n\to\infty} \left(\frac{(n!)^\frac{1}{n}}{n} - \frac{(n!)^\frac{1}{n}}{n+1}\right) = \frac{1}{\mathrm{e}} + 0 = \frac{1}{\mathrm{e}}
\end{align}
这就证明了数列$\displaystyle \{\frac{(n!)^\frac{1}{n}}{n}\}$的极限是$\displaystyle \frac{1}{\mathrm{e}}$.

\exercise
1. 对任意给定的$\epsilon > 0$,存在$N \in \mathbb{N}^\star$,当$n > N$时,有
\begin{equation}
    |a_n - a_N|<\epsilon
\end{equation}
问$\{a_n\}$是不是基本列?

思路:在实数轴上,所有$a_N$附件的项都落在以它为中心的半径不超过$\epsilon$的一个范围内,那么,假设$a_n, a_m$也是在这个范围内,$a_n$与$a_m$的距离只会更加近,因为这个$\epsilon$可以任意小,从而这个以$a_N$为中心的范围可以是任意地``拥挤''.由此我们猜测$\{a_n\}$是基本列,去证$|a_m - a_n|<\epsilon$.

\begin{proof}
设$m, n>N$,那么由题设,
\begin{equation}
    |a_m - a_N|<\epsilon, \; |a_n - a_N| <\epsilon
\end{equation}
也就是
\begin{align}
    a_N - \epsilon < a_m < a_N + \epsilon \\
    a_N - \epsilon < a_n < a_N + \epsilon 
\end{align}
从而
\begin{equation}
    |a_m - a_n| < 2\epsilon
\end{equation}
由于$2\epsilon$可以是任意的正数,所以按定义,$\{a_n\}$是基本列.
\end{proof}

2. (1)数列$\{a_n\}$满足
\begin{equation}
    |a_{n+p}-a_n|\leq\frac{p}{n}
\end{equation}
且对一切$n,p\in\mathbb{N}^\star$成立,问$\{ a_n\}$是不是基本列?

(2) 当$|a_{n+p} - a_n| \leq p/n^2$时,上述结论又如何?

思路:对于题(1),当$n\to\infty$时$\displaystyle\frac{p}{n}$显然趋于$0$,但我们从直觉上能够感觉到这个趋于$0$的速度还不够快,由此猜想$\{a_n\}$应该不是基本列,可以去举反例.
\begin{proof}
(1) 考虑
\begin{equation}
    a_n = 1 + \frac{1}{2} + \frac{1}{3} + \cdots + \frac{1}{n} 
\end{equation}
于是
\begin{align}
    a_{n+p} - a_n &= \frac{1}{n+1} + \frac{1}{n+2} + \cdots + \frac{1}{n+p}\\
    &\leq \frac{1}{n+1} + \frac{1}{n+1} + \cdots + \frac{1}{n+1} \\
    &= \frac{p}{n+1} < \frac{p}{n}
\end{align}
数列$\{a_n\}$满足$|a_{n+p}-a_n|\leq \displaystyle\frac{p}{n}, \, (n,p\in\mathbb{N}^\star)$,但是$\{a_n\}$显然不是基本列.

(2) 由题设,有
\begin{align}
    |a_{n+p} - a_n| &\leq |a_{n+1} - a_n| + |a_{n+2} - a_{n+1}| + \cdots + |a_{n+p} - a_{n+p-1}| \\
    &\leq \frac{1}{n^2} + \frac{1}{(n+1)^2} + \cdots + \frac{1}{(n+p-1)^2}
\end{align}
令
\begin{equation}
    b_n = 1 + \frac{1}{2^2} + \frac{1}{3^2} + \cdots + \frac{1}{n^2} = \sum_{i=1}^n \frac{1}{i^2}
\end{equation}
于是
\begin{equation}
    |a_{n+p} - a_n| \leq b_{n+p-1} - b_n = |b_{n+p-1} - b_n| < |b_{n+p} - b_n|
\end{equation}
显然,数列$\{b_n\}$是收敛的,因此数列$\{ b_n \}$是基本列,因此,对任意$\epsilon > 0$,存在$N \in \mathbb{N}^\star$,使得当$n > N$时,
\begin{equation}
    |a_{n+p} - a_n| < |b_{n+p} - b_n| < \epsilon
\end{equation}
对任意$n, p \in \mathbb{N}^\star$成立.所以,由定义,$\{ a_n \}$是基本列.
\end{proof}

3. 证明下列数列收敛:

\begin{table}[H]
    \centering
    \begin{tabularx}{\textwidth} { >{\raggedright\arraybackslash}X }
    (1) $\displaystyle a_n = 1-\frac{1}{2^2}+\frac{1}{3^2}-\cdots+(-1)^{n-1}\frac{1}{n^2}\, (n\in\mathbb{N}^\star)$; \\ [0.7em]
    (2) $\displaystyle b_n = a_0 + a_1 q + \cdots + a_n q^n \, (n \in \mathbb{N}^\star)$,其中$\{a_0,a_1,\cdots\}$为一有界数列,$|q|<1$; \\ [0.7em]
    (3) $a_n = \displaystyle \sin x + \frac{\sin 2x}{2^2} + \cdots + \frac{\sin nx}{n^2}\, (n \in \mathbb{N}^\star, x \in \mathbb{R})$; \\ [0.7em]
    (4) $a_n = \displaystyle \frac{\sin 2x}{2(2+\sin 2x)} + \frac{\sin 3x}{3(3+\sin 3x)} + \cdots + \frac{\sin nx}{n(n+\sin nx)} \, (n\in\mathbb{N}^\star, x \in \mathbb{R})$.
    \end{tabularx}
\end{table}

\begin{proof}
(1). 将$a_n$写成
\begin{equation}
    a_n = \sum_{i=1}^n (-1)^{i-1} \frac{1}{i^2}
\end{equation}
于是
\begin{align}
    |a_{n+p}-a_n| = \Bigg\lvert\sum_{i=n+1}^{n+p} (-1)^{i-1}\frac{1}{i^2} \Bigg\rvert < \Bigg\lvert \sum_{i=n+1}^{n+p} \frac{1}{i^2} \Bigg\rvert
\end{align}
令
\begin{equation}
    b_n = \sum_{i=1}^n \frac{1}{i^2}
\end{equation}
于是
\begin{equation}
    |a_{n+p}-a_n|<b_{n+p} - b_n =|b_{n+p}-b_n|
\end{equation}
并且数列$\{ b_n \}$是基本列,从而对任意$\epsilon > 0$,存在$N \in \mathbb{N}^\star$,使得当$n > N$时,
\begin{equation}
    |a_{n+p}-a_n|<|b_{n+p}-b_n|<\epsilon
\end{equation}
对任意$n,p \in \mathbb{N}^\star$都成立.于是,按定义,数列$\{a_n\}$是基本列,从而数列$\{a_n\}$收敛.
\end{proof}

\begin{proof}
(2). 设$l$是$\{a_n\}$的下界,$u$是$\{a_n\}$的上界,令$a = \max\{ |l|, |u| \}$:
\begin{align}
    |b_{n+p} - b_n| &= \Bigg\lvert \sum_{i=0}^{n+p} a_i q^i - \sum_{i=0}^{n} a_i q^i \Bigg\rvert \\
    &= \Bigg\lvert \sum_{i=n+1}^{n+p} a_i q^i \Bigg\rvert < a |q|^{n+1} \sum_{i=0}^{p-1} |q|^i = |q|^{n+1} \frac{a(1-|q|^p)}{1-|q|} \\
    &< |q|^{n+1} \frac{a}{1-|q|}
\end{align}
那么,对任意$\epsilon > 0$,我们可以取
\begin{equation}
    N = \lceil \log_{|q|} \left(\frac{\epsilon(1-|q|)}{a}\right) - 1\rceil
\end{equation}
则当$n > N$时
\begin{align}
    |b_{n+p}-b_n| &< |q|^{n+1}\frac{a}{1-|q|} < |q|^{N+1} \frac{a}{1-|q|} \\
    &< \frac{\epsilon(1-|q|)}{a} \frac{a}{1-|q|} = \epsilon
\end{align}
对任意$n, p\in\mathbb{N}^\star$都成立,于是由定义,$\{b_n\}$是一个基本列,于是$\{b_n\}$收敛.
\end{proof}
\begin{proof}
(3). 设$n,p\in\mathbb{N}^\star$:
\begin{align}
    |a_{n+p}-a_n| &= \Bigg\lvert \sum_{i=1}^{n+p} \frac{\sin ix}{i^2} - \sum_{i=1}^n \frac{\sin ix}{i^2} \Bigg\rvert \\
    &= \Bigg\lvert \sum_{i=n+1}^{n+p} \frac{\sin ix}{i^2} \Bigg\rvert \\
    &< \sum_{i=n+1}^{n+p} \frac{1}{i^2}
\end{align}
令
\begin{equation}
    b_n = \sum_{i=1}^n \frac{1}{i^2}
\end{equation}
易知数列$\{b_n\}$是收敛数量,故$\{b_n\}$是基本列,于是,对任意正数$\epsilon > 0$,存在$N \in \mathbb{N}^\star$,当$n>N$时,
\begin{equation}
    |b_{n+p}-b_{n}|<\epsilon
\end{equation}
对任意$n,p\in\mathbb{N}^\star$成立,又由于
\begin{equation}
    |a_{n+p}-a_n|<\sum_{i=n+1}^{n+p} \frac{1}{i^2}= |b_{n+p}-b_n|
\end{equation}
所以同样也有
\begin{equation}
    |a_{n+p}-a_n|<|b_{n+p}-b_n|<\epsilon
\end{equation}
对任意$n,p\in\mathbb{N}^\star$成立.所以由定义$\{a_n\}$是基本列,所以$\{a_n\}$收敛.
\end{proof}
\begin{proof}
(4). 将$a_n$写作
\begin{equation}
    a_n = \sum_{i=2}^n \frac{\sin ix}{i(i+\sin ix)}
\end{equation}
于是,对于$n,p\in\mathbb{N}^\star, n \geq 2$,有
\begin{align}
    |a_{n+p}-a_n| &= \Bigg\lvert \sum_{i=2}^{n+p}\frac{\sin ix}{i(i+\sin ix)} - \sum_{i=2}^n \frac{\sin ix}{i(i+\sin ix)} \Bigg\rvert \\
    &= \Bigg\lvert \sum_{i=n+1}^{n+p} \frac{\sin ix}{i(i+\sin ix)} \Bigg\rvert = \Bigg\lvert \sum_{i=n+1}^{n+p} \frac{\sin ix}{i^2 + i \sin ix} \Bigg\rvert \\
    &< \sum_{i=n+1}^{n+p} \bigg\lvert \frac{\sin ix}{i^2 + i \sin ix} \bigg\rvert < \sum_{i=n+1}^{n+p}\frac{1}{i^2-i}  = \sum_{i=n+1}^{n+p} \frac{1}{i(i-1)} \\
    &< \sum_{i=n+1}^{n+p}\frac{1}{(i-1)^2}
\end{align}
令
\begin{equation}
    b_n = 1+\frac{1}{2^2}+\cdots+\frac{1}{n^2}=\sum_{i=1}^n\frac{1}{i^2}
\end{equation}
则
\begin{equation}
    |a_{n+p}-a_n|<\sum_{i=n+1}^{n+p} = |b_{n+p-1}-b_{n}|
\end{equation}
由于数列$\{b_n\}$收敛,所以$\{b_n\}$是基本列,所以,对任意$\epsilon>0$,存在$N\in\mathbb{N}^\star$,使得当$n>N$时,
\begin{equation}
    |a_{n+p}-a_n|<|b_{n+p-1}-b_n|<\epsilon
\end{equation}
对任意$p\in\mathbb{N}^\star$都成立,从而按照定义,$\{a_n\}$是基本列,于是$\{a_n\}$收敛.
\end{proof}

4. 设数列
\begin{equation}
    \{|a_2-a_1|+|a_3-a_2|+\cdots+|a_n-a_{n-1}|\}
\end{equation}
有界,求证$\{a_n\}$收敛.
\begin{proof}
令
\begin{equation}
    b_n = |a_2-a_1|+|a_3-a_2|+\cdots+|a_n-a_{n-1}| =\sum_{i=2}^n |a_{i}-a_{i-1}|
\end{equation}
于是
\begin{align}
    b_{n+1}-b_n &= \sum_{i=2}^{n+1} |a_i - a_{i-1}| -\sum_{i=2}^{n} |a_i-a_{i-1}| \\
    &= |a_{n+1}-a_{n}| > 0
\end{align}
从而,数列$\{b_n\}$,也就是数列$\{|a_2-a_1|+\cdots+|a_{n}-a_{n-1}|\}$是单调递增的,又由题设知数列$\{b_n\}$有界,所以数列$\{b_n\}$收敛.

现在我们来证明数列$\{a_n\}$是一个基本列:
\begin{align}
    |a_{n+p}-a_n| &\leq |a_{n+1}-a_n|+|a_{n+2}-a_{n+1}|+\cdots+|a_{n+p}-a_{n+p-1}| \\
    &= b_{n+p} - b_{n-1} = b_{n+p} - b_n + b_{n} - b_{n-1} = |b_{n+p}-b_n|+|b_n-b_{n-1}|
\end{align}
由于数列$\{b_n\}$收敛,所以$\{b_n\}$是基本列,所以,对任意$\epsilon > 0$,存在$N\in\mathbb{N}^\star$,使得当$n > N$时
\begin{equation}
    |b_{n+p}-b_n|<\epsilon, \; |b_n-b_{n-1}|<\epsilon
\end{equation}
对任意$p \in \mathbb{N}^\star$都成立,而又因为$|a_{n+p}-a_n|<|b_{n+p} - b_n|+|b_{n}-b_{n-1}|$,所以
\begin{equation}
    |a_{n+p}-a_n|<|b_{n+p}-b_n|+|b_n-b_{n-1}|<\epsilon+\epsilon=2\epsilon
\end{equation}
对任意$p\in\mathbb{N}^\star$都成立,所以,由定义知$\{a_n\}$是基本列,所以$\{a_n\}$收敛.
\end{proof}
5. 用精确语言表述``数列$\{a_n\}$不是基本列''.

答:

一个数列$\{ a_n \}$不是基本列,如果存在$\epsilon_0 > 0$,使得对任意$N \in \nat$,存在$n, m > N$,使得$|a_m - a_n| \geq \epsilon_0$.

6. 设$a_n \in [a,b]\,(n\in\mathbb{N}^\star)$.证明:如果$\{a_n\}$发散,则$\{a_n\}$必有两个子列收敛于不同的数.
\begin{proof}
任取$\{ a_n \}$的两个收敛子列,分别记为$\{ a_{i_n} \}$和$\{ a_{j_n} \}$,由于$\{ a_n \}$发散,所以$\{ a_n \}$不是基本列,那么就存在$\epsilon_0 > 0$,使得对任意$N \in \nat$,都存在$n, m > N$,使得$|a_n - a_m| \geq \epsilon_0$,于是就存在$i_n, j_m > N$,使得$|a_{i_n} - a_{j_m}| \geq \epsilon_0$,这就说明了$\{ a_{i_n} \}$和$\{ a_{j_m} \}$分别收敛到不同的极限.
\end{proof}

\exercise
2. 求数列$\{ (1+1/n)^n : n \in \mathbb{N}^\star \}$和$\{(1+1/n)^{n+1}:n\in\mathbb{N}^\star\}$的下确界和上确界.

思路:注意到$\{ (1+1/n)^n : n \in \mathbb{N}^\star \}$是严格单调递增数列并且极限是$\mathrm{e}$,所以我们去证它的上确界是$\mathrm{e}$,而数列$\{(1+1/n)^{n+1}:n\in\mathbb{N}^\star\}$的极限也是$\mathrm{e}$并且是严格单调递减数列,我们去证它的下确界是$\mathrm{e}$.
\begin{proof}
令$a_n = (1+1/n)^n$,易证,数列$\{ a_n \}$是严格单调递增的,并且极限是$\mathrm{e}$,从而$\mathrm{e} < a_n$对所有$n \in \mathbb{N}^\star$成立,也就是对任意$x \in \{a_n:n\in\mathbb{N}^\star\}$都有$\mathrm{e}<x$;由于$\{a_n\}$的极限是$\mathrm{e}$,所以,对任意$\epsilon >0$,都存在$N \in \mathbb{N}^\star$,使得当$n>N$时,有
\begin{equation}
    |a_n - \mathrm{e}| = \mathrm{e}-a_n < \epsilon
\end{equation}
令$x_\epsilon = a_n$,该不等式可以写成$x_\epsilon > e - \epsilon$,并且$x_\epsilon = a_n \in \{a_n : n \in \mathbb{N}^\star \}$,因此$\mathrm{e}$是集合$\{ a_n : n \in \mathbb{N}^\star \}$也就是集合$\{ (1+1/n)^n : n \in \mathbb{N}^\star \}$的上确界.易知$\{(1+1/n)^n:n\in\mathbb{N}\}$的下确界是$a_1$也就是$3/2$.

再来证$\mathrm{e}$是集合$\{(1+1/n)^{n+1}:n\in\mathbb{N}\}$的下确界.令$b_n=(1+1/n)^{n+1}$,易证数列$\{b_n\}$是严格单调递减的并且极限是$\mathrm{e}$,从而对任意$n \in \mathbb{N}^\star$都有$b_n > \displaystyle\lim_{n\to\infty} b_n = \mathrm{e}$,也就是对任意$ x \in \{ b_n : n \in \mathbb{N}^\star \}$都有$x > \mathrm{e}$.

由$\displaystyle\lim_{n\to\infty}b_n = e$知,对任意$\epsilon > 0$,存在$N \in \mathbb{N}^\star$,使得当$n > N$时,有
\begin{equation}
    |b_n - \mathrm{e}| = b_n - \mathrm{e} < \epsilon 
\end{equation}
令$y_\epsilon = b_n$,该不等式可以写成$y_\epsilon = b_n < \mathrm{e} + \epsilon$(对任意$n\in\mathbb{N}^\star$),从而$\mathrm{e}$正是集合$\{b_n:n\in\mathbb{N}^\star\}$的下确界.

显然$\{b_n:n\in\mathbb{N}^\star\}$的上确界是$b_1$,也就是$\displaystyle \left(\frac{3}{2}\right)^2$.
\end{proof}

3. 求数列$\{n^{1/n}:n\in\mathbb{N}^\star\}$的下确界和上确界.

解:先求上确界.通过计算可知$1^{1/1} < 2^{1/2} <3^{1/3}$,令$a_n = n^{1/n}$,则$a_n$的增减性等价于$\ln \, a_n$的增减性,为了判断$\ln \, a_n$的增减性,我们对数列$\{ \ln \, a_n \}$的任意相邻两项做差:
\begin{align}
    \ln \, a_{n+1} - \ln \, a_n &= \frac{1}{n+1} \ln \, \left(n+1\right) - \frac{1}{n} \ln \, n\\
    &= \frac{1}{n(n+1)} \left( n \ln \, \left(n+1\right) - (n+1) \ln \, n\right)
\end{align}
并且当$n \geq 3$时,有
\begin{align}
    n \ln \, \left(n+1\right) - (n+1) \ln \, n &= n \left(\ln \, \left(n+1\right) - \ln \, n \right) - \ln \, n \\
    &< n \cdot \frac{1}{n} - \ln \, n = 1 - \ln \, n \leq 1 - \ln 3 < 0
\end{align}
即$\ln \, a_{n+1} - \ln \, a_n < 0$,由此可知,当$n \geq 3$时,有$a_{n+1} < a_n$.因此,对任意$n \in \mathbb{N}^\star$,都有$a_3 \geq a_n$,并且,对任意$\epsilon > 0$,可以取$x_\epsilon = a_3$,这时有
\begin{equation}
    a_3 - x_\epsilon = a_3 - a_3 = 0 < \epsilon
\end{equation}
也就是$x_\epsilon > a_3 - \epsilon$
所以$a_3$也就是$\displaystyle 3^\frac{1}{3}$,是$\{a_n:n\in\mathbb{N}^\star\}$的上确界.

再来证$1$是$\{n^\frac{1}{n}\}$的下确界,由于$\displaystyle\lim_{n\to\infty}n^\frac{1}{n}=1$,所以,对任意$\epsilon>0$,存在$N \in \mathbb{N}^\star$,使得当$n > N$时,有
\begin{equation}
    |n^\frac{1}{n} - 1| = n^\frac{1}{n} - 1 < \epsilon
\end{equation}
令$y_\epsilon = n^\frac{1}{n}$,则对所有$n > N$,都有$y_\epsilon = n^\frac{1}{n} < 1 + \epsilon$成立,并且显然$y_\epsilon \in \{ n^\frac{1}{n}:n \in \mathbb{N}^\star \}$,因此,由定义,$1$是$\{\displaystyle n^\frac{1}{n}:n\in\mathbb{N}^\star\}$的下确界.

4. 设在数列$\{ a_n:n\in\mathbb{N}^\star\}$中,既没有最小值,也没有最大值.求证:数列$\{a_n\}$发散.
\begin{proof}
如果数列$\{ a_n \}$是无界的,譬如说,不存在上界,那么我们可以从中找出一个趋于正无穷的子列,于是$\{a_n \}$发散.

否则根据确界原理,$\{a_n\}$有上确界和下确界,分别记为$u$和$l$,由于$u$是$\{ a_n \}$的上确界,所以对任意$\epsilon > 0$,存在$i_1 \in \nat$,使得$u - a_{i_1} < \epsilon$,又因为$\{ a_n \}$是不存在最大值的,于是就存在$i_2 \in \nat$,使得$a_{i_1} < a_{i_2} < u$,类似地,我们可以找出一系列的下标$i_1, i_2, i_3, i_4, \cdots \in \nat$,它们满足
\begin{equation}
    a_{i_1} < a_{i_2} < a_{i_3} < a_{i_4} < \cdots < u
\end{equation}
并且对以上这两个不等式做放缩可得
\begin{equation}
    0 < \cdots < u - a_{i_4} < u - a_{i_3} < u - a_{i_2} < u - a_{i_1} < \epsilon
\end{equation}
这样一来,我们就找到了一个趋于$u$的子列,它就是$\{ a_{i_n} \}$.

由于$l$是下确界,所以对任意$\epsilon > 0$,都存在$j_1 \in \nat$,使得$a_{j_1} - l < \epsilon$,由于$\{ a_n \}$不存在最小值,所以一定存在$j_2 \in \nat$使得
\begin{equation}
    a_{j_1} > a_{j_2} > l
\end{equation}
类似地,总能找出一系列的下标$j_1, j_2, j_3, \cdots$,满足
\begin{equation}
    a_{j_1} > a_{j_2} > a_{j_3} > \cdots > l
\end{equation}
对以上这两个不等式做放缩,得到
\begin{equation}
    \epsilon > a_{j_1} - l > a_{j_2} - l > a_{j_3} - l > \cdots > 0
\end{equation}
这样我们就找到了一个趋于$l$的子列,它就是$\{ a_{j_m} \}$.如果$u = l$,也就是$\{ a_n \}$的上确界等于下确界,那么将可以推出$\{ a_n \}$是常数列,这时每一个$a_n$都可以当成最大值(最小值),这与题设矛盾,所以$u \neq l$,所以$\{ a_{i_n} \}$和$\{ a_{j_m} \}$是$\{ a_n \}$的两个趋向不同极限的子列,所以$\{ a_n \}$发散.
\end{proof}

% 5. 试用定理1.8.1证明中使用的``二分法'',证明定理1.7.1(Bolzano---Weierstrass定理).

\exercise
\exercise

1. 求$\displaystyle\lim_{n\to \infty} \inf a_n$和$\displaystyle\lim_{n \to \infty} \sup a_n$,设:
\begin{table}[H]
    \centering
    \begin{tabularx}{0.8\textwidth} {  >{\raggedright\arraybackslash}X >{\raggedright\arraybackslash}X  }
       (1) ~ $a_n = \displaystyle \frac{(-1)^n}{n} + \frac{1+(-1)^n}{2}$; & (2)~$a_n=\displaystyle n^{(-1)^n}$; \\ [1em]
       (3) ~ $a_n = \arctan \, \left( n^{(-1)^n} \right)$;
      \end{tabularx}
\end{table}

(1) 解:利用上、下极限的等价定义:
\begin{align}
    a^\star = \lim_{n \to \infty} \sup_{k \geq n} \, \{ a_k \}, \; a_\star = \lim_{n \to \infty} \inf_{k \geq n} \, \{a_k\}
\end{align}
我们首先求$\displaystyle\sup_{k \geq n} \, \{a_k\}$,它自身可以看成是一个数列:
\begin{equation}
    \sup_{k \geq n} \, a_k = \begin{cases}
        a_n, & n\text{为偶数} \\
        a_{n+1}, & n\text{为奇数} 
    \end{cases}
\end{equation}
这样我们就得到
\begin{equation}
    \{ \sup_{k \geq n} \, a_k \} = \{ a_2, a_2, a_4, a_4, a_6, a_6, \cdots \}
\end{equation}
它可以看成是$\{ a_n \}$所有偶数项组成的子列,再求$\displaystyle \inf_{k \geq n} \, \{ a_k \}$,它等于
\begin{equation}
    \inf_{k \geq n} \, a_k = \begin{cases}
        a_n, & n\text{为奇数} \\
        a_{n+1}, & n\text{为偶数} 
    \end{cases}
\end{equation}
这样我们就得到
\begin{equation}
    \{ \inf_{k \geq n} \, a_k \} = \{ a_1, a_1, a_3, a_3, a_5, a_5, \cdots \}
\end{equation}
它可以看成是$\{ a_n \}$所有奇数项组成的子列.于是$\displaystyle \lim_{n \to \infty} \sup_{k \geq n} \, \{ a_k \}$就等于$\{ a_{2n} \}$的极限,就等于
\begin{equation}
    \lim_{n \to \infty} a_{2n} = \lim_{n \to \infty} \left( \frac{1}{n} + 1 \right) = \left(\lim_{n\to\infty} \frac{1}{n}\right) + 1= 0 + 1 = 1
\end{equation}
而$\displaystyle \lim_{n \to \infty} \inf_{k \geq n} \, \{ a_k \}$等于$\{ a_{2n-1} \}$的极限,就等于
\begin{equation}
    \lim_{n \to \infty} a_{2n - 1} = \lim_{n \to \infty} -\frac{1}{n} = - \left(\lim_{n\to\infty} \frac{1}{n}\right) = -\left( 0 \right) = -0 = 0.
\end{equation}

(2)解:首先对$a_n$的幂次进行计算,也就是,对$(-1)^n$进行计算,我们发现:
\begin{equation}
    (-1)^n = \begin{cases}
        -1, & n\text{为奇数} \\
        1, & n\text{为偶数}
    \end{cases}
\end{equation}
这样一来就得到
\begin{equation}
    a_n = \begin{cases}
        n^{-1}, & n\text{为奇数} \\
        n, & n\text{为偶数}
    \end{cases}
\end{equation}
进而有
\begin{equation}
    \sup_{k \geq n} \, \{ a_k \} = n, \; \inf_{k \geq n} \, \{ a_k \} = n^{-1}
\end{equation}
从而得
\begin{align}
    &a^\star = \lim_{n\to\infty} \sup_{k \geq n} \, \{ a_k \} = \lim_{n \to \infty} n = +\infty \\
    &a_\star = \lim_{n\to\infty} \inf_{k \geq n} \, \{ a_k \} = \lim_{n \to \infty} n^{-1} = \lim_{n \to \infty} \frac{1}{n} = 0.
\end{align}

(3) 解:$\arctan$是定义在$(-\infty, +\infty)$上的严格单调递增函数,利用这个性质可以得到
\begin{align}
    &\ulimit \{ \arctan \, \left(n^{(-1)^n}\right) \} = \arctan \, \left( \ulimit \{ n^{(-1)^n}\}\right) = \arctan \, +\infty = \frac{\pi}{2} \\
    &\llimit \{ \arctan \, \left(n^{(-1)^n}\right) \} = \arctan \, \left( \llimit \{ n^{(-1)^n}\}\right) = \arctan \, 0 = 0
\end{align}

2. 试证下面诸式当两边有意义时成立:

(1) 若$\displaystyle\lim_{n\to\infty}b_n=b$,则
\begin{align}
    &\llimit \left( a_n + b_n \right) = \llimit a_n + b, \\
    &\ulimit \left( a_n + b_n \right) = \ulimit a_n + b;
\end{align}

(1)
\begin{proof}
由例3的结论:
\begin{align}
    &\llimit \left( a_n + b_n \right) \leq \llimit a_n + \ulimit b_n = \llimit a_n + b \\
    &\llimit \left( a_n + b_n \right) \geq \llimit a_n + \llimit b_n = \llimit a_n + b 
\end{align}
所以这只能是
\begin{equation}
    \llimit \left( a_n + b_n \right) = \llimit a_n + b.
\end{equation}
再次利用例3的结论:
\begin{align}
    &\ulimit \left( b_n + a_n \right) \leq \ulimit b_n + \ulimit a_n = b + \ulimit a_n \\
    &\ulimit \left( b_n + a_n \right) \geq \llimit b_n + \ulimit a_n = b + \ulimit a_n
\end{align}
所以这只能是
\begin{equation}
    \ulimit \left( a_n + b_n \right) = \ulimit a_n + b.
\end{equation}
\end{proof}

\exercise

1. 计算下列极限:
\begin{table}[H]
    \centering
    \begin{tabularx}{\textwidth} {  >{\raggedright\arraybackslash}X >{\raggedright\arraybackslash}X  }
       (1)~$\displaystyle\lim_{n\to\infty}\displaystyle\frac{1+\displaystyle\frac{1}{2}+\cdots+\displaystyle\frac{1}{n}}{\ln \, n}$; & (2)~$\displaystyle\lim_{n\to\infty}\displaystyle\frac{1+\displaystyle\frac{1}{3}+\cdots+\displaystyle\frac{1}{2n-1}}{\ln \, 2\sqrt{n}}$; 
    \end{tabularx}
\end{table}

(1) 解:
令
\begin{align}
    a_n &= 1+\frac{1}{2}+\cdots+\frac{1}{n} \\
    b_n &= \ln \, n
\end{align}
则
\begin{align}
    \text{原式} &= \lim_{n\to\infty} \frac{a_n}{b_n} = \lim_{n \to \infty} \frac{a_{n+1}-a_n}{b_{n+1}-b_n} = \lim_{n \to \infty} \frac{\displaystyle\frac{1}{n+1}}{\ln \, \left( n+1 \right) - \ln \, n} \\
    &= \lim_{n \to \infty} \frac{1}{(n+1) \ln \, \left( 1 + \displaystyle\frac{1}{n}\right)} = \lim_{n \to \infty} \frac{1}{\ln \, \left( \left(1+\displaystyle\frac{1}{n}\right)^{n+1}\right)} \\
    &= \frac{1}{\ln \, \left( \displaystyle\lim_{n \to \infty} \left( \left(1 + \displaystyle\frac{1}{n}\right)^{n+1}\right)\right)} = \frac{1}{\ln \, \mathrm{e}} = \frac{1}{1} = 1.
\end{align}

(2) 解:
令
\begin{align}
    a_n &= 1 + \frac{1}{3} + \frac{1}{5} + \cdots + \frac{1}{2n-1} \\
    b_n &= \ln \, 2\sqrt{n}
\end{align}
则
\begin{align}
    \text{原式} &= \lim_{n \to \infty} \frac{a_n}{b_n} = \lim_{n \to \infty} \frac{a_{n+1} - a_n}{b_{n+1} - b_n} = \lim_{n \to \infty} \frac{\displaystyle \frac{1}{2n+1}}{\ln \, 2\sqrt{n+1} - \ln \, 2 \sqrt{n}} \\
    &= \lim_{n \to \infty} \frac{\displaystyle\frac{1}{2n+1}}{\ln \, \displaystyle\frac{2\sqrt{n+1}}{2\sqrt{n}}} = \lim_{n \to \infty} \frac{\displaystyle\frac{1}{2n+1}}{\ln \sqrt{\displaystyle\frac{n+1}{n}}} = \lim_{n \to \infty} \frac{1}{(2n+1) \ln \, \left(\left(1 + \displaystyle\frac{1}{n}\right)^{\frac{1}{2}} \right)} \\
    &= \lim_{n \to \infty} \frac{1}{\ln \, \left( \left(1+\displaystyle\frac{1}{n}\right)^{\frac{2n+1}{2}}\right)} = \lim_{n \to \infty} \frac{1}{\ln \, \left(\left(1+\displaystyle\frac{1}{n}\right)^{n+\frac{1}{2}}\right)} \\
    &= \frac{1}{\displaystyle\lim_{n \to \infty} \ln \, \left(\left(1+\displaystyle\frac{1}{n}\right)^{n+\frac{1}{2}}\right)} = \frac{1}{\ln \, \displaystyle\lim \left( \left( 1 + \displaystyle\frac{1}{n}\right)^{n+\frac{1}{2}}\right)} = \frac{1}{\ln \, \mathrm{e}} = \frac{1}{1} = 1.
\end{align}

2. 计算极限$\displaystyle\lim_{n \to \infty} \left( n! \right)^{1/n^2}$.(提示:取对数).

2. 解:
\begin{align}
    \text{原式} &= \lim_{n \to \infty} \exp \{ \, \ln \, \left( (n!)^{1/n^2}\right) \} = \lim_{n \to \infty} \exp \{ \, \displaystyle\frac{1}{n^2} \displaystyle\sum_{i=1}^{n} \ln \, i \, \}
\end{align}
令
\begin{align}
    a_n = \sum_{i=1}^n \ln \, i,  \;\; b_n = n^2
\end{align}
那么就有
\begin{align}
    \text{原式} &= \lim_{n \to \infty} \exp \, \{ \, \displaystyle\frac{1}{n^2} \displaystyle\sum_{i=1}^{n} \ln \, i \, \} = \exp \, \{ \, \displaystyle\lim_{n \to \infty} \displaystyle\frac{a_n}{b_n} \,\, \} \\
    &= \exp \, \{ \, \displaystyle\lim_{n \to \infty} \displaystyle\frac{a_{n+1}-a_n}{b_{n+1}-b_n} \,\, \} = \exp \, \{ \, \displaystyle\lim_{n \to \infty} \displaystyle\frac{\ln \, \left(n+1\right) - \ln \, n}{2n + 1} \,\, \} \\
    &= \exp \, \{ \, \displaystyle\lim_{n \to \infty} \ln \, \left( 1 + \displaystyle\frac{1}{n} \right)^{\frac{1}{2n+1}} \,\, \} = \exp \, \{ \, \ln \, \displaystyle\lim_{n \to \infty} \left( 1 + \displaystyle\frac{1}{n} \right)^{\frac{1}{2n+1}} \,\, \} \\
    &= \exp \, \{ \, \ln \, 1 \,\, \} = \exp \, 0 = 1.
\end{align}

3. 计算极限
\begin{equation}
    \lim_{n\to\infty} \frac{1^2 + 3^2 + \cdots + (2n-1)^2}{n^3}.
\end{equation}

3. 解:令
\begin{align}
    a_n &= 1^2 + 3^2 + \cdots + (2n-1)^2 \\
    b_n &= n^3
\end{align}
那么
\begin{align}
    \text{原式} &= \lim_{n \to \infty} \frac{a_{n+1}-a_n}{b_{n+1}-b_n} = \lim_{n \to \infty} \frac{(2n+1)^2}{3n^2 + 3n + 1} = \lim_{n \to \infty} \frac{4n^2 + 4n + 1}{3n^2 + 3n + 1} \\
    &= \lim_{n \to \infty} \frac{4 + \displaystyle\frac{4}{n} + \displaystyle\frac{1}{n^2}}{3 + \displaystyle\frac{3}{n} + \frac{1}{n^2}} = \frac{\displaystyle\lim_{n \to \infty} \left(4 + \displaystyle\frac{4}{n} + \displaystyle\frac{1}{n^2}\right)}{\displaystyle\lim_{n \to \infty} \left(3 + \displaystyle\frac{3}{n} + \displaystyle\frac{1}{n^2}\right)} = \frac{4}{3}.
\end{align}

4. 设$\displaystyle\lim_{n \to \infty} a_n = a$.证明:
\begin{equation}
    \lim_{n \to \infty} \frac{a_1 + 2a_2 + \cdots + na_n}{n^2} = \frac{a}{2}.
\end{equation}

\begin{proof}
令
\begin{align}
    b_n &= a_1 + 2a_2 + \cdots + na_n \\
    c_n &= n^2
\end{align}
于是
\begin{align}
    \text{左边} &= \lim_{n \to \infty} \frac{b_n}{c_n} = \lim_{n \to \infty} \frac{b_{n} - b_{n-1}}{c_{n} - c_{n-1}} = \lim_{n \to \infty} \frac{n a_n}{2n - 1} = \lim_{n \to \infty} \frac{a_n}{2 - \displaystyle\frac{1}{n}} \\
    &= \frac{\displaystyle\lim_{n \to \infty} a_n}{\displaystyle\lim_{n \to \infty} \left(2 - \displaystyle\frac{1}{n}\right)} = \frac{a}{2} = \text{右边}.
\end{align}
\end{proof}

\begin{theo}[Stolz]
设$\{b_n \}$是严格递增且趋于$+\infty$的数列,那么
\begin{equation}
    \lim_{n \to \infty} \frac{a_n}{b_n} = A
\end{equation}
的充分不必要条件是
\begin{equation}
    \lim_{n \to \infty} \frac{a_n - a_{n-1}}{b_{n} - b_{n-1}} = A.
\end{equation}
\end{theo}

5. 举例说明Stolz定理的逆命题不成立.

\begin{proof}
取
\begin{align}
    a_n &= n^2 + n \cdot (-1)^n \\
    b_n &= n^2
\end{align}
那么$\{ a_n \}, \{ b_n \}$符合Stolz条件,并且
\begin{equation}
    \lim_{n \to \infty} \frac{a_n}{b_n} = 1
\end{equation}
但是
\begin{equation}
    \lim_{n \to \infty} \frac{a_{n} - a_{n-1}}{b_{n} - b_{n-1}} = +\infty
\end{equation}
这就证明了,在Stolz定理中:$\displaystyle\lim_{n \to \infty} \frac{a_n - a_{n-1}}{b_n - b_{n-1}} = A$不是$\displaystyle\lim_{n \to \infty} \frac{a_n}{b_n} = A$的必要条件.
\end{proof}