\exercise

1. 证明:对任意的实数$c$,方程$x^3-3x+c=0$在$[0,1]$上无相异的根.

\prove 采用反证法.令$f(x)=x^3-3x+c$,设存在$c_1, c_2 \in \real$满足$0<c_1<c_2<1$并且$f(c_1)=f(c_2)=0$.那么在开区间$(c_1,c_2)$应用Rolle中值定理,可得存在$\xi \in (c_1,c_2)$使得$f^\prime (\xi)=0$.

与此同时我们求得$f^\prime (x) = 3x^2-3$,得知$f^\prime$只有$0,1$这两个零点,这与$\xi \in (c_1,c_2) \subset (0,1)$是矛盾的.\qed\bigskip

2. 设函数$f$在$(a,b)$上可导,且$f(a+)=f(b-)$是有限的或为$\infty$.求证:存在一点$\xi \in (a,b)$,使得$f^\prime (\xi) = 0$.

\prove 

\textbf{情形1:}$f(a+)=f(b-)=c$为有限数.构造函数
\begin{equation}
    g(h) = \frac{f(b-h)-f(a+h)}{b-h-(a+h)}
\end{equation}
对每一个$h \in (0, \displaystyle\frac{b-a}{2})$,都可以在区间$(a+h,b-h)$应用Lagrange定理:可知存在$\xi \in (a+h,b-h)$使得
\begin{equation}
    g(h) = \frac{f(b-h)-f(a+h)}{b-h-(a+h)} = f^\prime (\xi)
\end{equation}
令$h \to 0^+$,则$g(h)$的分子趋于$0$,而分母趋于$b-a$,得
\begin{equation}
    \lim_{h \to 0^+} \frac{f(b-h)-f(a+h)}{b-h-(a+h)} = 0
\end{equation}
因此$\displaystyle\lim_{h \to 0^+} f^{\prime}(\xi) = f^{\prime}(\xi)= 0$.

\textbf{情形2:}$f(a+)=f(b-)=\infty$,不失一般性假设$f(a+)=f(b-)=+\infty$.任取$x_1, x_2 \in (a,b)$,并且要求$x_1<x_2$.若$f(x_1)=f(x_2)$,那么就在区间$(x_1,x_2)$应用Rolle中值定理,定理已经得证.如果$f(x_1) \neq f(x_2)$,不失一般性假设$f(x_1) < f(x_2)$,由于$f(a+)=+\infty$,可知存在$x_3 \in (a,x_1)$使得$f(x_3)>f(x_2)$,这时我们有$f(x_1) < f(x_2) < f(x_3)$,而$x_3 < x_1 < x_2$,依介值性,可在$(x_3,x_1)$找到一个$x_4$使得$f(x_4)=f(x_2)$,现在让我们在区间$(x_4,x_2)$应用Rolle中值定理,可知存在$x_5 \in (x_4,x_2)$使得$f^\prime (x_5) = 0$.容易验证这个$x_5$仍然是在$(a,b)$内的.\qed\bigskip

3. 证明下列不等式:
\begin{tasks}(1)
    \task $\lvert \sin \, x - \sin \, y \rvert \leq \lvert x - y \rvert \; \left(x, y \in \real \right)$;
    \task $py^{p-1}\left(x-y\right)\leq x^p - y^p \leq px^{p-1}\left(x-y\right) \; (0 < y < x, p > 1)$;
    \task $\displaystyle\frac{a-b}{a}<\ln \, \displaystyle\frac{a}{b} < \displaystyle\frac{a-b}{b} \; (0<b<a)$;
    \task $\displaystyle\frac{a-b}{\sqrt{1+a^2}\sqrt{1+b^2}} < \arctan \, a - \arctan \, b < a-b \; (0 < b < a)$.
\end{tasks}

(1) \prove 不失一般性设$x<y$,在区间$(x,y)$中应用Lagrange定理:存在$\xi \in (x,y)$使得
\begin{equation}
    \frac{\sin \, x - \sin \, y}{x-y} = \left(\sin \, x\right)^\prime \bigg\vert_{x = \xi} = \cos \, \xi
\end{equation}
再对上式两端都取绝对值,根据函数$\cos $的有界性即证.\qed\bigskip

(2) \prove 依Lagrange定理:存在$\xi \in (y, x)$使得
\begin{equation}
    \frac{x^p-y^p}{x-y} = p \xi^{p-1}
\end{equation}
对上式中的$\xi$在$(x,y)$内进行放缩即证.\qed\bigskip

(3) \prove 依$\ln$函数的性质等价于证
\begin{equation}
    \frac{a-b}{a} < \ln \, a - \ln \, b < \frac{a-b}{b}
\end{equation}
即证
\begin{equation}
    \frac{1}{a} < \frac{\ln \, a - \ln \, b}{a-b} < \frac{1}{b}
\end{equation}
而根据Lagrange定理存在$\xi \in (b,a)$使得
\begin{equation}
    \frac{\ln \, a - \ln \, b}{a-b} = \frac{1}{\xi}
\end{equation}
对$\xi$放缩即证.\qed\bigskip

(4) \prove 令$a = \tan \, x$,再令$b = \tan \, y$,不等式右边是显然的,我们来证左边,我们要证的是:
\begin{equation}
    \frac{\tan \, x - \tan \, y}{\displaystyle\frac{1}{\cos \, x}\cdot \displaystyle\frac{1}{\cos \, y}} < x - y
\end{equation}
也就是
\begin{equation}
    \frac{\tan \, x - \tan \, y}{x-y} < \frac{1}{\cos \, x \cos \, y}
\end{equation}
将和差化积公式代入$\tan \, x - \tan \, y$得
\begin{equation}
    \frac{\sin \, \left(x-y\right)}{\left(x-y\right)(\cos \, x \cos \, y)} < \frac{1}{\cos \, x \cos \, y}
\end{equation}
即证
\begin{equation}
    \frac{\sin \, \left(x-y\right)}{x-y} < 1
\end{equation}
依几何关系知:该式对于任何$x>y$都是成立的.\qed\bigskip

4. 设函数$f$在$\real$上有$n$阶导数,且$p$是一个$n$次多项式,其最高次项系数为$a_0$.如果有互不相同的$x_i$,使得$f(x_i)=p(x_i) \; (i=0,1,\cdots)$.求证:存在$\xi$,满足$a_0 = f^{(n)}(\xi)/n!$.

\prove 令$g(x)=f(x)-p(x)$,不失一般性假设$x_0<x_1<\cdots<x_n$,由题意知
\begin{equation}
    g(x_i)=0,\quad i = 0,1,\cdots,n
\end{equation}
于是在区间
\begin{equation}
(x_i,x_{i+1}), \quad i=0,1,\cdots,n-1
\end{equation}
上分别应用Rolle定理,可知存在
\begin{equation}
x^{(1)}_i \in (x_{i},x_{i+1}), \quad i = 0,1,\cdots,n-1
\end{equation}
使得
\begin{equation}
g^{(1)}(x^{(1)}_i)=0, \quad i=0,1,2,\cdots,n-1
\end{equation}
于是在区间
\begin{equation}
(x^{(1)}_{i},x^{(1)}_{i+1}),\quad i=0,1,2,\cdots,n-2
\end{equation}
上分别应用Rolle定理,可知存在
\begin{equation}
x^{(2)}_{i}\in (x^{(1)}_{i},x^{(1)}_{i+1}), \quad i = 0,1,\cdots,n-2
\end{equation}
使得
\begin{equation}
g^{(2)}(x^{(2)}_i)=0, \quad i=0,1,2,\cdots,n-2
\end{equation}
于是在区间
\begin{equation}
    (x^{(2)}_{i},x^{(2)}_{i+1}),\quad i=0,1,2,\cdots,n-3
\end{equation}
上分别应用Rolle定理,可知存在
\begin{equation}
    x^{(3)}_{i}\in (x^{(2)}_{i},x^{(2)}_{i+1}), \quad i = 0,1,\cdots,n-3
\end{equation}
使得
\begin{equation}
g^{(3)}(x^{(3)}_i)=0, \quad i=0,1,2,\cdots,n-3
\end{equation}
照此规律最终得到:存在
\begin{equation}
    x^{(n)}_i \in (x^{(n-1)}_i, x^{(n-1)}_{i+1}), \quad i = 0
\end{equation}
使得
\begin{equation}
    g^{(n)}(x^{(n)}_i) = 0, \quad i = 0
\end{equation}
另一方面
\begin{equation}
    g^{(n)}(x) = f^{(n)}(x)-n!a_0
\end{equation}
把$x^{(n)}_{0}$代入得
\begin{equation}
    g^{(n)}(x^{(n)}_0) = f^{(n)}(x^{(n)}_0)-n!a_0 = 0
\end{equation}
由此可知
\begin{equation}
    a_0 = f^{(n)}(x^{(n)}_0)/n!.
\end{equation}
取$\xi = x^{(n)}_0$即证.\qed\bigskip

5. \begin{minipage}[t]{0.8\textwidth}
设常数$a_0,a_1,a_2,\cdots, a_n$满足
\begin{equation*}
    \frac{a_0}{n+1}+\frac{a_1}{n}+\cdots +\frac{a_{n-1}}{2}+a_n = 0.
\end{equation*}
求证:多项式$a_0 x^n + a_1 x^{n-1} + \cdots + a_{n-1} + a_n$在$(0,1)$内有一个零点.
\end{minipage}
\bigskip

\prove 令
\begin{equation}
    g(x) = \frac{a_0}{n+1} x^{n+1} + \frac{a_1}{n} x^n + \cdots + \frac{a_{n-1}}{2} x^2 + a_n x
\end{equation}
则
\begin{align}
    g(1) &= \frac{a_0}{n+1}+\frac{a_1}{n}+\cdots +\frac{a_{n-1}}{2}+a_n = 0 \\
    g(0) &= 0
\end{align}
故可在区间$(0,1)$对函数$g$应用Rolle定理,可知存在$\xi \in (0,1)$使得$g^{\prime}(\xi)=0$,而正好有
\begin{equation}
    g^{\prime}(x) = a_0 x^n + a_1 x^{n-1} + \cdots + a_{n-1} + a_n
\end{equation}
由此即证.\qed\bigskip

6. 设函数$f$在开区间$(0,a)$上可导,且$f(0+)=+\infty$.证明:$f^\prime$在$x=0$的右旁无下界.

\prove 任取$x_1 \in (0,a)$,设$x \in (0, x_1)$,在区间$(x,x_1)$应用Lagrange定理,可得
\begin{equation}
    \frac{f(x_1)-f(x)}{x_1-x} = f^{\prime}(\xi), \quad \xi \in (x, x_1)
\end{equation}
由于$0 < x_1-x < a$,又因为$f(x) \to +\infty \; (x \to 0^+)$,所以存在$0<\delta<x_1$使得对每一个$x \in (0,\delta)$都有$f(x_1)-f(x)<0$,所以有
\begin{equation}
    f^{\prime}(\xi) = \frac{f(x_1)-f(x)}{x_1-x} < \frac{f(x_1)-f(x)}{a}, \quad \xi \in (x, x_1)
\end{equation}
令$x \to 0^+$,则$\displaystyle\frac{f(x_1)-f(x)}{a} \to -\infty$,从而$f^{\prime}(\xi) \to -\infty$,命题得证.\qed\bigskip