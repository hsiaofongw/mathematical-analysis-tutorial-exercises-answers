\exercise

1. 设函数$f$在$x=0$可导并且$f(0) = 0$.求极限
\begin{equation*}
    \lim_{x \to 0} \frac{f(x)}{x}.
\end{equation*}

\solve 易知
\begin{align}
    \lim_{x \to 0}\frac{f(x)}{x} &= \lim_{x \to 0} \frac{f(x)-0}{x} \\
    &= \lim_{x \to 0} \frac{f(x)-f(0)}{x} 
\end{align}
这是是导数的定义,又因为$f$在$x=0$可导,因此
\begin{equation}
    \lim_{x \to 0} \frac{f(x)}{x} = f^\prime (0).
\end{equation}
\qed\bigskip

2. 设$f$在$x = 0$可导,$a_n \to 0^-$,$b_n \to 0^+  \; (n \to \infty)$.证明:
\begin{equation*}
    \lim_{n \to \infty} \frac{f(b_n) - f(a_n)}{b_n - a_n} = f^\prime (0).
\end{equation*}

\prove 变换得
\begin{align}
    \frac{f(b_n)-f(a_n)}{b_n-a_n} &= \frac{f(b_n)-f(0)}{b_n-a_n}+\frac{f(0)-f(a_n)}{b_n-a_n} \\
    &= \frac{f(b_n)-f(0)}{b_n} \cdot \frac{b_n}{b_n - a_n} + \frac{f(0)-f(a_n)}{0 - a_n} \cdot \frac{0 - a_n}{b_n - a_n}
\end{align}
令
\begin{align}
    g_1 : \nat &\longrightarrow \real \\
    n &\longmapsto \frac{f(b_n)-f(0)}{b_n}
\end{align}
再令
\begin{align}
    g_2 : \nat &\longrightarrow \real \\
    n &\longmapsto \frac{f(0)-f(a_n)}{0-a_n}
\end{align}
则依归结原则
\begin{align}
    \lim_{n \to \infty} g_1(n) &= \lim_{h \to 0^+} \frac{f(h)-f(0)}{h} = f^{\prime}_+(0) \\
    \lim_{n \to \infty} g_2(n) &= \lim_{h \to 0^-} \frac{f(h)-f(0)}{h} = f^{\prime}_-(0)
\end{align}
令$o(n) = g_2(n)-g_1(n)$,则$o(n) \to 0 \; (n \to \infty)$.于是原式可以写成
\begin{align}
    \frac{f(b_n)-f(a_n)}{b_n - a_n} &= g_1(n) \cdot \frac{b_n}{b_n-a_n} + g_2(n) \cdot \frac{0-a_n}{b_n-a_n} \\
    &= g_1(n) \cdot \frac{b_n}{b_n-a_n} + g_1(n) \cdot \frac{0-a_n}{b_n-a_n} + o(n) \cdot \frac{0-a_n}{b_n-a_n} \\
    &=g_1(n) \cdot \frac{b_n + 0 - a_n}{b_n - a_n} + o(n) \cdot \frac{0 - a_n}{b_n - a_n} \\
    &= g_1(n) + o(n) \cdot \frac{0-a_n}{b_n - a_n}
\end{align}
由于$a_n \to 0^-, \, b_n \to 0^+, \; (n \to \infty)$,故存在足够大的$M > 0$,使得对于一切$n \in M, n \geq M$都有
\begin{equation}
    -a_n > 0, \quad b_n > 0
\end{equation}
并且
\begin{equation}
    0 < \frac{0 - a_n}{b_n - a_n} < \frac{b_n - a_n}{b_n - a_n} = 1
\end{equation}
由此可见
\begin{equation}
    \frac{0-a_n}{b_n - a_n}
\end{equation}
是一个有界量,从而当$n \to \infty$时,
\begin{equation}
    o(n) \cdot \frac{0 - a_n}{b_n - a_n} \to 0
\end{equation}
从而
\begin{align}
    \lim_{n \to \infty} \frac{f(b_n)-f(a_n)}{b_n-a_n} &= \lim_{n \to \infty} \left( g_1(n) + o(n) \cdot \frac{0 - a_n}{b_n-a_n}\right) \\
    &= \lim_{n \to \infty} g_1(n) + \lim_{n \to \infty} \left( o(n) \cdot \frac{0 - a_n}{b_n - a_n} \right) \\
    &= f^{\prime}_{+}(0) + 0 = f^{\prime}_{+}(0) = f^{\prime}(0).
\end{align}
\qed\bigskip

3. 设$f$是偶函数且在$x=0$可导.证明:$f^{\prime}(0)=0$.

\prove 不失一般性设$f^{\prime}(0)>0$,因为$f$在$x=0$可导,故$f^{\prime}_{-}(0)=f^{\prime}_{+}(0)=f^{\prime}(0) > 0$,也就是说
\begin{align}
    \lim_{s \to 0^-} \frac{f(s)-f(0)}{s} &= f^{\prime}_{-}(0) > 0 \\
    \lim_{h \to 0^+} \frac{f(h)-f(0)}{h} &= f^{\prime}_{+}(0) > 0 
\end{align}
从而存在足够小的正数$\delta > 0$,使得对任意$x \in (0,0+\delta)$,都有
\begin{equation}
    \frac{f(x)-f(0)}{x} > 0 \implies f(x) - f(0) > 0 \implies f(x) > 0
\end{equation}
并且对任意$-x \in (0-\delta, 0)$都有
\begin{equation}
    \frac{f(-x)-f(0)}{-x} > 0 \implies f(-x)-f(0) < 0 \implies f(-x) < f(0)
\end{equation}
从而有
\begin{equation}
    f(-x) < f(0) < f(x)
\end{equation}
可是这与$f$作为偶函数是矛盾的.\qed\bigskip

4. 设函数$f$在$x_0$可导.证明:
\begin{equation*}
    \lim_{h \to 0} \frac{f(x_0+h)-f(x_0-h)}{2h} = f^{\prime}(x_0).
\end{equation*}
举例说明,即使上式左边的极限存在且有限,$f$在$x_0$也未必可导.

\prove 易知
\begin{align}
    \text{左边} &= \lim_{h \to 0} \frac{f(x_0+h)-f(x_0-h)}{2h} \\
    &= \lim_{h \to 0} \frac{f(x_0 + h) - f(x_0) + f(x_0) - f(x_0 - h)}{2h} \\
    &= \lim_{h \to 0} \frac{f(x_0 + h) - f(x_0)}{2h} + \lim_{h \to 0} \frac{f(x_0)-f(x_0-h)}{2h} \\
    &= \frac{1}{2} \lim_{h \to 0} \frac{f(x_0 + h) - f(x_0)}{h} + \frac{1}{2} \lim_{-h \to 0} \frac{f(x_0 + (- h)) - f(x_0)}{-h} \\
    &= \frac{1}{2} f^{\prime} (x_0) + \frac{1}{2} f^{\prime} (x_0) = f^{\prime} (x_0) = \text{右边}.
\end{align}
由此即证.\qed\bigskip

\textbf{举例.}令
\begin{align}
    f: \real \setminus \{ 0 \} &\longrightarrow \real^+ \\
    x &\longmapsto \bigg\lvert \frac{1}{x} \bigg\rvert
\end{align}
取$x_0 = 0$,那么可以算得
\begin{equation}
    \lim_{h \to 0} \frac{f(0 + h) - f(0-h)}{2h} = \lim_{h \to 0} \frac{0}{2h} = 0.
\end{equation}
可是极限
\begin{equation}
    \lim_{h \to 0^+} \bigg\lvert \frac{1}{x} \bigg\lvert = -\infty, \quad \lim_{h \to 0^+} \bigg\lvert \frac{1}{x} \bigg\rvert = + \infty
\end{equation}
这就说明$f$在$x = 0$处的导数不存在.\qed\bigskip

5. 在抛物线$y=x^2$上:
\begin{tasks}(1)
    \task 哪一点的切线平行于直线$y=4x-5$? 
    \task 哪一点的切线垂直于直线$2x-6y+5=0$? 
    \task 哪一点的切线与直线$3x-y+1=0$交成$45^{\circ}$的角?
\end{tasks}

\solve 函数$x \mapsto x^2$是定义在$\real$上的,任取$x_0 \in \real$,设$h > 0$,我们有
\begin{equation}
    (x_0 + h)^2 - x_0^2 = h(2x_0 + h)
\end{equation}
从而
\begin{equation}
    \lim_{h \to 0} \frac{\left(x_0+h\right)^2-x_0^2}{h} = \lim_{h \to 0} \left( 2x_0 + h\right) = 2x_0
\end{equation}
当点$(x_0, x_0^2)$处的切线平行于$y=4x-5$时,它们的斜率相等,也就是
\begin{equation}
    2x_0 = 4 \implies x_0 = 2
\end{equation}
由此可见在点$(2, 4)$处的切线与$y=4x-5$平行.并且当两条直线垂直时,它们的斜率的乘积为$-1$,设在点$(x_1, x_1^2)$处的切线与$2x-6y+5=0$垂直,那么有
\begin{equation}
    2x_0 \cdot \frac{1}{3} = -1 \implies x_0 = -\frac{3}{2}
\end{equation}
直线$3x-y+1=0$的方向向量为$(1, 3)$,而任意一点$(x_2, x_2^2)$处的切线的方向向量为$(1, 2x_2)$,设这两条直线的夹角为$\theta$,那么
\begin{equation}
    (1, 3) \cdot (1, 2x_2) = \lVert (1, 3) \rVert \, \lVert (1, 2x_2) \rVert \, \cos \, \theta 
\end{equation}
若$\theta = 45^\circ$,那么$\cos \, \theta = \displaystyle\frac{1}{\sqrt{2}}$,那么
\begin{equation}
    1 + 6 x_2 = \frac{1}{\sqrt{2}} \sqrt{10} \sqrt{1 + 4x_2^2}
\end{equation}
解出$x_2 = \displaystyle\frac{1}{4}$.\qed\bigskip