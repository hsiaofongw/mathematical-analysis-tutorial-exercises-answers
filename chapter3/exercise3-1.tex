\exercise

1. 设函数$f$在$x=0$可导并且$f(0) = 0$.求极限
\begin{equation*}
    \lim_{x \to 0} \frac{f(x)}{x}.
\end{equation*}

\solve 易知
\begin{align}
    \lim_{x \to 0}\frac{f(x)}{x} &= \lim_{x \to 0} \frac{f(x)-0}{x} \\
    &= \lim_{x \to 0} \frac{f(x)-f(0)}{x} 
\end{align}
这是是导数的定义,又因为$f$在$x=0$可导,因此
\begin{equation}
    \lim_{x \to 0} \frac{f(x)}{x} = f^\prime (0).
\end{equation}
\qed\bigskip

2. 设$f$在$x = 0$可导,$a_n \to 0^-$,$b_n \to 0^+  \; (n \to \infty)$.证明:
\begin{equation*}
    \lim_{n \to \infty} \frac{f(b_n) - f(a_n)}{b_n - a_n} = f^\prime (0).
\end{equation*}

\prove 变换得
\begin{align}
    \frac{f(b_n)-f(a_n)}{b_n-a_n} &= \frac{f(b_n)-f(0)}{b_n-a_n}+\frac{f(0)-f(a_n)}{b_n-a_n} \\
    &= \frac{f(b_n)-f(0)}{b_n} \cdot \frac{b_n}{b_n - a_n} + \frac{f(0)-f(a_n)}{0 - a_n} \cdot \frac{0 - a_n}{b_n - a_n}
\end{align}
令
\begin{align}
    g_1 : \nat &\longrightarrow \real \\
    n &\longmapsto \frac{f(b_n)-f(0)}{b_n}
\end{align}
再令
\begin{align}
    g_2 : \nat &\longrightarrow \real \\
    n &\longmapsto \frac{f(0)-f(a_n)}{0-a_n}
\end{align}
则依归结原则
\begin{align}
    \lim_{n \to \infty} g_1(n) &= \lim_{h \to 0^+} \frac{f(h)-f(0)}{h} = f^{\prime}_+(0) \\
    \lim_{n \to \infty} g_2(n) &= \lim_{h \to 0^-} \frac{f(h)-f(0)}{h} = f^{\prime}_-(0)
\end{align}
令$o(n) = g_2(n)-g_1(n)$,则$o(n) \to 0 \; (n \to \infty)$.于是原式可以写成
\begin{align}
    \frac{f(b_n)-f(a_n)}{b_n - a_n} &= g_1(n) \cdot \frac{b_n}{b_n-a_n} + g_2(n) \cdot \frac{0-a_n}{b_n-a_n} \\
    &= g_1(n) \frac{b_n}{b_n-a_n} + g_1(n) \cdot \frac{0-a_n}{b_n-a_n} + o(n) \cdot \frac{0-a_n}{b_n-a_n} \\
    &=g_1(n) \cdot \frac{b_n + 0 - a_n}{b_n - a_n} + o(n) \cdot \frac{0 - a_n}{b_n - a_n} \\
    &= g_1(n) + o(n) \cdot \frac{0-a_n}{b_n - a_n}
\end{align}
由于$a_n \to 0^-, \, b_n \to 0^+, \; (n \to \infty)$,故存在足够大的$M > 0$,使得对于一切$n \in M, n \geq M$都有
\begin{equation}
    -a_n > 0, \quad b_n > 0
\end{equation}
并且
\begin{equation}
    0 < \frac{0 - a_n}{b_n - a_n} < \frac{b_n - a_n}{b_n - a_n} = 1
\end{equation}
由此可见
\begin{equation}
    \frac{0-a_n}{b_n - a_n}
\end{equation}
是一个有界量,从而当$n \to \infty$时,
\begin{equation}
    o(n) \cdot \frac{0 - a_n}{b_n - a_n} \to 0
\end{equation}
从而
\begin{align}
    \lim_{n \to \infty} \frac{f(b_n)-f(a_n)}{b_n-a_n} &= \lim_{n \to \infty} \left( g_1(n) + o(n) \cdot \frac{0 - a_n}{b_n-a_n}\right) \\
    &= \lim_{n \to \infty} g_1(n) + \lim_{n \to \infty} \left( o(n) \cdot \frac{0 - a_n}{b_n - a_n} \right) \\
    &= f^{\prime}_{+}(0) + 0 = f^{\prime}_{+}(0) = f^{\prime}(0).
\end{align}
\qed\bigskip

3. 设$f$是偶函数且在$x=0$可导.证明:$f^{\prime}(0)=0$.

\prove 不失一般性设$f^{\prime}(0)>0$,因为$f$在$x=0$可导,故$f^{\prime}_{-}(0)=f^{\prime}_{+}(0)=f^{\prime}(0) > 0$,也就是说
\begin{align}
    \lim_{s \to 0^-} \frac{f(s)-f(0)}{s} &= f^{\prime}_{-}(0) > 0 \\
    \lim_{h \to 0^+} \frac{f(h)-f(0)}{h} &= f^{\prime}_{+}(0) > 0 
\end{align}
从而存在足够小的正数$\delta > 0$,使得对任意$x \in (0,0+\delta)$,都有
\begin{equation}
    \frac{f(x)-f(0)}{x} > 0 \implies f(x) - f(0) > 0 \implies f(x) > 0
\end{equation}
并且对任意$-x \in (0-\delta, 0)$都有
\begin{equation}
    \frac{f(-x)-f(0)}{-x} > 0 \implies f(-x)-f(0) < 0 \implies f(-x) < f(0)
\end{equation}
从而有
\begin{equation}
    f(-x) < f(0) < f(x)
\end{equation}
可是这与$f$作为偶函数是矛盾的.\qed\bigskip