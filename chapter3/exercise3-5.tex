\exercise

1. 研究下列函数在指定区间上的增减性:
\begin{tasks}(1)
    \task $f(x) = \tan \, x , \; (-\pi/2,\pi/2)$;
    \task $f(x) = \left(\arctan \, x\right) - x, \; \real$
\end{tasks}

(1) \solve 求导得
\begin{equation}
    \left(\tan \, x\right)^\prime = 1 + \left(\tan \, x\right)^2 > 0
\end{equation}
因此$\tan$在$(-\pi/2, \pi/2)$是增的.\qed\bigskip

(2) \solve 求导得
\begin{equation}
    \left(\left(\arctan \, x\right) - x\right)^\prime = \frac{1}{1+x^2} - 1 < 0
\end{equation}
因此函数$x \mapsto \left(\arctan \, x\right) - x$在$\real$是减的.\qed\bigskip

2. 证明不等式:
\begin{tasks}(2)
    \task $x(x-\arctan \, x) > 0 \; (x \neq 0)$;
    \task $x - \displaystyle\frac{x^2}{2}<\ln \, \left(1+x\right)<x \; (x>0)$;
    \task $x-\displaystyle\frac{x^3}{6}<\sin \, x < x \; (x > 0)$;
    \task $\ln \, \left(1+x\right)>\displaystyle\frac{\arctan \, x}{1+x} \left(x>0\right)$.
\end{tasks}

(1) \prove 由于该函数是奇的,故我们只讨论$x > 0$的情形,由于
\begin{equation}
    \left(0 - \arctan \, 0\right) = 0
\end{equation}
并且
\begin{equation}
    \left(x - \arctan \, x\right)^\prime = 1 - \frac{1}{1+x^2} > 0
\end{equation}
故对一切$x>0$都有
\begin{equation}
    \left(x - \arctan \, x\right) > 0
\end{equation}
又因为当$x > 0$时$x > 0$,故当$x > 0$时
\begin{equation}
    x \left(x - \arctan \, x\right) > 0
\end{equation}
有根据函数的奇偶性,得证当$x < 0$时也有
\begin{equation}
    x \left(x - \arctan \, x\right) > 0
\end{equation}
这就证明了对一切$x \neq 0$都有
\begin{equation}
    x \left(x - \arctan \, x\right) > 0.
\end{equation}
\qed\bigskip

(2) \prove 先证不等式右边,求导可知
\begin{equation}
    \left(\ln \, \left(1+x\right) - x\right)^\prime = \frac{1}{1+x} - 1 < 0 , \quad ( x > 0)
\end{equation}
并且
\begin{equation}
    \left(\ln \, \left(1+x\right) - x\right) \bigg\vert_{x = 0} = 0
\end{equation}
所以当$x > 0$时,恒有$\ln \, \left(1+x\right) - x < 0$,也就是$\ln \, \left(1+x\right) < x$.现在来证不等式左边,求导可知
\begin{equation}
    \left(x-\frac{x^2}{2}-\ln \, \left(1+x\right)\right)^\prime = 1 - x - \frac{1}{1+x}
\end{equation}
我们知道
\begin{align}
    &\mathrel{\phantom{\implies}} 1 - x^2 < 1 \\
    &\implies (1-x)(1+x) < 1 \\
    &\implies 1-x < \frac{1}{1+x} \, \quad (x > 0) 
\end{align}
所以有
\begin{equation}
    \left(x - \frac{x^2}{2}-\ln \, \left(1+x\right)\right)^\prime = 1 - x - \frac{1}{1+x} < 0
\end{equation}
又因为
\begin{equation}
    \left(x - \frac{x^2}{2}-\ln \, \left(1+x\right)\right) \bigg\vert_{x = 0} = 0
\end{equation}
所以对一切$x > 0$恒有
\begin{equation}
    x - \frac{x^2}{2} - \ln \, \left(1+x\right) < 0
\end{equation}
也就是
\begin{equation}
    x - \frac{x^2}{2} < \ln \, \left(1+x\right).
\end{equation}
\qed\bigskip

(3) \prove 我们只证不等式左边,由于在$x=0$处等号成立,故只需考虑导数的大小
\begin{align}
    &\mathrel{\phantom{\impliedby}} x - \frac{x^3}{6} < \sin \, x \\
    &\impliedby 1 - \frac{x^2}{2} < \cos \, x \\
    &\impliedby - x < -\sin \, x \\
    &\impliedby x > \sin \, x.
\end{align}
\qed\bigskip

(4) \prove 易知
\begin{align}
    &\mathrel{\phantom{\impliedby}} \ln \, \left(1+x\right) > \frac{\arctan \, x}{1+x} \\
    &\impliedby \ln \, \left(1+x\right) > \left(\arctan \, x\right) \left(\ln \, \left(1+x\right)\right)^\prime \\
    &\impliedby \frac{1}{\left(\ln \, \left(1+x\right)\right)^\prime} > \frac{\arctan \, x}{\ln \, \left(1+x\right)} \\
    &\impliedby \frac{\displaystyle\frac{1}{1+0}}{\displaystyle\frac{1}{1+x}} > \frac{\arctan \, x - \arctan \, 0}{\ln \, \left(1+x\right) - \ln \, \left(1+0\right)}
\end{align}
在区间$(0,x)$运用Cauchy定理可知
\begin{equation}
    \frac{\arctan \, x - \arctan \, 0}{\ln \, \left(1+x\right) - \ln \, \left(1+0\right)} = \frac{\displaystyle\frac{1}{1+\xi^2}}{\displaystyle\frac{1}{1+\xi}}, \quad \xi \in (0, x)
\end{equation}
毫无疑问$\displaystyle\frac{1}{1+\xi^2} < \displaystyle\frac{1}{1+0}$并且$\displaystyle\frac{1}{1+x}<\displaystyle\frac{1}{1+\xi}$,所以
\begin{equation}
    \displaystyle\frac{\displaystyle\frac{1}{1+\xi^2}}{\displaystyle\frac{1}{1+\xi}} < \displaystyle\frac{\displaystyle\frac{1}{1+0}}{\displaystyle\frac{1}{1+x}}
\end{equation}
所以
\begin{equation}
    \displaystyle\frac{\arctan \, x - \arctan \, 0}{\ln \, \left(1+x\right) - \ln \, \left(1+0\right)} < \displaystyle\frac{\displaystyle\frac{1}{1+0}}{\displaystyle\frac{1}{1+x}}
\end{equation}
所以
\begin{equation}
    \ln \, \left(1+x\right)>\frac{\arctan \, x}{1+x}.
\end{equation}
\qed\bigskip

3. 证明不等式
\begin{tasks}(1)
    \task 当$0<x_1<x_2<\pi/2$时,
    \begin{equation*}
        \frac{\tan x_2}{\tan x_1} > \frac{x_2}{x_1};
    \end{equation*}
    \task 当$x,y>0$且$\beta>\alpha>0$时,
    \begin{equation*}
        \left(x^\alpha + y^\alpha \right)^{1/\alpha} > \left(x^\beta + y^\beta\right)^{1/\beta};
    \end{equation*}
    \task 若$x>-1$,则
    \begin{align*}
        \left(1+x\right)^\alpha &\leq 1 + \alpha x \quad \left(0 < \alpha \leq 1\right), \\
        \left(1+x\right)^\alpha &\geq 1 + \alpha x \quad \left(\alpha < 0, \text{或} \; \alpha \geq 1\right)
    \end{align*}
    \task 设$p \geq 2$,当$x \in [0,1]$时,
    \begin{equation*}
        \left(\frac{1+x}{2}\right)^p + \left(\frac{1-x}{2}\right)^p \leq \frac{1}{2} \left(1+x^p\right).
    \end{equation*}
\end{tasks}

(1) \prove 易知
\begin{align}
    &\mathrel{\phantom{\impliedby}} \frac{\tan \, x_2}{\tan \, x_1} > \frac{x_2}{x_1} \\
    &\impliedby \frac{\tan \, x_1 + \tan \, x_2 - \tan \, x_1}{\tan \, x_1} > \frac{x_1 + x_2 - x_1}{x_1} \\
    &\impliedby 1 + \frac{\tan \, x_2 - \tan \, x_1}{\tan \, x_1} > 1 + \frac{x_2 - x_1}{x_1} \\
    &\impliedby \frac{\tan \, x_2 - \tan \, x_1}{\tan \, x_1} > \frac{x_2-x_1}{x_1} \\
    &\impliedby \frac{\tan \, x_2 - \tan \, x_1}{x_2 - x_1} > \frac{\tan \, x_1}{x_1}
\end{align}
现在让我们在区间$(x_1,x_2)$应用Lagrange定理,可知存在$\xi \in (x_1,x_2)$,使得
\begin{equation}
    \frac{\tan \, x_2 - \tan \, x_1}{x_2 - x_1} = 1 + \left(\tan \, \xi\right)^2 = \frac{1}{\left(\cos \, \xi\right)^2}
\end{equation}
注意到
\begin{equation}
    \frac{\tan \, x_1}{x_1} = \frac{\sin \, x_1}{x_1 \cos \, x_1} = \frac{1}{\displaystyle\frac{x_1}{\sin \, x_1} \cos \, x_1}
\end{equation}
由于
\begin{equation}
    \frac{x_1}{\sin \, x_1} > 1 \geq \cos \, \xi, \quad \cos \, x_1 > \cos \, \xi
\end{equation}
所以
\begin{equation}
    \frac{1}{\displaystyle\frac{x_1}{\sin \, x_1}\cos \, x_1} < \frac{1}{\left(\cos \, \xi\right)^2} 
\end{equation}
所以
\begin{equation}
    \frac{\tan \, x_2 - \tan \, x_1}{x_2 - x_1} > \frac{\tan \, x_1}{x_1}
\end{equation}
所以
\begin{equation}
    \frac{\tan \, x_2}{\tan \, x_1} > \frac{x_2}{x_1}.
\end{equation}
\qed\bigskip

(1) \prove (证法2:利用凸性)令
\begin{equation}
    f(x) = \tan \, x
\end{equation}
则
\begin{equation}
    f^\prime (x) = \frac{1}{\left(\cos x\right)^2}, \quad f^{\prime\prime}(x) = \frac{2\sin \, x}{\left(\cos \, x\right)^3} > 0, \quad (0 < x < \pi/2)
\end{equation}
由此可知$f$在$(0,\pi/2)$是严格凸的,所以对任意$0 < x_1 < x_2 < \pi/2$有
\begin{equation}
    \frac{f(x_1)-f(0)}{x_1-0} < \frac{f(x_2)-f(0)}{x_2-0}
\end{equation}
也就是
\begin{equation}
    \frac{\tan \, x_1}{x_1} < \frac{\tan \, x_2}{x_2}
\end{equation}
从而得
\begin{equation}
    \frac{\tan \, x_2}{\tan \, x_1} > \frac{x_2}{x_1}.
\end{equation}
\qed\bigskip

\annotate 凸性使证明一下子变得简洁了好多.\bigskip

(2) \prove 假设$x,y>1$,那么有
\begin{align}
    &\mathrel{\phantom{\impliedby}} \left(x^\alpha + y^\alpha\right) > \left(x^\beta + y^\beta \right)^\beta \\
    &\impliedby \text{函数} \; t \mapsto \exp \, \{ \frac{\ln \, \left(x^t + y^t \right)}{t}\} \; \text{是递减的}, \; (0 < t) \\
    &\impliedby \text{它的导数} \; \exp \, \{ \frac{\ln \, \left(x^t+y^t\right)}{t} \}\left(\frac{x^t\ln \, x + y^t \ln \, y}{t\left(x^t+y^t\right)} - \frac{\ln \, \left(x^t + y^t\right)}{t^2}\right) < 0 \\
    &\impliedby \frac{x^t\ln \, x + y^t \ln \, y}{t\left(x^t+y^t\right)} - \frac{\ln \, \left(x^t + y^t\right)}{t^2} < 0 \\
    &\impliedby \frac{x^t}{x^t+y^t} \ln \, \left(x^t\right) + \frac{y^t}{x^t+y^t} \ln \, \left(y^t\right) < \ln \, \left(x^t+y^t\right) \\
    &\impliedby \text{令} \; a = x^t, b = y^t, \lambda_1 = \frac{a}{a+b}, \lambda_2 = \frac{b}{a+b} \; \text{则} \; \lambda_1 \ln \, a + \lambda_2 \ln \, b < \ln \, \left(a+b\right) \\
    &\impliedby \exp \, \{\lambda_1 \ln \, a + \lambda_2 \ln \, b\} < \exp \, \{\ln \, \left(a+b\right)\} \\
    &\impliedby \text{利用凸性得} \; \exp \, \{\lambda_1 \ln \, a + \lambda_2 \ln \, b\} < \lambda_1  \exp \, \{ \ln \, a \} + \lambda_2 \exp \, \{ \ln \, b \} \\
    &\mathrel{\phantom{\impliedby}} = \lambda_1 a + \lambda_2 b < a+b = \exp \, \{\ln \, \left(a+b\right)\}.
\end{align}
\qed\bigskip

\annotate 这充分体现了导数工具的强大.\bigskip

4. 设函数$f$在$[0,1]$上有三阶导函数,且$f(0)=f(1)=0$.令$F(x)=x^2 f(x)$,求证:存在$\xi \in (0,1)$,使得$F^{\prime\prime\prime}(\xi)=0$.
