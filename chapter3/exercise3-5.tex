\exercise

1. 研究下列函数在指定区间上的增减性:
\begin{tasks}(1)
    \task $f(x) = \tan \, x , \; (-\pi/2,\pi/2)$;
    \task $f(x) = \left(\arctan \, x\right) - x, \; \real$
\end{tasks}

(1) \solve 求导得
\begin{equation}
    \left(\tan \, x\right)^\prime = 1 + \left(\tan \, x\right)^2 > 0
\end{equation}
因此$\tan$在$(-\pi/2, \pi/2)$是增的.\qed\bigskip

(2) \solve 求导得
\begin{equation}
    \left(\left(\arctan \, x\right) - x\right)^\prime = \frac{1}{1+x^2} - 1 < 0
\end{equation}
因此函数$x \mapsto \left(\arctan \, x\right) - x$在$\real$是减的.\qed\bigskip

2. 证明不等式:
\begin{tasks}(2)
    \task $x(x-\arctan \, x) > 0 \; (x \neq 0)$;
    \task $x - \displaystyle\frac{x^2}{2}<\ln \, \left(1+x\right)<x \; (x>0)$;
    \task $x-\displaystyle\frac{x^3}{6}<\sin \, x < x \; (x > 0)$;
    \task $\ln \, \left(1+x\right)>\displaystyle\frac{\arctan \, x}{1+x} \left(x>0\right)$.
\end{tasks}

(1) \prove 由于该函数是奇的,故我们只讨论$x > 0$的情形,由于
\begin{equation}
    \left(0 - \arctan \, 0\right) = 0
\end{equation}
并且
\begin{equation}
    \left(x - \arctan \, x\right)^\prime = 1 - \frac{1}{1+x^2} > 0
\end{equation}
故对一切$x>0$都有
\begin{equation}
    \left(x - \arctan \, x\right) > 0
\end{equation}
又因为当$x > 0$时$x > 0$,故当$x > 0$时
\begin{equation}
    x \left(x - \arctan \, x\right) > 0
\end{equation}
有根据函数的奇偶性,得证当$x < 0$时也有
\begin{equation}
    x \left(x - \arctan \, x\right) > 0
\end{equation}
这就证明了对一切$x \neq 0$都有
\begin{equation}
    x \left(x - \arctan \, x\right) > 0.
\end{equation}
\qed\bigskip

(2) \prove 先证不等式右边,求导可知
\begin{equation}
    \left(\ln \, \left(1+x\right) - x\right)^\prime = \frac{1}{1+x} - 1 < 0 , \quad ( x > 0)
\end{equation}
并且
\begin{equation}
    \left(\ln \, \left(1+x\right) - x\right) \bigg\vert_{x = 0} = 0
\end{equation}
所以当$x > 0$时,恒有$\ln \, \left(1+x\right) - x < 0$,也就是$\ln \, \left(1+x\right) < x$.现在来证不等式左边,求导可知
\begin{equation}
    \left(x-\frac{x^2}{2}-\ln \, \left(1+x\right)\right)^\prime = 1 - x - \frac{1}{1+x}
\end{equation}
我们知道
\begin{align}
    &\mathrel{\phantom{\implies}} 1 - x^2 < 1 \\
    &\implies (1-x)(1+x) < 1 \\
    &\implies 1-x < \frac{1}{1+x} \, \quad (x > 0) 
\end{align}
所以有
\begin{equation}
    \left(x - \frac{x^2}{2}-\ln \, \left(1+x\right)\right)^\prime = 1 - x - \frac{1}{1+x} < 0
\end{equation}
又因为
\begin{equation}
    \left(x - \frac{x^2}{2}-\ln \, \left(1+x\right)\right) \bigg\vert_{x = 0} = 0
\end{equation}
所以对一切$x > 0$恒有
\begin{equation}
    x - \frac{x^2}{2} - \ln \, \left(1+x\right) < 0
\end{equation}
也就是
\begin{equation}
    x - \frac{x^2}{2} < \ln \, \left(1+x\right).
\end{equation}
\qed\bigskip

(3) \prove 我们只证不等式左边,由于在$x=0$处等号成立,故只需考虑导数的大小
\begin{align}
    &\mathrel{\phantom{\impliedby}} x - \frac{x^3}{6} < \sin \, x \\
    &\impliedby 1 - \frac{x^2}{2} < \cos \, x \\
    &\impliedby - x < -\sin \, x \\
    &\impliedby x > \sin \, x.
\end{align}
\qed\bigskip