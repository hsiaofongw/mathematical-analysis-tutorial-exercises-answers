\exercise

1. 回答下列问题:

\begin{minipage}[l]{0.8\textwidth}
    \setlength{\leftskip}{0.5em}

    \medskip
    (1) 设$f(x)=\sin \, \displaystyle\frac{1}{x}$,能否定义$f(0)$,使得$f$在$x=0$处是连续的?
    \medskip

    (2) 设$f(x)=x \sin \, \displaystyle\frac{1}{x}$,如何理解函数$f$的定义域?
    \medskip

    (3) 设函数$f$在点$x_0$的近旁有定义,并且
    \begin{equation*}
        \lim_{h \to 0} f(x_0 + h) = f(x_0),
    \end{equation*}
    ~~~~$f$是否在$x_0$处连续?
    \medskip

    (4) 设函数$f$在点$x_0$的近旁有定义,并且
    \begin{equation*}
        \lim_{h \to 0} \left(f(x_0+h)-f(x_0-h)\right)=0,
    \end{equation*}
    ~~~~$f$是否在$x_0$处连续?
    \medskip

    (5) 设连续函数$f$在区间$(a,b)$中的全体有理点上取零值,$f$是怎样的函数?
\end{minipage}
\bigskip

(1) \answer 假设可以通过将$f(0)$定义为$y_0 \;(y_0 \in \real)$,使得$f$在$x=0$处连续,那么由于函数$\sin$的值域是$[0,1]$,所以$y_0 \in [0,1]$,可是,由于当$x \to 0$时,$\displaystyle\frac{1}{x} \to \infty$,这使得在$[0,1]$中的任意一点$y$都不是$\sin \, \displaystyle\frac{1}{x}$的极限,故不可能能够在$[0,1]$中找到这样的$y_0$使得$f$在$x=0$处连续.
\bigskip

(2) \answer 所有使得表达式$\displaystyle\frac{1}{x}$有意义的$x$取值的全体.
\bigskip

(3) \answer 可做换元令$x = x_0 + h$,则
\begin{equation}
    \lim_{h \to 0} f(x_0 + h) = \lim_{x \to x_0} f(x) = f(x_0)
\end{equation}
从而$f$在$x_0$处是连续的.

(4) \answer 这样的$f$在$x_0$处不一定连续.考虑所有图像关于$x = x_0$对称却又在$x_0$处没有定义的函数作为反例.
\bigskip

2. 研究下列函数在$x=0$处的连续性:
\begin{enumerate}
    \item $f(x)=\lvert x \rvert$; 
    \item $f(x)=\lfloor x \rfloor$;
    \item $f(x)=\begin{cases}
       \expe^{-1/x^2}, & x \neq 0, \\
       0, & x = 0;
    \end{cases}$ \\
    \item $f(x) = \begin{cases}
        \displaystyle\frac{\sin \, x}{\lvert x \rvert}, & x \neq 0, \\
        1, & x = 0;
    \end{cases}$
    \item $f(x)=\begin{cases}
        (1+x^2)^{1/x^2}, & x \neq 0, \\
        2.7, & x = 0.
    \end{cases}$
\end{enumerate}

(1) \prove 对任意$\epsilon > 0$,取$\delta = \epsilon$,则只要$0 < \lvert x - 0 \rvert < \delta$,就有
\begin{equation}
    \lvert x - 0 \rvert = \lvert x \rvert  < \delta = \epsilon
\end{equation}
从而
\begin{equation}
    \lvert \lvert x \rvert - 0 \rvert < \epsilon
\end{equation}
从而$\displaystyle\lim_{x \to 0} \lvert x \rvert = 0$,从而函数$f: x \mapsto \lvert x \rvert$在$x=0$处是连续的.\qed\bigskip

(2) 函数$f: x \mapsto \lfloor x \rfloor$在$x=0$处不连续.证明如下:\smallskip

\prove 先计算右极限
\begin{equation}
    \lim_{x \to 0^+} \lfloor x \rfloor = \lim_{x \to 0^+} 0 = 0
\end{equation}
再计算左极限
\begin{equation}
    \lim_{x \to 0^-} \lfloor x \rfloor = \lim_{x \to 0^-} -1 = -1
\end{equation}
从而函数$f: x \mapsto \lfloor x \rfloor$在$x=0$处的极限不存在,从而在$x=0$处不连续.\qed\bigskip
