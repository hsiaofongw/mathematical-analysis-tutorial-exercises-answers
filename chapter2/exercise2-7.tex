\exercise

1. 回答下列问题:

\begin{minipage}[l]{0.8\textwidth}
    \setlength{\leftskip}{0.5em}

    \medskip
    (1) 设$f(x)=\sin \, \displaystyle\frac{1}{x}$,能否定义$f(0)$,使得$f$在$x=0$处是连续的?
    \medskip

    (2) 设$f(x)=x \sin \, \displaystyle\frac{1}{x}$,如何理解函数$f$的定义域?
    \medskip

    (3) 设函数$f$在点$x_0$的近旁有定义,并且
    \begin{equation*}
        \lim_{h \to 0} f(x_0 + h) = f(x_0),
    \end{equation*}
    ~~~~$f$是否在$x_0$处连续?
    \medskip

    (4) 设函数$f$在点$x_0$的近旁有定义,并且
    \begin{equation*}
        \lim_{h \to 0} \left(f(x_0+h)-f(x_0-h)\right)=0,
    \end{equation*}
    ~~~~$f$是否在$x_0$处连续?
    \medskip

    (5) 设连续函数$f$在区间$(a,b)$中的全体有理点上取零值,$f$是怎样的函数?
\end{minipage}
\bigskip

(1) \answer 假设可以通过将$f(0)$定义为$y_0 \;(y_0 \in \real)$,使得$f$在$x=0$处连续,那么由于函数$\sin$的值域是$[0,1]$,所以$y_0 \in [0,1]$,可是,由于当$x \to 0$时,$\displaystyle\frac{1}{x} \to \infty$,这使得在$[0,1]$中的任意一点$y$都不是$\sin \, \displaystyle\frac{1}{x}$的极限,故不可能能够在$[0,1]$中找到这样的$y_0$使得$f$在$x=0$处连续.
\bigskip

(2) \answer 所有使得表达式$\displaystyle\frac{1}{x}$有意义的$x$取值的全体.
\bigskip

(3) \answer 可做换元令$x = x_0 + h$,则
\begin{equation}
    \lim_{h \to 0} f(x_0 + h) = \lim_{x \to x_0} f(x) = f(x_0)
\end{equation}
从而$f$在$x_0$处是连续的.

(4) \answer 这样的$f$在$x_0$处不一定连续.考虑所有图像关于$x = x_0$对称却又在$x_0$处没有定义的函数作为反例.
\bigskip

2. 研究下列函数在$x=0$处的连续性:
\begin{enumerate}
    \item $f(x)=\lvert x \rvert$; 
    \item $f(x)=\lfloor x \rfloor$;
    \item $f(x)=\begin{cases}
       \expe^{-1/x^2}, & x \neq 0, \\
       0, & x = 0;
    \end{cases}$ \\
    \item $f(x) = \begin{cases}
        \displaystyle\frac{\sin \, x}{\lvert x \rvert}, & x \neq 0, \\
        1, & x = 0;
    \end{cases}$
    \item $f(x)=\begin{cases}
        (1+x^2)^{1/x^2}, & x \neq 0, \\
        2.7, & x = 0.
    \end{cases}$
\end{enumerate}

(1) \prove 对任意$\epsilon > 0$,取$\delta = \epsilon$,则只要$0 < \lvert x - 0 \rvert < \delta$,就有
\begin{equation}
    \lvert x - 0 \rvert = \lvert x \rvert  < \delta = \epsilon
\end{equation}
从而
\begin{equation}
    \lvert \lvert x \rvert - 0 \rvert < \epsilon
\end{equation}
从而$\displaystyle\lim_{x \to 0} \lvert x \rvert = 0$,从而函数$f: x \mapsto \lvert x \rvert$在$x=0$处是连续的.\qed\bigskip

(2) 函数$f: x \mapsto \lfloor x \rfloor$在$x=0$处不连续.证明如下:\smallskip

\prove 先计算右极限
\begin{equation}
    \lim_{x \to 0^+} \lfloor x \rfloor = \lim_{x \to 0^+} 0 = 0
\end{equation}
再计算左极限
\begin{equation}
    \lim_{x \to 0^-} \lfloor x \rfloor = \lim_{x \to 0^-} -1 = -1
\end{equation}
从而函数$f: x \mapsto \lfloor x \rfloor$在$x=0$处的极限不存在,从而在$x=0$处不连续.\qed\bigskip

(3) \prove 对任意$\epsilon > 0$,取$\delta = - \displaystyle\frac{1}{\sqrt{\lvert \ln \, \epsilon \rvert}}$,那么只要$\lvert x - 0 \rvert < \epsilon$,就有
\begin{equation}
    \lvert \expe^{-1/x^2} \rvert = \expe^{-1/x^2} < \expe^{-1/\delta^2} = \epsilon
\end{equation}
这就说明$\displaystyle\lim_{x \to 0} \expe^{-1/x^2} = 0$.又因为$f(0) = 0$,所以函数$f$在$x=0$处连续.\qed\bigskip

(4) \prove 由于
\begin{equation}
    \lim_{x \to 0^-} \frac{\sin \, x}{\lvert x \rvert} = -1 \neq \lim_{x \to 0^+} \frac{\sin \, x}{\lvert x \rvert} = 1
\end{equation}
故函数$f$在$x=0$处极限不存在,故函数$f$在$x=0$处不连续.\qed\bigskip

(5) \prove 做代换,令$t=1/x$,得
\begin{equation}
    \lim_{x \to 0} \left(1+x^2\right)^{1/x^2} = \lim_{t \to +\infty} \left(1+\displaystyle\frac{1}{t^2}\right)^{t^2}
\end{equation}
再做代换,令$u=t^2$,得
\begin{equation}
    \lim_{t \to +\infty} \left(1+\displaystyle\frac{1}{t^2}\right)^{t^2}=\lim_{u \to \infty} \left(1+\displaystyle\frac{1}{u}\right)^u = \expe
\end{equation}
近似计算$\expe$精确到小数点前2位得
\begin{equation}
    \expe \approx 2.71
\end{equation}
这说明$\displaystyle\lim_{x \to 0} f(x) = \expe \neq f(0) = 2.7$,从而这说明$f$在$x=0$处不连续.\qed\bigskip

3. 定出$a,b$和$c$,使得函数
\begin{equation}
    f(x)=\begin{cases}
        -1, & x \leq -1, \\
        ax^2+bx+c, & \lvert x \rvert < 1 \, \text{且} \, x \neq 0, \\
        0, & x = 0, \\
        1, &x \geq 1
    \end{cases}
\end{equation}
在$(-\infty,+\infty)$上连续.

\bigskip
\solve 只需使$y = ax^2+bx+c$的图像同时经过$(-1,-1), (0,0), (1,1)$这三点,于是我们有方程组:
\begin{equation}
    \begin{cases}
        a-b+c=-1 \\
        c = 0 \\
        a+b+c=1
    \end{cases}
\end{equation}
解出$a=0,b=-1,c=0$.\qed\bigskip

4. 讨论函数$f+g$和$fg$在$x_0$处的连续性,如果:
\begin{enumerate}
    \item $f$在$x_0$处连续,但$g$在$x_0$处不连续;
    \item $f$和$g$在$x_0$处都不连续.
\end{enumerate}
\bigskip

\answer 不确定.\qed\bigskip

5. 设函数$f$在$x_0$处连续,$f(x_0) > 0$.证明:当$x$充分靠近$x_0$时,有
\begin{equation}
    f(x) > \frac{f(x_0)}{2}.
\end{equation}
\bigskip

\prove 由题意知
\begin{equation}
    \lim_{x \to x_0} f(x) = f(x_0) > 0
\end{equation}
故对任意$\epsilon > 0$,都存在$\delta > 0$,使得对一切满足$\lvert x - x_0 \rvert < \delta$的$x$,都有
\begin{equation}
    \lvert f(x) - f(x_0) \rvert < \epsilon
\end{equation}
取$\epsilon_0 = \displaystyle\frac{f(x_0)}{2}$,则存在$\delta_1 > 0$,使得对一切满足$\lvert x - x_0 \rvert < \delta$的$x$,都有
\begin{equation}
    \lvert f(x) - f(x_0) \rvert < \epsilon_0 = \frac{f(x_0)}{2}
\end{equation}
也就是
\begin{equation}
    f(x_0) - \frac{f(x_0)}{2} < f(x) < f(x_0) + \frac{f(x_0)}{2}
\end{equation}
从而命题得证.\qed\bigskip

6. 设函数$f$在$x_0$处连续.证明$\lvert f \rvert$和$f^2$都在$x_0$处连续.反之是否成立?
\bigskip

\prove 由于$f$在$x_0$处连续,所以$f$在$x_0$处有定义,所以$f(x_0)$有定义,所以$\lvert f(x_0) \rvert$有定义,所以$\lvert f \rvert$在$x_0$处有定义.同理,函数$f^2$在$x_0$处有定义.又因为
\begin{align}
    \lim_{x \to x_0} f(x) = f(x_0) \implies \lim_{x \to x_0} \lvert f(x) \rvert = \lvert f(x_0) \rvert
\end{align}
所以函数$\lvert f \rvert$在$x_0$处连续.又因为
\begin{align}
    \lim_{x \to x_0} f(x) = f(x_0) &\implies \lim_{x \to x_0} f(x) \lim_{x \to x_0} f(x) = (f(x_0))^2 \\
    &\implies \lim_{x \to x_0} \left(f(x)\right)^2 = \left(f(x_0)\right)^2
\end{align}
所以函数$f^2$在$x_0$处连续.

反之未必成立,令
\begin{equation}
    f(x) = \begin{cases}
        1, & x \geq 0 \\
        -1, & x < 0
    \end{cases}
\end{equation}
则$\lvert f(x) \rvert = 1, \left(f(x)\right)^2 = 1, \; \forall x \in \real$,并且$\lvert f \rvert, f^2$在$\real$上每一点都是连续的,可是$f$在$x=0$处并不连续.\qed\bigskip

7. 设函数$f, g$在$(a,b)$连续,又令
\begin{align}
    F(x) &= \max(f(x),g(x)) \quad (a < x < b), \\
    G(x) &= \min(f(x),g(x)) \quad (a < x < b).
\end{align}
证明:$F$和$G$是$(a,b)$上的连续函数.\bigskip

\prove 任取$x_0 \in (a,b)$,不失一般性,假设$f(x_0)\geq g(x_0)$,从而$\max(f(x_0),g(x_0))=f(x_0)$,由于
\begin{equation}
\displaystyle\lim_{x \to x_0} f(x) = f(x_0) > \lim_{x \to x_0} g(x) = g(x_0)
\end{equation}
所以存在一个足够小的$\delta_1 > 0$,使得对于$x_0$的某足够小邻域$B(\check{x_0},\delta_1)$内的每一点$x$都有
\begin{equation}
    f(x) \geq g(x)
\end{equation}
也就是说在该邻域内恒有$f(x)=\max(f(x),g(x))$,又因为$\displaystyle\lim_{x \to x_0} f(x) = f(x_0)$,所以,对任意$\epsilon > 0$,存在$\delta > 0$,只有$0 < \lvert x - x_0 \rvert < \min \{ \delta_1, \delta \}$,就有
\begin{equation}
    \lvert f(x) - f(x_0) \rvert = \lvert \max(f(x),g(x)) - \max(f(x_0), g(x_0)) \rvert < \epsilon
\end{equation}
从而这就证明了函数$F$在$(a,b)$是连续的.同理可证$G$在$(a,b)$是连续的.\qed\bigskip

\annotate 关键的一点在于,任取$(a,b)$内一点$x_0$,如果$f(x_0) \geq g(x_0)$,那么由$f, g$的连续性,在$x_0$的\textbf{一个足够小的邻域}内,一点会有$f(x) \geq g(x)$,这时$\max \{ f(x), g(x) \} = f(x)$,从而在该足够小的邻域内讨论$\max \{ f(x), g(x) \}$的连续性等价于讨论$f$的连续性,而$f$是连续的,故$\max \{ f(x), g(x) \}$连续.\bigskip

8. 设函数$f$只有可去间断点,又令
\begin{equation}
    g(x) = \lim_{t \to x} f(t).
\end{equation}
证明:$g$是连续函数.\bigskip

\prove 设$f$是定义在$(a,b)$上的函数,且$f$在$(a,b)$上只含有可去间断点.于是,任取$x_0 \in (a,b)$,$\displaystyle\lim_{t \to x_0} f(t)$都存在,也就是说$g(x_0)$有定义.

再证$\displaystyle\lim_{x \to x_0}\lim_{t \to x} f(t) = \lim_{t \to x_0} f(t)$.

由于$\displaystyle\lim_{x \to x_0} f(x_0)$存在,故对任意$\epsilon > 0$,存在$\delta_1 > 0$,使得只要$\lvert x - x_0 \rvert < \delta_1$,就有
\begin{equation}
    \lvert f(x) - \lim_{x \to x_0} f(x) \rvert < \epsilon
\end{equation}
也就是
\begin{equation}
    \lim_{x \to x_0} f(x) - \epsilon < f(x) < \lim_{x \to x_0} f(x) + \epsilon
\end{equation}
由于$\lvert x - x_0 \rvert < \delta$,故可以找到足够小的正数$\delta_2$使得$B(\check{x}, \delta_2) \subset B(\check{x_0}, \delta_1)$,并且使得对一切$t \in B(\check{x},\delta_2)$,都有$t \in B(\check{x_0},\delta_1)$,从而有
\begin{equation}
    \lim_{x \to x_0} f(x) - \epsilon < f(t) < \lim_{x \to x_0} f(x) + \epsilon
\end{equation}
由于该不等式对一切$t \in B(\check{x}, \delta_2)$都成立,故
\begin{equation}
    \lim_{t \to x}\lim_{x \to x_0} f(x) - \epsilon < \lim_{t \to x} f(t) < \lim_{t \to x}\lim_{x \to x_0} f(x) + \epsilon
\end{equation}
上式中$\displaystyle\lim_{x \to x_0} f(x)$为常数,故上式又可写为
\begin{equation}
    \lim_{x \to x_0} f(x) - \epsilon < \lim_{t \to x} f(t) < \lim_{x \to x_0} f(x) + \epsilon
\end{equation}
也就是
\begin{equation}
    \lvert \lim_{t \to x} f(t) - \lim_{x \to x_0} f(x) \rvert < \epsilon
\end{equation}
也就是$\displaystyle\lim_{x \to x_0} \lim_{t \to x} f(t) = \displaystyle\lim_{x \to x_0} f(x)$,即$\displaystyle\lim_{x \to x_0} g(x) = g(x_0)$,从而这就证得了$g$在$(a,b)$上的每一点都连续.\qed\bigskip

9. 设函数$f$在$\real$上递增(或递减).若定义$F(x)=f(x+)$.试证明:$F$在$\real$上右连续.\bigskip

\prove 任取$x_0 \in \real$,由于$f$是递增函数,依\textbf{定理2.7.5},$f$的间断点一定是跳跃间断点,并且$f$只有至多可数多个跳跃间断点,故$F(x_0) = f(x_0 +) = \displaystyle\lim_{x \to x_0^+} f(x)$总是存在.下证$\displaystyle\lim_{t \to x_0} F(t) = F(x_0)$.

由于$\displaystyle\lim_{t \to x_0^+} f(t)$存在,故对任意$\epsilon > 0$,存在$\delta_1 > 0$,使得对一切$0 < t - x_0 < \delta_1$都有
\begin{equation}
    0 < f(t) - \lim_{t \to x_0^+} f(t) < \epsilon
    \label{ieq:2734}
\end{equation}
当$0 < x - x_0 < \delta_1$时,存在$r$,使得$x+r < x_0+\delta_1$,从而一切$0< t-x  < r$不等式(\ref{ieq:2734})都成立,于是由\textbf{定理2.4.6},有
\begin{equation}
    0 < \lim_{t \to x^+} f(t) - \lim_{t \to x_0^+} f(t) < \epsilon
\end{equation}
故$\displaystyle\lim_{x \to x_0^+} \displaystyle\lim_{t \to x^+} f(t) = \lim_{t \to x_0^+} f(t)$.从而$\displaystyle\lim_{t \to x_0} F(t) = F(x_0)$.\qed\bigskip