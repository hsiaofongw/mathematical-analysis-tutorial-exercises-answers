\question

1. 设$f$在$(0,+\infty)$上满足函数方程$f(2x)=f(x) \, (x>0)$,并且$\displaystyle\lim_{x \to +\infty} f(x)$存在且有限.求证:$f$为常值函数.

\begin{proof}
采用反证法.假设存在$x_1, x_2 \in (0, +\infty), x_1 \neq x_2$,使得$f(x_1) \neq f(x_2)$.

由题设$f(2^n x_2) = f(x_2), \forall n \in \nat$,并且$f(2^n x_1) = f(x_1)$.设$\displaystyle\lim_{x \to +\infty} f(x) = c$,那么,对任意$\epsilon > 0$,存在$A > 0$,使得对一切$x > A$,都有
\begin{equation}
    \lvert f(x) - c \rvert < \epsilon
\end{equation}
由于当$n \to +\infty$时$2^n x_2, 2^n x_1 \to \infty$,所以存在$N \in \nat$,使得对一切$n \geq N$都有$2^n x_1, 2^n x_2 > A$,也就是说
\begin{align}
    \lvert f(2^n x_1) - c \rvert < \epsilon, \quad \lvert f(2^n x_2) - c \rvert < \epsilon
\end{align}
而这与$f(2^n x_1) \neq f(2^n x_2)$矛盾.
\end{proof}

2. 设$a$和$b$是两个大于$1$的常数,函数$f: \real \to \real$在$x=0$的邻域内有界,并且对一切$x \in \real$,有$f(ax) = bf(x)$.求证:
\begin{equation}
    \lim_{x \to 0} f(x) = f(0).
\end{equation}

\begin{proof}
依题意得
\begin{align}
    &\mathrel{\phantom{\implies}} \forall x \in \real, f(ax) = bf(x) \\
    &\implies f(a \cdot 0) = bf(0) \\
    &\implies f(0) = bf(0) \\
    &\implies 0 = (b-1)f(0) \\
    &\implies 0 = f(0)
\end{align}
采用反证法.假设结论不成立,也就是说,假设$\displaystyle\lim_{x \to 0} f(x)$不存在或者$\displaystyle\lim_{x \to 0} \neq f(0) = 0$.依题意,$f$在$x=0$的邻域有界,也就是说,存在$\delta_0, M > 0$,使得对一切$x \in B(\check{0}, \delta_0)$都有
\begin{equation}
    \lvert f(x) \rvert \leq M
\end{equation}
由函数极限定义的否定形式,存在$\epsilon_0 > 0$,使得对任意$\delta > 0$,都存在$x \in B(\check{0}, \delta)$,使得
\begin{equation}
    \lvert f(x) - 0 \rvert \geq \epsilon_0 \implies \lvert f(x) \rvert \geq \epsilon_0
\end{equation}
事实上,由于$b > 1$,故存在足够大的正整数$N$,使得
\begin{equation}
    b^N \epsilon_0 > M
\end{equation}
与此同时,存在足够小的$\delta_1 > 0$,满足$\delta_1 a^N < \delta$,由函数极限定义的否定形式,存在$x_0 \in B(\check{0},\delta_1)$,使得
\begin{equation}
    \lvert f(x_0) \rvert \geq \epsilon_0
\end{equation}
依题意有
\begin{equation}
    \lvert f(a^N x_0) \rvert = \lvert b^N f(x_0) \rvert \geq b^N \epsilon_0 > M
\end{equation}
由于$0 < x_0 < \delta_1$,所以$0 < a^N x_0 < a^N \delta_1$,又由于$\delta_1 a^N < \delta$,所以$0< a^N x_0 < \delta$,也就是说$a^N x_0 \in B(\check{0}, \delta)$,但是却有$\lvert f(a^N x_0) \rvert > M$,这与$f$在$B(\check{0}, \delta)$有界的题设矛盾.
\end{proof}

3. 设$f$和$g$是两个周期函数,且满足:
\begin{equation}
    \lim_{x \to +\infty} \left(f(x) - g(x)\right) = 0.
\end{equation}
证明:$f=g$.

\begin{proof}
设$l_1>0$是$f$的一个周期,设$l_2>0$是$g$的一个周期.设$f$与$g$的公共定义域是$D$,任取$a \in D \, (a > 0)$,并且令
\begin{align}
    x_n = a + n l_1, \quad y_n = a + n l_2
\end{align}
由周期性,对任意$n \in \nat$,恒有$f(x_n) = f(a)$,以及$g(y_n) = g(a)$.由于$l_1, l_2 > 0$,所以$\{ x_n \}$与$\{ y_n \}$都是趋于正无穷的数列.由归结原则
\begin{align}
    &\mathrel{\phantom{\implies}} \lim_{n \to +\infty} \left(f(x_n) - g(x_n)\right) = 0 \\
    &\implies \lim_{n \to +\infty} \left(f(a) - g(x_n)\right) = 0 \\
    &\implies \lim_{n \to +\infty} g(x_n) = f(a) \label{eq:gxnconverge}
\end{align}
同理有
\begin{equation}
    \lim_{n \to +\infty} f(y_n) = g(a) \label{eq:fxnconverge}
\end{equation}
由于$\displaystyle\lim_{x \to +\infty} \left(f(x) - g(x)\right) = 0$,所以,对任意$\epsilon > 0$,存在$A > 0$,使得只要$x > A$,就有
\begin{equation}
    \lvert f(x) - g(x) \rvert < \epsilon
\end{equation}
由于当$n \to +\infty$时$x_n, y_n \to +\infty$,所以,存在$N \in \nat$,使得对一切$n > N$,都满足
\begin{equation}
    x_n, y_n \geq \min \{ x_n, y_n \} > A
\end{equation}
所以有
\begin{equation}
\lvert f(x_n) - g(x_n) \rvert < \epsilon, \quad \lvert f(y_n) - g(y_n) \rvert < \epsilon
\end{equation}
这说明
\begin{align}
    \lim_{n \to +\infty} \left( f(x_n) - g(x_n) \right) &= 0 \label{eq:fxnequalgxn} \\
    \lim_{n \to +\infty} \left(f(y_n) - g(y_n)\right) &= 0 \label{eq:fynequalgyn}
\end{align}
让式(\ref{eq:gxnconverge})与式(\ref{eq:fxnequalgxn})两端相加,由极限的线性性质我们得到
\begin{equation}
    \lim_{n \to +\infty} g(x_n) + \lim_{n \to +\infty} \left(f(x_n) - g(x_n)\right) = f(a) + 0
\end{equation}
化简得
\begin{equation}
    \lim_{n \to +\infty} f(x_n) = f(a)
\end{equation}
再利用式(\ref{eq:gxnconverge})就有
\begin{equation}
    \lim_{n \to +\infty} g(x_n) = \lim_{n \to +\infty} f(x_n)
\end{equation}
于是根据式(\ref{eq:gxnconverge})与式(\ref{eq:fxnconverge}),我们得
\begin{equation}
    f(a) = g(a)
\end{equation}
又由于$a$的任意性,我们知道对任意$a \in D$,都有$f(a) = g(a)$,从而这就证得了$f = g$.
\end{proof}