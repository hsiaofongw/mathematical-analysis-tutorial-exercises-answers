\exercise

1. 求下列函数的定义域:

\begin{table}[H]
    \centering
    \begin{tabularx}{\textwidth} {  >{\raggedright\arraybackslash}X >{\raggedright\arraybackslash}X  }
        (1)~$f(x)=\displaystyle\sqrt{1-x^2}$; & (2)~$f(x)=\displaystyle\sqrt[3]{\displaystyle\frac{1+x}{1-x}}$; \\ [1em]
        (3)~$f(x)=\displaystyle\frac{x+1}{x^2 + x -2}$; & (4)~$f(x) = \ln \, \displaystyle\frac{1+\sin \, x}{1-\cos \, x}$.
      \end{tabularx}
\end{table}
(1) 解:设定义域为$D$:
\begin{align}
    &\mathrel{\phantom{\implies}} \forall x \in D, \exists y \in \real, y = \displaystyle\sqrt{1-x^2} \\
    &\implies 1 - x^2 \geq 0 \\
    &\implies 0 \leq x^2 \leq 1 \implies -1 \leq x \leq 1 \\
    &\implies D = [-1, 1].
\end{align}
(2) 解:设定义域为$D$:
\begin{align}
    &\mathrel{\phantom{\implies}} \forall x \in D, \exists y \in \real, y = \displaystyle\sqrt[3]{\displaystyle\frac{1+x}{1-x}} \\
    &\implies 1 - x \neq 0 \implies x \neq 1 \implies D = \real \setminus \{ 1 \}.
\end{align}
(3) 解:设定义域为$D$:
\begin{align}
    &\mathrel{\phantom{\implies}} \forall x \in D, \exists y \in \real, y = \displaystyle\frac{x+1}{x^2 +x - 2} \\
    &\implies x^2 + x - 2 \neq 0 \implies (x+2)(x-1) \neq 0 \implies x \neq 1 \; \land \; x \neq -2 \\
    &\implies D = \real \setminus \{ 1, 2 \}.
\end{align}
(4) 解:设定义域为$D$:
\begin{align}
    &\mathrel{\phantom{\implies}} \forall x \in \real, \exists y \in \real, y = \ln \, \frac{1+\sin \, x}{1+ \cos \, x} \\
    &\implies \frac{1+\sin \, x}{1+\cos \, x} > 0 \implies (1+\sin \, x)(1+\cos \, x) > 0  \\
    &\implies 1 + \cos \, x \neq 0 \implies \cos \, x \neq -1 \implies x \not\in \{x : \cos \, x = -1, x \in \real \} \\
    &\implies x \not\in \{ 2\pi n + \frac{3\pi}{2} : n \in \integer \} \implies D = \real \setminus \{ 2 \pi n + \frac{3 \pi}{2} : n \in \integer \}
\end{align}

2. 给定函数$f: \real \longrightarrow \real$,如果$x \in R$使$f(x)=x$,则称$x$为$f$的一个不动点.若$f \circ  f$有唯一的不动点,求证:$f$也有唯一的不动点.

\begin{proof}
设$x_0$是$f \circ f$的不动点,依题意,这个$x_0$是唯一的仅此一个.如果$x_0$是$f \circ f$的不动点,那么
\begin{equation}
    f(f(f(x_0))) = f(x_0)
\end{equation}
这说明$f(x_0)$是$f \circ f$的不动点,而作为$f \circ f$的不动点的$x_0$又是唯一的,所以$f(x_0) = x_0$,所以$f$有唯一的不动点$x_0$.命题得证.
\end{proof}

3. 设$f: \real \longrightarrow \real$.若$f \circ f$有且仅有两个不动点$a, b\,(a \neq b)$,求证只有一下两种情况:
\begin{enumerate}
    \item $a, b$都是$f$的不动点;
    \item $f(a) = b, \, f(b) = a$.
\end{enumerate}

\begin{proof}
设$a, b (a \neq b)$是$f \circ f$仅有的为数不多的两个不动点.依题2的证明结论,如果$x$是$f \circ f$的不动点那么$f(x)$必定也是$f \circ f$的不动点,因为$a$是$f \circ f$的不动点,所以$f(a)$是$f \circ f$的不动点,而$f \circ f$的全部不动点构成一个有限集合
\begin{equation}
    \{ a, b \}
\end{equation}
所以
\begin{equation}
    f(a) \in \{a, b \}
\end{equation}
如果说$f(a) = a$,那么同理可推出$f(b)=b$,那么$a, b$都是$f$的不动点,否则如果$f(a) = b$那么可同理推出$f(b)=a$.
\end{proof}

4. 设函数$f: \real \longrightarrow \real$,且每一个实数都是$f \circ f$的不动点.试问:
\begin{enumerate}
    \item 有几个这样的函数?
    \item 若$f$在$\real$上递增,有几个这样的函数?
\end{enumerate}

(1) 解:4个,分别是:
\begin{align}
    f_1(x) &= x \\
    f_2(x) &= -x \\
    f_3(x) &= \begin{cases}
        \displaystyle\frac{1}{x}, & x \neq 0 \\
        0, & x = 0
    \end{cases} \\
    f_4(x) &= \begin{cases}
        \displaystyle-\frac{1}{x}, & x \neq 0 \\
        0, & x = 0
    \end{cases}
\end{align}

(2) 解:1个,是
\begin{equation}
    f(x) = x.
\end{equation}

5. 求下列函数$f$的$n$次复合$f^n$:

\begin{enumerate}
    \item $f(x) = \displaystyle\frac{x}{\displaystyle\sqrt{1+x^2}}$;
    \item $f(x) = \displaystyle\frac{x}{1+bx}$.
\end{enumerate}

(1) 解:
当$n = 0$时,$f^0 (x) = I (x) = x = \displaystyle\frac{x}{\displaystyle\sqrt{1+0 x^2}}$;

当$n=1$时,$f^1 (x) = f(x) = \displaystyle\frac{x}{\displaystyle\sqrt{1+x^2}}=\displaystyle\frac{x}{\displaystyle\sqrt{1+1 x^2}}$;

当$n=2$时,$f^2 (x) = f(f(x)) = \displaystyle\frac{\displaystyle\frac{x}{\displaystyle\sqrt{1+x^2}}}{\displaystyle\sqrt{1+\left(\displaystyle\frac{x}{\displaystyle\sqrt{1+x^2}}\right)^2}} = \displaystyle\frac{x}{\displaystyle\sqrt{1+2x^2}}$;

采用数学归纳法来证明:假设对于某个非负整数$k$有:
\begin{equation}
    f^k (x) = \frac{x}{\sqrt{1+kx^2}}
\end{equation}
成立,那么
\begin{equation}
    f^{k+1}(x) = f(f^k(x)) = \frac{\displaystyle\frac{x}{\sqrt{1+kx^2}}}{\sqrt{1+\left(\displaystyle\frac{x}{\sqrt{1+kx^2}}\right)^2}} = \frac{x}{\sqrt{1+(k+1)x^2}}
\end{equation}
依数学归纳法原理,对任意非负整数$n$,都有$f^n (x) = \displaystyle\frac{x}{\displaystyle\sqrt{1+nx^2}}$成立.

(2) 解:当$n=0$时:
\begin{equation}
    f^0 (x) = I(x) = x = \frac{x}{1 + 0bx}
\end{equation}
当$n=1$时:
\begin{equation}
    f^1 (x) = f(x) = \frac{x}{1+bx} = \frac{x}{1+1bx}
\end{equation}
当$n=2$时:
\begin{equation}
    f^2 (x) = f(f(x)) = \frac{\displaystyle\frac{x}{1+bx}}{1+\displaystyle\frac{bx}{1+bx}} = \displaystyle\frac{x}{1+2bx}
\end{equation}
猜测对一般的$n \in \nnat$都有$f^n (x) = \displaystyle\frac{x}{1+nbx}$成立,采用数学归纳法来证明,假设对于某个$k \in \nnat$,当$n=k$时命题是成立的,那么当$n=k+1$时:
\begin{equation}
    f^{k+1}(x) = f(f^k (x)) = \displaystyle\frac{\displaystyle\frac{x}{1+kbx}}{1+\displaystyle\frac{bx}{1+kbx}} = \displaystyle\frac{x}{1+(k+1)bx}
\end{equation}
依数学归纳法原理,对任意$n \in \nnat$命题都成立.

6. 设$f: \real \longrightarrow \real$满足方程
\begin{equation}
    f(x+y)=f(x)+f(y) \quad (x,y \in \real)
\end{equation}
试证:对一切有理数$x$,有$f(x) = xf(1)$.

\begin{proof}
设$x$是一个非零有理数,那么存在$p, q \in \integer, p, q \neq 0$使得$x = \displaystyle\frac{p}{q}$,于是
\begin{equation}
    f(x) = f(\displaystyle\frac{p}{q}) = f(\underbrace{\displaystyle\frac{1}{q}+\cdots+\displaystyle\frac{1}{q}}_{p\text{个}}) = \underbrace{f(\displaystyle\frac{1}{q})+\cdots+f(\displaystyle\frac{1}{q})}_{p\text{个}} = p f(\displaystyle\frac{1}{q})
    \label{eq:rational1}
\end{equation}
让$q$个$f(\displaystyle\frac{1}{q})$相加,得到
\begin{equation}
    \underbrace{f(\displaystyle\frac{1}{q})+\cdots+f(\displaystyle\frac{1}{q})}_{q\text{个}} = f(\underbrace{\displaystyle\frac{1}{q}+\cdots+\displaystyle\frac{1}{q}}_{q\text{个}})=f(q \cdot \displaystyle\frac{1}{q}) = f(1)
\end{equation}
从而得$\displaystyle\frac{1}{q}=\displaystyle\frac{1}{q}f(1)$,于是代入式(\ref{eq:rational1})得
\begin{equation}
    f(x) = p f(\displaystyle\frac{1}{q}) = p \cdot \displaystyle\frac{1}{q}f(1) = \displaystyle\frac{p}{q}f(1)=xf(1)
\end{equation}
以上是当$x \neq 0$的情形,如果$x=0$,那么依题意
\begin{equation}
    f(0 + 0) = f(0) + f(0)
\end{equation}
从而
\begin{equation}
    f(0) = 0 = 0 f(1)
\end{equation}
于是对一切$x \in \rational$命题成立.
\end{proof}

7. 设函数$f: \real \longrightarrow \real$,$l$为一正数,如果$f(x+l)=f(x)$对一切$x$成立,则称$f$是周期为$l$的{\bfseries{周期函数}}.如果$f$以任何正数为周期,求证:$f$为常值函数.
\begin{proof}
采用反证法,假设$f$不是常值函数,那么就存在$x_1, x_2 \in \real, x_1 < x_2$使得$f(x_1) \neq f(x_2)$,令$d = x_2 - x_1$,那么就有$f(x_1)\neq f(x_2)=f(x_1 + d)$,这与周期函数的定义不符,矛盾.
\end{proof}

8. 试证$\sin \, \left(x^2\right), \sin \, x + \cos \, \sqrt{2}x$均不是周期函数.
\begin{proof}
采用反证法,设存在$l > 0$,使得$\sin \, \left(x+l\right)^2=\sin \, \left(x\right)^2$对一切$x \in \real$成立.那么对任意$x \in \real$,有
\begin{align}
    \sin \, \left(x+l\right)^2 = \sin \, \left(x\right)^2 &\implies \sin \, \left(x+l\right)^2-\sin \, \left(x\right)^2 = 0 \\
    &\implies \sin \, \left(x^2 + 2xl + l^2\right) - \sin \, \left(x^2\right) = 0 \\
    &\implies 2xl+l^2 = 2\pi n, \, n \in \integer \\
    &\implies x = \frac{\pi n}{l} - \frac{l}{2}
\end{align}
可是当$x \not\in \{ \displaystyle\frac{\pi n}{l} - \displaystyle\frac{l}{2} : n \in \integer \}$时,会有$2xl+l^2 \neq 2\pi n$,于是$\sin \, \left(x^2+2xl+l^2\right)^2 \neq \sin \left( x \right)^2$,矛盾.
\end{proof}
\begin{proof}
采用反证法,设存在$l > 0$,使得$\sin \, x + \cos \, \sqrt{2} x$成为一周期为$l$的周期函数.那么,对任意$x \in \real$都有$\sin \, (x + l) + \cos \, (\sqrt{2}(x+l)) = \sin \, x + \cos \, \sqrt{2}x$成立,也就是
\begin{align}
    &\mathrel{\phantom{\implies}} \sin \, (x+l) + \cos \, (\sqrt{2}x + \sqrt{2}l) - \sin \, x - \cos \, \sqrt{2} x  = 0 \\
    &\implies \sin \, x (\cos \, l - 1) + \sin \, l \cos \, x + \cos \, \sqrt{2} x ( \cos \, \sqrt{2} l - 1) - \sin \, \sqrt{2} x \sin \, \sqrt{2} l = 0 \label{eq:irrelavant}
\end{align}
考虑函数列$\sin \, x, \cos \, x, \cos \, \sqrt{2}x , \sin \, \sqrt{2} x$的Wronskian行列式:
\begin{equation}
    W(x) = \begin{vmatrix}
        \sin \, x & \cos \, x & \cos \, \sqrt{2} x & \sin \, \sqrt{2} x \\
        \cos \, x & -\sin \, x & -\sqrt{2} \sin \, \sqrt{2} x & \sqrt{2} \cos \, \sqrt{2} x \\
        -\sin \, x & -\cos \, x & -2 \cos \sqrt{2} x & -2 \sin \, \sqrt{2} x \\
        -\cos \, x & \sin \, x & 2\sqrt{2} \sin \, \sqrt{2} x & -2\sqrt{2} \cos \, \sqrt{2} x 
    \end{vmatrix}
\end{equation}
取$x_0 = 0$,去计算$W(x_0)$:
\begin{align}
    &W(x_0) = \begin{vmatrix}
        0 & 1 & 1 & 0  \\
        1 & 0 & 0 & \sqrt{2} \\
        0 & -1 & -2 & 0 \\
        -1 & 0 & 0 & -2\sqrt{2}
    \end{vmatrix} \xrightarrow{[3] + 2 \times [1]} \begin{vmatrix}
        0 & 1 & 1 & 0 \\
        1 & 0 & 0 & \sqrt{2} \\
        0 & 1 & 0 & 0 \\
        -1 & 0 & 0 & -2\sqrt{2}
    \end{vmatrix} \xrightarrow{[4]+[2]} \begin{vmatrix}
            0 & 1 & 1 & 0 \\
            1 & 0 & 0 & \sqrt{2} \\
            0 & 1 & 0 & 0 \\
            0 & 0 & 0 & -\sqrt{2}
    \end{vmatrix} \\
    &\xrightarrow{[1] + (-1) \times [3]} \begin{vmatrix}
        0 & 0 & 1 & 0 \\
        1 & 0 & 0 & \sqrt{2} \\
        0 & 1 & 0 & 0 \\
        0 & 0 & 0 & -\sqrt{2}
    \end{vmatrix} \xrightarrow{[1] \longleftrightarrow [2]} \begin{vmatrix}
        1 & 0 & 0 & \sqrt{2} \\
        0 & 0 & 1 & 0 \\
        0 & 1 & 0 & 0 \\
        0 & 0 & 0 & -\sqrt{2}
    \end{vmatrix} \xrightarrow{[2] \longleftrightarrow [3]} \begin{vmatrix}
            1 & 0 & 0 & \sqrt{2} \\
            0 & 1 & 0 & 0 \\
            0 & 0 & 1 & 0 \\
            0 & 0 & 0 & -\sqrt{2}
    \end{vmatrix} \\
    &\xrightarrow{[1] + [4]} \begin{vmatrix}
        1 & 0 & 0 & 0 \\
            0 & 1 & 0 & 0 \\
            0 & 0 & 1 & 0 \\
            0 & 0 & 0 & -\sqrt{2}
    \end{vmatrix} = -\sqrt{2} \neq 0
\end{align}
因此函数列$\sin \, x, \cos \, x, \cos \, \sqrt{2}x , \sin \, \sqrt{2} x$线性无关,因此从式(\ref{eq:irrelavant})右端等于$0$可推出
\begin{equation}
    \begin{cases}
        (\cos \, l) - 1 &= 0 \\
        \sin \, l &= 0 \\
        (\cos \, \sqrt{2} l) - 1 &= 0 \\
        \sin \, \sqrt{2} l &= 0
    \end{cases}
\end{equation}
这说明$l = 0$,从而与$l > 0$矛盾,从而不存在这样的$l > 0$使得题设函数成为一周期函数.
\end{proof}

9. 设函数$f: (-a, a) \longrightarrow \real$.如果对任何$x \in (-a, a)$,有$f(x) = f(-x)$,则称$f$为偶函数;若$f(x) = -f(-x)$,则称$f$为奇函数.求证:$(-a, a)$上的任何函数均可表示为一个奇函数和一个偶函数之和.

\begin{proof}
令
\begin{align}
    p: (-a, a) \;\; &\longrightarrow \;\; \real \\
    x \;\; &\longmapsto \;\; \displaystyle\frac{f(x)+f(-x)}{2}
\end{align}
再令
\begin{align}
    q: (-a, a) \;\; &\longrightarrow \;\; \real \\
    x \;\; &\longmapsto \;\; \displaystyle\frac{f(x)-f(-x)}{2}
\end{align}
则
\begin{equation}
    p(-x) = \displaystyle\frac{f(-x)+f(-(-x))}{2} = \displaystyle\frac{f(-x)+f(x)}{2} = \displaystyle\frac{f(x)+f(-x)}{2} = p(x)
\end{equation}
这说明$p$在$(-a, a)$上成为一偶函数.又有
\begin{equation}
    q(-x) = \displaystyle\frac{f(-x)-f(-(-x))}{2} = \displaystyle\frac{f(-x)-f(x)}{2} = -\displaystyle\frac{f(x)-f(-x)}{2} = -q(x)
\end{equation}
这说明$q$在$(-a, a)$上成为一奇函数.并且
\begin{equation}
    p(x)+q(x) = \displaystyle\frac{f(x)+f(-x)}{2} + \displaystyle\frac{f(x)-f(-x)}{2} = f(x)
\end{equation}
这就将函数$f$表示成了一个偶函数$p$和一个奇函数$q$的和.
\end{proof}

10. 函数
\begin{equation}
    \cosh \, x = \frac{\expe^x + \expe^{-x}}{2}, \quad \sinh \, x = \frac{\expe^{x} - \expe^{-x}}{2}
\end{equation}
分别称为{\bfseries{双曲正弦}}和{\bfseries{双曲余弦}}.证明:
\begin{enumerate}
    \item $\cosh$是偶函数,$\sinh$是奇函数;
    \item $(\cosh \, x)^2 - (\sinh \, x)^2 = 1 \; (x \in \real)$.
\end{enumerate}

\begin{proof}
计算得
\begin{equation}
    \cosh \, (-x) = \frac{\expe^{-x} + \expe^{-(-x)}}{2} = \frac{\expe^{-x}+\expe^{x}}{2} = \frac{\expe^x + \expe^{-x}}{2} = \cosh \, x
\end{equation}
这说明$\cosh$是偶函数.又有
\begin{equation}
    \sinh \, (-x) = \frac{\expe^{-x} - \expe^{-(-x)}}{2} = \frac{\expe^{-x}-\expe^{x}}{2} = - \frac{\expe^{x} - \expe^{-x}}{2} = - \sinh \, x
\end{equation}
这说明$\sinh$是奇函数.
\end{proof}

\begin{proof}
    \begin{align}
        \text{左边} &= (\cosh \, x)^2 - (\sinh \, x)^2 = (\cosh \, x + \sinh \, x)(\cosh \, x - \sinh \, x) \\
        &= \expe^x \cdot \expe^{-x} = \expe^0 = 1 = \text{右边}
    \end{align}
\end{proof}