\exercise

1. 计算下列极限:
\begin{enumerate}
    \item $\displaystyle\lim_{x \to 1} \displaystyle\frac{x^2-1}{2x^2-x-1}$; 
    \item $\displaystyle\lim_{x \to 0} \displaystyle\frac{(1+x)^5-(1+5x)}{x^2+x^5}$;
\end{enumerate}
\pagebreak
\begin{enumerate}
\setcounter{enumi}{2}
    \item $\displaystyle\lim_{x \to 0} \displaystyle\frac{(1+x)(1+2x)(1+3x)-1}{x}$;
    \item $\displaystyle\lim_{x \to a} \displaystyle\frac{(x^n-a^n)-na^{n-1}(x-a)}{(x-a)^2} \; (n \in \nat)$;
    \item $\displaystyle\lim_{x \to 1} \displaystyle\frac{x^{n+1}-(n+1)x+n}{(x-1)^2} \; (n\in\nat)$.
\end{enumerate}
\bigskip

(1) \solve 因式分解得
\begin{equation}
    \lim_{x \to 1} \frac{x^2-1}{2x^2-x-1} = \lim_{x \to 1} \frac{(x-1)(x+1)}{(x-1)(2x+1)} = \lim_{x \to 1} \frac{x+1}{2x+1} = \frac{2}{3}.
\end{equation}
\qed\bigskip

(2) \solve 使用二项式定理,将$(1+x)^5$的高次项之和表为高阶无穷小:
\begin{equation}
    \lim_{x \to 0} \frac{1+5x+\binom{5}{2}x^2+o(x^2)-(1+5x)}{x^2+x^5} = \lim_{x \to 0} \frac{10x^2+o(x^2)}{x^2+x^5} = \lim_{x \to 0} \frac{10+\displaystyle\frac{o(x^2)}{x^2}}{1+x^3} = 10.
\end{equation}
\qed\bigskip

(3) \solve 只考虑分子的常数项和$1$次项,并将分子的其他项表为高阶无穷小:
\begin{equation}
    \text{原式} = \lim_{x \to 0} \frac{1 + 6x + o(x) - 1}{x} = \lim_{x \to 0} \frac{6x + o(x)}{x} = \lim_{x \to 0} \frac{6 + \displaystyle\frac{o(x)}{x}}{1} = 6.
\end{equation}
\qed\bigskip

(4) \solve 令$t = x-a$,那么$x^n = (t+a)^n$,从而我们可以使用二项式定理:
\begin{align}
    \text{原式} &= \lim_{t \to 0} \frac{(t+a)^n-a^n-na^{n-1}t}{t^2} \\
    &= \lim_{t \to 0} \frac{a^n + nt a^{n-1} + \binom{n}{2}t^2 a^{n-2} + o(t^2) - a^n - na^{n-1}t}{t^2}  \\
    &= \lim_{t \to 0} \frac{\binom{n}{2}t^2 a^{n-2}+o(t^2)}{t^2}  = \binom{n}{2}a^{n-2} , \quad (n \geq 2)
\end{align}
\qed\bigskip

(5) \solve 令$t = x-1$,那么
\begin{align}
    \text{原式} &= \lim_{t \to 0} \frac{(t+1)^{n+1}-(n+1)(t+1)+n}{t^2} \\
    &= \lim_{t \to 0} \frac{1 + (n+1)t + \binom{n+1}{2}t^2 + o(t^2) - (n+1)(t+1)+n}{t^2} \\
    &= \lim_{t \to 0} \frac{\binom{n+1}{2}t^2+o(t^2)}{t^2} = \binom{n+1}{2}.
\end{align}
\qed\bigskip

\pagebreak
2. 计算下列极限:
\begin{enumerate}
    \item $\displaystyle\lim_{x \to 1}\displaystyle\frac{\sqrt[m]{x}-1}{\sqrt[n]{x}-1} \; (m,n \in \nat)$;
    \item $\displaystyle\lim_{x \to 0}\displaystyle\frac{\sqrt[m]{1+\alpha x}-1}{x} \; (m \in \nat)$; 
    \item $\displaystyle\lim_{x \to 0}\displaystyle\frac{\sqrt[m]{1+\alpha x}-\sqrt[n]{1+\beta x}}{x} \; (m,n \in \nat)$;
    \item $\displaystyle\lim_{x \to 1}\displaystyle\frac{(1-\sqrt{x})(1-\sqrt[3]{x})\cdots(1-\sqrt[n]{x})}{(1-x)^{n-1}} \; (n \in \nat)$.
\end{enumerate}
\bigskip

(1) \solve 利用等比级数求和公式
\begin{equation}
    1+q^2+q^3+\cdots+q^{n-1}=\frac{q^n-1}{q-1}
\end{equation}
我们有
\begin{equation}
    \sqrt[m]{x}-1=\frac{x-1}{\displaystyle\sum_{i=0}^{m-1}x^{i/m}}
\end{equation}
同理有
\begin{equation}
    \sqrt[n]{x}-1=\frac{x-1}{\displaystyle\sum_{i=0}^{n-1}x^{i/n}}
\end{equation}
所以有
\begin{equation}
    \lim_{x \to 1}\frac{\sqrt[m]{x}-1}{\sqrt[n]{x}-1}=\lim_{x\to 1}\frac{\displaystyle\frac{x-1}{\displaystyle\sum_{i=0}^{m-1}x^{i/m}}}{\displaystyle\frac{x-1}{\displaystyle\sum_{i=0}^{n-1}x^{i/n}}} = \lim_{x \to 1}\frac{\displaystyle\sum_{i=0}^{n-1}x^{i/n}}{\displaystyle\sum_{i=0}^{m-1}x^{i/m}} = \frac{n}{m}.
\end{equation}
\qed\bigskip

(2) \solve 利用公式
\begin{equation}
    \sum_{i=0}^{m-1}(1+\alpha x)^{i/m} = \frac{(1+\alpha x)^{m/m}-1}{(1+\alpha x)^{1/m}-1}=\frac{\alpha x}{\sqrt[m]{1+\alpha x}-1}
\end{equation}
我们有
\begin{equation}
    \sqrt[m]{1+\alpha x} - 1 = \frac{\alpha x}{\displaystyle\sum_{i=0}^{m-1}(1+\alpha x)^{i/m}}
\end{equation}
从而得
\begin{equation}
    \text{原式} = \lim_{x \to 0} \frac{\displaystyle\frac{\alpha x}{\displaystyle\sum_{i=0}^{m-1}(1+\alpha x)^{i/m}}}{x} = \lim_{x \to 0} \frac{\alpha}{\displaystyle\sum_{i=0}^{m-1}(1+\alpha x)^{i/m}} = \frac{\alpha}{m} .
\end{equation}
\qed\bigskip

(3) \solve 利用第(2)题结论
\begin{equation}
    \text{原式} = \lim_{x \to 0} \frac{\sqrt[m]{1+\alpha x}-1}{x} - \lim_{x \to 0} \frac{\sqrt[n]{1+\beta x}-1}{x} = \frac{\alpha}{m} - \frac{\beta}{n}. 
\end{equation}
\qed\bigskip

(4) \solve 利用第(2)题结论,我们知
\begin{equation}
    \sqrt[m]{1+\alpha x} - 1 \sim \frac{\alpha}{m} x, \quad (x \to 0)
\end{equation}
因此
\begin{equation}
    1 - \sqrt[m]{1-\alpha x} \sim \frac{\alpha}{m} x, \quad (x \to 0)
\end{equation}
因此
\begin{equation}
    1 - \sqrt[m]{1-x} \sim \frac{1}{m}x, \quad (x \to 0)
\end{equation}
因此,令$t = 1-x$,那么
\begin{equation}
    \text{原式} = \lim_{t \to 0} \prod_{i=2}^{n} \frac{1-\sqrt[i]{1-t}}{t} = \prod_{i=2}^{n}\lim_{t \to 0} \frac{1-\sqrt[i]{1-t}}{t} = \prod_{i=2}^n \frac{1}{i} = \frac{1}{n!}.
\end{equation}
\qed\bigskip

3. 计算下列极限:
\begin{tasks}(2)
    \task $\displaystyle\lim_{x \to 0}\left(\displaystyle\frac{1+\tan \, x}{1+\sin \, x}\right)^{1/ \sin \, x}$;
    \task $\displaystyle\lim_{x \to 0} \left(\displaystyle\frac{\cos \, x}{\cos \, 2x}\right)^{1/x^2}$;
    \task $\displaystyle\lim_{x \to \pi /4} \left(\tan \, x\right)^{\tan \, 2x}$ 
    \task $\displaystyle\lim_{x \to \pi /2} \left(\sin \, x\right)^{\tan \, x}$;
    \task $\displaystyle\lim_{x \to +\infty} \left(\sin \, \displaystyle\frac{1}{x} + \cos \, \displaystyle\frac{1}{x}\right)^x$;
    \task $\displaystyle\lim_{x \to 0^+} \left(\cos \, \sqrt{x}\right)^{1/x}$;
    \task $\displaystyle\lim_{n \to \infty} \left(\cos \, \displaystyle\frac{x}{\sqrt{n}}\right)^n$;
    \task $\displaystyle\lim_{x \to 0} \left(2 \expe^{x/(1+x)}-1\right)^{(x^2+1)/x}$;
    \task $\displaystyle\lim_{x \to a}\left(\displaystyle\frac{\sin \, x}{\sin \, a}\right)^{1/(x-a)} \; (a \neq \pm k \pi, k \in \nat)$.
\end{tasks}
\bigskip

(1) \solve 我们令
\begin{equation}
    u(x) = \frac{1+\tan \, x}{1+\sin \, x} , \quad v(x) = \frac{1}{\sin \, x}
\end{equation}
于是
\begin{align}
    \text{原式} &= \lim_{x \to 0} \left(\left(1+(u(x)-1)\right)^{1/(u(x)-1)}\right)^{v(x)(u(x)-1)}
\end{align}
由于
\begin{align}
    \lim_{x \to 0} v(x)(u(x)-1) &= \lim_{x \to 0} \frac{\tan \, x - \sin \, x}{1 + \sin \, x} \cdot \frac{1}{\sin \, x} \\
    &= \lim_{x \to 0} \frac{\tan \, x - \sin \, x}{\sin \, x} \cdot \frac{1}{1+\sin \, x} \\
    &= \lim_{x \to 0} \frac{\tan \, x - \sin \, x}{\sin \, x} \cdot \lim_{x \to 0} \frac{1}{1 + \sin \, x} \\
    &= 0 \cdot 1 = 0
\end{align}
从而
\begin{equation}
    \text{原式} = \expe^{0} = 1.
\end{equation}
\qed\bigskip

(2) \solve 令
\begin{equation}
    u(x) = \frac{\cos \, x}{\cos \, 2x}, \quad v(x) = \frac{1}{x^2}
\end{equation}
于是
\begin{align}
    \text{原式} &= \lim_{x \to 0} u(x)^{v(x)} = \lim_{x \to 0} \left(\left(1 + (u(x)-1)\right)^{1/(u(x)-1)}\right)^{v(x)(u(x)-1)}
\end{align}
由于
\begin{align}
    \lim_{x \to 0}v(x)(u(x)-1) &= \lim_{x \to 0} \left(\frac{\cos \, x - \cos \, 2x}{\cos \, 2x}\right) \cdot \frac{1}{x^2} \\
    &= \lim_{x \to 0} \left(\frac{-2 \sin \, \displaystyle\frac{3x}{2} \sin \, \displaystyle \frac{-x}{2}}{\cos \, 2x}\right) \cdot \frac{1}{x^2} \\
    &= \lim_{x \to 0} \frac{2 \cdot \displaystyle\frac{3x}{2} \cdot \displaystyle\frac{x}{2}}{x^2(1-2x^2)} \\
    &= \lim_{x \to 0} \frac{3}{2} \cdot \frac{1}{1-2x^2} = \frac{3}{2}.
\end{align}
于是
\begin{equation}
    \text{原式} = \expe^{3/2}.
\end{equation}
\qed\bigskip

(3) \solve 令
\begin{align}
    u(x) = \tan \, x, \quad v(x) = \tan \, 2x
\end{align}
那么
\begin{equation}
    \lim_{x \to \pi /4} \left(\tan \, x - 1\right) \tan \, 2x = \lim_{x \to \pi /4} - \frac{2 \tan \, x}{1+\tan \, x} = -1
\end{equation}
那么
\begin{equation}
    \lim_{x \to \pi /4} \left(\tan \, x\right)^{\tan \, 2x} = \lim_{x \to \pi /4} \left(u(x)\right)^{v(x)} = \expe^{-1}.
\end{equation}
\qed\bigskip

(4) \solve 令
\begin{align}
    u(x) = \sin \, x , \quad v(x) = \tan \, x
\end{align}
则
\begin{align}
    \lim_{x \to \pi / 2} \left(u(x)-1\right)v(x) &= \lim_{x\to \pi /2} \left(\sin \, x - 1\right) \tan \, x \\
    &= \lim_{x \to \pi /2} \frac{\left(\sin \, x\right)^2 - \sin \, x}{\cos \, x} \\
    &= \lim_{x \to \pi /2} \frac{1 - \left(\cos \, x\right)^2 - \sqrt{1-\left(\cos \, x\right)^2}}{\cos \, x} \\
    &= \lim_{x \to \pi /2} \frac{1-\sqrt{1-\left(\cos \, x\right)^2}}{\cos \, x} - \lim_{x \to \pi /2} \cos \, x
\end{align}
\textbf{看好了},我令$t = \cos \, x$,那么当$x \to \pi /2$的时候,$t = \cos \, x$不是$\to 0$么?
\begin{align}
    \text{原式} &= \lim_{t \to 0} \frac{1-\sqrt{1-t^2}}{t} - \cos \, \lim_{x \to \pi/2} x \\
    &= \lim_{t \to 0}\frac{1/2 \, t^2}{t} - \cos \, \pi /2 \\
    &= \lim_{t \to 0} 1/2 \, t - 0 \\
    &= 1/2 \lim_{t \to 0} t  = 1/2 \cdot 0 = 0.
\end{align}
因此
\begin{equation}
    \text{原式} = \expe^0 = 1
\end{equation}
\qed\bigskip

(5) \solve 令
\begin{align}
    u(x) = \sin \, \frac{1}{x} + \cos \, \frac{1}{x} , \quad v(x) = x
\end{align}
则
\begin{align}
    \lim_{x \to +\infty} \left(u(x)-1\right)v(x) &= \lim_{x \to +\infty} x \sin \, \frac{1}{x} + x \cos \, \frac{1}{x} - x \\
    &= \lim_{x \to +\infty} \frac{\sin \, \displaystyle\frac{1}{x}}{\displaystyle\frac{1}{x}} + x \left(\cos \, \displaystyle\frac{1}{x} - 1\right) \\
    &= \lim_{x \to +\infty} \frac{\sin \, \displaystyle\frac{1}{x}}{\displaystyle\frac{1}{x}} + \frac{\cos \, \displaystyle\frac{1}{x} - 1}{\displaystyle\frac{1}{x}} \\
    &= \lim_{1/x \to 0} \frac{\sin \, \displaystyle\frac{1}{x}}{\displaystyle\frac{1}{x}} + \lim_{1/x \to 0} \frac{\cos \, \displaystyle\frac{1}{x}-1}{\displaystyle\frac{1}{x}} \\
    &= 1 + \lim_{1/x \to 0} \frac{- 1/2 \, \left(\displaystyle\frac{1}{x}\right)^2}{\displaystyle\frac{1}{x}} \\
    &= 1 + \lim_{1/x \to 0} - \frac{1}{2}\cdot\frac{1}{x} \\
    &= 1 - \frac{1}{2} \lim_{1/x \to 0} \frac{1}{x} = 1- \frac{1}{2} \cdot 0 = 1 - 0 = 1.
\end{align}
于是
\begin{equation}
    \text{原式} = \expe^1 = \expe.
\end{equation}
\qed\bigskip

(6) \solve 令
\begin{align}
    u(x) = \cos \, \sqrt{x}, \quad v(x) = \frac{1}{x}
\end{align}
则
\begin{align}
    \lim_{x \to 0^+} \left(u(x)-1\right)v(x) &= \lim_{x \to 0^+} \frac{\cos \, \sqrt{x} - 1}{x} \\
    &= \lim_{x \to 0^+} \frac{-1/2 \left(\sqrt{x}\right)^2}{x} = \lim_{x \to 0^+} - \frac{1}{2} \cdot \frac{x}{x} = \lim_{x \to 0^+} -\frac{1}{2} = - \frac{1}{2}.
\end{align}
则
\begin{equation}
    \text{原式} = \expe^{-1/2}.
\end{equation}
\qed\bigskip

(7) \solve 考虑函数
\begin{align}
    f: \real^+ &\longrightarrow  \real \\
    y &\longmapsto \left(\cos \frac{x}{\sqrt{y}} \right)^y
\end{align}
的极限
\begin{equation}
    \lim_{y \to +\infty} \left(\cos \, \frac{x}{\sqrt{y}}\right)^y
\end{equation}
这里$x$是任一给定的实数,令
\begin{equation}
    u(y) = \cos \, \frac{x}{\sqrt{y}}, \quad v(y) = y
\end{equation}
我们计算出
\begin{align}
    \lim_{y \to +\infty} \left(u(y) - 1\right) v(y) &= \lim_{y \to +\infty} \frac{\cos \, \displaystyle\frac{x}{\sqrt{y}} - 1}{\displaystyle\frac{1}{y}} \\
    &= \lim_{y \to +\infty} \frac{-1/2 \left(x/\sqrt{y}\right)^2}{1/y} = \lim_{y \to +\infty} -x^2/2 x^2 = -x^2/2
\end{align}
从而有
\begin{equation}
    \lim_{y \to +\infty} \left(\cos \, \frac{x}{\sqrt{y}}\right)^y = \expe^{-x^2/2}
\end{equation}
从而依归结原则
\begin{equation}
    \lim_{n \to \infty} \left(\cos \, \frac{x}{\sqrt{n}}\right)^n = \expe^{-x^2/2}.
\end{equation}
\qed\bigskip

(8) \solve 令
\begin{equation}
    u(x) = 2\expe^{x/(1+x)} - 1, \quad v(x) \frac{x^2+1}{x}
\end{equation}
那么
\begin{align}
    \lim_{x \to 0} \left(u(x) - 1\right) v(x) &= \lim_{x \to 0} 2\left(\expe^{x/(1+x)}-1\right) \frac{x^2+1}{x} \\
    &= 2\lim_{x \to 0} \frac{x}{1+x} \cdot \frac{x^2+1}{x} = 2 \lim_{x \to 0} \frac{x^2+1}{1+x} = 2 \cdot 1 = 2
\end{align}
于是
\begin{equation}
    \text{原式} = \expe^2.
\end{equation}
\qed\bigskip