
\exercise

1. 设$A = \{ a, b, c, d \}$.证明有唯一的映射:$f: A \to A$满足下列条件:
\begin{enumerate}
    \item $f(a) = b, f(c) = d$;
    \item $( f \circ f ) (x) = x$对一切$x \in A$成立.
\end{enumerate}

\begin{proof}
任取$A$到$A$的两个映射:$f: A \to A, \, g: A \to A$,设$f, g$都满足条件(1)、(2),我们去验证$f$与$g$相等:
\begin{align}
    &f(a) = b, \, g(a) = b \, \implies f(a) = g(a) \\
    &f(b) = f(f(a)) = (f \circ f)(a) = a, \, g(b) = g(g(a)) = (g \circ g)(a) = a \, \implies f(b) = g(b) \\
    &f(c) = d, \, g(c) = d \, \implies f(c) = g(c) \\
    &f(d) = f(f(c)) = (f \circ f)(c) = c, \, g(d) = g(g(c)) = (g \circ g)(c) = c \, \implies f(d) = g(d)
\end{align}
从而$\forall x \in A, f(x) = g(x)$,从而$f = g$,从而这样的映射是唯一的.
\end{proof}

2. 设映射$f$满足$(f \circ f)(a) = a$,求$f^n (a)$.
\begin{proof}
当$n=2$时
\begin{equation}
    f^2 (a) = (f \circ f)(a) = a
\end{equation}
由此可知当$n=2$时$f^2 (a) = a$成立,现假设对某个$k \in \nat$,当$n = 2(k-1)$时$f^n (a) = a$成立,于是
\begin{align}
f^{2k} (a) &= (f \circ f^{2k-1}) (a) = (f \circ (f \circ f^{2(k-1)}))(a) = ((f \circ f) \circ f^{2(k-1)})(a) = (f^2 \circ f^{2(k-1)})(a) \\
&= f^2 (f^{2(k-1)}(a)) = f^2 (a) = a
\end{align}
这说明对于$n = 2k$命题也成立,那么根据数学归纳法原理,对任意正偶数$n$,$f^n (a) = a$成立.
\end{proof}

解:对于$n=1$,有$f^1 (a) = f(a)$,对于任意正偶数$n$,有$f^n (a) = a$,对任意大于等于$3$的正奇数$n$,$n$可被表示为$n=2k+1, k \in \nat$,于是有
\begin{equation}
    f^{n}(a) = \begin{cases}
        a, & n\text{为正偶数}; \\
        f(a), & n\text{为正奇数}.
    \end{cases}
\end{equation}

3. 定义映射$D: \real \to \{ 0, 1 \}$如下:
\begin{equation}
    D(x) = \begin{cases}
        1, & \text{当}x\text{为有理数时}, \\
        0, & \text{当}x\text{为无理数时}.
    \end{cases}
\end{equation}
\begin{enumerate}
    \item 求复合映射$D \circ D$;
    \item 求$D^{-1}(\{0\}), D^{-1}(\{1\}), D^{-1}(\{0,1\})$.
\end{enumerate}

(1) 解:任取$x \in \real$,如果$x$为有理数,那么
\begin{equation}
    D(x) = 1, \quad (x\text{为有理数})
\end{equation}
那么由于$1$是有理数,所以
\begin{equation}
    D(D(x)) = D(1) = 1, \quad (x\text{为有理数})
\end{equation}
如果$x$为无理数,那么
\begin{equation}
    D(x) = 0, \quad (x\text{为无理数})
\end{equation}
那么,由于$0$为有理数,所以
\begin{equation}
    D(D(x)) = D(0) = 1, \quad (x\text{为无理数})
\end{equation}
所以无论$x$是有理数还是无理数,都有
\begin{equation}
    D(D(x)) = (D \circ D)(x) = 1, \quad \forall x \in \real
\end{equation}
所以$D \circ D$可写为$D \circ D: \real \to \{ 1 \}$.

(2) 解:$\real = \mathbb{Q} \bigcup (\real \setminus \mathbb{Q})$并且$\rational \bigcap (\real \setminus \rational) = \emptyset$,$\forall x \in \rational, D(x) = 1$,$\forall x \in \real \setminus \rational, D(x) = 0$,所以$D^{-1}(\{0\}) = \irrational$.同理,$D^{-1}(\{1\}) = \mathbb{Q}$.$D^{-1}(\{0, 1\}) = \real$.

4. 设$A$是由$n$个元素组成的集合.若映射$f: A \to A$是单射,则称$f$是$A$的一个排列.
\begin{enumerate}
    \item 证明:$f(A) = A$. 
    \item 证明:$f^{-1}$存在.
    \item $A$共有多少个排列?
\end{enumerate}

(1)
\begin{proof}
由于$f: A \to A$是集合$A$到自身的映射,所以$f(A) \subset A$.

对任意$x \in A$都应有$x \in f(A)$,否则,假设存在$x_0 \in A$但是$x_0 \not\in f(A)$,那么$| f(A) | < |A| = n$,但是由于$f$是单射并且$|A| = n$,所以$|f(A)| \geq n$,矛盾.所以$A \subset f(A)$.因为$f(A) \subset A$以及$A \subset f(A)$,所以$f(A) = A$.
\end{proof}

(2)
\begin{proof}
任取$x_0 \in A$,让$x_{1} = f(x_0), x_{m+1} = f(x_m)$,让$m$取遍全体正整数,我们得到一个数列$\{ x_m \}$,它是
\begin{equation}
    \{ x_m \} = \{ f(x_0), f(f(x_0)), f(f(f(x_0))), \cdots \}
\end{equation}
采用反证法,假设对任意$m \in \nat$,都有$x_m \neq x_0$.那么任取$\{ x_m \}$中的两项$x_i, x_j, i \neq j$,设$ i < j$,事实上,$x_i, x_j$可写成$x_i = f^{i}(x_0), x_j = f^j (x_0)$,如果$x_i = x_j$,那么就有$f^{i}(x_0) = f^{j}(x_0)$,由于$f$是单射,所以就有$f^{i-1}(x_0) = f^{j-1}(x_0)$,反复应用$f$的单射性质,最终会得到$f^{0}(x_0) = f^{j-i}(x_0)$也就是$x_0 = f^{j-i}(x_0)$,但根据假设不可能有$x_0 = f^{j-i}(x_0)$成立,所以也不可能有$x_i = x_j$成立,所以$\{ x_m \}$中任意两项都是两两不等的,从而$\{ x_m \}$有无穷多项,但是,由于$x_{m+1} = f(x_m), x_1 = f(x_0), \forall m \in \nat$,可知$x_m \in f(A) = A, \forall m \in \nat$,也就是说$|A|$有无穷多个元素,矛盾.

这就说明,对每一个$x \in A$,都存在$m \in \nat$,使得$f^m (x_0) = x_0$,这就证明了$f^{-1}$是存在的.
\end{proof}

点评:从直觉收到启发,我们发现在$\{ x_n \}$的前$n$项中,应该必然有一项与$x_0$相等,否则经过无数次将$f$反复应用于$x_0$将会``创造''出无穷多个不同的元素,这与$A$是有限集矛盾.

或者我们还可以这样理解:想象$A$中的每一个元素是荷塘上的一片荷叶,想象这样一只青蛙,它最开始在编号为$x_0$的荷叶上,青蛙按照$f$的指示从编号为$x_0$的荷叶跳到编号为$f(x_0)$的荷叶,又从编号为$f(x_0)$的荷叶跳到编号为$f(f(x_0))$的荷叶,也就是说第$t$次跳跃就是从编号为$f^{t-1}(x_0)$的荷叶跳到编号为$f^{t}(x_0)$的荷叶,从直觉上理解,我们一定有信心认为,青蛙在经过有限次这样的跳跃后必定会回到最初的编号为$x_0$的荷叶上,否则就说明荷塘中有无穷多片荷叶,这是荒谬的.

在群论中,设$S_n$是$n$元置换群,容易证明它是循环群,因此它有生成元,$f$与自身的复合可以看做是$S_n$中任意一个元素自己与自己做加法运算,譬如说,任取$a_0 \in S_n$,我们可以这样定义$f$:
\begin{align}
    f: \{1,2,3,\cdots,n\} &\to \{ 1,2,3,\cdots, n\} \\
    x &\mapsto a_0 x
\end{align}
于是$f \circ f$等价于$a \oplus a$,其中$\oplus$是群$S_n$的二元代数运算.这样就将代数结构与排列映射联系了起来.

(3) 确定了$f$所描述的具体的对应法则,也就知道了$A$有多少个排列.将$A$中的$n$个元素分别记为$x_1, x_2, \cdots, x_n$,为了构造出一种可能的$f$,我们要依次确定$x_1, x_2, \cdots, x_m$的映象,任取其中一个,记做$x_{i_1}$,那么有$n$个元素可作为$x_{i_1}$的映象,从剩下的$n-1$个没有确定映象的元素中再任取一个,记为$x_{i_2}$,有$n-1$个元素可作为$x_{i_2}$的映象,类似地,有$n-2$个元素可作为$x_{i_3}$的映项,这样进行下去,有$2$个元素可作为$x_{i_{n-1}}$的映象,有$1$个元素可作为$x_{i_n}$的映象,按照计数原理,确定$x_{i_1}, x_{i_2}, \cdots, x_{i_n}$这$n$个元素的映象的方式总共有$n \times (n-1) \times \cdots 2 \times 1 = n!$种.因此所有可能的$f$只有$n!$种.因此$A$有$n!$个排列.

5. 设$A$是由$n$个元素组成的集合.若映射$f$满足$f(a) = a$对一切$a \in G$成立,则称$f$为$A$的恒等排列.求证:当$n \geq 2$时,存在非恒等排列$f$,使得$f \circ f$为恒等排列.

\begin{proof}
当$n=2$时,设
\begin{equation}
    A = \{ x_1 ,x_2 \}
\end{equation}
其中$x_1, x_2$是集合$A$的全部元素并且$x_1 \neq x_2$.我们令:
\begin{equation}
    f(x) = \begin{cases}
        x_2, & x = x_1; \\
        x_1, & x = x_2. 
    \end{cases}
\end{equation}
那么$f(x_1) = x_2$,显然它是非恒等映射,并且$f(f(x_1)) = f(x_2) = x_1$,以及$f(f(x_2)) = f(x_1) = x_2$,所以$f \circ f$是恒等映射,也就是说当$n=2$时命题成立.当$n \geq 3$时,设
\begin{equation}
    A = \{ x_1, x_2, \cdots , x_n \}
\end{equation}
其中$x_1, x_2, \cdots, x_n$是$A$的全部元素并且它们两两不相等.令
\begin{equation}
    f(x) = \begin{cases}
        x_2, & x = x_1; \\
        x_1, & x = x_2; \\
        x, & \text{otherwise.}
    \end{cases}
\end{equation}
因为$f(x_1) = x_2$,所以$f$是非恒等映射,但是对于$x = x_1$或者$x = x_2$,可验证$f \circ f$是恒等映射,对于$x \neq x_1$并且$x \neq x_2$,由于$f(x) = x$,所以$f \circ f(x) = x$,所以$f \circ f$对于任意$x \in A$都是恒等映射.于是对每一个$n \in \nat, n \geq 2$都可构造出这样的非恒等映射$f$使得$f \circ f$为恒等映射.命题得证.
\end{proof}