\exercise

1. 如果$f$在$(a,b)$上一致连续,证明:$f$在$(a,b)$上有界.

\prove 采用反证法.假设结论不成立.不失一般性,假设$f$在$(a,b)$上没有上界.由此我们可以找到一个数列$\{ x_n \} \in (a,b)$并且$f(x_n) \to +\infty, \; (n \to \infty)$,由于$\{ x_n \}$是有界的,所以我们可以找到一个$\{ x_n \}$的收敛子列,不妨记为$\{ x_{k_n}\}$,由于$f(x_n) \to +\infty$,又因为$\{ x_{k_n} \}$是$\{ x_n \}$的子列所以
\begin{equation}
\lim_{n \to \infty} f(x_{k_n} ) = +\infty
\label{eq:fxkndiverge}
\end{equation}
由于$f$在$(a,b)$上是一致收敛的,所以,对任意$\epsilon > 0$,存在$\delta > 0$,使得对任意$x_1, x_2 \in (a,b)$,只有$\lvert x_1 - x_2 \rvert < \delta$,就有
\begin{equation}
    \lvert f(x_1) - f(x_2) \rvert < \epsilon
\end{equation}
由于$\{ x_{k_n} \}$收敛,故它是一个柯西列,故存在足够大的正整数$M$,使得对一切$n_1, n_2 > M, n_1, n_2 \in \nat$,都有
\begin{equation}
    \lvert x_{k_{n_1}} - x_{k_{n_2}} \rvert < \delta
\end{equation}
故
\begin{equation}
    \lvert f(x_{k_{n_1}}) - f(x_{k_{n_2}}) \rvert < \epsilon
\end{equation}
故数列$\{ f(x_{k_n}) \}$为一柯西列因而是收敛的,可是这与式(\ref{eq:fxkndiverge})矛盾.\qed\bigskip

2. 若函数$f$和$g$都在区间$I$上一致连续,问$fg$是否在$I$上一致连续?试就$I$为有限区间或无限区间分别讨论之.

\textbf{情形1:}$I$为有限区间.我们将证明$fg$在有限区间内是一致连续的.

\prove 对任意$x_1 , x_2 \in (a,b)$,有
\begin{align}
    \lvert f(x_1)g(x_1)-f(x_2)g(x_2)\rvert &= \lvert f(x_1)g(x_1) - f(x_2)g(x_1) + f(x_2)g(x_1) - f(x_2)g(x_2)\rvert \\
    &\leq \lvert f(x_1)g(x_1) - f(x_2)g(x_1) \rvert + \lvert f(x_2)g(x_1) - f(x_2)g(x_2) \rvert \\
    &= \lvert g(x_1) \rvert \lvert f(x_1) - f(x_2) \rvert + \lvert f(x_2) \rvert \lvert g(x_1) - g(x_2) \rvert
\end{align}
由于$f, g$在$(a,b)$上是有界的,不妨设$\lvert f(x) \rvert, \lvert g(x) \rvert \leq M, 0< M < +\infty, \forall x \in (a,b)$,由此我们得
\begin{equation}
    \lvert f(x_1)g(x_1) - f(x_2)g(x_2) \rvert \leq M \left(\lvert f(x_1) - f(x_2) \rvert + \lvert g(x_1)-g(x_2) \rvert\right)
\end{equation}
又因为$f, g$在$(a,b)$上一致连续,故对任意$\epsilon > 0$,存在$\delta_1 > 0$,使得对任意$x_1 , x_2 \in (a,b)$,只要$0 < \lvert x_1 - x_2 \rvert < \delta$,就有
\begin{equation}
    \lvert f(x_1) - f(x_2) \rvert < \frac{\epsilon}{M}
\end{equation}
同理,存在$\delta_2 > 0$,使得对任意$x_1, x_2 \in (a,b)$,只要$0 < \lvert x_1 - x_2 \rvert < \delta$,就有
\begin{equation}
    \lvert g(x_1) - g(x_2) \rvert < \frac{\epsilon}{M}
\end{equation}
因此,对任意$\epsilon > 0$,只需取$\delta = \min \{ \delta_1, \delta_2 \}$,以上两不等式就同时成立,并且此时有
\begin{equation}
    \lvert f(x_1)g(x_1) - f(x_2)g(x_2) \rvert \leq M \left(\lvert f(x_1)-f(x_2) \rvert + \lvert g(x_1)-g(x_2)\rvert \right) < \epsilon
\end{equation}
从而这就证明了$fg$在$(a,b)$上是一致连续的.\qed\bigskip

\textbf{情形2:}$I$为无限区间.我们可以举出反例,令
\begin{align}
    f: \real &\longrightarrow \real \\
    x &\longmapsto x
\end{align}
再令$g = f$.显然$f, g$在$\real$上是一致连续的,可是$fg: x \mapsto x^2$在$\real$上却不是一致连续的,这是因为,取
\begin{equation}
    s_n = n, t_n = n + \frac{1}{2n}
\end{equation}
则$\lvert s_n - t_n \rvert < 1/n, \forall n \in \nat$,与此同时
\begin{equation}
    \lvert f(s_n)g(s_n) - f(t_n)g(t_n) \rvert = \lvert n^2  - (n + \frac{1}{2n})^2 \rvert = 1 + \frac{1}{4n^2} > 1
\end{equation}
从而函数$fg$在$\real$上未必是一致连续的.\qed\bigskip

3. 设函数$f$在开区间$(a,b)$上连续,$f(a+)$和$f(b+)$存在且有限.证明:$f$在$(a,b)$上一致连续.

\prove 采用反证法.假设结论不成立.那么存在$\epsilon_0 > 0$,存在两个数列$\{ s_n \}$和$\{ t_n \}$,使得虽有$ \lvert s_n - t_n \rvert < 1/n, \forall n \in \nat$,但是
\begin{equation}
    \lvert f(s_n) - f(t_n) \rvert \geq \epsilon_0, \forall n \in \nat
\end{equation}
由于数列$\{ s_n \}$是有界的,所以可以从中找出一个收敛子列记做$\{ s_{k_n} \}$,不妨设$ s_{k_n} \to \xi$,下面我们来证明$t_{k_n} \to \xi$,注意到:
\begin{align}
    \lvert t_{k_n} - \xi \rvert &= \lvert t_{k_n} - s_{k_n} + s_{k_n} - \xi \rvert \\
    &\leq \lvert t_{k_n} - s_{k_n} \rvert + \lvert s_{k_n} - \xi \rvert \\
    &< \frac{1}{k_n} + \lvert s_{k_n} - \xi \rvert \label{eq:2-10-3-1}
\end{align}
由此即证$t_{k_n} \to \xi \; (n \to \infty)$.分两种情况进行讨论.

\textbf{情形1:} $\xi \in (a, b)$,也就是说$\xi \neq a, \xi \neq b$.由于$\{ s_{k_n} \}, \{t_{k_n} \}$是子列,故同样有
\begin{equation}
    \lvert f(s_{k_n}) - f(t_{k_n}) \rvert \geq \epsilon_0 > 0, \forall n \in \nat
    \label{eq:2-10-3-2}
\end{equation}
令$n \to \infty$,由于$s_{k_n}, t_{k_n} \to \xi$,依$f$在$(a,b)$的连续性,我们有
\begin{equation}
    \lim_{n \to \infty} \lvert f(s_{k_n}) - f(t_{k_n}) \rvert = \lvert f(\xi) - f(\xi) \rvert = 0
\end{equation}
可是依式(\ref{eq:2-10-3-2}),我们有
\begin{equation}
    \lim_{n \to \infty} \lvert f(s_{k_n}) - t_{k_n} \rvert \geq \lim_{n \to \infty} \epsilon_0 = \epsilon_0 > 0
\end{equation}
这就出现了矛盾.

\textbf{情形2:} $\xi = a$或者$\xi = b$.不失一般性假设$\xi = a$,由于$f(a+)$是存在的,故
\begin{equation}
    \lim_{n \to \infty} \left( f(s_{k_n}) - f(t_{k_n}) \right) = \lim_{n \to \infty} f(s_{k_n}) - \lim_{n \to \infty} f(t_{k_n}) = f(a+) - f(a+) = 0
\end{equation}
从而
\begin{equation}
    \lim_{n \to \infty} \lvert f(s_{k_n}) - f(t_{k_n}) \rvert = 0
\end{equation}
可是依式(\ref{eq:2-10-3-2}),我们有
\begin{equation}
    \lim_{n \to \infty} \lvert f(s_{k_n}) - f(t_{k_n}) \rvert \geq \lim_{n \to \infty} \epsilon_0 = \epsilon_0 > 0
\end{equation}
这就出现了矛盾.\qed\bigskip

4. 设函数$f$在区间$[a, +\infty)$上连续,$f(+\infty)$存在且有限.证明:$f$在$[a, +\infty)$上一致连续.

\prove 采用反证法,设存在数列$\{ s_n \}, \{ t_n \}$,满足$\lvert s_n - t_n \rvert < 1/n, \forall n \in \nat$,并且存在$\epsilon_0 > 0$,满足
\begin{equation}
    \lvert f(s_n) - f(t_n) \rvert \geq \epsilon_0, \quad \forall n \in \nat
\end{equation}
分两种情形考虑:

\textbf{情形1:} 数列$\{ s_n \}$是收敛的.那么我们可以找到$\{ s_n \}$的一个收敛子列$\{ s_{k_n} \}$,设$\displaystyle\lim_{n \to \infty} s_{k_n} = \xi$,易证$\displaystyle\lim_{n \to \infty} t_{k_n} = \xi$,则依函数$f$在$[a,+\infty)$上的连续性
\begin{equation}
    \lim_{n \to \infty} \lvert f(s_{k_n}) - f(t_{k_n}) \rvert = \lvert f(\xi) - f(\xi) \rvert = 0
\end{equation}
可是由于$\{s_{k_n}\}, \{t_{k_n}\}$是子列,故
\begin{equation}
    \lvert f(s_{k_n}) - f(t_{k_n}) \rvert \geq \epsilon_0 > 0
\end{equation}
从而
\begin{equation}
    \lim_{n \to \infty} \lvert f(s_{k_n}) - f(t_{k_n}) \rvert > 0
\end{equation}
矛盾.

\textbf{情形2:} 数列$\{ s_n \}$是发散的,从而数列$\{ t_n \}$也是发散的.由于$\displaystyle\lim_{x \to +\infty} f(x)$存在,故依柯西收敛准则,对任意$\epsilon > 0$,存在$M > 0$,使得对任意$x_1, x_2 \in [a, +\infty)$,只要$x_1, x_2 > M$,就有
\begin{equation}
    \lvert f(x_1) - f(x_2) \rvert < \epsilon
\end{equation}
从数列$\{ s_n \}$中可找到一个子列$\{ s_{k_n}\}$满足$s_{k_n} \to +\infty$,又由于$\lvert s_{k_n} - t_{k_n} \rvert < 1/k_n$,故同样有$t_{k_n} \to +\infty$,故给定$M > 0$,存在足够大的正整数$N \in \nat$,使得对一切$n > N, n \in \nat$,都有
\begin{equation}
    s_{k_n}, t_{k_n} > M
\end{equation}
从而
\begin{equation}
    \lvert f(s_{k_n}) - f(t_{k_n}) \rvert < \epsilon
\end{equation}
可是这与
\begin{equation}
    \lvert f(s_{k_n}) - f(t_{k_n}) \rvert \geq \epsilon_0, \forall n \in \nat
\end{equation}
矛盾.\qed\bigskip

5. 设$f$在区间$[a, +\infty)$上连续.若存在常数$b$和$c$,使得$\displaystyle\lim_{x \to +\infty} \left(f(x) - bx - c\right) = 0$,证明:$f$在$[a, +\infty)$上一致连续.

\prove 由$\displaystyle\lim_{x \to +\infty} (f(x) - bx - c) = 0$可推出$\displaystyle\lim_{x \to +\infty} (f(x) - bx) = c$,令$g(x) = f(x) - bx$,则依题4的结论,函数$g$在$[a, +\infty)$上一致连续.又由于函数$x \mapsto bx$在$[a, +\infty)$上一致连续,故函数$x \mapsto bx$与$g$的和函数$f(x)$在$[a, +\infty)$上一致连续.\qed\bigskip

6. 求证:三次方程$x^3 + 2x - 1 = 0$只有唯一的实根,且次根在$(0,1)$内.

\prove 

存在性.

令$f(x) = x^3 + 2x - 1$,分别将$0, 1$代入算得
\begin{align}
    f(0) &= 0^3 + 2\times 0 - 1 = -1 \\
    f(1) &= 1^3 + 2\times 1 - 1 = 2
\end{align}
由此可见$f(0)f(1) < 0$,又由于函数$f$在$[0,1]$连续,所以依零点定理,存在$c \in (0,1)$,使得$f(c) = 0$,由此存在性即证.

唯一性.

任取$d \in \real \; (d > 0)$,则对任意$x \in \real$,有
\begin{equation}
    f(x+d) - f(x) = (x+d)^3 - x^3 + 2d
\end{equation}
由于函数$x \mapsto x^3$在$\real$上是严格单调递增的,故$(x+d)^3-x^3>0$,故$f(x+d)-f(x)>0$.从而函数$f$在$\real$上是严格单调递增的.于是对任意$x > c$,都有$f(x) > 0$,对任意$x < c$,都有$f(x) < 0$,由此可见,只存在唯一的$c \in \real$,使得$f(c) = 0$.从而唯一性得证.\qed\bigskip
