\exercise

1. 如果$f$在$(a,b)$上一致连续,证明:$f$在$(a,b)$上有界.

\prove 采用反证法.假设结论不成立.不失一般性,假设$f$在$(a,b)$上没有上界.由此我们可以找到一个数列$\{ x_n \} \in (a,b)$并且$f(x_n) \to +\infty, \; (n \to \infty)$,由于$\{ x_n \}$是有界的,所以我们可以找到一个$\{ x_n \}$的收敛子列,不妨记为$\{ x_{k_n}\}$,由于$f(x_n) \to +\infty$,又因为$\{ x_{k_n} \}$是$\{ x_n \}$的子列所以
\begin{equation}
\lim_{n \to \infty} f(x_{k_n} ) = +\infty
\label{eq:fxkndiverge}
\end{equation}
由于$f$在$(a,b)$上是一致收敛的,所以,对任意$\epsilon > 0$,存在$\delta > 0$,使得对任意$x_1, x_2 \in (a,b)$,只有$\lvert x_1 - x_2 \rvert < \delta$,就有
\begin{equation}
    \lvert f(x_1) - f(x_2) \rvert < \epsilon
\end{equation}
由于$\{ x_{k_n} \}$收敛,故它是一个柯西列,故存在足够大的正整数$M$,使得对一切$n_1, n_2 > M, n_1, n_2 \in \nat$,都有
\begin{equation}
    \lvert x_{k_{n_1}} - x_{k_{n_2}} \rvert < \delta
\end{equation}
故
\begin{equation}
    \lvert f(x_{k_{n_1}}) - f(x_{k_{n_2}}) \rvert < \epsilon
\end{equation}
故数列$\{ f(x_{k_n}) \}$为一柯西列因而是收敛的,可是这与式(\ref{eq:fxkndiverge})矛盾.\qed\bigskip

2. 若函数$f$和$g$都在区间$I$上一致连续,问$fg$是否在$I$上一致连续?试就$I$为有限区间或无限区间分别讨论之.

\textbf{情形1:}$I$为有限区间.我们将证明$fg$在有限区间内是一致连续的.

\prove 对任意$x_1 , x_2 \in (a,b)$,有
\begin{align}
    \lvert f(x_1)g(x_1)-f(x_2)g(x_2)\rvert &= \lvert f(x_1)g(x_1) - f(x_2)g(x_1) + f(x_2)g(x_1) - f(x_2)g(x_2)\rvert \\
    &\leq \lvert f(x_1)g(x_1) - f(x_2)g(x_1) \rvert + \lvert f(x_2)g(x_1) - f(x_2)g(x_2) \rvert \\
    &= \lvert g(x_1) \rvert \lvert f(x_1) - f(x_2) \rvert + \lvert f(x_2) \rvert \lvert g(x_1) - g(x_2) \rvert
\end{align}
由于$f, g$在$(a,b)$上是有界的,不妨设$\lvert f(x) \rvert, \lvert g(x) \rvert \leq M, 0< M < +\infty, \forall x \in (a,b)$,由此我们得
\begin{equation}
    \lvert f(x_1)g(x_1) - f(x_2)g(x_2) \rvert \leq M \left(\lvert f(x_1) - f(x_2) \rvert + \lvert g(x_1)-g(x_2) \rvert\right)
\end{equation}
又因为$f, g$在$(a,b)$上一致连续,故对任意$\epsilon > 0$,存在$\delta_1 > 0$,使得对任意$x_1 , x_2 \in (a,b)$,只要$0 < \lvert x_1 - x_2 \rvert < \delta$,就有
\begin{equation}
    \lvert f(x_1) - f(x_2) \rvert < \frac{\epsilon}{M}
\end{equation}
同理,存在$\delta_2 > 0$,使得对任意$x_1, x_2 \in (a,b)$,只要$0 < \lvert x_1 - x_2 \rvert < \delta$,就有
\begin{equation}
    \lvert g(x_1) - g(x_2) \rvert < \frac{\epsilon}{M}
\end{equation}
因此,对任意$\epsilon > 0$,只需取$\delta = \min \{ \delta_1, \delta_2 \}$,以上两不等式就同时成立,并且此时有
\begin{equation}
    \lvert f(x_1)g(x_1) - f(x_2)g(x_2) \rvert \leq M \left(\lvert f(x_1)-f(x_2) \rvert + \lvert g(x_1)-g(x_2)\rvert \right) < \epsilon
\end{equation}
从而这就证明了$fg$在$(a,b)$上是一致连续的.\qed\bigskip

\textbf{情形2:}$I$为无限区间.我们可以举出反例,令
\begin{align}
    f: \real &\longrightarrow \real \\
    x &\longmapsto x
\end{align}
再令$g = f$.显然$f, g$在$\real$上是一致连续的,可是$fg: x \mapsto x^2$在$\real$上却不是一致连续的,这是因为,取
\begin{equation}
    s_n = n, t_n = n + \frac{1}{2n}
\end{equation}
则$\lvert s_n - t_n \rvert < 1/n, \forall n \in \nat$,与此同时
\begin{equation}
    \lvert f(s_n)g(s_n) - f(t_n)g(t_n) \rvert = \lvert n^2  - (n + \frac{1}{2n})^2 \rvert = 1 + \frac{1}{4n^2} > 1
\end{equation}
从而函数$fg$在$\real$上未必是一致连续的.\qed\bigskip

3. 设函数$f$在开区间$(a,b)$上连续,$f(a+)$和$f(b+)$存在且有限.证明:$f$在$(a,b)$上一致连续.

\prove 采用反证法.假设结论不成立.那么存在$\epsilon_0 > 0$,存在两个数列$\{ s_n \}$和$\{ t_n \}$,使得虽有$ \lvert s_n - t_n \rvert < 1/n, \forall n \in \nat$,但是
\begin{equation}
    \lvert f(s_n) - f(t_n) \rvert \geq \epsilon_0, \forall n \in \nat
\end{equation}
由于数列$\{ s_n \}$是有界的,所以可以从中找出一个收敛子列记做$\{ s_{k_n} \}$,不妨设$ s_{k_n} \to \xi$,下面我们来证明$t_{k_n} \to \xi$,注意到:
\begin{align}
    \lvert t_{k_n} - \xi \rvert &= \lvert t_{k_n} - s_{k_n} + s_{k_n} - \xi \rvert \\
    &\leq \lvert t_{k_n} - s_{k_n} \rvert + \lvert s_{k_n} - \xi \rvert \\
    &< \frac{1}{k_n} + \lvert s_{k_n} - \xi \rvert \label{eq:2-10-3-1}
\end{align}
由此即证$t_{k_n} \to \xi \; (n \to \infty)$.分两种情况进行讨论.

\textbf{情形1:} $\xi \in (a, b)$,也就是说$\xi \neq a, \xi \neq b$.由于$\{ s_{k_n} \}, \{t_{k_n} \}$是子列,故同样有
\begin{equation}
    \lvert f(s_{k_n}) - f(t_{k_n}) \rvert \geq \epsilon_0 > 0, \forall n \in \nat
    \label{eq:2-10-3-2}
\end{equation}
令$n \to \infty$,由于$s_{k_n}, t_{k_n} \to \xi$,依$f$在$(a,b)$的连续性,我们有
\begin{equation}
    \lim_{n \to \infty} \lvert f(s_{k_n}) - f(t_{k_n}) \rvert = \lvert f(\xi) - f(\xi) \rvert = 0
\end{equation}
可是依式(\ref{eq:2-10-3-2}),我们有
\begin{equation}
    \lim_{n \to \infty} \lvert f(s_{k_n}) - t_{k_n} \rvert \geq \lim_{n \to \infty} \epsilon_0 = \epsilon_0 > 0
\end{equation}
这就出现了矛盾.

\textbf{情形2:} $\xi = a$或者$\xi = b$.不失一般性假设$\xi = a$,由于$f(a+)$是存在的,故
\begin{equation}
    \lim_{n \to \infty} \left( f(s_{k_n}) - f(t_{k_n}) \right) = \lim_{n \to \infty} f(s_{k_n}) - \lim_{n \to \infty} f(t_{k_n}) = f(a+) - f(a+) = 0
\end{equation}
从而
\begin{equation}
    \lim_{n \to \infty} \lvert f(s_{k_n}) - f(t_{k_n}) \rvert = 0
\end{equation}
可是依式(\ref{eq:2-10-3-2}),我们有
\begin{equation}
    \lim_{n \to \infty} \lvert f(s_{k_n}) - f(t_{k_n}) \rvert \geq \lim_{n \to \infty} \epsilon_0 = \epsilon_0 > 0
\end{equation}
这就出现了矛盾.\qed\bigskip

4. 设函数$f$在区间$[a, +\infty)$上连续,$f(+\infty)$存在且有限.证明:$f$在$[a, +\infty)$上一致连续.

\prove 采用反证法,设存在数列$\{ s_n \}, \{ t_n \}$,满足$\lvert s_n - t_n \rvert < 1/n, \forall n \in \nat$,并且存在$\epsilon_0 > 0$,满足
\begin{equation}
    \lvert f(s_n) - f(t_n) \rvert \geq \epsilon_0, \quad \forall n \in \nat
\end{equation}
分两种情形考虑:

\textbf{情形1:} 数列$\{ s_n \}$是收敛的.那么我们可以找到$\{ s_n \}$的一个收敛子列$\{ s_{k_n} \}$,设$\displaystyle\lim_{n \to \infty} s_{k_n} = \xi$,易证$\displaystyle\lim_{n \to \infty} t_{k_n} = \xi$,则依函数$f$在$[a,+\infty)$上的连续性
\begin{equation}
    \lim_{n \to \infty} \lvert f(s_{k_n}) - f(t_{k_n}) \rvert = \lvert f(\xi) - f(\xi) \rvert = 0
\end{equation}
可是由于$\{s_{k_n}\}, \{t_{k_n}\}$是子列,故
\begin{equation}
    \lvert f(s_{k_n}) - f(t_{k_n}) \rvert \geq \epsilon_0 > 0
\end{equation}
从而
\begin{equation}
    \lim_{n \to \infty} \lvert f(s_{k_n}) - f(t_{k_n}) \rvert > 0
\end{equation}
矛盾.

\textbf{情形2:} 数列$\{ s_n \}$是发散的,从而数列$\{ t_n \}$也是发散的.由于$\displaystyle\lim_{x \to +\infty} f(x)$存在,故依柯西收敛准则,对任意$\epsilon > 0$,存在$M > 0$,使得对任意$x_1, x_2 \in [a, +\infty)$,只要$x_1, x_2 > M$,就有
\begin{equation}
    \lvert f(x_1) - f(x_2) \rvert < \epsilon
\end{equation}
从数列$\{ s_n \}$中可找到一个子列$\{ s_{k_n}\}$满足$s_{k_n} \to +\infty$,又由于$\lvert s_{k_n} - t_{k_n} \rvert < 1/k_n$,故同样有$t_{k_n} \to +\infty$,故给定$M > 0$,存在足够大的正整数$N \in \nat$,使得对一切$n > N, n \in \nat$,都有
\begin{equation}
    s_{k_n}, t_{k_n} > M
\end{equation}
从而
\begin{equation}
    \lvert f(s_{k_n}) - f(t_{k_n}) \rvert < \epsilon
\end{equation}
可是这与
\begin{equation}
    \lvert f(s_{k_n}) - f(t_{k_n}) \rvert \geq \epsilon_0, \forall n \in \nat
\end{equation}
矛盾.\qed\bigskip

5. 设$f$在区间$[a, +\infty)$上连续.若存在常数$b$和$c$,使得$\displaystyle\lim_{x \to +\infty} \left(f(x) - bx - c\right) = 0$,证明:$f$在$[a, +\infty)$上一致连续.

\prove 由$\displaystyle\lim_{x \to +\infty} (f(x) - bx - c) = 0$可推出$\displaystyle\lim_{x \to +\infty} (f(x) - bx) = c$,令$g(x) = f(x) - bx$,则依题4的结论,函数$g$在$[a, +\infty)$上一致连续.又由于函数$x \mapsto bx$在$[a, +\infty)$上一致连续,故函数$x \mapsto bx$与$g$的和函数$f(x)$在$[a, +\infty)$上一致连续.\qed\bigskip

6. 求证:三次方程$x^3 + 2x - 1 = 0$只有唯一的实根,且次根在$(0,1)$内.

\prove 

存在性.

令$f(x) = x^3 + 2x - 1$,分别将$0, 1$代入算得
\begin{align}
    f(0) &= 0^3 + 2\times 0 - 1 = -1 \\
    f(1) &= 1^3 + 2\times 1 - 1 = 2
\end{align}
由此可见$f(0)f(1) < 0$,又由于函数$f$在$[0,1]$连续,所以依零点定理,存在$c \in (0,1)$,使得$f(c) = 0$,由此存在性即证.

唯一性.

任取$d \in \real \; (d > 0)$,则对任意$x \in \real$,有
\begin{equation}
    f(x+d) - f(x) = (x+d)^3 - x^3 + 2d
\end{equation}
由于函数$x \mapsto x^3$在$\real$上是严格单调递增的,故$(x+d)^3-x^3>0$,故$f(x+d)-f(x)>0$.从而函数$f$在$\real$上是严格单调递增的.于是对任意$x > c$,都有$f(x) > 0$,对任意$x < c$,都有$f(x) < 0$,由此可见,只存在唯一的$c \in \real$,使得$f(c) = 0$.从而唯一性得证.\qed\bigskip

7. 设$\phi \in C(\real)$,且
\begin{equation}
    \lim_{x \to +\infty} \frac{\phi(x)}{x^n} = \lim_{x \to -\infty} \frac{\phi(x)}{x^n} = 0.
\end{equation}

\begin{tasks}(1)
    \task 证明:当$n$为奇数时,方程$x^n + \phi(x) = 0$有一个实根.
    \task 证明:当$n$为偶数时,存在$y$,使得对所有的$x \in \real$,有
    \begin{equation*}
        y^n + \phi(y) \leq x^n + \phi(x)
    \end{equation*}
\end{tasks}

(1) \prove 令$p(x)=x^n + \phi (x)$,将方程左端写成
\begin{equation}
    p(x) = x^n + \phi(x) = x^n (1 + \frac{\phi(x)}{x^n})
\end{equation}
由于$n$是奇数,故
\begin{equation}
    \lim_{x \to -\infty} p(x) = \lim_{x \to -\infty} x^n(1+\frac{\phi(x)}{x^n}) = -\infty
\end{equation}
同理得
\begin{equation}
    \lim_{x \to +\infty} p(x) = +\infty
\end{equation}
依连续性,可知$\phi$在$\real$上每点都有定义,又因为$\displaystyle\lim_{x \to -\infty} p(x) = -\infty$并且$\displaystyle\lim_{x \to +\infty} p(x) = +\infty$,我们可以找到足够大的正数$M$,使得对一切$x$只只要满足$x < -M$就有$f(x) < 0$,只要满足$x > M$就有$f(x) > 0$,设$x_1 < -M, x_2 > M$,则$f(x_1)f(x_2) < 0$,由于函数$\phi$在$\real$上是连续的,故函数$p$在$\real$上是连续的,故$p$在$[x_1, x_2] \subset \real$上也是连续的,故存在$c \in (x_1, x_2)$使得$p(c) = 0$,也就是$p(c) = c^n + \phi(c) = 0$,由此可知这个$c$的确是方程$x^n+\phi(x)=0$的一个实根.由此可知方程$x^n+\phi(x)=0$在$\real$上的确起码有一个根.\qed\bigskip

(2) \prove 令$p(x) = x^n + \phi(x)$,当$n$为偶数时
\begin{equation}
    \lim_{x \to -\infty} p(x) = \lim_{x \to +\infty} p(x) = \lim_{x \to +\infty} x^n(1+\frac{\phi(x)}{x^n}) = +\infty
\end{equation}
任取$r_1 \in \real \; (r_1 > 0)$,令$m_1 = \displaystyle\inf_{x \in [-r_1, r_1]} p(x)$,那么由于$\displaystyle\lim_{x \to \infty} p(x) = +\infty$,存在足够大的正数$M > 0$,使得对一切$\lvert x \rvert > M$有$p(x) > m_1$,再令$m_2 = \displaystyle\inf_{x \in [-M, M]} p(x)$,那么$m_2 \leq m_1$,那么对任意$x \in (-\infty, M) \bigcup (M, +\infty)$,都有$f(x) > m_1 \geq m_2$.

而由于$m_2 = \displaystyle\inf_{x \in [-M,M]} p(x)$,故对一切$x \in [-M, M]$,都有$p(x) \geq m_2$,并且依\textbf{定理2.10.3},存在$c \in [-M, M]$,使得$p(c) = m_2$.故对任意$x \in \real$都有$p(x) \geq m_2$,当$x = c$时,等号成立.也就是说
\begin{equation}
    \exists c \in \real, \forall x \in \real, \quad p(x) = x^n + \phi(x) \geq m_2 = p(c) = c^n + \phi(c) 
\end{equation}
取$y=c$命题即证.\qed\bigskip

\annotate 题(1)仿造的是例题的证法.题(2)的证明思路是将整个实数轴划分为三个部分:左边的无限区间,中间的闭区间,右边的无限区间,并且注意到$p(x) \to +\infty$这个事实,找到那个$M$使得对一切$\lvert x \rvert > M$都有$p(x) > m_2= \displaystyle\inf_{\text{中间那个闭区间}} p(x)$.最后再引用\textbf{定理2.10.3}来说明这个$m_2$必可在中间那个闭区间中被取到就可以了.\bigskip

8. 设$f \in C[0,1]$且$f(0)=f(1)$.求证:对任何$n \in \nat$,存在$x_n \in [0,1]$,使得$f(x_n) = f(x_n + 1/n)$.

\prove 给定$n \in \nat$,令$g_n(x) = f(x + 1/n) - f(x)$,于是$g \in C[0,\displaystyle\frac{n-1}{n}]$,我们要证明对每一个$n \in \nat$,方程$g_n(x) = 0$在$[0,\displaystyle\frac{n-1}{n}]$都有根.来考虑两种情形:

\textbf{情形1:}对任意$x \in [0, \displaystyle\frac{n-1}{n}]$,$g(x) = f(x+1/n)-f(x)=0$,那么命题已经得证.

\textbf{情形2:}存在$x_1 \in [0, \displaystyle\frac{n-1}{n}]$,使得$g(x_1) = f(x_1+1/n)-f(x_1)>0$,那么就必定存在$x_2 \in [0, \displaystyle\frac{n-1}{n}], x_2 \neq x_1$,使得$g(x_2) = f(x_2+1/n)-f(x_2)<0$,否则就会推出与$f(1)-f(0)=0$矛盾的事实.依零值定理,存在$c \in (x_1,x_2)$,使得$g(c)=f(c+1/n)-f(c)=0$.也就是$f(c+1/n)=f(c)$,由于$n$是任取的,因此对每一个$n \in \nat$都存在这样的$c \in [0,\displaystyle\frac{n-1}{n}]\subset[0,1]$使得$f(c)=f(c+1/n)$,因而命题得证.\qed\bigskip

\annotate 题8仔细阅读后不难发现只是\textbf{例3}的另一种等价表述.这个题目充分表达了仔细阅读并理解例题的重要性,也告诉我们要学会举一反三、灵活应变,切不可总想着用套路来解题.\bigskip

9. 设$f \in C(\real)$且$\displaystyle\lim_{x \to +\infty}f(x) = \displaystyle\lim_{x \to -\infty}f(x) = +\infty$,又设$f$的最小值$f(a)<a$.求证:$f \circ f$至少在两个点上取到最小值.

\prove 由于$\displaystyle\lim_{x \to +\infty} f(x) = +\infty$,我们可以找到足够正的正数$b$,满足$b > a$,并且对一切$x \geq b$都有$f(x) > a$.从而$a \in f([a,b])$.又由于$\displaystyle\lim_{x \to -\infty}f(x) = +\infty$,所以我们还可以找到足够负的负数$c$,满足$c < a$,并且对一切$x \leq c$都有$f(x) > a$.从而$a \in f([c,a])$.

因为$a \in f([a,b])$,所以存在$x_1 \in [a,b]$,使得$f(x_1) = a$,从而$f(f(x_1))=f(a)$.同理存在$x_2 \in [c,a]$,使得$f(x_2) = a$,从而$f(f(x_2))=f(a)$.\qed\bigskip

10. 设函数$f: [a,b] \to \real$在有理点上取无理数值,在无理点上取有理数值.证明:$f$不是$[a,b]$上的连续函数.

\prove 由于$[a,b]$中的有理点是可数的,因此$f([a,b])$中的无理点也是可数的,因此$f([a,b])$是可数的,然而,如果$f$为连续函数,则$f([a,b])$为一闭区间,从而$f([a,b])$为不可数集,矛盾.\qed\bigskip

11. 设对任意$x,y \in (-\infty, +\infty)$,函数$f$满足$\lvert f(x) - f(y) \rvert \leq k \lvert x - y \rvert \; (0 < k < 1)$.求证:

\begin{tasks}(1)
    \task 函数$kx - f(x)$递增;
    \task 存在唯一的$\xi \in (-\infty, +\infty)$,使得$f(\xi) = \xi$.
\end{tasks}

(1) \prove 设$y > x > 0$.由题意:
\begin{align}
    &\mathrel{\phantom{\implies}} \lvert f(x) - f(y) \rvert \leq k \lvert x - y \rvert \\
    &\implies 0 \leq \lvert f(x) - f(y) \rvert \leq ky - kx \\ 
    &\implies 0 \leq ky - kx + f(x) - f(y) \\
    &\implies 0 \leq ky - f(y) - (kx - f(x))
\end{align}
从而函数$x \mapsto kx - f(x)$递增.\qed\bigskip

(2) \prove 令
\begin{equation}
    g(x) = x - f(x)
\end{equation}
则$g(x) = x - kx + kx - f(x) = x(1-k) + kx - f(x)$.由于$0 < k < 1$,所以$1-k > 0$,所以函数$x \mapsto x(1-k)$是严格递增的,又因为函数$x \mapsto kx - f(x)$是递增的,所以函数$g$是严格递增的.所以有
\begin{equation}
    \lim_{x \to +\infty} g(x) = +\infty, \quad \lim_{x \to -\infty} g(x) = -\infty
\end{equation}
所以存在$c \in (-\infty, +\infty)$,使得$g(c) = 0$.又因为$g$的严格单调性,所以这个$c$唯一.\qed\bigskip

12. 证明:在$(-\infty, +\infty)$上连续的周期函数必在$(-\infty, +\infty)$上一致连续.由此证明:$\left(\sin \, x\right)^2 + \sin \, \left(x\right)^2$不是周期函数.

\prove 设$f$为一个在$(-\infty,+\infty)$上连续的周期函数.由于$f$在$(-\infty,+\infty)$上连续,所以$f$在任何有界闭区间上连续,从而$f$在任何有界闭区间上一致连续.又根据周期性,可知$f$在$(-\infty,+\infty)$上一致连续.

由于函数$x \mapsto \left( \sin \, x\right)^2 + \sin \, \left(x\right)^2$是初等函数的有限次复合,并且定义域为$(-\infty, +\infty)$,从而函数$x \mapsto \left(\sin \, x\right)^2 + \sin \, \left(x\right)^2$是$(-\infty, +\infty)$上的连续函数.我们要证$x \mapsto \left(\sin \, x\right)^2 + \sin \, \left(x\right)^2$不是周期函数.为此,采用反证法,假设$x \mapsto \left(\sin \, x\right)^2 + \sin \, \left(x\right)^2$是周期函数,从而它在$(-\infty,+\infty)$上是一致连续的,然而这是不可能的,因为函数$x \mapsto \sin \, \left(x\right)^2$不一致连续,因而函数$x \mapsto \left(\sin \, x\right)^2 + \sin \, \left(x\right)^2$也不一致连续,矛盾.\qed\bigskip