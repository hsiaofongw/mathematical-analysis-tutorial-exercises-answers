\exercise

1. 对下列函数$f$,求$\displaystyle\limsup_{x \to \infty} f(x)$,$\displaystyle\liminf_{x \to \infty} f(x)$:

\begin{tasks}(1)
    \task $f(x) = \sin \, x$; 
    \task $f(x) = x^2 \cos \, x$;
    \task $f(x) = \displaystyle\frac{x}{1+x^2 \left(\sin \, x\right)^2}$.
\end{tasks}

(1) \solve 令
\begin{equation}
    x(n) = 2\pi n + \frac{\pi}{2}, \quad y(n) = 2 \pi n - \frac{\pi}{2}
\end{equation}
则
\begin{align}
    &\lim_{n \to +\infty} \sin \, x(n) = \lim_{n \to -\infty} \sin \, x(n) = 1 \\
    &\lim_{n \to +\infty} \sin \, y(n) = \lim_{n \to -\infty} \sin \, y(n) = -1
\end{align}
从而
\begin{equation}
    \limsup_{x \to \infty} \sin \, x = 1, \quad \liminf_{y \to \infty} \sin \, y = -1
\end{equation}
\qed\bigskip

(2) \solve 令
\begin{equation}
    x(n) = 2\pi n - \pi, \quad y(n) = 2\pi n
\end{equation}
于是有
\begin{align}
    \lim_{n \to +\infty} x(n)^2 \cos \, x(n) = \lim_{n \to \infty} -(x(n)^2) = -\infty
\end{align}
以及
\begin{align}
    \lim_{n \to +\infty} y(n)^2 \cos \, y(n) = \lim_{n \to -\infty} y(n)^2 \cos \, y(n) = \lim_{n \to \infty} y(n)^2 = +\infty
\end{align}
从而
\begin{align}
    \limsup_{x \to \infty} x^2 \cos \, x = +\infty, \quad \liminf_{x \to \infty} x^2 \cos \, x = - \infty
\end{align}
\qed\bigskip

(3) \solve 令
\begin{equation}
    x(n) = 2\pi n
\end{equation}
则
\begin{align}
    \lim_{n \to +\infty} \frac{x(n)}{1+x(n)^2 \left(\sin \, x(n)\right)^2} &= +\infty \\
    \lim_{n \to -\infty} \frac{x(n)}{1+x(n)^2 \left(\sin \, x(n)\right)^2} &= -\infty 
\end{align}
所以
\begin{align}
    \limsup_{x \to \infty} \frac{x}{1+x^2 \left(\sin \, x\right)^2} &= +\infty \\
    \liminf_{x \to \infty} \frac{x}{1+x^2 \left(\sin \, x\right)^2} &= -\infty 
\end{align}
\qed\bigskip

2. 设函数$f$在一定$x_0$的左边$(x_0-\delta,x_0)$上连续,且
\begin{equation*}
    \liminf_{x \to x_0^-} f(x) < \alpha < \limsup_{x \to x_0^-} f(x).
\end{equation*}
求证:一定有数列$x_n \to x_0^-$,使得$f(x_n)=\alpha \; (n=1,2,3,\cdots)$.

\prove 记
\begin{equation}
    \alpha_\star = \liminf_{x \to x_0^-} f(x), \quad \alpha^\star = \limsup_{x \to x_0^-} f(x)
\end{equation}
那么依\textbf{定理2.11.1},存在趋于$x_0$的数列$\{ a_n \}, \{ b_n \} \subset (x_0 - \delta, x_0)$,并且满足$f(a_n) \to \alpha_\star, f(b_n) \to \alpha^\star$.由于
\begin{equation}
    \alpha_\star < \alpha < \alpha^\star
\end{equation}
故存在足够大的正整数$M$,使得对一切$n \geq M$,都有
\begin{equation}
    f(a_n) < \alpha < f(b_n)
\end{equation}
依介值定理,存在
\begin{equation}
    x_n \in (a_n, b_n), \quad n = 1,2,3,\cdots
\end{equation}
使得
\begin{equation}
    f(x_n) = \alpha, \quad n = 1,2,3,\cdots
\end{equation}
并且由于
\begin{equation}
    a_n < x_n < b_n, \quad n = 1,2,3,\cdots
\end{equation}
并且$a_n, b_n \to x_0$,所以依夹逼定理我们有$x_n \to x_0$.\qed\bigskip

3. 设函数$f$在$x_0$的空心邻域内有定义,令
\begin{align*}
    \varphi\left(\delta\right) &= \inf \, \{ f(x) : 0 < \lvert x - x_0 \rvert < \delta \}, \\
    \psi\left(\delta\right) &= \sup \, \{ f(x) : 0 < \lvert x - x_0 \rvert < \delta \} .
\end{align*}
求证:
\begin{tasks}(1)
    \task $\varphi$和$\psi$分别是递减和递增函数;
    \task 令$\alpha = \displaystyle\lim_{\delta \to 0^+} \varphi (\delta), \beta = \displaystyle\lim_{\delta \to 0^+} \psi(\delta)$,则
    \begin{equation*}
        \alpha = \liminf_{x \to x_0} f(x), \quad \beta = \limsup_{x \to x_0} f(x).
    \end{equation*}
\end{tasks}

(1) \prove 设$\delta_2 > \delta_1 > 0$.于是
\begin{equation}
    \inf \, \{ f(x) : 0 < \lvert x - x_0 \rvert < \delta_1 \} \leq \inf \, \{ f(x) : 0 < \lvert x - x_0 \rvert < \delta_2 \}
\end{equation}
于是
\begin{equation*}
    \varphi (\delta_2) \leq \varphi (\delta_1)
\end{equation*}
这说明$\varphi$在$(0, +\infty)$是单调递减的.同理可证,$\psi$在$(0,+\infty)$是单调递增的.\qed\bigskip

(2) \prove 任取$l \in E$,依定义,存在数列$\{ x_n \} \subset B(\check{x_0}, \delta)$,并且满足$x_n \to x_0$,使得$f(x_n) \to l$.任取趋于$0$的数列$\{ \delta_n \} \subset B(\check{0}+, \delta)$,由于$x_n \to x_0$,故存在$i_1 \in \nat$,使得对一切$n \geq i_1$,都有
\begin{equation}
    \lvert x_n - x_0 \rvert < \delta_1
\end{equation}
从而
\begin{equation}
    \lvert x_{i_1} - x_0 \lvert < \delta_1
\end{equation}
从而
\begin{equation}
    f(x_{i_1}) < \sup \, \{ f(x) : 0 < \lvert x - x_0 \rvert < \delta_1 \} = \psi (\delta_1)
\end{equation}
类似地存在$i_2, i_3, i_4, \cdots \in \nat$,使得对一般的$n \in \nat$,都有
\begin{equation}
    f(x_{i_n}) < \psi (\delta_n)
\end{equation}
令$n \to \infty$,得
\begin{equation}
    \lim_{n \to \infty} f(x_{i_n}) \leq \lim_{n \to \infty} \phi (\delta_n)
\end{equation}
也就是
\begin{equation}
    l \leq \beta
\end{equation}
由于$l$是任取的,故对任意$x \in E$,都有$x \leq \beta$,而依\textbf{定理2.11.1},有$a^\star \in E$,因此
\begin{equation}
    a^\star \leq \beta
\end{equation}
下面我们再来证明$\beta \leq a^\star$,依\textbf{定理2.11.1},存在$h_1 > 0$,使得当$0 < \lvert x - x_0 \rvert < h_1$时,有
\begin{equation}
    f(x) < a^\star + 1
\end{equation}
从而
\begin{equation}
    \psi (h_1) = \sup \, \{ f(x) : 0 < \lvert x - x_0 \rvert < h_1 \} < a^\star + 1
\end{equation}
类似地我们可以找到一系列的$h_2, h_3, h_4, \cdots > 0$并且满足$h_n \to 0$,使得对一般的$n \in \nat$都有
\begin{equation}
    \psi (h_n) < a^\star + \frac{1}{n}
\end{equation}
令$n \to \infty$,有
\begin{equation}
    \lim_{n \to \infty} \psi (h_n) \leq \lim_{n \to \infty} \left( a^\star + \frac{1}{n} \right)
\end{equation}
也就是
\begin{equation}
   \lim_{\delta \to 0^+} \psi (\delta)  \leq \lim_{n \to \infty} \left(a^\star + \frac{1}{n}\right)
\end{equation}
也就是
\begin{equation}
    \beta \leq a^\star
\end{equation}
因为$a^\star \leq \beta$与$\beta \leq a^\star$同时成立,所以只能是$\beta = a^\star$.可以类似地证明$\alpha = a_\star$.\qed\bigskip