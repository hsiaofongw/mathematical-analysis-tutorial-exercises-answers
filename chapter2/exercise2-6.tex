\exercise

1. 求下列无穷小或无穷大的阶:

\begin{table}[H]
    \centering
    \begin{tabularx}{\textwidth} {  >{\raggedright\arraybackslash}X >{\raggedright\arraybackslash}X  }
       (1)~$x-5x^3+x^{10} \, (x \to 0)$; & (2)~$x-5x^3+x^{10} \, (x \to \infty)$; \\[1em]
       (3)~$\displaystyle\frac{x+1}{x^4+1} \, (x \to \infty)$; & (4)~$x^3-3x+2 \, (x \to 1)$; \\[1em]
       (5)~$\displaystyle\frac{2x^5}{x^3-3x+1} \, (x \to +\infty)$; & (6)~$\displaystyle\frac{1}{\sin \, \pi x} \, (x \to 1)$;
    \end{tabularx}
\end{table}

\pagebreak
\begin{table}[H]
    \centering
    \begin{tabularx}{\textwidth} {  >{\raggedright\arraybackslash}X >{\raggedright\arraybackslash}X  }
       (7)~$\sqrt{x\sin \, x} \, (x \to 0)$; & (8)~$\sqrt{x^2+\sqrt[3]{x}} \, (x \to 0)$; \\[1em]
       (9)~$\sqrt{x^2+\sqrt[3]{x}} \, (x \to \infty)$; & (10)~$\sqrt{1+x}-\sqrt{1-x} \, (x \to 0)$; \\[1em]
       (11)~$\sin \, \left(\displaystyle\sqrt{1+\displaystyle\sqrt{1+\displaystyle\sqrt{x}}}-\displaystyle\sqrt{2}\right) \, (x \to 0^+)$; & (12)~$\sqrt{1+\tan \, x} - \sqrt{1 - \sin \, x} \, (x \to 0)$; \\[1em]
       (13)~$\displaystyle\sqrt{x+\displaystyle\sqrt{x+\displaystyle\sqrt{x}}} \, (x \to \infty)$; & (14)~$(1+x)(1+x^2)\cdots(1+x^n) \; (x \to +\infty)$.
    \end{tabularx}
\end{table}

\medskip
(1) 解:因为
\begin{equation}
    \lim_{x \to 0} \frac{x-5x^3+x^{10}}{x} = \lim_{x \to 0} 1 - 5x^2+x^9 = 1
\end{equation}
又因为
\begin{equation}
    \lim_{x \to 0} x-5x^3+x^{10} = 10, \quad \lim_{x \to 0} x = 0
\end{equation}
所以$x-5x^3+x^{10}$与$x$是同阶无穷小.所以$x-5x^3+x^{10}$是$1$阶无穷小.\qed

\medskip
(2) 解:因为
\begin{equation}
    \lim_{x \to \infty} x^{10} = \infty
\end{equation}
且
\begin{equation}
    \lim_{x \to \infty} x-5x^3+x^{10} = \infty
\end{equation}
且
\begin{equation}
    \lim_{x \to \infty} \frac{x-5x^3+x^{10}}{x^{10}} = 1
\end{equation}
所以$x-5x^3+x^{10}$与$x^{10}$是同阶无穷大,又因为$x^{10}$是$10$阶无穷大,所以$x-5x^3+x^{10}$是$10$阶无穷大.\qed

\medskip
(3) 解:因为
\begin{equation}
    \lim_{x \to \infty} \frac{x+1}{x^4+1} = 0
\end{equation}
并且
\begin{equation}
    \lim_{x \to \infty} \frac{1}{\left( x+1 \right)^3} = 0
\end{equation}
并且
\begin{equation}
    \lim_{x \to \infty} \displaystyle\frac{\displaystyle\frac{x+1}{x^4+1}}{\displaystyle\frac{1}{\left(x+1\right)^3}} = l, \; (0 < l < \infty)
\end{equation}
所以当$x \to \infty$时$\displaystyle\frac{x+1}{x^4+1}$与$\displaystyle\frac{1}{\left(x+1\right)^3}$是同阶无穷小,而$\displaystyle\frac{1}{(x+1)^3}$是$3$阶无穷小,所以$\displaystyle\frac{x+1}{x^4+1}$是$3$阶无穷小.\qed

\medskip
(4) 解:对原式做因式分解得
\begin{equation}
    x^3 - 3x + 2 = (x-1)^2 (x+2)
\end{equation}
于是有
\begin{equation}
    \lim_{x \to 1} \frac{x^3-3x+2}{(x-1)^2} = \lim_{x \to 1} \frac{(x-1)^2(x+2)}{(x-1)^2} = \lim_{x \to 1} x+2 = 3
\end{equation}
又由于
\begin{equation}
    \lim_{x \to 1} x^3 - 3x + 2 = \lim_{x \to 1} (x-1)^2 (x+2) = 0
\end{equation}
以及
\begin{equation}
    \lim_{x \to 1} (x-1)^2 = 0
\end{equation}
所以当$x \to 1$时,我们说$x^3-3x+2$与$(x-1)^2$是等价无穷小.而$(x-1)^2$是$2$阶无穷小,所以$x^3-3x+2$是$2$阶无穷小.\qed

\medskip
(5) 解:由于
\begin{equation}
    \lim_{x \to +\infty} \frac{2x^5}{x^3-3x+1} = +\infty
\end{equation}
所以$\displaystyle\frac{2x^5}{x^3-3x+1}$是无穷大.又由于
\begin{equation}
    \lim_{x \to +\infty} \displaystyle\frac{\displaystyle\frac{2x^5}{x^3-3x+1}}{x^2} = l, \; (0 < l < +\infty)
\end{equation}
以及
\begin{equation}
    \lim_{x \to +\infty} x^2 = +\infty
\end{equation}
所以当$x \to +\infty$时,$\displaystyle\frac{2x^5}{x^3-3x+1}$与$x^2$是同阶无穷大,又因为$x^2$是$2$阶无穷大,所以$\displaystyle\frac{2x^5}{x^3-3x+1}$是$2$阶无穷大.\qed

\medskip
(6) 解:先将式子转化成我们熟悉的$\sin$的自变量趋于$0$的形式
\begin{align}
    \sin \, \left(\pi x\right) &= \sin \, \left(\pi(x-1)+\pi\right) \\
    &= \sin \left(\pi \left(x-1\right) \right) \cos \, \pi + \sin \, \pi \cos \, \left(\pi \left(x-1\right)\right) \\
    &= - \sin \, \left(\pi\left(x-1\right)\right)
\end{align}
于是有
\begin{equation}
    \lim_{x \to 1} \frac{1}{\sin \, \pi x} = -\lim_{x \to 1} \frac{1}{\sin \, \pi \left(x-1\right)}
\end{equation}
由于
\begin{equation}
    \lim_{x \to 1} \displaystyle\frac{\displaystyle\frac{1}{\sin \, \pi \left(x-1\right)}}{\displaystyle\frac{1}{\pi \left(x-1\right)}} = \lim_{x \to 1} \frac{\pi \left(x-1\right)}{\sin \, \pi \left(x-1\right)} = 1
\end{equation}
并且
\begin{equation}
    \lim_{x \to 1} \frac{1}{\pi \left(x-1\right)} = \infty
\end{equation}
以及
\begin{equation}
    \lim_{x \to 1} \frac{1}{\sin \, \pi \left(x-1\right)} = \infty
\end{equation}
所以当$x \to 1$时,$\displaystyle\frac{1}{\sin \, \pi \left(x-1\right)}$与$\displaystyle\frac{1}{\pi \left(x-1\right)}$是等价无穷大,并且由于$\displaystyle\frac{1}{\pi \left(x-1\right)}$是$1$阶无穷大,所以$\displaystyle\frac{1}{\sin \, \pi \left(x-1\right)}$是$1$阶无穷大.又由于
\begin{equation}
    \lim_{x \to 1} \frac{1}{\sin \, \pi x} = - \lim_{x \to 1} \frac{1}{\sin \, \pi \left(x-1\right)} 
\end{equation}
所以$\displaystyle\frac{1}{\sin \, \pi x}$是$1$阶无穷大.\qed

\medskip
(7) 解:因为
\begin{equation}
    \lim_{x \to 0} \frac{\sqrt{x \sin \, x}}{x} = \lim_{x \to 0} \displaystyle\sqrt{\displaystyle\frac{\sin \, x}{x}} = 1
\end{equation}
又因为
\begin{equation}
    \lim_{x \to 0} x = 0
\end{equation}
所以,$\displaystyle\sqrt{x \sin \, x}$与$x$是同阶无穷小,又因为$x$是$1$阶无穷小,所以$\displaystyle\sqrt{x \sin \, x}$是$1$阶无穷小.\qed

\medskip
(8) 解:因为
\begin{equation}
    \lim_{x \to 0} \sqrt{x^2+\sqrt[3]{x}} = 0
\end{equation}
所以$\sqrt{x^2+\sqrt[3]{x}}$是当$x \to 0$时的无穷小.又因为
\begin{equation}
    \lim_{x \to 0} {\displaystyle x^{1/6}} = \lim_{x \to 0} \sqrt{\sqrt[3]{x}} = 0
\end{equation}
所以$x^{1/6}$是当$x \to 0$时的无穷小.又因为
\begin{equation}
    \lim_{x \to 0} \displaystyle\frac{\sqrt{x^2+\sqrt[3]{x}}}{\sqrt{\sqrt[3]{x}}} = \lim_{x \to 0} \displaystyle\sqrt{\displaystyle\frac{x}{\sqrt[3]{x}}+1} = 1
\end{equation}
所以当$x \to 0$时$x^{1/6}$与$\displaystyle\sqrt{x^2+\sqrt[3]{x}}$是同阶无穷小.又因为$x^{1/6}$是$1/6$阶无穷小,所以$\displaystyle\sqrt{x^2+\sqrt[3]{x}}$是$1/6$阶无穷小.\qed

\medskip
(9) 解:因为
\begin{equation}
    \lim_{x \to \infty} \sqrt{x^2+\sqrt[3]{x}} = \infty
\end{equation}
所以当$x \to \infty$时,$\displaystyle\sqrt{x^2+\sqrt[3]{x}}$是无穷大.又因为
\begin{equation}
    \lim_{x \to \infty} x = \infty
\end{equation}
所以当$x \to \infty$时,$x$是无穷大.又因为
\begin{equation}
    \lim_{x \to \infty} \frac{\sqrt{x^2+\sqrt[3]{x}}}{x} = \lim_{x \to \infty} \sqrt{1 + \displaystyle\frac{\sqrt[3]{x}}{x}} = 1
\end{equation}
所以当$x \to \infty$时,$\sqrt{x^2+\sqrt[3]{x}}$与$x$是同阶无穷大,又因为$x$是$1$阶无穷大,所以$\sqrt{x^2 + \sqrt[3]{x}}$是$1$阶无穷大.\qed

\medskip
(10) 解:因为
\begin{equation}
    \lim_{x \to 0} \left( \sqrt{1+x} - \sqrt{1-x} \right) = \lim_{x \to 0} \frac{2x}{\sqrt{1+x}+\sqrt{1-x}} = 0
\end{equation}
所以当$x \to 0$时$\sqrt{1+x}-\sqrt{1-x}$是无穷小.又因为
\begin{equation}
    \lim_{x \to 0} 2x = 0
\end{equation}
并且
\begin{equation}
    \lim_{x \to 0} \displaystyle\frac{\sqrt{1+x}-\sqrt{1-x}}{2x}=\lim_{x \to 0} \displaystyle\frac{\displaystyle\frac{2x}{\sqrt{1+x}+\sqrt{1-x}}}{2x} = \lim_{x \to 0} \frac{1}{\sqrt{1+x}+\sqrt{1-x}} = \frac{1}{2}
\end{equation}
所以当$x \to 0$时$\sqrt{1+x}-\sqrt{1-x}$与$2x$是同阶无穷小,又因为$2x$是$1$阶无穷小,所以$\sqrt{1+x}-\sqrt{1-x}$是$1$阶无穷小.\qed

\medskip
(11) 解:因为
\begin{equation}
    \lim_{x \to 0^+} \displaystyle\frac{\sin \, \left(\displaystyle\sqrt{1+\displaystyle\sqrt{1+\displaystyle\sqrt{x}}}-\displaystyle\sqrt{2}\right)}{\displaystyle\sqrt{1+\displaystyle\sqrt{1+\displaystyle\sqrt{x}}}-\displaystyle\sqrt{2}} = 1
\end{equation}
所以当$x \to 0^+$时$\sin \, \left(\displaystyle\sqrt{1+\displaystyle\sqrt{1+\displaystyle\sqrt{x}}}-\displaystyle\sqrt{2}\right)$与$\displaystyle\sqrt{1+\displaystyle\sqrt{1+\displaystyle\sqrt{x}}}-\displaystyle\sqrt{2}$是同阶无穷小,又因为
\begin{align}
    \lim_{x \to 0^+} \displaystyle\frac{\displaystyle\sqrt{1+\displaystyle\sqrt{1+\displaystyle\sqrt{x}}}-\displaystyle\sqrt{2}}{\displaystyle\sqrt{1+\sqrt{x}}-1} &= \lim_{x \to 0^+} \displaystyle\frac{\displaystyle\frac{\displaystyle\sqrt{1+\displaystyle\sqrt{x}}-1}{\displaystyle\sqrt{1+\displaystyle\sqrt{1+\displaystyle\sqrt{x}}}+\displaystyle\sqrt{2}}}{\displaystyle\sqrt{1+\displaystyle\sqrt{x}}-1} \\
    &= \lim_{x \to 0^+} \displaystyle\frac{1}{\displaystyle\sqrt{1+\displaystyle\sqrt{1+\displaystyle\sqrt{x}}}+\displaystyle\sqrt{2}} \\
    &= \frac{1}{2\displaystyle\sqrt{2}}
\end{align}
所以$\displaystyle\sqrt{1+\displaystyle\sqrt{1+\displaystyle\sqrt{x}}}-\displaystyle\sqrt{2}$与$\displaystyle\sqrt{1+\displaystyle\sqrt{x}}-1$是同阶无穷小.又因为
\begin{align}
    \lim_{x \to 0^+} \displaystyle\frac{\displaystyle\sqrt{1+\displaystyle\sqrt{x}}-1}{\displaystyle\sqrt{x}} &= \lim_{x \to 0^+} \displaystyle\frac{\displaystyle\frac{\displaystyle\sqrt{x}}{\displaystyle\sqrt{1+\displaystyle\sqrt{x}}+1}}{\displaystyle\sqrt{x}} \\
    &= \lim_{x \to 0^+} \displaystyle\frac{1}{\displaystyle\sqrt{1+\displaystyle\sqrt{x}}+1} = \frac{1}{2}
\end{align}
所以$\displaystyle\sqrt{1+\displaystyle\sqrt{x}}-1$与$\displaystyle\sqrt{x}$是同阶无穷小.所以$\displaystyle\sqrt{1+\displaystyle\sqrt{1+\displaystyle\sqrt{x}}}-\displaystyle\sqrt{2}$与$\sqrt{x}$是同阶无穷小.又因为$\displaystyle\sqrt{x}$是$1/2$阶无穷小,所以$\displaystyle\sqrt{1+\displaystyle\sqrt{1+\displaystyle\sqrt{x}}}-\displaystyle\sqrt{2}$是$1/2$阶无穷小.\qed

\medskip
(12) 解:由于
\begin{align}
    \lim_{x \to 0} \frac{\sqrt{1+\tan \, x} - \sqrt{1-\sin \, x}}{\tan \, x - \sin \, x} &= \lim_{x \to 0} \displaystyle\frac{\displaystyle\frac{\tan \, x - \sin \, x}{\sqrt{1+\tan \, x}+\sqrt{1 - \sin \, x}}}{\tan \, x - \sin \, x} \\
    &= \lim_{x \to 0} \frac{1}{\sqrt{1+\tan \, x} + \sqrt{1- \sin \, x}} = \frac{1}{2}
\end{align}
于是当$x \to 0$时,$\sqrt{1+\tan \, x}-\sqrt{1-\sin \, x}$与$\tan \, x - \sin \, x$是同阶无穷小,又因为
\begin{align}
    \lim_{x \to 0} \frac{\tan \, x - \sin \, x}{\displaystyle\frac{1}{2}x^3} &= \lim_{x \to 0} \frac{\displaystyle\frac{\sin \, x}{\cos \, x} - \sin \, x}{\displaystyle\frac{1}{2} x^3} \\
    &= \lim_{x \to 0} \frac{\displaystyle\frac{1}{\cos \, x} \sin \, x \left(1- \cos \, x\right)}{\displaystyle\frac{1}{2}x^3} \\
    &= \lim_{x \to 0} \frac{1}{\cos \, x} \lim_{x \to 0} \frac{x \cdot \displaystyle\frac{1}{2}x^2}{\displaystyle\frac{1}{2}x^3} = 1 \cdot 1 = 1.
\end{align}
由此可见当$x \to 0$时$\tan \, x - \sin \, x$与$x^3$是同阶无穷小,也就是说$\tan \, x - \sin \, x$是$3$阶无穷小,从而$\sqrt{1+\tan \, x} - \sqrt{1 - \sin \, x}$是$3$阶无穷小.\qed

\medskip
(13) 解:由于
\begin{align}
    \displaystyle\sqrt{x+\displaystyle\sqrt{x+\displaystyle\sqrt{x}}} &= \displaystyle\sqrt{x+\displaystyle\sqrt{x+\displaystyle\sqrt{x}}} - \displaystyle\sqrt{x+\displaystyle\sqrt{x}} + \displaystyle\sqrt{x+\displaystyle\sqrt{x}} \\
    &= \displaystyle\sqrt{x+\displaystyle\sqrt{x+\displaystyle\sqrt{x}}} - \displaystyle\sqrt{x+\displaystyle\sqrt{x}} + \displaystyle\sqrt{x+\displaystyle\sqrt{x}} - \sqrt{x} + \sqrt{x} \\
    &= \frac{\displaystyle\sqrt{x+\displaystyle\sqrt{x}}-\displaystyle\sqrt{x}}{\displaystyle\sqrt{x+\displaystyle\sqrt{x+\displaystyle\sqrt{x}}}+\displaystyle\sqrt{x+\displaystyle\sqrt{x}}} + \displaystyle\frac{\displaystyle\sqrt{x}}{\displaystyle\sqrt{x+\displaystyle\sqrt{x}}+\displaystyle\sqrt{x}} + \sqrt{x} \\
    &= \frac{\displaystyle\frac{\displaystyle\sqrt{x}}{\displaystyle\sqrt{x+\displaystyle\sqrt{x}}+\displaystyle\sqrt{x}}}{\displaystyle\sqrt{x+\displaystyle\sqrt{x+\displaystyle\sqrt{x}}}+\displaystyle\sqrt{x+\displaystyle\sqrt{x}}} + \displaystyle\frac{\displaystyle\sqrt{x}}{\displaystyle\sqrt{x+\displaystyle\sqrt{x}}+\displaystyle\sqrt{x}} + \sqrt{x} \\
\end{align}
所以
\begin{equation}
    \lim_{x \to \infty} \frac{\displaystyle\sqrt{x+\displaystyle\sqrt{x+\displaystyle\sqrt{x}}}}{\displaystyle\sqrt{x}} = \lim_{x \to \infty} \left( \frac{\displaystyle\frac{1}{\displaystyle\sqrt{x+\displaystyle\sqrt{x}}+\displaystyle\sqrt{x}}}{\displaystyle\sqrt{x+\displaystyle\sqrt{x+\displaystyle\sqrt{x}}}+ \displaystyle\sqrt{x+\displaystyle\sqrt{x}}} + \frac{1}{\displaystyle\sqrt{x+\displaystyle\sqrt{x}}+\displaystyle\sqrt{x}} + 1 \right) = 1
\end{equation}
所以当$x \to \infty$时$\displaystyle\sqrt{x+\displaystyle\sqrt{x+\displaystyle\sqrt{x}}}$与$\sqrt{x}$是同阶无穷大,而且$\displaystyle\sqrt{x}$是$1/2$阶无穷大,所以$\displaystyle\sqrt{x+\displaystyle\sqrt{x+\displaystyle\sqrt{x}}}$是$1/2$阶无穷大.\qed

(14) 解:令
\begin{equation}
    P(x) = \sum_{i=1}^n (1+x^i)
\end{equation}
那么$P(x)$是一个多项式,它的阶数为$\deg \, P(x)$,并且因为
\begin{equation}
    \deg \, P(x) = \deg \, \sum_{i=1}^n (1+x^i) = \sum_{i=1}^n \deg \, (1+x^i) = \sum_{i=1}^n i = \frac{n(n+1)}{2}
\end{equation}
所以,$P(x)$是一个$\displaystyle\frac{n(n+1)}{2}$次多项式,记为
\begin{equation}
    P(x) = A x^{n(n+1)/2} + B R(x)
\end{equation}
式中,$A$为一非零常数,$R(x)$由多项式$P(x)$中所有次数小于$\displaystyle\frac{n(n+1)}{2}$的项构成,因此$\deg \, \displaystyle\frac{BR(x)}{x^{n(n+1)/2}} < 0$,因此有
\begin{equation}
    \lim_{x \to \infty} \frac{BR(x)}{x^{n(n+1)/2}} = 0
\end{equation}
因此有
\begin{equation}
    \lim_{x \to \infty} \frac{P(x)}{x^{n(n+1)/2}} = \lim_{x \to \infty} \frac{A X^{n(n+1)/2}}{x^{n(n+1)/2}} + \lim_{x \to \infty} \frac{BR(x)}{x^{n(n+1)/2}} = A + 0 = A < \infty
\end{equation}
因此$P(x)$与$x^{n(n+1)/2}$是同阶无穷大,因此$P(x)$的阶数是$n(n+1)/2$.\qed

\bigskip
2. 当$x \to x_0$时,$\alpha = o(1)$.求证:
\begin{table}[H]
    \centering
    \begin{tabularx}{\textwidth} {  >{\raggedright\arraybackslash}X >{\raggedright\arraybackslash}X  }
        (1)~$o(\alpha) + o(\alpha) = o(\alpha)$; & (2)~$o(c\alpha) = o(\alpha)$~($c$ 为常数); \\ [1em]
        (3)~$(o(\alpha))^k = o(\alpha^k)$; & (4)~$\displaystyle\frac{1}{1+\alpha} = 1 - \alpha + o(\alpha)$.
    \end{tabularx}
\end{table}

(1) \textbf{证明:}任取$f \in o(\alpha)$.由题意知
\begin{equation}
    \lim_{x \to x_0} \frac{f(x)}{\alpha} = 0
\end{equation}
对该式两边乘以$2$得
\begin{equation}
    2\lim_{x \to x_0} \frac{f(x)}{\alpha} = 0
\end{equation}
也就是
\begin{equation}
    \lim_{x \to x_0} \frac{2f(x)}{\alpha} = 0
\end{equation}
也就是
\begin{equation}
    \lim_{x \to x_0} \frac{f(x) + f(x)}{\alpha} = 0
\end{equation}
这说明
\begin{equation}
    f(x) + f(x) \in o(\alpha)
\end{equation}
所以说
\begin{equation}
    o(\alpha) + o(\alpha) \subset o(\alpha)
\end{equation}
由于
\begin{equation}
    o(\alpha) + o(\alpha) = \{ f+f: \lim_{x \to x_0} \displaystyle\frac{f(x)}{\alpha} \}
\end{equation}
任取$g \in o(\alpha)$,显然$0 \in o(\alpha)$,于是$g + 0 \in o(\alpha) + o(\alpha)$,于是$g \in o(\alpha) + o(\alpha)$,所以有
\begin{equation}
    o(\alpha) \subset o(\alpha) + o(\alpha)
\end{equation}
因为$o(\alpha) \subset o(\alpha) + o(\alpha)$并且$o(\alpha) + o(\alpha) \subset o(\alpha)$,所以
\begin{equation}
    o(\alpha) + o(\alpha) = o(\alpha)
\end{equation}
\qed

(2) \textbf{证明:}任取$f \in o(c\alpha)$,由定义
\begin{equation}
    \lim_{x \to x_0} \frac{f(x)}{c \alpha} = 0
\end{equation}
显然$c \neq 0$,两边同乘$c$得
\begin{equation}
    \lim_{x \to x_0} \frac{f(x)}{\alpha} = 0
\end{equation}
于是$f(x) \in o(\alpha)$,于是$o(c\alpha) \subset o(\alpha)$.任取$g \in o(\alpha)$,由定义
\begin{equation}
    \lim_{x \to x_0} \frac{g(x)}{\alpha} = 0
\end{equation}
两边同除$c$得
\begin{equation}
    \lim_{x \to x_0} \frac{g(x)}{c \alpha} = 0
\end{equation}
于是有$o(\alpha) \subset o(c\alpha)$,于是有
\begin{equation}
    o(\alpha) = o(c\alpha)
\end{equation}
\qed

(3) \textbf{证明:}任取$f \in o(\alpha^k)$,由定义
\begin{equation}
    \lim_{x \to x_0} \frac{f(x)}{\alpha^k} = 0
\end{equation}
于是
\begin{equation}
    \left( \lim_{x \to x_0} \frac{f(x)}{\alpha^k} \right)^k = 0^k = 0
\end{equation}
并且由极限的乘法运算法则
\begin{equation}
    \left( \lim_{x \to x_0} \frac{f(x)}{\alpha^k} \right)^k = \lim_{x \to x_0} \left( \displaystyle\frac{f(x)}{a^k}  \right)^k = \lim_{x \to x_0} \frac{\left(f(x)\right)^k}{\alpha^{2k}}
\end{equation}
也就是说
\begin{equation}
    \lim_{x \to x_0} \frac{\left(f(x)\right)^{k}}{\alpha^{2k}} = 0
\end{equation}
对上式两边同乘$\alpha^k$得
\begin{equation}
    \lim_{x \to x_0} \frac{\left(f(x)\right)^k}{\alpha^k} = 0
\end{equation}
这说明$f \in \left(o(\alpha)\right)^k$,从而$o(\alpha^k) \subset \left(o(\alpha)\right)^k$.任取$g \in \left(o(\alpha)\right)^k$,则
\begin{equation}
    \lim_{x \to x_0} \frac{g(x)^k}{\alpha^k} = 0
\end{equation}
推出
\begin{equation}
    \lim_{x \to x_0} \frac{g(x)}{\alpha} = 0
\end{equation}
推出
\begin{equation}
    \lim_{x \to x_0} \frac{g(x)}{\alpha^k} = 0
\end{equation}
推出$g \in o(\alpha^k)$,从而$\left(o(\alpha)\right)^k \subset o(\alpha^k)$,从而
\begin{equation}
    o(\alpha^k) = \left(o(\alpha)\right)^k
\end{equation}
\qed

(4) \textbf{证明:}
\begin{align}
    &\mathrel{\phantom{\iff}} \frac{1}{1+\alpha} = 1 - \alpha + o(\alpha) \\
    &\iff \frac{1}{1+\alpha} - 1 + \alpha = o(\alpha) \\
    &\iff \frac{\alpha^2}{1+\alpha} = o(\alpha) \\
    &\iff \text{任取} \, f \in o(\alpha), \lim_{x \to x_0} \frac{\displaystyle\frac{\alpha^2}{1+\alpha}}{\alpha} = \lim_{x \to x_0} \frac{\alpha}{1+\alpha} = \lim_{x \to x_0} \frac{f(x)}{\alpha} = 0.
\end{align}
\qed

\bigskip
3. 利用等价无穷小替换,求下列极限:
\begin{table}[H]
    \centering
    \begin{tabularx}{\textwidth} {  >{\raggedright\arraybackslash}X >{\raggedright\arraybackslash}X  }
        (1)~$\displaystyle\lim_{x \to 0} \displaystyle\frac{x (\tan \, x)^4}{(\sin \, x)^3 \left(1-\cos \, x\right)}$; & (2)~$\displaystyle\lim_{x \to 0} \displaystyle\frac{\sqrt{1+x^2}-1}{1-\cos \, x}$; \\[1em]
        (3)~$\displaystyle\lim_{x \to 0} \displaystyle\frac{\sqrt{1+x^4}-1}{1-(\cos \, x)^2}$; & (4)~$\displaystyle\lim_{x \to 0}\displaystyle\frac{\tan \, \tan \, x}{\sin \, x}$; \\[1em]
        (5)~$\displaystyle\lim_{x \to 0} \displaystyle\frac{(1+x+x^2)^{1/n}-1}{\sin \, 2x} \; (n \in \nat)$.
    \end{tabularx}
\end{table}
