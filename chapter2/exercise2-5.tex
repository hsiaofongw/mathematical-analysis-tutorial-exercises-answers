\exercise

1. 用$\epsilon$-$A$语言证明:
\begin{table}[H]
    \centering
    \begin{tabularx}{\textwidth} {  >{\raggedright\arraybackslash}X >{\raggedright\arraybackslash}X  }
       (1)~$\displaystyle\lim_{x \to -\infty} \displaystyle\frac{x^2+1}{3x^2-x+1}=\displaystyle\frac{1}{3}$; & (2)~$\displaystyle\lim_{x \to \infty} \displaystyle\frac{3x+2}{2x+3} = \displaystyle\frac{3}{2}$; \\ [1em]
       (3)~$\displaystyle\lim_{x \to +\infty} (x - \sqrt{x^2-a}) = 0$; & (4)~$\displaystyle\lim_{x \to +\infty} (\sqrt{x+1}-\sqrt{x-1}) = 0$.
    \end{tabularx}
\end{table}

(1) 证明:对任意$\epsilon > 0$,取$A = \max \{ 5, \displaystyle\frac{2}{\epsilon} \}$,那么对一切$x < -A$,都有$x^2+2x<3x^2-x+1$,并且有
\begin{align} 
    \bigg\lvert \displaystyle\frac{x^2+1}{3x^2-x+1} - \displaystyle\frac{1}{3} \bigg\rvert &= \bigg\lvert \displaystyle\frac{x+2}{3x^2-x+1} \bigg\rvert = \displaystyle\frac{-(x+2)}{3x^2-x+1} < \displaystyle\frac{-(x+2)}{x^2+2x} = -\displaystyle\frac{1}{x} < \frac{1}{A} < \epsilon
\end{align}
这就证明了$\displaystyle\lim_{x \to -\infty} \displaystyle\frac{x^2+1}{3x^2-x+1}=\displaystyle\frac{1}{3}$. \qed

(2) 证明:对任意$\epsilon > 0$,取$A = \max \{10, \displaystyle\frac{10}{\epsilon} \}$,那么当$|x| > |A|$时有$|2x+5|>|x|$,并且有
\begin{align}
    \bigg\lvert \displaystyle\frac{3x+2}{2x+3} - \displaystyle\frac{3}{2} \bigg\rvert = \bigg\lvert \displaystyle \frac{-5}{2x+3} \bigg\rvert = 5 \cdot \displaystyle\frac{1}{|2x+3|} < 5 \cdot \displaystyle\frac{1}{|x|} < 5 \cdot \displaystyle\frac{1}{|A|} \leq \displaystyle\frac{\epsilon}{2} < \epsilon
\end{align}
这就证明了$\displaystyle\lim_{x \to \infty} \displaystyle\frac{3x+2}{2x+3} = \displaystyle\frac{3}{2}$. \qed

(3) 证明:对任意$\epsilon > 0$,取$A = \displaystyle\frac{2|a|}{\epsilon}$,那么当$x > A$时
\begin{align}
    |x-\sqrt{x^2-a}|=\bigg\lvert \displaystyle\frac{a}{x+\sqrt{x^2-a}} \bigg\rvert < \displaystyle\frac{|a|}{x} < \displaystyle\frac{\epsilon}{2} < \epsilon
\end{align}
这就证明了$\displaystyle\lim_{x \to +\infty}(x-\sqrt{x^2-a})=0$. \qed

(4) 证明:对任意$\epsilon > 0$,取$A=\displaystyle\frac{4}{\epsilon^2}$,那么当$x > A$时
\begin{align}
    | \sqrt{x+1}-\sqrt{x-1} | = \bigg\lvert \displaystyle\frac{2}{\sqrt{x+1}+\sqrt{x-1}} \bigg\rvert < 2 \cdot \displaystyle\frac{1}{\sqrt{x}} < 2 \cdot \displaystyle\frac{1}{\sqrt{A}} = \epsilon
\end{align}
这就证明了$\displaystyle\lim_{x \to +\infty} (\sqrt{x+1}-\sqrt{x-1})=0$. \qed

2. 定出常数$a$和$b$,使得下列等式成立:
\begin{enumerate}
    \item $\displaystyle\lim_{x\to\infty}\left(\displaystyle\frac{x^2+1}{x+1}-ax-b\right)=0$; 
    \item $\displaystyle\lim_{x\to +\infty}\left(\sqrt{x^2-x+1}-ax-b\right)=0$;
    \item $\displaystyle\lim_{x\to -\infty}\left(\sqrt{x^2-x+1}-ax-b\right)=0$.
\end{enumerate}

(1) 证明:因为
\begin{align}
    \bigg\lvert\displaystyle\frac{x^2+1}{x+1}-a x - b\bigg\rvert=\bigg\lvert \displaystyle\frac{(1-a)x^2 - (a+b)x +(1-b)}{x+1} \bigg\rvert
\end{align}
所以要使得$\displaystyle\lim_{x\to\infty}\left(\displaystyle\frac{x^2+1}{x+1}-ax-b\right)=0$,$a,b$应满足
\begin{equation}
    \begin{cases}
        1-a &= 0\\
        a+b &= 0
    \end{cases}
\end{equation}
解出$a=1,b=-1$.下面我们证明这样的$a,b$确实使得极限等于$0$.因为
\begin{equation}
    \bigg\lvert\displaystyle\frac{x^2+1}{x+1}-ax-b\bigg\rvert = \bigg\lvert\displaystyle\frac{2}{x+1}\bigg\rvert
\end{equation}
所以对任意$\epsilon > 0$,取$A=\displaystyle\frac{2}{\epsilon}$,那么当$x>A$时有
\begin{equation}
    \bigg\lvert\displaystyle\frac{2}{x+1}\bigg\rvert=\frac{2}{|x+1|}<\frac{2}{x}<\frac{2}{A}=\epsilon
\end{equation}
这说明$\displaystyle\lim_{x\to +\infty}\left(\displaystyle\frac{x^2+1}{x+1}-ax-b\right)=0$.

又对任意$\epsilon>0$,取$A=\displaystyle\frac{1}{\epsilon}+1$,那么当$x<-A$时
\begin{equation}
    \bigg\lvert\displaystyle\frac{2}{x+1}\bigg\rvert=\frac{2}{-x-1}<\frac{1}{A-1}=\frac{1}{\displaystyle\frac{1}{\epsilon}}=\epsilon
\end{equation}
这说明$\displaystyle\lim_{x\to -\infty}\left(\displaystyle\frac{x^2+1}{x+1}-ax-b\right)=0$.

这就证明了$\displaystyle\lim_{x \to \infty}\left(\displaystyle\frac{x^2+1}{x+1}-ax-b\right)=0$. \qed

从而$\displaystyle\lim_{x\to\infty}\left(\displaystyle\frac{x^2+1}{x+1}-ax-b\right)=0$.\qed

(2) 证明:当$a=1,b=-\displaystyle\frac{1}{2}$时
\begin{equation}
    |\sqrt{x^2-x+1}-ax-b|=\bigg\lvert\sqrt{x^2-x+1}-(x-\displaystyle\frac{1}{2})\bigg\rvert=\displaystyle\frac{\displaystyle\frac{3}{4}}{\sqrt{x^2-x+1}+x-\displaystyle\frac{1}{2}}
\end{equation}
从而对任意$\epsilon > 0$,取$A=\displaystyle\frac{1}{\epsilon}+\displaystyle\frac{1}{2}$,那么当$x>A$时
\begin{align}
    |\sqrt{x^2-x+1}-ax-b|=\displaystyle\frac{\displaystyle\frac{3}{4}}{\sqrt{x^2-x+1}+x-\displaystyle\frac{1}{2}} < \displaystyle\frac{1}{x-\displaystyle\frac{1}{2}}<\displaystyle\frac{1}{A-\displaystyle\frac{1}{2}}=\epsilon
\end{align}
从而当$a=1,b=-\displaystyle\frac{1}{2}$时$\displaystyle\lim_{x\to +\infty}\sqrt{x^2-x+1}-ax-b=0$. \qed

(3) 证明:做换元$t=-x$,那么当$x \to -\infty$时$t \to +\infty$,等价于寻找$a,b$使得
\begin{equation}
    \displaystyle\lim_{x \to +\infty}\sqrt{t^2+t+1}+at-b=0
\end{equation}
成立.取$a=-1,b=\displaystyle\frac{1}{2}$,那么我们有
\begin{align}
    |\sqrt{t^2}+t+1+at-b|=\frac{3}{4} \cdot \Bigg\lvert\displaystyle\frac{1}{\displaystyle\sqrt{t^2+t+1}+t+\displaystyle\frac{1}{2}} \Bigg\rvert
\end{align}
对任意$\epsilon>0$,取$A=\max\{0, \displaystyle\frac{1}{\epsilon}-\displaystyle\frac{1}{2}\}$,那么当$t>A$的时候就有
\begin{equation}
    \displaystyle\frac{3}{4}\Bigg\lvert\displaystyle\frac{1}{\sqrt{t^2+t+1}+t}+t+\displaystyle\frac{1}{2}\Bigg\rvert<\frac{1}{t+\displaystyle\frac{1}{2}}<\frac{1}{A+\displaystyle\frac{1}{2}} \leq \epsilon
\end{equation}
这就证明了当$a=-1,b=\displaystyle\frac{1}{2}$时$\displaystyle\lim_{t\to +\infty}\sqrt{t^2+t+1}+at-b=\displaystyle\lim_{x\to -\infty}\sqrt{x^2-x+1}-ax-b=0$.\qed

3. 证明:
\begin{equation}
    \lim_{x \to +\infty} \left(\sin \, \sqrt{x+1} - \sin \, \sqrt{x-1}\right) = 0.
\end{equation}

证明:利用三角和差化积公式,我们有
\begin{align}
    \lim_{x \to +\infty} \left( \sin \, \sqrt{x+1} - \sin \, \sqrt{x-1} \right) &= 2 \lim_{x \to +\infty} \cos \, \displaystyle \frac{\sqrt{x+1}+\sqrt{x-1}}{2} \sin \, \displaystyle\frac{\sqrt{x+1}-\sqrt{x-1}}{2}
\end{align}
利用函数$\cos $的有界性,得到
\begin{align}
    \Bigg\lvert \cos \, \displaystyle\frac{\sqrt{x+1}+\sqrt{x-1}}{2} \sin \, \displaystyle\frac{\sqrt{x+1}-\sqrt{x-1}}{2} \Bigg\rvert &< \Bigg\lvert \sin \, \displaystyle\frac{\sqrt{x+1}-\sqrt{x-1}}{2} \Bigg\rvert \\ 
    &= \Bigg\lvert \sin \, \displaystyle\frac{1}{\sqrt{x+1}+\sqrt{x-1}} \Bigg\rvert
\end{align}
由于
\begin{equation}
    \lim_{x \to +\infty} \displaystyle\frac{\sin \, \displaystyle\frac{1}{\sqrt{x+1}+\sqrt{x-1}}}{\displaystyle\frac{1}{\sqrt{x+1}+\sqrt{x-1}}} = 1
\end{equation}
而
\begin{equation}
    \lim_{x \to +\infty} \displaystyle\frac{1}{\sqrt{x+1}+\sqrt{x-1}} = 0
\end{equation}
所以
\begin{equation}
    \lim_{x \to +\infty} \sin \, \displaystyle\frac{1}{\sqrt{x+1}+\sqrt{x-1}} = 0
\end{equation}
所以对任意$\epsilon > 0$,都存在$A > 0$,使得当$x > A$时,有
\begin{equation}
    \Bigg\lvert\sin \, \displaystyle\frac{1}{\sqrt{x+1}+\sqrt{x-1}}\Bigg\rvert<\epsilon
\end{equation}
再考虑已经推出的不等式关系,得
\begin{equation}
    \Bigg\lvert\cos\,\displaystyle\frac{\sqrt{x+1}+\sqrt{x-1}}{2}\sin\,\displaystyle\frac{\sqrt{x+1}-\sqrt{x-1}}{2} \Bigg\rvert<\epsilon
\end{equation}
从而
\begin{equation}
    \lim_{x \to +\infty} \cos \, \displaystyle\frac{\sqrt{x+1}+\sqrt{x-1}}{2}\sin\,\displaystyle\frac{\sqrt{x+1}-\sqrt{x-1}}{2}=0
\end{equation}
从而
\begin{eqnarray}
    \lim_{x \to +\infty} \left(\sin \, \sqrt{x+1}-\sin \, \sqrt{x-1}\right) = 2 \cdot 0 = 0.
\end{eqnarray}
\qed

4. 设常数$a_1,a_2,\cdots,a_n$满足$a_1+a_2+\cdots+a_n=0$.求证:
\begin{equation}
    \lim_{x \to +\infty}\sum_{k=1}^{n}a_k\sin \, \sqrt{x+k}=0.
\end{equation}

证明:若$n=1$,那么$a_1=0$,那么
\begin{eqnarray}
    \lim_{x \to +\infty} \sum_{k=1}^n a_k \sin \, \sqrt{x+k} = \lim_{x \to +\infty} 0 = 0
\end{eqnarray}
从而当$n=1$时结论是成立的.现考虑$n \geq 2$的情形.

当$n=2$时
\begin{align}
    &\mathrel{\phantom{=}} a_1 \sin \, \sqrt{x+1} + a_2 \sin \, \sqrt{x+2} \\
    &= a_1 \sin \, \sqrt{x+1} + a_2 \sin \, \sqrt{x+1} + a_2 \sin \, \sqrt{x+2} - a_2 \sin \, \sqrt{x+1} \\
    &= (a_1 + a_2) \sin \, \sqrt{x+1} + a_2 \left(\sin \, \sqrt{x+2} - \sin \, \sqrt{x+1} \right)
\end{align}

当$n=3$时
\begin{align}
    &\mathrel{\phantom{=}} a_1 \sin \, \sqrt{x+1} + a_2 \sin \, \sqrt{x+2} + a_3 \sin \, \sqrt{x+3} \\
    &= a_1 \sin \, \sqrt{x+1} + \left( a_2 + a_3 \right) \sin \, \sqrt{x+2} + a_3 \left( \sin \, \sqrt{x+3} - \sin \, \sqrt{x+2} \right) \\
    &= \left(a_1 + a_2 + a_3\right) \sin \, \sqrt{x+1} \\
    &+ \left(a_2 + a_3\right) \left(\sin \, \sqrt{x+2} - \sin \, \sqrt{x+1}\right) + a_3 \left(\sin \, \sqrt{x+3} - \sin \, \sqrt{x+2}\right)
\end{align}
我们猜测对于每一个$n \in \nat, n \geq 2$,都有
\begin{equation}
    \sum_{k = 1}^n a_k \sin \, \sqrt{x+k} = \left( \sum_{k = 1}^{n} a_k \sin \, \sqrt{x+1} \right) + \sum_{i=2}^n \sum_{j=i}^n a_j \left( \sin \, \sqrt{x+j} - \sin \sqrt{x+j-1} \right) 
\end{equation}
证明从略.

于是
\begin{align}
    &\mathrel{\phantom{=}} \lim_{x \to +\infty} \sum_{k=1}^{n} a_k \sin \, \sqrt{x+k} \\
    &= \lim_{x \to +\infty} \left( \left( \sum_{k = 1}^{n} a_k \sin \, \sqrt{x+1} \right) + \sum_{i=2}^n \left(\sum_{j=i}^n a_j \left( \sin \, \sqrt{x+j} - \sin \sqrt{x+j-1} \right) \right) \right) \\
    &= \lim_{x \to +\infty} \sum_{i=2}^n \left(\sum_{j=i}^n a_j \left( \sin \, \sqrt{x+j} - \sin \sqrt{x+j-1} \right) \right) \\
    &= \sum_{i=2}^n \sum_{j=i}^n a_j \lim_{x \to +\infty} \left( \sin \, \sqrt{x+j} - \sin \sqrt{x+j-1} \right) \\
    &= \sum_{i=2}^n \sum_{j=i}^n a_j \cdot 0 = 0.
\end{align}
待证得证.\qed

5. 求极限$\displaystyle\lim_{n \to \infty} \sin \, \left(\pi \, \sqrt{n^2+1}\right)$.

解:先将$\pi \sqrt{n^2+1}$拆开写,再利用和角正弦公式,得
\begin{align}
    \lim_{n \to \infty} \sin \, \left(\pi \, \sqrt{n^2+1}\right) &= \lim_{n \to \infty} \sin \, \left( n \pi + \sqrt{n^2+1} \, \pi  - n \pi \right) \\
    &= \lim_{n \to \infty} \cos \, n \pi \sin \, \left(\sqrt{n^2+1} \, \pi - n \pi\right)
\end{align}
下证$\displaystyle\lim_{n \to \infty} \cos \, n \pi \sin \, \left( \pi \sqrt{n^2+1} - \pi n \right) = 0$.

证明:令
\begin{equation}
    a_n = \sqrt{n^2+1} \pi - n \pi
\end{equation}
先证数列$\{ a_n \}$的极限是$0$:对任意$\epsilon > 0$,取$N = \bigg\lceil \displaystyle\frac{\pi}{\epsilon} \bigg\rceil$,那么对每一个$n \geq N$就都有
\begin{align}
    |a_n-0|=\pi|\sqrt{n^2+1}-n|=\pi\bigg\lvert\displaystyle\frac{1}{\sqrt{n^2+1}+n}\bigg\rvert < \frac{\pi}{n} < \frac{\pi}{N} \leq \epsilon
\end{align}
这里证明了$\displaystyle\lim_{n \to \infty} a_n = \displaystyle\lim_{n \to \infty} \pi \sqrt{n^2+1} - \pi n = 0$,后面会用到.

又由于函数$x \mapsto \sin \, x$在$x=0$处的极限$\displaystyle\lim_{x \to 0} \sin \, x = 0$,所以,对任意$\epsilon > 0$,存在$\delta > 0$,使得对一切$x \in B(\check{0}, \delta)$,都有
\begin{equation}
    |\sin \, x - 0| < \epsilon
\end{equation}

由于数列$\{ \sqrt{n^2+1} \, \pi - \pi n \}$收敛,故,存在$N_1 \in \nat$,使得对每一个$n > N_1$,都有
\begin{equation}
    \big\lvert \pi \sqrt{n^2+1} - \pi n - 0 \big\rvert < \delta
\end{equation}
由此可知,数列$\{ \pi \sqrt{n^2+1}\}$第$N_1$项之后的点全部落入$B(\check{0}, \delta)$内,也就是说
\begin{equation}
\{ \pi \sqrt{n^2+1} - \pi n : n \geq N_1 \} \subset B(\check{0}, \delta)
\end{equation}
由此可知,对每一个$n > N_1$,都有
\begin{equation}
    \bigg\lvert \sin \, \left( \pi \sqrt{n^2+1} - \pi n \right) - 0 \bigg\rvert < \epsilon
\end{equation}
从而数列$\{ \sin \, \left( \pi \sqrt{n^2+1} - \pi n \right) \}$的极限是$0$,从而对任意$\epsilon > 0$,存在$N_2 \in \nat$,使得对每一个$n > N_2$,都有
\begin{align}
    \bigg\lvert \cos \, n \pi \sin \, \left( \pi \sqrt{n^2+1} - \pi n \right) \bigg\rvert &= |\cos \, n \pi| \bigg\lvert \sin \, \left(\pi \sqrt{n^2+1} - \pi n\right) \bigg\rvert \\
    &< \bigg\lvert \sin \, \left(\pi \sqrt{n^2+1} - \pi n \right) \bigg\rvert < \epsilon
\end{align}
从而$\displaystyle\lim_{n \to \infty} \cos \, n \pi \sin \, \left(\pi \sqrt{n^2+1} - \pi n \right) = 0$.从而$\displaystyle\lim_{n \to \infty} \sin \, \left(\pi \sqrt{n^2+1}\right) = 0$.\qed

6. 求下列极限:

\begin{table}[H]
    \centering
    \begin{tabularx}{\textwidth} {  >{\raggedright\arraybackslash}X >{\raggedright\arraybackslash}X  }
       (1)~$\displaystyle\lim_{x \to \infty} \left(\displaystyle\frac{1+x}{3+x}\right)^x$; & (2)~$\displaystyle\lim_{x\to\infty}\left(\displaystyle\frac{x+a}{x-a}\right)^x$; \\[1em]
       (3)~$\displaystyle\lim_{x\to 0}\left(1-2x\right)^{1/x}$; & (4)~$\displaystyle\lim_{n \to \infty}\left(\displaystyle\frac{n+x}{n-2}\right)^n$.
    \end{tabularx}
\end{table}

(1) 解:观察到分式$\displaystyle\frac{1+x}{3+x}$应有极限$1$,这提醒我们将它拆成$1$与一个分式的和:
\begin{align}
    \lim_{x \to \infty} \left(\displaystyle\frac{1+x}{3+x}\right)^x &= \lim_{x \to \infty} \left(1 - \displaystyle\frac{2}{3+x} \right)^x
\end{align}
再令$t = 3+x$,那么原式变为
\begin{align}
    \lim_{x \to \infty} \left(1 - \displaystyle\frac{2}{3+x} \right)^x &= \lim_{t \to \infty} \left(1 - \displaystyle\frac{2}{t}\right)^{t-3} = \displaystyle\lim_{t \to \infty} \left(1-\displaystyle\frac{2}{t}\right)^t \cdot \lim_{t \to \infty} \left(1-\displaystyle\frac{2}{t}\right)^{-3} \\
    &= \displaystyle\frac{1}{\expe^2} \cdot 1 = \frac{1}{\expe^2}.
\end{align}

(2) 解:
\begin{align}
    \text{原式} &= \lim_{x \to \infty} \left(\displaystyle\frac{x+a}{x-a}\right)^x \\
    &= \lim_{x \to \infty} \left(1 + \displaystyle\frac{x+a}{x-a} - 1\right)^x \\
    &= \lim_{x \to \infty} \left(1+\displaystyle\frac{2a}{x-a}\right)^x \\
    &= \expe^{2a}.
\end{align}

(3) 解:令$2x=\displaystyle\frac{1}{t}$,那么当$x\to 0$时应有$t \to \infty$,那么
\begin{align}
    \text{原式} &= \lim_{x \to 0} \left(1-2x\right)^{1/x} \\
    &= \lim_{t \to \infty}\left(1-\displaystyle\frac{1}{t}\right)^{2t} = \expe^{-2}.
\end{align}

(4) 解:
\begin{align}
    \text{原式} &= \lim_{n \to \infty} \left(1 + \displaystyle\frac{n+x}{n-2} - 1\right)^n \\
    &= \lim_{n \to \infty} \left(1 + \displaystyle\frac{x+2}{n-2}\right)^n = \expe^{x+2}.
\end{align}

7. 用极限定义函数
\begin{equation}
    f(x) = \lim_{n \to \infty} n^x \left(\left(1+\frac{1}{n}\right)^{n+1}-\left(1+\frac{1}{n}\right)^n\right).
\end{equation}
求$f$的定义域,并写出$f$的表达式.

解:
\begin{align}
   \text{原式} &= \lim_{n \to \infty} \frac{\left(1+\displaystyle\frac{1}{n}\right)^{n}\left(\displaystyle\frac{1}{n}\right)}{n^{-x}} \\
   &= \lim_{n \to \infty} \left(1+\frac{1}{n}\right)^n \cdot n^{x-1}
\end{align}
式中,$\left(1+\displaystyle\frac{1}{n}\right)^n$有界,而当$x-1>0$时$\{ n^{x-1} \}$发散,故
\begin{equation}
    x-1 \leq 0
\end{equation}
也就是$x \leq 1$.定义域为$\{x : x \leq 1, x \in \real \}$.

当$x=1$,原式变为
\begin{equation}
    \lim_{n \to \infty} \left(1+\frac{1}{n}\right)^n n^0 = \lim_{n\to\infty}\left(1+\frac{1}{n}\right)^n=\expe
\end{equation}
当$x<1$时,有$x-1<0$,故
\begin{equation}
    \lim_{n \to \infty} n^{x-1} = 0
\end{equation}
故
\begin{equation}
    \lim_{n \to \infty} \left(1+\frac{1}{n}\right)^n n^{x-1}= \expe \cdot 0 = 0
\end{equation}
故函数$f$可写为
\begin{equation}
    f(x)=\begin{cases}
        \expe, & x = 1, \\
        0, & x < 1.
    \end{cases}
\end{equation}

8. 如果对$x \in (-1,1)$,有$\bigg\lvert \displaystyle\sum_{k=1}^n a_k \sin \, kx \bigg\rvert \leq \lvert \sin \, x \rvert$,求证:
\begin{equation}
    \bigg\lvert \sum_{k=1}^n k a_k \bigg\rvert \leq 1..
\end{equation}

\begin{proof}
    由题意得
    \begin{align}
        &\mathrel{\phantom{\implies}} \bigg\lvert \sum_{k=1}^n a_k \sin \, kx \bigg\rvert \leq \lvert \sin \, x \rvert \\
        &\implies \displaystyle\frac{\bigg\lvert\displaystyle\sum_{k=1}^n a_k \sin \, kx \bigg\rvert}{\lvert \sin \, x \rvert} \leq 1 \\
        &\implies \Bigg\lvert \displaystyle\frac{\displaystyle\sum_{k=1}^n a_k \sin \, kx}{\sin \, x} \Bigg\rvert \leq 1 \\
        &\implies \Bigg\lvert \sum_{k=1}^n \displaystyle\frac{a_k \sin \, kx}{\sin \, x} \Bigg\rvert \leq 1
    \end{align}

    只需令$x \to 0$,我们得到
    \begin{equation}
        \lim_{x \to 0} \frac{\sin \, kx}{x} = k
    \end{equation}
    并且得到
    \begin{equation}
        \Bigg\lvert \sum_{k=1}^n k a_k \Bigg\rvert \leq 1.
    \end{equation}
    待证得证.
\end{proof}

9. 证明:$\displaystyle\lim_{x \to +\infty} f(x)$存在的充分必要条件是,对任意的$\epsilon > 0$,存在一个正数$A$,只要$x_1, x_2$满足$x_1 > A, x_2 > A$,便有$\lvert f(x_1) - f(x_2) \rvert < \epsilon$.

\begin{proof}
    必要性.

    设$\displaystyle\lim_{x \to +\infty} f(x) = A > 0$.且这个$A$为一有限数.

    那么由于$\displaystyle\lim_{x \to +\infty} f(x) = A$,所以,对任意$\epsilon > 0$,存在$A_1 > 0$,使得只要$x_1 > A_1$,就有
    \begin{equation}
        \lvert f(x_1) - A \rvert < \frac{1}{2} \epsilon
    \end{equation}
    并且存在$A_2 > 0$,使得只要$x_2 > A_2$,就有
    \begin{equation}
        \lvert f(x_2) - A \rvert < \frac{1}{2} \epsilon 
    \end{equation}
    那么只要$x_1, x_2$都大于$\max \{ A_1, A_2 \}$,就有
    \begin{equation}
        \lvert f(x_1) - A \rvert < \frac{1}{2} \epsilon , \quad \lvert f(x_2) - A \rvert < \frac{1}{2} \epsilon 
    \end{equation}
    同时成立,根据数轴上的几何事实,立刻有
    \begin{equation}
        \lvert f(x_1) - f(x_2) \rvert \leq \lvert f(x_1) - A \rvert + \lvert f(x_2) - A \rvert = \frac{1}{2}\epsilon+\frac{1}{2}\epsilon=\epsilon.
    \end{equation}
    这就证得了必要性.

    充分性.

    任取一趋于正无穷的数列$\{ x_n \}$,我们要证$\{ f(x_n) \}$的极限存在且都相等.由题设,对任意$\epsilon$,存在$A>0$,使得对任意$x_1,x_2>A$,都有
    \begin{equation}
        \lvert f(x_1) - f(x_2) \rvert < \epsilon
    \end{equation}
    又由于$\{ x_n \}$是趋于正无穷的,所以,存在$M\in\nat$,使得对一切$n_1, n_2 \geq M$,都有$x_{n_1}, x_{n_2} > A$,也就是
    \begin{equation}
        \lvert f(x_{n_1}) - f(x_{n_2}) \rvert < \epsilon
    \end{equation}
    由此可知数列$\{ f(x_n) \}$是一柯西列,由此可知数列$\{ f(x_n) \}$收敛.由此可知,对于任意一个趋于无穷的数列$\{ x_n \}$,数列$\{ f(x_n) \}$都收敛.

    假设有两个趋于正无穷的数列$\{ x_{n} \}$,$\{ y_{n} \}$,使得$\displaystyle\lim_{n \to +\infty} f(x_n) \neq \displaystyle\lim_{n \to +\infty} f(y_n)$,不妨设$\displaystyle\lim_{n\to +\infty}f(x_n)>\displaystyle\lim_{n \to +\infty}f(y_n)$,那么
    \begin{equation}
        \lim_{n \to +\infty} f(x_n) - f(y_n) > 0
    \end{equation}
    设$\displaystyle\lim_{n\to +\infty} f(x_n) - f(y_n) = c > 0$,那么任取$d \in (0, c)$,都存在$N \in \nat$,使得对一切$n \geq N$,都有
    \begin{equation}
        f(x_n) - f(y_n) > d
    \end{equation}
    可是,由题设,存在$A_2 > 0$,使得对一切$x_n, y_n > A_2$,都有
    \begin{equation}
        f(x_n) - f(y_n) < d
    \end{equation}
    随着$n$的增长,$x_n, y_n$迟早都会大过$A_2$,也就是说迟早都会有$f(x_n)-f(y_n) < d$,矛盾.

    由此可知,对任意趋于无穷的数列$\{ x_n \}$,数列$\{ f(x_n) \}$的极限都存在并且相等.由归结原则,函数极限
    \begin{equation}
        \lim_{x \to +\infty} f(x)
    \end{equation}
    存在.

    这证得了充分性.
\end{proof}

10. 设$f$是$(-\infty,+\infty)$上的周期函数,且$\displaystyle\lim_{x \to +\infty} f(x) = 0$.求证:$f=0$.

\begin{proof}
采用反证法.假设存在某一点$x_0 \in \real$使得$f(x_0) \neq 0$,不失一般性,假设$f(x_0) > 0$.

由于$f$为一周期函数,故存在$c \in \real$且$c \neq 0$,使得对一切$x \in \real$,都有$f(x+c)=f(x)$.构造数列
\begin{equation}
    \{ x_0, x_0+c, x_0+2c, \cdots, x_0+nc, x_0+(n+1)c, \cdots \}
\end{equation}
记做$\{ x_n \}$,由于对每一项$x_n$都有$f(x_n)=c$,所以随着$n \to +\infty$,$f(x_n) \to c$.可是,由归结原则,$\displaystyle\lim_{x \to +\infty} f(x) = 0$等价于说对任意趋于$+\infty$的数列$\{ x_n \}$,数列$\{ f(x_n) \}$都趋于同一个极限也就是都趋于$\displaystyle\lim_{x \to +\infty}f(x)=0$.可如今$\{ f(x_n) \}$的极限是$c$不等于$0$.

你看这矛盾了吧?
\end{proof}