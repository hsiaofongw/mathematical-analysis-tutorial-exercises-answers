\documentclass[a4paper, titlepage]{ctexbook}
\usepackage[
    top=0.8in,
    left=0.8in,
    right=0.8in,
    bottom=1in
]{geometry}
\usepackage{amsmath}
\usepackage{amssymb}
\usepackage{float}
\usepackage{booktabs}
\usepackage{amsthm}
\usepackage{float}
\usepackage{IEEEtrantools}
\usepackage{array}
\usepackage{tabularx}
\usepackage{mdframed}
\usepackage{lipsum}
\usepackage{tikz-cd}
\usepackage{lipsum}
\usepackage{hyperref}
\usepackage{tasks}
\usepackage{siunitx}

\settasks{
    counter-format = (tsk[1]),
    label-offset = 0.5em
}

\hypersetup{
    colorlinks = false,
    bookmarks = true,
    bookmarksopen = false,
    bookmarksnumbered = false,
    pdfborder = {0 0 0}
}

\pagestyle{plain}

\newmdtheoremenv{theo}{定理}

\theoremstyle{definition}
\newtheorem{definition}{定义}

\theoremstyle{definition}
\newtheorem{example}{例}

\theoremstyle{plain}
\newtheorem{prop}{命题}

\theoremstyle{plain}
\newtheorem{theorem}{定理}

\theoremstyle{plain}
\newtheorem{lemma}{引理}

\theoremstyle{definition}
\newtheorem{property}{性质}

\newcommand{\T}{\mathsf{T}}
\newcommand{\nat}{\mathbb{N}^\star}
\newcommand{\pnat}{\mathbb{N}^\star}
\newcommand{\real}{\mathbb{R}}
\newcommand{\preal}{\mathbb{R}^{+}}
\newcommand{\integer}{\mathbb{Z}}
\newcommand{\ulimit}{\mathop{\underset{n \to \infty}{\lim \sup}} \,}
\newcommand{\llimit}{\mathop{\underset{n \to \infty}{\lim \inf}} \,}
\newcommand{\rational}{\mathbb{Q}}
\newcommand{\irrational}{\mathbb{R\setminus Q}}
\newcommand{\nnat}{\mathbb{N}}
\newcommand{\expe}{\mathrm{e}}
\newcommand{\trans}{\mathsf{T}}

\renewcommand{\labelenumi}{(\arabic{enumi})}

% \hspace*{\fill} $\square$

\ctexset{
    chapter/format ={\Large\bfseries\centering},
    chapter/name = {第,章},
    chapter/number = \arabic{chapter},
    appendix/name = {附录,}
}

\newcounter{exercise}[section]

\renewcommand{\theexercise}{\arabic{chapter}.\arabic{exercise}}
\newcommand{\exercise}{\refstepcounter{exercise}{\par\bigskip\centering{\bfseries\large 练习题~\theexercise}\par\bigskip}\addcontentsline{toc}{section}{\textbf{练习题~\theexercise}}}
\newcommand{\question}{{\par\bigskip\centering{\bfseries\large 问题~\theexercise}\par\bigskip}\addcontentsline{toc}{section}{\textbf{问题~\theexercise}}}

\setlength{\parindent}{0ex}

\counterwithin*{equation}{exercise}
\renewcommand{\theequation}{\arabic{chapter}.\arabic{exercise}.\arabic{equation}}

\newenvironment{abstract}{
    \centerline{\zihao{4}\textbf{摘要}\medskip}
    \begin{center}\begin{minipage}[c]{38em}
}{\end{minipage}\end{center}}

\renewenvironment{proof}{\smallskip\textbf{证明.}}{\hfill\qedsymbol\smallskip}

\newcommand{\solve}{\textbf{解.}}
\newcommand{\prove}{\textbf{证明.}}
\newcommand{\answer}{\textbf{答.}}
\newcommand{\annotate}{\textbf{点评.}}

\begin{document}
\section*{练习题1.1}
\noindent 1. 设$a$为有理数,$b$为无理数.求证:$a+b$与$a-b$都是无理数;当$a\neq 0$时,$ab$与$b/a$也是无理数.
\begin{proof}
采用反证法.设$a+b$是有理数,那么就存在互质的整数$p,q$,使得
\begin{equation}
    a+b = \frac{p}{q}
\end{equation}
由于$a$都是有理数,所以存在$p_1,q_1 \in \mathbb{Z}$使得$a = \displaystyle \frac{p_1}{q_1}$,其中$p_1,q_1$互质.于是
\begin{equation}
    b = a+b - a = \frac{p}{q} - \frac{p_1}{q_1} = \frac{p q_1 - q p_1}{q q_1}
\end{equation}
这说明$b$是有理数,矛盾.于是$a+b$是无理数.

\noindent 再证$a-b$是无理数,采用反证法,先假设$a-b$是有理数,即,存在互质的整数$p_2,q_2$使得$\displaystyle a-b=\frac{p_2}{q_2}$,那么
\begin{equation}
    b = a - (a-b) = \frac{p_1}{q_1} - \frac{p_2}{q_2} = \frac{p_1 q_2 - q_1 p_2}{q_1 q_2}
\end{equation}
这说明$b$是有理数,矛盾.于是$a-b$是无理数.
\end{proof}

\noindent 2. 证明:两个不同的有理数之间有无限多个有理数,也有无限多个无理数.
\begin{proof}
设$\displaystyle a_1 = \frac{p_1}{q_1}, b_1 = \frac{p_2}{q_2}$是有理数,且$a_1 \neq b_1$,令
\begin{equation}
    c_1 = \frac{a_1+b_1}{2} = \frac{1}{2} \left( \frac{p_1}{q_1} + \frac{p_2}{q_2} \right) = \frac{p_1 q_2 + p_2 q_1}{2 q_1 q_2}
\end{equation}
则显然$c_1$是无理数,且$a_1 < c_1 < b_1$,对任意的$n\geq 2, n \in \mathbb{N}$,我们令
\begin{equation}
    c_{n-1} = \frac{a_{n-1}+b_{n-1}}{2}, \; a_n = c_{n-1}, \; b_n = b_{n-1}
\end{equation}
由于有理数集关于加法运算封闭,所以按此方法可计算出无穷多个有理数$c_n$满足$a_1 < c_1 < c_2 < \cdots < b$,这就证明了两个不同的有理数之间有无穷多个有理数.

\noindent 设$t_1$是一个无理数,那么我们可以找到两个整数$n_1, m_1$,满足$n_1 < m_1$,并且使得
\begin{equation}
    n_1 < t_1 < m_1
    \label{ieq:n1t1m1}
\end{equation}
在$0$到$b_1 - a_1$之间必定存在$N_1 \in \mathbb{N}^\star$,使得
\begin{equation}
    0 < \frac{m_1}{N_1} - \frac{n_1}{N_1} = \frac{m_1-n_1}{N_1} < b_1 - a_1
\end{equation}
再结合不等式(\ref{ieq:n1t1m1})也就是
\begin{equation}
    a_1 < \frac{t_1}{N_1} - \frac{n_1}{N_1} + a_1 < b_1
\end{equation}
令
\begin{equation}
    s_1 = \frac{t_1}{N_1} - \frac{n_1}{N_1} + a_1
\end{equation}
显然,由于$a_1$是有理数,而$\displaystyle \frac{t_1}{N_1} - \frac{n_1}{N_1}$是无理数,所以$s_1$是无理数并且介于$a_1$到$b_1$之间.对任意$n \in \mathbb{N}^\star$,令
\begin{equation}
    s_{n+1} = \frac{s_n+b_1}{2}, 
\end{equation}
则可构造出无穷多个无理数$s_{n}$,并且满足$a_1 < s_1 < s_2 < \cdots < b_1$,这就证明了任意两个有理数之间有无限多个无理数.
\end{proof}

\noindent 16. 设$n=2,3,\cdots, \, x > -1$且$x\neq 0$.求证:$(1+x)^n > 1+nx$.
\begin{proof}
当$n=2$时,
\begin{equation}
    (1+x)^2 = 1 + 2x + x^2 > 1 + 2x + 0 = 1 + 2x
\end{equation}
于是对于$n=2$命题成立.假设对某个$k \in \mathbb{N}^\star, \, k \geq 2$,当$n=k$时命题成立,则
\begin{align}
    (1+x)^{k+1} &= (1+x)^k (1+x) > (1+kx)(1+x) \\
    &= 1 + (k+1)x + kx^2 > 1+(k+1)x
\end{align}
这说明如果对$n=k$命题成立,那么对$n=k+1$命题也成立.根据数学归纳法原理,命题对一切$n\in\mathbb{N}^\star$都成立.
\end{proof}

\section*{练习题1.2}
\noindent 5. 用精确语言表达``数列$\{a_n\}$不以$a$为极限''这一陈述.

\noindent 答:存在$\epsilon_0 > 0$,使得对任意$N \in \nat$,都存在$n > N$并且$|a_n - a| \geq \epsilon_0$.

\noindent 顺便一提函数极限的否定定义:存在$\epsilon_0 > 0$,使得对任意$\delta > 0$,都存在$|x - x_0| < \delta$并且$|f(x_0) - y_0| \geq \epsilon_0$.

\noindent 7. 设$a,b,c$是三个给定的实数,令$a_0 = a, b_0 = b, c_0 = c$,并归纳地定义
\begin{equation}
    \begin{cases}
        a_n = \displaystyle\frac{b_{n-1}+c_{n-1}}{2} \\
        b_n = \displaystyle\frac{a_{n-1}+c_{n-1}}{2} \\
        c_n = \displaystyle\frac{a_{n-1}+b_{n-1}}{2}
    \end{cases}, \quad n = 1,2,3,\cdots
\end{equation}
求证:
\begin{equation}
    \lim_{n\to\infty} a_n = \lim_{n\to\infty} b_n = \lim_{n\to\infty} c_n = \frac{1}{3} \left( a_0+b_0+c_0 \right).
\end{equation}

\begin{proof}
    对每一个$n \in \nat$,将$r_n,s_n,t_n$分别定义为$a_n,b_n,c_n$中最小的、次小的和最大的数.$a_n,b_n,c_n$收敛到同一个数等价于它们之中的最小数和最大数都收敛到同一个数,也就是说,我们要去证明$r_n,t_n$都收敛到同一个数.

\noindent 由
\begin{align}
    &\mathrel{\phantom{\implies}} s_n \leq t_n \\
    &\implies r_n + s_n \leq r_n + t_n \\
    &\implies \frac{r_n+s_n}{2} \leq \frac{r_n+t_n}{2} 
\end{align}
以及
\begin{align}
    &\mathrel{\phantom{\implies}} r_n \leq s_n \\
    &\implies r_n + t_n \leq s_n + t_n \\
    &\implies \frac{r_n+t_n}{2} \leq \frac{s_n+t_n}{2}
\end{align}
得到
\begin{equation}
    \frac{r_n+s_n}{2} \leq \frac{r_n+t_n}{2} \leq \frac{s_n+t_n}{2}
\end{equation}
因此
\begin{equation}
    r_{n+1} = \frac{r_n+s_n}{2}, \; s_{n+1} = \frac{r_n+t_n}{2}, \; t_{n+1}=\frac{s_n+t_n}{2}
\end{equation}
于是
\begin{equation}
    t_{n+1} - r_{n+1} = \frac{s_n+t_n}{2} - \frac{r_n+s_n}{2} = \frac{t_n - r_n}{2}
\end{equation}
\noindent 并且由此得
\begin{equation}
t_n - r_n = \frac{t_{n-1}-r_{n-1}}{2} = \frac{t_{n-2}-r_{n-2}}{4} = \cdots = \frac{1}{2^{n}}\left(t_0 - s_0\right)
\end{equation}
于是对任意$\epsilon > 0$,我们可以取$N = \lceil \log_2 \left(t_0-s_0\right) - \log_2 \epsilon \rceil$,于是当$n > N$的时候就有
\begin{equation}
    |t_n-r_n|=\frac{1}{2^n}(t_0-s_0) < \frac{1}{2^N}(t_0-s_0) \leq \epsilon
\end{equation}
为了把$r_n$也就是$t_n$的共同极限找出来,我们要寻找一个始终位于$r_n$和$t_n$中间的常数,为此,我们要去证明$\displaystyle\frac{r_0+s_0+t_0}{3}$就是这个数.首先,根据
\begin{equation}
    r_{n+1} = \frac{r_n+s_n}{2}, \; s_{n+1} = \frac{r_n+t_n}{2}, \; t_{n+1}=\frac{s_n+t_n}{2} \; (\forall n \in \integer)
\end{equation}
我们得
\begin{equation}
    r_{n+1}+s_{n+1}+t_{n+1} = \frac{2r_n+2s_n+2t_n}{2} = r_n + s_n + t_n \; (\forall n \in \integer)
\end{equation}
这说明
\begin{equation}
    r_0+s_0+t_0=r_n+s_n+t_n, \; (\forall n \in \integer)
\end{equation}
于是放缩得
\begin{equation}
    r_n \leq \frac{r_0+s_0+t_0}{3} \leq t_n, \; (\forall n \in \integer)
\end{equation}
这说明,在数轴上,不管$n$取哪一个整数,$\displaystyle\frac{r_0+s_0+t_0}{3}$这个数,总是插在$r_n$与$t_n$之间,又根据$r_n,s_n,t_n$的定义,有$r_n \leq \min \{a_n,b_n,c_n\}$以及$t_n \geq \max \{a_n, b_n, c_n \}$,于是,依几何事实,有
\begin{align}
    |a_n - \frac{r_0+s_0+t_0}{3}| < |r_n - t_n| < \epsilon, \\
    |b_n - \frac{r_0+s_0+t_0}{3}| < |r_n - t_n| < \epsilon, \\
    |c_n - \frac{r_0+s_0+t_0}{3}| < |r_n - t_n| < \epsilon
\end{align}
又由于$\displaystyle\frac{r_0+s_0+t_0}{3}=\frac{a_0+b_0+c_0}{3}$,所以
\begin{equation}
    \lim_{n\to\infty} a_n = \lim_{n\to\infty} b_n = \lim_{n\to\infty} c_n = \frac{a_0+b_0+c_0}{3}
\end{equation}
于是命题得证.
\end{proof}


\section*{练习题1.5}
\noindent 5. 令$\displaystyle a_n = (n!)^\frac{1}{n}$,求证:$\{ a_n \}$是一个递增数列.
\begin{proof}
猜测$\{ a_n \}$是递增的,为此,我们尝试去证$\displaystyle \frac{a_{n+1}}{a_n} > 1$,由题设:
\begin{equation}
    \frac{a_{n+1}}{a_n} = \frac{(1 \times 2 \times \cdots \times n+1)^\frac{1}{n+1}}{(1 \times 2 \times \cdots \times n)^\frac{1}{n}}
    \label{eq:a1}
\end{equation}
对式(\ref{eq:a1})左右两端同时自乘$n+1$次,得:
\begin{equation}
    \left( \frac{a_{n+1}}{a_n} \right)^{n+1} = \frac{1 \times 2 \times \cdots \times n+1}{1 \times 2 \times \cdots \times n \times (1 \times 2 \times \cdots \times n)^\frac{1}{n}} = \frac{n+1}{(1 \times 2 \times \cdots \times n)^\frac{1}{n}}
    \label{eq:a2}
\end{equation}
对式(\ref{eq:a2})左右两端同时自乘$n$次,得:
\begin{equation}
    \left(\left(\frac{a_{n+1}}{a_n} \right)^{n+1}\right)^{n} = \frac{(n+1)^n}{1 \times 2 \times \cdots \times n} = \frac{\prod_{i=1}^{n} \left( n+1 \right) }{\prod_{i=1}^{n} i} = \prod_{i=1}^n \frac{n+1}{i} > \prod_{i=1}^{n} 1 = 1
\end{equation}
也就是
\begin{equation}
   \left( \frac{a_{n+1}}{a_n} \right)^{n (n+1)} > 1
\end{equation}
从而$\displaystyle \frac{a_{n+a}}{a_n} >1$,又根据$a_n > 0$,于是数列$\{ a_n \}$是递增的.
\end{proof}

\section*{练习题1.6}
\noindent 1. 求下列极限
\begin{table}[H]
    \centering
    \begin{tabularx}{0.8\textwidth} {  >{\raggedright\arraybackslash}X >{\raggedright\arraybackslash}X  }
       (1)$\displaystyle \lim_{n\to\infty} \left(1+\frac{1}{n-2}\right)^n$; &  (2)$\displaystyle \lim_{n \to \infty} \left(1-\frac{1}{n+3}\right)^n$; \\ [1.5em]
       (3)$\displaystyle \lim_{n \to \infty} \left(\frac{1+n}{2+n}\right)^n$; & (4) $\displaystyle \lim_{n \to \infty} \left(1+\frac{3}{n}\right)^n$; \\ [1.5em]
       (5)$\displaystyle \lim_{n \to \infty} \left(1+\frac{1}{2n^2}\right)^{4n^2}$.
      \end{tabularx}
\end{table}
\noindent (1). 解:令$m = n-2$,则
\begin{align}
    \lim_{n \to \infty} \left(1+\frac{1}{n-2}\right)^n &= \lim_{m \to \infty} \left(1+ \frac{1}{m}\right)^{m+2} \\
    &= \lim_{m \to \infty} \left(1+\frac{1}{m}\right)^m \left(1+\frac{1}{m}\right)^2 \\
    &= \lim_{m \to \infty} \left(1+\frac{1}{m}\right)^m \lim_{m \to \infty} \left(1+\frac{1}{m}\right)^2 \\
    &= \mathrm{e} \cdot 1 \\
    &= \mathrm{e}
\end{align}
\noindent (2). 解:令$m=n+3$,则
\begin{align}
    \lim_{n \to \infty} \left(1-\frac{1}{n+3}\right)^n &= \lim_{m \to \infty} \left(1-\frac{1}{m}\right)^{m+2} \\
    &= \lim_{m \to \infty} \left(1-\frac{1}{m}\right)^m \lim_{m \to \infty} \left(1-\frac{1}{m}\right)^2 \\
    &= \lim_{m \to \infty} \left(1-\frac{1}{m}\right)^m \\
    &= \mathrm{e}^{-1}
\end{align}
\noindent (3). 解:
\begin{align}
    \lim_{n \to \infty} \left(\frac{1+n}{2+n}\right)^n &= \lim_{n \to \infty} \left(1 - \frac{1}{2+n}\right)^n
\end{align}
令$m = 2+n$,那么
\begin{align}
    \text{原式} &= \lim_{m \to \infty} \left(1-\frac{1}{m}\right)^{m-2} \\
    &= \lim_{m\to\infty} \frac{\left(1-\frac{1}{m}\right)^m}{\left(1-\frac{1}{m}\right)^2} \\
    &= \frac{\displaystyle\lim_{n \to \infty}\left(1-\frac{1}{m}\right)^m}{\displaystyle\lim_{m\to\infty} \left(1-\frac{1}{m}\right)^2} \\
    &= \frac{\mathrm{e}^{-1}}{1} \\
    &= \mathrm{e}^{-1}
\end{align}
(4). 解:令$m=3n$,则
\begin{align}
    \lim_{n\to\infty}\left(1+\frac{3}{n}\right)^n &= \lim_{m\to\infty}\left(1+\frac{3}{3m}\right)^{3m} \\
    &= \lim_{m\to\infty}\left(1+\frac{1}{m}\right)^{3m} \\
    &= \lim_{m\to\infty}\left(1+\frac{1}{m}\right)^{m}\left(1+\frac{1}{m}\right)^{m}\left(1+\frac{1}{m}\right)^{m} \\
    &= \lim_{m\to\infty}\left(1+\frac{1}{m}\right)^{m} \lim_{m\to\infty}\left(1+\frac{1}{m}\right)^{m}\lim_{m\to\infty}\left(1+\frac{1}{m}\right)^{m} \\
    &= \mathrm{e}\cdot\mathrm{e}\cdot\mathrm{e}\\
    &=\mathrm{e}^3
\end{align}
(5). 解:令$m=2n^2$,则
\begin{align}
    \lim_{n\to\infty} \left(1+\frac{1}{2n^2}\right)^{4n^2} &= \lim_{m\to\infty}\left(1+\frac{1}{m}\right)^{2m} \\
    &= \lim_{m\to\infty}\left(1+\frac{1}{m}\right)^{m} \left(1+\frac{1}{m}\right)^{m} \\
    &= \lim_{m\to\infty}\left(1+\frac{1}{m}\right)^{m} \lim_{m\to\infty}\left(1+\frac{1}{m}\right)^{m} \\
    &=\mathrm{e} \cdot \mathrm{e} \\
    &=\mathrm{e}^2
\end{align}

\noindent 2. 设$k \in \mathbb{N}^\star$,求证:$\displaystyle \lim_{n \to \infty} (1+\frac{k}{n})^n = e^k$.
\begin{proof}
采用数学归纳法.当$k=1$时:
\begin{equation}
    \lim_{n\to \infty}(1+\frac{k}{n})^n = \lim_{n \to \infty}(1+\frac{1}{n})^n = e = e^1 = e^k
\end{equation}
因此当$k=1$时命题成立.现在假设当$k = m-1, \, (m-1\in \mathbb{N}^\star)$时命题成立.那么有
\begin{align}
    &\phantom{=} \lim_{n \to \infty} (1+\frac{m}{n})^n \\
    (\text{令} \; n = m t) \quad & = \lim_{m \to \infty} (1 + \frac{1}{t})^{m t} \\
    &= \lim_{t \to \infty} (1+\frac{1}{t})^{(m-1) t} (1+\frac{1}{t})^t \\
    &= \left( \lim_{t \to \infty} (1+\frac{1}{t})^{(m-1) t} \right) \cdot \left( \lim_{t \to \infty} (1+\frac{1}{t})^t \right)\\ 
    (\text{令} \; t = \frac{s}{m-1}) \quad &= \left( \lim_{s \to \infty} (1 + \frac{m-1}{s})^s \right) \cdot e \\
    (\text{由归纳假设} ) \quad &= \mathrm{e}^{m-1} \cdot \mathrm{e} \\
    &= \mathrm{e}^m
\end{align}
这里证明了对于$k = m$命题成立.从而根据数学归纳法原理,对任意$k \in \mathbb{N}^\star$命题都成立.
\end{proof}

\noindent 3. 求证:$\displaystyle \{ (1+\frac{1}{n})^n \}$是严格递增数列.
\begin{proof}
由算式---几何平均不等式:
\begin{equation}
(1+\frac{1}{n})^n = (\frac{n+1}{n})^n = 1 \cdot \prod_{i=1}^{n} \frac{n+1}{n} < \left( \frac{1 + n(\frac{n+1}{n})}{n+1} \right)^{n+1} = \left( 1 + \frac{1}{n+1} \right)^{n+1}
\end{equation}
从而$\displaystyle \{ (1+\frac{1}{n})^n\}$严格递增.
\end{proof}
\noindent 4. 求证:$\displaystyle \{ (1+\frac{1}{n})^{n+1}\}$是严格递减数列.
\begin{proof}
令$\displaystyle q_n = (1 + \frac{1}{n})^{n+1}$,原命题等价于$\{ \displaystyle \frac{1}{q_n}\}$是严格递增数列.由算式---几何平均不等式
\begin{equation}
    \frac{1}{q_n} = \left( \frac{n}{n+1} \right)^{n+1} = 1 \cdot \prod_{i = 1}^{n+1} \frac{n}{n+1} < \left( \frac{1 + (n+1) \frac{n}{n+1}}{n+2} \right)^{n+2} = \left( \frac{1+n}{n+2}\right)^{n+2} = \frac{1}{q_{n+1}}
\end{equation}
从而数列$\{ \displaystyle \frac{1}{q_n} \}$严格单调递增,因此数列$\{ q_n \}$也就是$\{ \displaystyle (1 + \frac{1}{n})^{n+1}\}$严格单调递减.
\end{proof}
\noindent 5. 证明不等式:
\begin{equation}
    \left( 1+\frac{1}{n} \right)^n < \mathrm{e} < \left( 1 + \frac{1}{n} \right)^{n+1}.
\end{equation}
\begin{proof}
令$\displaystyle p_n = \left( 1+\frac{1}{n} \right)^n, \, q_n = \left( 1+\frac{1}{n} \right)^{n+1}$.先证明不等式的左半部分.

\noindent 采用数学归纳法,当$n=1$时:
\begin{equation}
    p_1 = (1+\frac{1}{1})^1 = 2 < e
\end{equation}
显然成立.假设当$n=m-1, \, (m-1 \in \mathbb{N}^\star)$时命题成立,我们去证明当$p_{m-1} < \mathrm{e}$时有$p_m < \mathrm{e}$,为此,采用反证法,假设对某个$m$,当$p_{m-1} < \mathrm{e}$时有$p_m > \mathrm{e}$,记$\epsilon_0 = \min \{ |p_{m-1} - \mathrm{e}|, |p_m - \mathrm{e}| \}$,显然$\epsilon_0 > 0$,由数列$\{ p_n \}$的严格单调性,对任意$n \in \mathbb{N}^\star$都有$|p_n - \mathrm{e}| \geq \epsilon_0 > 0$,但是由于数列$\{ p_n \}$的极限是$\mathrm{e}$,所以由极限的定义,对任意$\epsilon > 0$,存在$N_0 \in \mathbb{N}^\star$,使得当$n > N_0$时有$|p_n - e| < \epsilon$,由此推出矛盾,故对于$p_m > e$的假设是错误的.又根据数列$\{ p_n \}$的严格单调性,$p_m \neq e$,因此只能是$p_m < e$. 从而当$p_{m-1} < e$成立时$p_m < e$也成立,根据数学归纳法原理,对任意$n \in \mathbb{N}^\star$,$p_n < e$都成立.

\noindent 利用数列$\{ q_n \}$的严格单调性质以及$\{ q_n \}$的极限也是$\mathrm{e}$的事实可类似地证明$e < q_n, \, (n \in \mathbb{N}^\star)$.
\end{proof}
\noindent 6.利用对数函数$\ln \, x$的严格递增性质,证明:
\begin{equation}
    \frac{1}{n+1} < \ln \, \left( 1+\frac{1}{n} \right) < \frac{1}{n}
\end{equation}
对一切$n \in \mathbb{N}^\star$成立.
\begin{proof}
先证不等式的左半部分.要证$\displaystyle \frac{1}{n+1} < \ln \, \left( 1+\frac{1}{n} \right)$等价于去证$\displaystyle 1 < (n+1) \ln \, \left( 1+\frac{1}{n} \right)$,利用对数函数$\ln \, x$的严格单调递增性质,也就是等价于$\ln \, \mathrm{e} < \ln \, \left(\left( 1 + \frac{1}{n} \right)^{n+1} \right)$,由题5得到的结论:$\displaystyle \mathrm{e} < \left( 1+\frac{1}{n} \right)^{n+1}$(对任意$n \in \mathbb{N}^\star$),以及对数函数$\ln \, x$的严格单调性质,可知$\ln \, \mathrm{e} < \ln \, \left( \left( 1 + \frac{1}{n} \right)^{n+1} \right)$成立,因此$\displaystyle 1 < (n+1) \ln \, \left( 1+\frac{1}{n} \right)$成立,因此$\displaystyle \frac{1}{n+1} < \ln \, \left( 1 + \frac{1}{n} \right)$成立,即,要证的不等式的左半部分成立.

\noindent 再用类似的方法来证明不等式的右半部分.由题5得到的结论:$\displaystyle \left( 1 + \frac{1}{n} \right)^n < \mathrm{e}$(对任意$n \in \mathbb{N}^\star$),对这个不等式两端取对数并且利用对数函数$\ln \, x$的单调递增性质,得到$\displaystyle n \ln \, \left( 1 + \frac{1}{n} \right) < \ln \, \mathrm{e} = 1$,再对这个不等式两端除以$n$即得$\displaystyle \ln \, \left( 1 + \frac{1}{n}\right) < \frac{1}{n}$.于是命题得证.
\end{proof}
\noindent 7. 设$n \in \mathbb{N}^\star$且$k = 1,2,\cdots$. 证明不等式:
\begin{equation}
    \frac{k}{n+k} < \ln \, \left( 1+\frac{k}{n} \right) < \frac{k}{n}
\end{equation}
\noindent 思路:回顾题4、题5以及题6的证明过程,我们受到启发:要证$\displaystyle \frac{k}{n+k} < \ln \, \left( 1 + \frac{k}{n}\right)$成立就要去证$\displaystyle \ln \, \left( \mathrm{e}^k \right) < (n+k) \ln \, \left( 1 + \frac{k}{n}\right)$成立,相应地就要去证$\displaystyle \mathrm{e}^k < \left(1+\frac{k}{n}\right)^{n+k}$并且要证数列$\displaystyle \{ \left(1+\frac{k}{n}\right)^{n+k} \}$严格单调递减,而对于数列$\displaystyle \{ \left( 1 + \frac{k}{n} \right)^{n+k}\}$的单调递减性的证明可以参考和借鉴题4的方法.
\begin{proof}
先证明数列$\displaystyle \{ \left(1+\frac{k}{n}\right)^{n+k}\}$是严格单调递减的.令$\displaystyle p_n =  \left(1+\frac{k}{n}\right)^{n+k}$,这等价于证明数列$\displaystyle \{ \frac{1}{p_n} \}$是严格单调递减的.利用算式---几何均值不等式,我们有:
\begin{equation}
    \frac{1}{p_n} =  \left(\frac{n}{n+k}\right)^{n+k} = 1 \cdot \prod_{i=1}^{n+k} \left(\frac{n}{n+k}\right) <(\frac{1+(n+k) \frac{n}{n+k}}{n+k+1})^{n+k+1} = \frac{1}{p_{n+1}}
\end{equation}
这就证明了数列$\displaystyle \{ \frac{1}{p_n}\}$是严格单调递减的,从而数列$\{p_n \}$也就是$\{ \displaystyle \left(1+\frac{k}{n}\right)^{n+k}\}$是严格单调递增的.又根据
\begin{equation}
    \lim_{n \to \infty} \left(1+\frac{k}{n}\right)^{n+k} = \mathrm{e}^k
\end{equation}
以及数列$\{ \displaystyle \left( 1 + \frac{k}{n} \right)^{n+k}\}$的严格单调递减性,以及$\mathrm{e}^k < p_1 = \displaystyle \left(1+\frac{k}{n}\right)^{n+k}$的事实,可证明
\begin{equation}
    \mathrm{e}^k < \left( 1 + \frac{k}{n}\right)^{n+k}
    \label{eq:ek}
\end{equation}
对每一个$k, n\in \mathrm{N}^{\star}$都成立.对式(\ref{eq:ek})两边取对数,再根据对数函数$\ln \, x$的严格单调递增性,得到
\begin{equation}
    k < (n+k) \ln \, \left(1+\frac{k}{n}\right)
\end{equation}
也就是
\begin{equation}
    \frac{k}{n+k} < \ln \, \left(1+\frac{k}{n}\right)
\end{equation}
从而要证明的不等式的左半部分成立.采用类似的方法可类似地证明不等式的右半部分成立.
\end{proof}
\noindent 8. 对$n \in \mathbb{N}^\star$,求证:
\begin{equation}
    \frac{1}{2} + \frac{1}{3} + \cdots + \frac{1}{n} < \ln \, \left( n+1 \right) < 1 + \frac{1}{2} + \cdots + \frac{1}{n}.
\end{equation}
\begin{proof}
利用题6的结论:$\displaystyle \frac{1}{n+1} < \ln \, \left( 1 + \frac{1}{n} \right) = \ln \, (n+1) - \ln \, n$,我们有
\begin{equation}
\sum_{i = 1}^{n} \frac{1}{i + 1} < \sum_{i=1}^n \ln \, \left( i+1 \right) - \ln \, i = \sum_{i=1}^n \ln \, \left( i + 1\right) - \sum_{i=1}^n \ln \, i = \sum_{i=1}^{n+1} \ln \, i - \sum_{i=1}^{n} \ln \, i = \ln \, \left( n+1 \right)
\end{equation}
于是左半部分得证.再利用题6的结论:$\displaystyle \ln \, \left( n+1 \right) - \ln \, n = \ln \, \left( 1 + \frac{1}{n} \right) < \frac{1}{n}$,可以得到
\begin{equation}
\sum_{i=1}^n \ln \, \left( i+1 \right) - \ln \, i = \sum_{i=1}^n \ln \, \left( i+1 \right) - \sum_{i=1}^n \ln \, i = \sum_{i=1}^{n+1} \ln \, i - \sum_{i=1}^n \ln \, i = \ln \, \left(n+1 \right) < \sum_{i=1}^n \frac{1}{i}
\end{equation}
于是右半部分得证.
\end{proof}
\noindent 9. 令
\begin{equation}
    x_n = 1 + \frac{1}{2} + \cdots + \frac{1}{n} - \ln \, \left( n+1 \right) \quad (n \in \mathbb{N}^\star).
\end{equation}
证明:$\displaystyle \lim_{n \to \infty} x_n $存在,此极限记为$\gamma$,叫做Euler(欧拉,1707\textasciitilde 1783)常数.
\begin{proof}
先证单调性.
\begin{align}
    x_{n+1} &= 1 + \frac{1}{2} + \cdots + \frac{1}{n} + \frac{1}{n+1} - \ln \, \left(n+2\right) \\
    &< 1 + \frac{1}{2} + \cdots + \frac{1}{n} + \ln \, \left( 1 + \frac{1}{n} \right) - \ln \, \left( n+2\right) \\
    &= 1 + \frac{1}{2} + \cdots + \frac{1}{n} + \ln \, \left( n+1 \right) - \ln \, n - \ln \, \left( n+2 \right) \\
    &< 1 + \frac{1}{2} + \cdots + \frac{1}{n} + \ln \, \left( n+1 \right) \\
    &= x_n \quad (n \in \mathbb{N}^\star)
\end{align}
再证有界性.利用题8的结论,有
\begin{equation}
    0 < 1 + \frac{1}{2} + \cdots + \frac{1}{n} - \ln \left( n + 1\right) 
\end{equation}
从而数列$\{ x_n \}$既单调又有界,所以数列$\{ x_n \}$的极限存在.
\end{proof}
\noindent 10. 利用第9题,证明:
\begin{equation}
    1+\frac{1}{2} + \frac{1}{3} + \cdots + \frac{1}{n} = \ln \, n + \gamma + \epsilon_n,
\end{equation}
其中$\displaystyle \lim_{n \to \infty} \epsilon_n = 0$.
\begin{proof}
令$\displaystyle p_n = 1 + \frac{1}{2} + \cdots + \frac{1}{n} - \ln \, n$,我们先来证明数列$\{ p_n \}$的极限存在.先证单调性:
\begin{align}
    p_{n+1} &= 1 + \frac{1}{2} + \cdots + \frac{1}{n} + \frac{1}{n+1} - \ln \, \left( n + 1 \right) \\
    &< 1 + \frac{1}{2} + \cdots + \frac{1}{n} + \ln \, \left( 1 + \frac{1}{n} \right) - \ln \, \left(n+1\right) \\
    &= 1 + \frac{1}{2} + \cdots + \frac{1}{n} - \ln \, n \\
    &= p_n \quad (n \in \mathbb{N}^\star)
\end{align}
再证有界性.利用题8的结论:$\displaystyle \ln \, n < \ln \, \left( n+1 \right) < 1 + \frac{1}{2} + \cdots +\frac{1}{n}$,于是
\begin{equation}
    p_n = 1 + \frac{1}{2} +\cdots + \frac{1}{n} - \ln \, n > 0
\end{equation}
从而数列$\{ p_n \}$有界,因此数列$\{ p_n \}$的极限存在,即$ \displaystyle \lim_{n \to \infty} p_n$存在.令$\displaystyle x_n = 1 + \frac{1}{2} + \cdots + \frac{1}{n} - \ln \, \left( n + 1\right)$,我们来证$\displaystyle \lim_{n \to \infty} p_n$的确等于$\displaystyle \lim_{n \to \infty} x_n$:
\begin{align}
    \lim_{n \to \infty} p_n &= \lim_{n \to \infty} \left( x_n + \ln \, \left( n+1 \right) - \ln \, n \right) \\
    &= \lim_{n \to \infty} x_n + \lim_{n \to \infty} \left( \ln \, \left(n+1\right) - \ln \, n \right) \\
    &= \lim_{n \to \infty} x_n + 0 \\
    &= \lim_{n \to \infty} x_n
\end{align}
其中$\displaystyle \lim_{n \to \infty} \left( \ln \, \left(n+1\right) - \ln \, n\right) = 0$的条件是从题6的结论:
\begin{equation}
\frac{1}{n+1} < \ln \, \left(1 + \frac{1}{n}\right) = \ln \, \left(n+1\right) - \ln \, n < \frac{1}{n}
\end{equation}
得到的.令$\displaystyle \epsilon_n = 1 + \frac{1}{2} +\cdots + \frac{1}{n} - \ln \, n - \gamma$,令$\displaystyle \gamma = \lim_{n \to \infty} x_n$,我们要证$\displaystyle \lim_{n \to \infty} \epsilon_n = 0$:
\begin{align}
    \lim_{n \to \infty} \epsilon_n &= \lim_{n \to \infty} \left(1 + \frac{1}{2} + \cdots + \frac{1}{n} - \ln \, n - \gamma \right) \\
    &= \lim_{n \to \infty} \left( p_n - \gamma \right) \\
    &= \lim_{n\to \infty} p_n - \lim_{n\to \infty} \gamma \\
    &= \left( \lim_{n\to \infty} p_n \right) - \gamma \\
    &= \lim_{n\to \infty} p_n - \lim_{n \to \infty} x_n \\ 
    &= 0
\end{align}
命题得证.
\end{proof}
\noindent 11. 证明不等式:
\begin{equation}
    \left(\frac{n+1}{\mathrm{e}}\right)^n < n! < \mathrm{e} \left(\frac{n+1}{\mathrm{e}}\right)^{n+1}
\end{equation}
\begin{proof}
先证不等式左半部分.先将要证的不等式改写成便于证明的等价形式:
\begin{align}
    \left( \frac{n+1}{\mathrm{e}}\right)^n < n! &\iff n \left(\ln \, \left(n+1\right) - 1\right) < \sum_{i=1}^n \ln \, i \\
    &\iff \sum_{i=1}^n \left(\ln \, \left(n+1\right) - \ln \, i\right) < n \label{ieq:1}
\end{align}
我们现采用数学归纳法证明不等式(\ref{ieq:1})成立,显然当$n=1$时不等式(\ref{ieq:1})成立,假设当$n = m-1$时($m-1 \in \mathbb{N}^\star$)不等式(\ref{ieq:1})成立.那么当$n=m$时:
\begin{align}
    \sum_{i=1}^m \left(\ln\,\left(m+1\right) - \ln \, i\right) &= \sum_{i=1}^{m-1} \left(\ln \, m - \ln \, i\right) + \sum_{i=1}^m \left(\ln\,\left(m+1\right) - \ln \, i\right) - \sum_{i=1}^{m-1} \left(\ln \, m - \ln \, i\right) \\
    &= \left(\sum_{i=1}^{m-1} \ln \, m - \ln \, i \right) + m \left( \ln \, \left(m+1\right) - \ln \, m \right) \\
    &< m-1 + m \cdot \frac{1}{m} \\
    &= 1
\end{align}
由此可见当$n=m$时不等式(\ref{ieq:1})也成立,根据数学归纳法原理,不等式(\ref{ieq:1})对每一个$n\in\mathbb{N}^\star$都成立,从而不等式(\ref{ieq:1})的等价形式
\begin{equation}
\left(\frac{n+1}{\mathrm{e}}\right)^n < n!
\end{equation}
对每一个$n\in\mathbb{N}^\star$都成立.

\noindent 现在再去证明不等式右半部分,先将它改写成便于证明的等价形式:
\begin{align}
    n! < \mathrm{e}\left(\frac{n+1}{\mathrm{e}}\right)^{n+1} &\iff \sum_{i=1}^n \ln \, i < (n+1)\ln \, \left(n+1\right) - n \\
    &\iff n < \left( \sum_{i=1}^n \ln \, \left(n+1\right) - \ln \, i \right) + \ln \, \left(n+1\right)
\end{align}
令
\begin{equation}
    p_n = \left( \sum_{i=1}^n \ln \, \left(n+1\right) - \ln \, i \right) + \ln \, \left(n+1\right)
\end{equation}
我们现在采用数学归纳法证明$n < p_n$对每一个$n\in\mathbb{N}^\star$成立.当$n=1$时:
\begin{align}
    p_n = p_1 = 2 \ln \, 2 = 2 \left( \ln \, 2 - \ln \, 1\right) = 2 \ln \, \left( 1 + \frac{1}{1} \right) > 2 \cdot \frac{1}{1+1} = 1 = n
\end{align}
从而当$n=1$时$n < p_n$成立,现在假设:当$n=m-1, \, (m-1 \in \mathbb{N}^\star)$时,$n < p_n$成立.于是
\begin{align}
    p_m &= p_{m-1} + p_{m} - p_{m-1} \\
    &= p_{m-1} + (m+1) \left( \ln \, \left(m+1\right) - \ln \, m \right) \\
    &= p_{m-1} + (m+1) \ln \, \left(1 + \frac{1}{m}\right) \\
    &> m-1 + (m+1) \frac{1}{m+1} \\
    &= m
\end{align}
从而对于$n=m$,$n < p_n$也成立,根据数学归纳法原理,$n < p_n$对每一个$n \in \mathbb{N}^\star$都成立,也就是
\begin{equation}
    n < \left( \sum_{i=1}^n \ln \, \left(n+1\right) - \ln \, i \right) + \ln \, \left(n+1\right)
\end{equation}
以及它的等价形式
\begin{equation}
    n! < \mathrm{e}\left(\frac{n+1}{\mathrm{e}}\right)^{n+1}
\end{equation}
对每一个$n \in \mathbb{N}^\star$都成立.
\end{proof}
\noindent 12. 证明:利用题11的结论:
\begin{align}
    &\phantom{\implies} \left(\frac{n+1}{\mathrm{e}}\right)^n < n! < \left(\frac{n+1}{\mathrm{e}}\right)^{n+1} \\
    &\implies \frac{1}{\mathrm{e}^n} < \frac{n!}{(n+1)^n} < \frac{n+1}{\mathrm{e}^n} \\
    &\implies \frac{1}{\mathrm{e}} < \frac{(n!)^\frac{1}{n}}{n+1} < \frac{(n+1)^\frac{1}{n}}{\mathrm{e}} \\
    &\implies \lim_{n\to\infty} \frac{1}{\mathrm{e}} < \lim_{n \to \infty} \frac{(n!)^\frac{1}{n}}{n+1} < \lim_{n\to\infty} \frac{(n+1)^\frac{1}{n}}{\mathrm{e}} \\
    &\implies \lim_{n\to\infty} \frac{(n!)^\frac{1}{n}}{n+1} = \frac{1}{\mathrm{e}}
\end{align}
再次利用题11的结论:
\begin{align}
    &\phantom{\implies} \left(\frac{n+1}{\mathrm{e}}\right)^n < n! < \left(\frac{n+1}{\mathrm{e}}\right)^{n+1} \\
    &\implies \frac{1}{\mathrm{e}^n} < \frac{n!}{(n+1)^n} < \frac{n+1}{\mathrm{e}^n} \\
    &\implies \frac{1}{\mathrm{e}} < \frac{(n!)^\frac{1}{n}}{n+1} < \frac{(n+1)^\frac{1}{n}}{\mathrm{e}} \\
    &\implies \frac{1}{n \mathrm{e}} < \frac{(n!)^\frac{1}{n}}{n(n+1)} < \frac{(n+1)^\frac{1}{n}}{\mathrm{e}} \\
    &\implies \lim_{n\to\infty} \frac{1}{n \mathrm{e}} = \lim_{n\to\infty} \frac{(n!)^\frac{1}{n}}{n(n+1)} = \lim_{n\to\infty} \frac{(n+1)^\frac{1}{n}}{\mathrm{e}} = 0 \\
    &\implies \lim_{n\to\infty} \left(\frac{(n!)^\frac{1}{n}}{n} - \frac{(n!)^\frac{1}{n}}{n+1}\right) = 0 \\
    &\implies \lim_{n\to\infty} \frac{(n!)^\frac{1}{n}}{n} = \lim_{n\to\infty} \frac{(n!)^\frac{1}{n}}{n+1} + \lim_{n\to\infty} \left(\frac{(n!)^\frac{1}{n}}{n} - \frac{(n!)^\frac{1}{n}}{n+1}\right) = \frac{1}{\mathrm{e}} + 0 = \frac{1}{\mathrm{e}}
\end{align}
这就证明了数列$\displaystyle \{\frac{(n!)^\frac{1}{n}}{n}\}$的极限是$\displaystyle \frac{1}{\mathrm{e}}$.

\section*{练习题1.7}
\noindent 1. 对任意给定的$\epsilon > 0$,存在$N \in \mathbb{N}^\star$,当$n > N$时,有
\begin{equation}
    |a_n - a_N|<\epsilon
\end{equation}
问$\{a_n\}$是不是基本列?

\noindent 思路:在实数轴上,所有$a_N$附件的项都落在以它为中心的半径不超过$\epsilon$的一个范围内,那么,假设$a_n, a_m$也是在这个范围内,$a_n$与$a_m$的距离只会更加近,因为这个$\epsilon$可以任意小,从而这个以$a_N$为中心的范围可以是任意地``拥挤''.由此我们猜测$\{a_n\}$是基本列,去证$|a_m - a_n|<\epsilon$.

\begin{proof}
设$m, n>N$,那么由题设,
\begin{equation}
    |a_m - a_N|<\epsilon, \; |a_n - a_N| <\epsilon
\end{equation}
也就是
\begin{align}
    a_N - \epsilon < a_m < a_N + \epsilon \\
    a_N - \epsilon < a_n < a_N + \epsilon 
\end{align}
从而
\begin{equation}
    |a_m - a_n| < 2\epsilon
\end{equation}
由于$2\epsilon$可以是任意的正数,所以按定义,$\{a_n\}$是基本列.
\end{proof}

\noindent 2. (1)数列$\{a_n\}$满足
\begin{equation}
    |a_{n+p}-a_n|\leq\frac{p}{n}
\end{equation}
且对一切$n,p\in\mathbb{N}^\star$成立,问$\{ a_n\}$是不是基本列?

\noindent (2) 当$|a_{n+p} - a_n| \leq p/n^2$时,上述结论又如何?

\noindent 思路:对于题(1),当$n\to\infty$时$\displaystyle\frac{p}{n}$显然趋于$0$,但我们从直觉上能够感觉到这个趋于$0$的速度还不够快,由此猜想$\{a_n\}$应该不是基本列,可以去举反例.
\begin{proof}
(1) 考虑
\begin{equation}
    a_n = 1 + \frac{1}{2} + \frac{1}{3} + \cdots + \frac{1}{n} 
\end{equation}
于是
\begin{align}
    a_{n+p} - a_n &= \frac{1}{n+1} + \frac{1}{n+2} + \cdots + \frac{1}{n+p}\\
    &\leq \frac{1}{n+1} + \frac{1}{n+1} + \cdots + \frac{1}{n+1} \\
    &= \frac{p}{n+1} < \frac{p}{n}
\end{align}
数列$\{a_n\}$满足$|a_{n+p}-a_n|\leq \displaystyle\frac{p}{n}, \, (n,p\in\mathbb{N}^\star)$,但是$\{a_n\}$显然不是基本列.

\noindent (2) 由题设,有
\begin{align}
    |a_{n+p} - a_n| &\leq |a_{n+1} - a_n| + |a_{n+2} - a_{n+1}| + \cdots + |a_{n+p} - a_{n+p-1}| \\
    &\leq \frac{1}{n^2} + \frac{1}{(n+1)^2} + \cdots + \frac{1}{(n+p-1)^2}
\end{align}
令
\begin{equation}
    b_n = 1 + \frac{1}{2^2} + \frac{1}{3^2} + \cdots + \frac{1}{n^2} = \sum_{i=1}^n \frac{1}{i^2}
\end{equation}
于是
\begin{equation}
    |a_{n+p} - a_n| \leq b_{n+p-1} - b_n = |b_{n+p-1} - b_n| < |b_{n+p} - b_n|
\end{equation}
显然,数列$\{b_n\}$是收敛的,因此数列$\{ b_n \}$是基本列,因此,对任意$\epsilon > 0$,存在$N \in \mathbb{N}^\star$,使得当$n > N$时,
\begin{equation}
    |a_{n+p} - a_n| < |b_{n+p} - b_n| < \epsilon
\end{equation}
对任意$n, p \in \mathbb{N}^\star$成立.所以,由定义,$\{ a_n \}$是基本列.
\end{proof}

\noindent 3. 证明下列数列收敛:

\begin{table}[H]
    \centering
    \begin{tabularx}{\textwidth} { >{\raggedright\arraybackslash}X }
    (1) $\displaystyle a_n = 1-\frac{1}{2^2}+\frac{1}{3^2}-\cdots+(-1)^{n-1}\frac{1}{n^2}\, (n\in\mathbb{N}^\star)$; \\ [0.7em]
    (2) $\displaystyle b_n = a_0 + a_1 q + \cdots + a_n q^n \, (n \in \mathbb{N}^\star)$,其中$\{a_0,a_1,\cdots\}$为一有界数列,$|q|<1$; \\ [0.7em]
    (3) $a_n = \displaystyle \sin x + \frac{\sin 2x}{2^2} + \cdots + \frac{\sin nx}{n^2}\, (n \in \mathbb{N}^\star, x \in \mathbb{R})$; \\ [0.7em]
    (4) $a_n = \displaystyle \frac{\sin 2x}{2(2+\sin 2x)} + \frac{\sin 3x}{3(3+\sin 3x)} + \cdots + \frac{\sin nx}{n(n+\sin nx)} \, (n\in\mathbb{N}^\star, x \in \mathbb{R})$.
    \end{tabularx}
\end{table}

\begin{proof}
(1). 将$a_n$写成
\begin{equation}
    a_n = \sum_{i=1}^n (-1)^{i-1} \frac{1}{i^2}
\end{equation}
于是
\begin{align}
    |a_{n+p}-a_n| = \Bigg\lvert\sum_{i=n+1}^{n+p} (-1)^{i-1}\frac{1}{i^2} \Bigg\rvert < \Bigg\lvert \sum_{i=n+1}^{n+p} \frac{1}{i^2} \Bigg\rvert
\end{align}
令
\begin{equation}
    b_n = \sum_{i=1}^n \frac{1}{i^2}
\end{equation}
于是
\begin{equation}
    |a_{n+p}-a_n|<b_{n+p} - b_n =|b_{n+p}-b_n|
\end{equation}
并且数列$\{ b_n \}$是基本列,从而对任意$\epsilon > 0$,存在$N \in \mathbb{N}^\star$,使得当$n > N$时,
\begin{equation}
    |a_{n+p}-a_n|<|b_{n+p}-b_n|<\epsilon
\end{equation}
对任意$n,p \in \mathbb{N}^\star$都成立.于是,按定义,数列$\{a_n\}$是基本列,从而数列$\{a_n\}$收敛.
\end{proof}

\begin{proof}
(2). 设$l$是$\{a_n\}$的下界,$u$是$\{a_n\}$的上界,令$a = \max\{ |l|, |u| \}$:
\begin{align}
    |b_{n+p} - b_n| &= \Bigg\lvert \sum_{i=0}^{n+p} a_i q^i - \sum_{i=0}^{n} a_i q^i \Bigg\rvert \\
    &= \Bigg\lvert \sum_{i=n+1}^{n+p} a_i q^i \Bigg\rvert < a |q|^{n+1} \sum_{i=0}^{p-1} |q|^i = |q|^{n+1} \frac{a(1-|q|^p)}{1-|q|} \\
    &< |q|^{n+1} \frac{a}{1-|q|}
\end{align}
那么,对任意$\epsilon > 0$,我们可以取
\begin{equation}
    N = \lceil \log_{|q|} \left(\frac{\epsilon(1-|q|)}{a}\right) - 1\rceil
\end{equation}
则当$n > N$时
\begin{align}
    |b_{n+p}-b_n| &< |q|^{n+1}\frac{a}{1-|q|} < |q|^{N+1} \frac{a}{1-|q|} \\
    &< \frac{\epsilon(1-|q|)}{a} \frac{a}{1-|q|} = \epsilon
\end{align}
对任意$n, p\in\mathbb{N}^\star$都成立,于是由定义,$\{b_n\}$是一个基本列,于是$\{b_n\}$收敛.
\end{proof}
\begin{proof}
(3). 设$n,p\in\mathbb{N}^\star$:
\begin{align}
    |a_{n+p}-a_n| &= \Bigg\lvert \sum_{i=1}^{n+p} \frac{\sin ix}{i^2} - \sum_{i=1}^n \frac{\sin ix}{i^2} \Bigg\rvert \\
    &= \Bigg\lvert \sum_{i=n+1}^{n+p} \frac{\sin ix}{i^2} \Bigg\rvert \\
    &< \sum_{i=n+1}^{n+p} \frac{1}{i^2}
\end{align}
令
\begin{equation}
    b_n = \sum_{i=1}^n \frac{1}{i^2}
\end{equation}
易知数列$\{b_n\}$是收敛数量,故$\{b_n\}$是基本列,于是,对任意正数$\epsilon > 0$,存在$N \in \mathbb{N}^\star$,当$n>N$时,
\begin{equation}
    |b_{n+p}-b_{n}|<\epsilon
\end{equation}
对任意$n,p\in\mathbb{N}^\star$成立,又由于
\begin{equation}
    |a_{n+p}-a_n|<\sum_{i=n+1}^{n+p} \frac{1}{i^2}= |b_{n+p}-b_n|
\end{equation}
所以同样也有
\begin{equation}
    |a_{n+p}-a_n|<|b_{n+p}-b_n|<\epsilon
\end{equation}
对任意$n,p\in\mathbb{N}^\star$成立.所以由定义$\{a_n\}$是基本列,所以$\{a_n\}$收敛.
\end{proof}
\begin{proof}
(4). 将$a_n$写作
\begin{equation}
    a_n = \sum_{i=2}^n \frac{\sin ix}{i(i+\sin ix)}
\end{equation}
于是,对于$n,p\in\mathbb{N}^\star, n \geq 2$,有
\begin{align}
    |a_{n+p}-a_n| &= \Bigg\lvert \sum_{i=2}^{n+p}\frac{\sin ix}{i(i+\sin ix)} - \sum_{i=2}^n \frac{\sin ix}{i(i+\sin ix)} \Bigg\rvert \\
    &= \Bigg\lvert \sum_{i=n+1}^{n+p} \frac{\sin ix}{i(i+\sin ix)} \Bigg\rvert = \Bigg\lvert \sum_{i=n+1}^{n+p} \frac{\sin ix}{i^2 + i \sin ix} \Bigg\rvert \\
    &< \sum_{i=n+1}^{n+p} \bigg\lvert \frac{\sin ix}{i^2 + i \sin ix} \bigg\rvert < \sum_{i=n+1}^{n+p}\frac{1}{i^2-i}  = \sum_{i=n+1}^{n+p} \frac{1}{i(i-1)} \\
    &< \sum_{i=n+1}^{n+p}\frac{1}{(i-1)^2}
\end{align}
令
\begin{equation}
    b_n = 1+\frac{1}{2^2}+\cdots+\frac{1}{n^2}=\sum_{i=1}^n\frac{1}{i^2}
\end{equation}
则
\begin{equation}
    |a_{n+p}-a_n|<\sum_{i=n+1}^{n+p} = |b_{n+p-1}-b_{n}|
\end{equation}
由于数列$\{b_n\}$收敛,所以$\{b_n\}$是基本列,所以,对任意$\epsilon>0$,存在$N\in\mathbb{N}^\star$,使得当$n>N$时,
\begin{equation}
    |a_{n+p}-a_n|<|b_{n+p-1}-b_n|<\epsilon
\end{equation}
对任意$p\in\mathbb{N}^\star$都成立,从而按照定义,$\{a_n\}$是基本列,于是$\{a_n\}$收敛.
\end{proof}

\noindent 4. 设数列
\begin{equation}
    \{|a_2-a_1|+|a_3-a_2|+\cdots+|a_n-a_{n-1}|\}
\end{equation}
有界,求证$\{a_n\}$收敛.
\begin{proof}
令
\begin{equation}
    b_n = |a_2-a_1|+|a_3-a_2|+\cdots+|a_n-a_{n-1}| =\sum_{i=2}^n |a_{i}-a_{i-1}|
\end{equation}
于是
\begin{align}
    b_{n+1}-b_n &= \sum_{i=2}^{n+1} |a_i - a_{i-1}| -\sum_{i=2}^{n} |a_i-a_{i-1}| \\
    &= |a_{n+1}-a_{n}| > 0
\end{align}
从而,数列$\{b_n\}$,也就是数列$\{|a_2-a_1|+\cdots+|a_{n}-a_{n-1}|\}$是单调递增的,又由题设知数列$\{b_n\}$有界,所以数列$\{b_n\}$收敛.

\noindent 现在我们来证明数列$\{a_n\}$是一个基本列:
\begin{align}
    |a_{n+p}-a_n| &\leq |a_{n+1}-a_n|+|a_{n+2}-a_{n+1}|+\cdots+|a_{n+p}-a_{n+p-1}| \\
    &= b_{n+p} - b_{n-1} = b_{n+p} - b_n + b_{n} - b_{n-1} = |b_{n+p}-b_n|+|b_n-b_{n-1}|
\end{align}
由于数列$\{b_n\}$收敛,所以$\{b_n\}$是基本列,所以,对任意$\epsilon > 0$,存在$N\in\mathbb{N}^\star$,使得当$n > N$时
\begin{equation}
    |b_{n+p}-b_n|<\epsilon, \; |b_n-b_{n-1}|<\epsilon
\end{equation}
对任意$p \in \mathbb{N}^\star$都成立,而又因为$|a_{n+p}-a_n|<|b_{n+p} - b_n|+|b_{n}-b_{n-1}|$,所以
\begin{equation}
    |a_{n+p}-a_n|<|b_{n+p}-b_n|+|b_n-b_{n-1}|<\epsilon+\epsilon=2\epsilon
\end{equation}
对任意$p\in\mathbb{N}^\star$都成立,所以,由定义知$\{a_n\}$是基本列,所以$\{a_n\}$收敛.
\end{proof}
\noindent 5. 用精确语言表述``数列$\{a_n\}$不是基本列''.

\noindent 答:

\noindent 一个数列$\{ a_n \}$不是基本列,如果存在$\epsilon_0 > 0$,使得对任意$N \in \nat$,存在$n, m > N$,使得$|a_m - a_n| \geq \epsilon_0$.

\noindent 6. 设$a_n \in [a,b]\,(n\in\mathbb{N}^\star)$.证明:如果$\{a_n\}$发散,则$\{a_n\}$必有两个子列收敛于不同的数.
\begin{proof}
任取$\{ a_n \}$的两个收敛子列,分别记为$\{ a_{i_n} \}$和$\{ a_{j_n} \}$,由于$\{ a_n \}$发散,所以$\{ a_n \}$不是基本列,那么就存在$\epsilon_0 > 0$,使得对任意$N \in \nat$,都存在$n, m > N$,使得$|a_n - a_m| \geq \epsilon_0$,于是就存在$i_n, j_m > N$,使得$|a_{i_n} - a_{j_m}| \geq \epsilon_0$,这就说明了$\{ a_{i_n} \}$和$\{ a_{j_m} \}$分别收敛到不同的极限.
\end{proof}

\section*{练习题1.8}
\noindent 2. 求数列$\{ (1+1/n)^n : n \in \mathbb{N}^\star \}$和$\{(1+1/n)^{n+1}:n\in\mathbb{N}^\star\}$的下确界和上确界.

\noindent 思路:注意到$\{ (1+1/n)^n : n \in \mathbb{N}^\star \}$是严格单调递增数列并且极限是$\mathrm{e}$,所以我们去证它的上确界是$\mathrm{e}$,而数列$\{(1+1/n)^{n+1}:n\in\mathbb{N}^\star\}$的极限也是$\mathrm{e}$并且是严格单调递减数列,我们去证它的下确界是$\mathrm{e}$.
\begin{proof}
令$a_n = (1+1/n)^n$,易证,数列$\{ a_n \}$是严格单调递增的,并且极限是$\mathrm{e}$,从而$\mathrm{e} < a_n$对所有$n \in \mathbb{N}^\star$成立,也就是对任意$x \in \{a_n:n\in\mathbb{N}^\star\}$都有$\mathrm{e}<x$;由于$\{a_n\}$的极限是$\mathrm{e}$,所以,对任意$\epsilon >0$,都存在$N \in \mathbb{N}^\star$,使得当$n>N$时,有
\begin{equation}
    |a_n - \mathrm{e}| = \mathrm{e}-a_n < \epsilon
\end{equation}
令$x_\epsilon = a_n$,该不等式可以写成$x_\epsilon > e - \epsilon$,并且$x_\epsilon = a_n \in \{a_n : n \in \mathbb{N}^\star \}$,因此$\mathrm{e}$是集合$\{ a_n : n \in \mathbb{N}^\star \}$也就是集合$\{ (1+1/n)^n : n \in \mathbb{N}^\star \}$的上确界.易知$\{(1+1/n)^n:n\in\mathbb{N}\}$的下确界是$a_1$也就是$3/2$.

\noindent 再来证$\mathrm{e}$是集合$\{(1+1/n)^{n+1}:n\in\mathbb{N}\}$的下确界.令$b_n=(1+1/n)^{n+1}$,易证数列$\{b_n\}$是严格单调递减的并且极限是$\mathrm{e}$,从而对任意$n \in \mathbb{N}^\star$都有$b_n > \displaystyle\lim_{n\to\infty} b_n = \mathrm{e}$,也就是对任意$ x \in \{ b_n : n \in \mathbb{N}^\star \}$都有$x > \mathrm{e}$.

\noindent 由$\displaystyle\lim_{n\to\infty}b_n = e$知,对任意$\epsilon > 0$,存在$N \in \mathbb{N}^\star$,使得当$n > N$时,有
\begin{equation}
    |b_n - \mathrm{e}| = b_n - \mathrm{e} < \epsilon 
\end{equation}
令$y_\epsilon = b_n$,该不等式可以写成$y_\epsilon = b_n < \mathrm{e} + \epsilon$(对任意$n\in\mathbb{N}^\star$),从而$\mathrm{e}$正是集合$\{b_n:n\in\mathbb{N}^\star\}$的下确界.

\noindent 显然$\{b_n:n\in\mathbb{N}^\star\}$的上确界是$b_1$,也就是$\displaystyle \left(\frac{3}{2}\right)^2$.
\end{proof}

\noindent 3. 求数列$\{n^{1/n}:n\in\mathbb{N}^\star\}$的下确界和上确界.

\noindent 解:先求上确界.通过计算可知$1^{1/1} < 2^{1/2} <3^{1/3}$,令$a_n = n^{1/n}$,则$a_n$的增减性等价于$\ln \, a_n$的增减性,为了判断$\ln \, a_n$的增减性,我们对数列$\{ \ln \, a_n \}$的任意相邻两项做差:
\begin{align}
    \ln \, a_{n+1} - \ln \, a_n &= \frac{1}{n+1} \ln \, \left(n+1\right) - \frac{1}{n} \ln \, n\\
    &= \frac{1}{n(n+1)} \left( n \ln \, \left(n+1\right) - (n+1) \ln \, n\right)
\end{align}
并且当$n \geq 3$时,有
\begin{align}
    n \ln \, \left(n+1\right) - (n+1) \ln \, n &= n \left(\ln \, \left(n+1\right) - \ln \, n \right) - \ln \, n \\
    &< n \cdot \frac{1}{n} - \ln \, n = 1 - \ln \, n \leq 1 - \ln 3 < 0
\end{align}
即$\ln \, a_{n+1} - \ln \, a_n < 0$,由此可知,当$n \geq 3$时,有$a_{n+1} < a_n$.因此,对任意$n \in \mathbb{N}^\star$,都有$a_3 \geq a_n$,并且,对任意$\epsilon > 0$,可以取$x_\epsilon = a_3$,这时有
\begin{equation}
    a_3 - x_\epsilon = a_3 - a_3 = 0 < \epsilon
\end{equation}
也就是$x_\epsilon > a_3 - \epsilon$
所以$a_3$也就是$\displaystyle 3^\frac{1}{3}$,是$\{a_n:n\in\mathbb{N}^\star\}$的上确界.

\noindent 再来证$1$是$\{n^\frac{1}{n}\}$的下确界,由于$\displaystyle\lim_{n\to\infty}n^\frac{1}{n}=1$,所以,对任意$\epsilon>0$,存在$N \in \mathbb{N}^\star$,使得当$n > N$时,有
\begin{equation}
    |n^\frac{1}{n} - 1| = n^\frac{1}{n} - 1 < \epsilon
\end{equation}
令$y_\epsilon = n^\frac{1}{n}$,则对所有$n > N$,都有$y_\epsilon = n^\frac{1}{n} < 1 + \epsilon$成立,并且显然$y_\epsilon \in \{ n^\frac{1}{n}:n \in \mathbb{N}^\star \}$,因此,由定义,$1$是$\{\displaystyle n^\frac{1}{n}:n\in\mathbb{N}^\star\}$的下确界.

\noindent 4. 设在数列$\{ a_n:n\in\mathbb{N}^\star\}$中,既没有最小值,也没有最大值.求证:数列$\{a_n\}$发散.
\begin{proof}
如果数列$\{ a_n \}$是无界的,譬如说,不存在上界,那么我们可以从中找出一个趋于正无穷的子列,于是$\{a_n \}$发散.

\noindent 否则根据确界原理,$\{a_n\}$有上确界和下确界,分别记为$u$和$l$,由于$u$是$\{ a_n \}$的上确界,所以对任意$\epsilon > 0$,存在$i_1 \in \nat$,使得$u - a_{i_1} < \epsilon$,又因为$\{ a_n \}$是不存在最大值的,于是就存在$i_2 \in \nat$,使得$a_{i_1} < a_{i_2} < u$,类似地,我们可以找出一系列的下标$i_1, i_2, i_3, i_4, \cdots \in \nat$,它们满足
\begin{equation}
    a_{i_1} < a_{i_2} < a_{i_3} < a_{i_4} < \cdots < u
\end{equation}
并且对以上这两个不等式做放缩可得
\begin{equation}
    0 < \cdots < u - a_{i_4} < u - a_{i_3} < u - a_{i_2} < u - a_{i_1} < \epsilon
\end{equation}
这样一来,我们就找到了一个趋于$u$的子列,它就是$\{ a_{i_n} \}$.

\noindent 由于$l$是下确界,所以对任意$\epsilon > 0$,都存在$j_1 \in \nat$,使得$a_{j_1} - l < \epsilon$,由于$\{ a_n \}$不存在最小值,所以一定存在$j_2 \in \nat$使得
\begin{equation}
    a_{j_1} > a_{j_2} > l
\end{equation}
类似地,总能找出一系列的下标$j_1, j_2, j_3, \cdots$,满足
\begin{equation}
    a_{j_1} > a_{j_2} > a_{j_3} > \cdots > l
\end{equation}
对以上这两个不等式做放缩,得到
\begin{equation}
    \epsilon > a_{j_1} - l > a_{j_2} - l > a_{j_3} - l > \cdots > 0
\end{equation}
这样我们就找到了一个趋于$l$的子列,它就是$\{ a_{j_m} \}$.如果$u = l$,也就是$\{ a_n \}$的上确界等于下确界,那么将可以推出$\{ a_n \}$是常数列,这时每一个$a_n$都可以当成最大值(最小值),这与题设矛盾,所以$u \neq l$,所以$\{ a_{i_n} \}$和$\{ a_{j_m} \}$是$\{ a_n \}$的两个趋向不同极限的子列,所以$\{ a_n \}$发散.
\end{proof}

% \noindent 5. 试用定理1.8.1证明中使用的``二分法'',证明定理1.7.1(Bolzano---Weierstrass定理).

\section*{练习题1.10}
\noindent 1. 求$\displaystyle\lim_{n\to \infty} \inf a_n$和$\displaystyle\lim_{n \to \infty} \sup a_n$,设:
\begin{table}[H]
    \centering
    \begin{tabularx}{0.8\textwidth} {  >{\raggedright\arraybackslash}X >{\raggedright\arraybackslash}X  }
       (1) ~ $a_n = \displaystyle \frac{(-1)^n}{n} + \frac{1+(-1)^n}{2}$; & (2)~$a_n=\displaystyle n^{(-1)^n}$; \\ [1em]
       (3) ~ $a_n = \arctan \, \left( n^{(-1)^n} \right)$;
      \end{tabularx}
\end{table}

\noindent (1) 解:利用上、下极限的等价定义:
\begin{align}
    a^\star = \lim_{n \to \infty} \sup_{k \geq n} \, \{ a_k \}, \; a_\star = \lim_{n \to \infty} \inf_{k \geq n} \, \{a_k\}
\end{align}
我们首先求$\displaystyle\sup_{k \geq n} \, \{a_k\}$,它自身可以看成是一个数列:
\begin{equation}
    \sup_{k \geq n} \, a_k = \begin{cases}
        a_n, & n\text{为偶数} \\
        a_{n+1}, & n\text{为奇数} 
    \end{cases}
\end{equation}
这样我们就得到
\begin{equation}
    \{ \sup_{k \geq n} \, a_k \} = \{ a_2, a_2, a_4, a_4, a_6, a_6, \cdots \}
\end{equation}
它可以看成是$\{ a_n \}$所有偶数项组成的子列,再求$\displaystyle \inf_{k \geq n} \, \{ a_k \}$,它等于
\begin{equation}
    \inf_{k \geq n} \, a_k = \begin{cases}
        a_n, & n\text{为奇数} \\
        a_{n+1}, & n\text{为偶数} 
    \end{cases}
\end{equation}
这样我们就得到
\begin{equation}
    \{ \inf_{k \geq n} \, a_k \} = \{ a_1, a_1, a_3, a_3, a_5, a_5, \cdots \}
\end{equation}
它可以看成是$\{ a_n \}$所有奇数项组成的子列.于是$\displaystyle \lim_{n \to \infty} \sup_{k \geq n} \, \{ a_k \}$就等于$\{ a_{2n} \}$的极限,就等于
\begin{equation}
    \lim_{n \to \infty} a_{2n} = \lim_{n \to \infty} \left( \frac{1}{n} + 1 \right) = \left(\lim_{n\to\infty} \frac{1}{n}\right) + 1= 0 + 1 = 1
\end{equation}
而$\displaystyle \lim_{n \to \infty} \inf_{k \geq n} \, \{ a_k \}$等于$\{ a_{2n-1} \}$的极限,就等于
\begin{equation}
    \lim_{n \to \infty} a_{2n - 1} = \lim_{n \to \infty} -\frac{1}{n} = - \left(\lim_{n\to\infty} \frac{1}{n}\right) = -\left( 0 \right) = -0 = 0.
\end{equation}

\noindent (2)解:首先对$a_n$的幂次进行计算,也就是,对$(-1)^n$进行计算,我们发现:
\begin{equation}
    (-1)^n = \begin{cases}
        -1, & n\text{为奇数} \\
        1, & n\text{为偶数}
    \end{cases}
\end{equation}
这样一来就得到
\begin{equation}
    a_n = \begin{cases}
        n^{-1}, & n\text{为奇数} \\
        n, & n\text{为偶数}
    \end{cases}
\end{equation}
进而有
\begin{equation}
    \sup_{k \geq n} \, \{ a_k \} = n, \; \inf_{k \geq n} \, \{ a_k \} = n^{-1}
\end{equation}
从而得
\begin{align}
    &a^\star = \lim_{n\to\infty} \sup_{k \geq n} \, \{ a_k \} = \lim_{n \to \infty} n = +\infty \\
    &a_\star = \lim_{n\to\infty} \inf_{k \geq n} \, \{ a_k \} = \lim_{n \to \infty} n^{-1} = \lim_{n \to \infty} \frac{1}{n} = 0.
\end{align}

\noindent (3) 解:$\arctan$是定义在$(-\infty, +\infty)$上的严格单调递增函数,利用这个性质可以得到
\begin{align}
    &\ulimit \{ \arctan \, \left(n^{(-1)^n}\right) \} = \arctan \, \left( \ulimit \{ n^{(-1)^n}\}\right) = \arctan \, +\infty = \frac{\pi}{2} \\
    &\llimit \{ \arctan \, \left(n^{(-1)^n}\right) \} = \arctan \, \left( \llimit \{ n^{(-1)^n}\}\right) = \arctan \, 0 = 0
\end{align}

\noindent 2. 试证下面诸式当两边有意义时成立:

\noindent (1) 若$\displaystyle\lim_{n\to\infty}b_n=b$,则
\begin{align}
    &\llimit \left( a_n + b_n \right) = \llimit a_n + b, \\
    &\ulimit \left( a_n + b_n \right) = \ulimit a_n + b;
\end{align}

\noindent (1)
\begin{proof}
由例3的结论:
\begin{align}
    &\llimit \left( a_n + b_n \right) \leq \llimit a_n + \ulimit b_n = \llimit a_n + b \\
    &\llimit \left( a_n + b_n \right) \geq \llimit a_n + \llimit b_n = \llimit a_n + b 
\end{align}
所以这只能是
\begin{equation}
    \llimit \left( a_n + b_n \right) = \llimit a_n + b.
\end{equation}
再次利用例3的结论:
\begin{align}
    &\ulimit \left( b_n + a_n \right) \leq \ulimit b_n + \ulimit a_n = b + \ulimit a_n \\
    &\ulimit \left( b_n + a_n \right) \geq \llimit b_n + \ulimit a_n = b + \ulimit a_n
\end{align}
所以这只能是
\begin{equation}
    \ulimit \left( a_n + b_n \right) = \ulimit a_n + b.
\end{equation}
\end{proof}

\section*{练习题1.11}
\noindent 1. 计算下列极限:
\begin{table}[H]
    \centering
    \begin{tabularx}{\textwidth} {  >{\raggedright\arraybackslash}X >{\raggedright\arraybackslash}X  }
       (1)~$\displaystyle\lim_{n\to\infty}\displaystyle\frac{1+\displaystyle\frac{1}{2}+\cdots+\displaystyle\frac{1}{n}}{\ln \, n}$; & (2)~$\displaystyle\lim_{n\to\infty}\displaystyle\frac{1+\displaystyle\frac{1}{3}+\cdots+\displaystyle\frac{1}{2n-1}}{\ln \, 2\sqrt{n}}$; 
    \end{tabularx}
\end{table}

\noindent (1) 解:
令
\begin{align}
    a_n &= 1+\frac{1}{2}+\cdots+\frac{1}{n} \\
    b_n &= \ln \, n
\end{align}
则
\begin{align}
    \text{原式} &= \lim_{n\to\infty} \frac{a_n}{b_n} = \lim_{n \to \infty} \frac{a_{n+1}-a_n}{b_{n+1}-b_n} = \lim_{n \to \infty} \frac{\displaystyle\frac{1}{n+1}}{\ln \, \left( n+1 \right) - \ln \, n} \\
    &= \lim_{n \to \infty} \frac{1}{(n+1) \ln \, \left( 1 + \displaystyle\frac{1}{n}\right)} = \lim_{n \to \infty} \frac{1}{\ln \, \left( \left(1+\displaystyle\frac{1}{n}\right)^{n+1}\right)} \\
    &= \frac{1}{\ln \, \left( \displaystyle\lim_{n \to \infty} \left( \left(1 + \displaystyle\frac{1}{n}\right)^{n+1}\right)\right)} = \frac{1}{\ln \, \mathrm{e}} = \frac{1}{1} = 1.
\end{align}

\noindent (2) 解:
令
\begin{align}
    a_n &= 1 + \frac{1}{3} + \frac{1}{5} + \cdots + \frac{1}{2n-1} \\
    b_n &= \ln \, 2\sqrt{n}
\end{align}
则
\begin{align}
    \text{原式} &= \lim_{n \to \infty} \frac{a_n}{b_n} = \lim_{n \to \infty} \frac{a_{n+1} - a_n}{b_{n+1} - b_n} = \lim_{n \to \infty} \frac{\displaystyle \frac{1}{2n+1}}{\ln \, 2\sqrt{n+1} - \ln \, 2 \sqrt{n}} \\
    &= \lim_{n \to \infty} \frac{\displaystyle\frac{1}{2n+1}}{\ln \, \displaystyle\frac{2\sqrt{n+1}}{2\sqrt{n}}} = \lim_{n \to \infty} \frac{\displaystyle\frac{1}{2n+1}}{\ln \sqrt{\displaystyle\frac{n+1}{n}}} = \lim_{n \to \infty} \frac{1}{(2n+1) \ln \, \left(\left(1 + \displaystyle\frac{1}{n}\right)^{\frac{1}{2}} \right)} \\
    &= \lim_{n \to \infty} \frac{1}{\ln \, \left( \left(1+\displaystyle\frac{1}{n}\right)^{\frac{2n+1}{2}}\right)} = \lim_{n \to \infty} \frac{1}{\ln \, \left(\left(1+\displaystyle\frac{1}{n}\right)^{n+\frac{1}{2}}\right)} \\
    &= \frac{1}{\displaystyle\lim_{n \to \infty} \ln \, \left(\left(1+\displaystyle\frac{1}{n}\right)^{n+\frac{1}{2}}\right)} = \frac{1}{\ln \, \displaystyle\lim \left( \left( 1 + \displaystyle\frac{1}{n}\right)^{n+\frac{1}{2}}\right)} = \frac{1}{\ln \, \mathrm{e}} = \frac{1}{1} = 1.
\end{align}

\noindent 2. 计算极限$\displaystyle\lim_{n \to \infty} \left( n! \right)^{1/n^2}$.(提示:取对数).

\noindent 2. 解:
\begin{align}
    \text{原式} &= \lim_{n \to \infty} \exp \{ \, \ln \, \left( (n!)^{1/n^2}\right) \} = \lim_{n \to \infty} \exp \{ \, \displaystyle\frac{1}{n^2} \displaystyle\sum_{i=1}^{n} \ln \, i \, \}
\end{align}
令
\begin{align}
    a_n = \sum_{i=1}^n \ln \, i,  \;\; b_n = n^2
\end{align}
那么就有
\begin{align}
    \text{原式} &= \lim_{n \to \infty} \exp \, \{ \, \displaystyle\frac{1}{n^2} \displaystyle\sum_{i=1}^{n} \ln \, i \, \} = \exp \, \{ \, \displaystyle\lim_{n \to \infty} \displaystyle\frac{a_n}{b_n} \,\, \} \\
    &= \exp \, \{ \, \displaystyle\lim_{n \to \infty} \displaystyle\frac{a_{n+1}-a_n}{b_{n+1}-b_n} \,\, \} = \exp \, \{ \, \displaystyle\lim_{n \to \infty} \displaystyle\frac{\ln \, \left(n+1\right) - \ln \, n}{2n + 1} \,\, \} \\
    &= \exp \, \{ \, \displaystyle\lim_{n \to \infty} \ln \, \left( 1 + \displaystyle\frac{1}{n} \right)^{\frac{1}{2n+1}} \,\, \} = \exp \, \{ \, \ln \, \displaystyle\lim_{n \to \infty} \left( 1 + \displaystyle\frac{1}{n} \right)^{\frac{1}{2n+1}} \,\, \} \\
    &= \exp \, \{ \, \ln \, 1 \,\, \} = \exp \, 0 = 1.
\end{align}

\noindent 3. 计算极限
\begin{equation}
    \lim_{n\to\infty} \frac{1^2 + 3^2 + \cdots + (2n-1)^2}{n^3}.
\end{equation}

\noindent 3. 解:令
\begin{align}
    a_n &= 1^2 + 3^2 + \cdots + (2n-1)^2 \\
    b_n &= n^3
\end{align}
那么
\begin{align}
    \text{原式} &= \lim_{n \to \infty} \frac{a_{n+1}-a_n}{b_{n+1}-b_n} = \lim_{n \to \infty} \frac{(2n+1)^2}{3n^2 + 3n + 1} = \lim_{n \to \infty} \frac{4n^2 + 4n + 1}{3n^2 + 3n + 1} \\
    &= \lim_{n \to \infty} \frac{4 + \displaystyle\frac{4}{n} + \displaystyle\frac{1}{n^2}}{3 + \displaystyle\frac{3}{n} + \frac{1}{n^2}} = \frac{\displaystyle\lim_{n \to \infty} \left(4 + \displaystyle\frac{4}{n} + \displaystyle\frac{1}{n^2}\right)}{\displaystyle\lim_{n \to \infty} \left(3 + \displaystyle\frac{3}{n} + \displaystyle\frac{1}{n^2}\right)} = \frac{4}{3}.
\end{align}

\noindent 4. 设$\displaystyle\lim_{n \to \infty} a_n = a$.证明:
\begin{equation}
    \lim_{n \to \infty} \frac{a_1 + 2a_2 + \cdots + na_n}{n^2} = \frac{a}{2}.
\end{equation}

\begin{proof}
令
\begin{align}
    b_n &= a_1 + 2a_2 + \cdots + na_n \\
    c_n &= n^2
\end{align}
于是
\begin{align}
    \text{左边} &= \lim_{n \to \infty} \frac{b_n}{c_n} = \lim_{n \to \infty} \frac{b_{n} - b_{n-1}}{c_{n} - c_{n-1}} = \lim_{n \to \infty} \frac{n a_n}{2n - 1} = \lim_{n \to \infty} \frac{a_n}{2 - \displaystyle\frac{1}{n}} \\
    &= \frac{\displaystyle\lim_{n \to \infty} a_n}{\displaystyle\lim_{n \to \infty} \left(2 - \displaystyle\frac{1}{n}\right)} = \frac{a}{2} = \text{右边}.
\end{align}
\end{proof}

\begin{theo}[Stolz]
设$\{b_n \}$是严格递增且趋于$+\infty$的数列,那么
\begin{equation}
    \lim_{n \to \infty} \frac{a_n}{b_n} = A
\end{equation}
的充分不必要条件是
\begin{equation}
    \lim_{n \to \infty} \frac{a_n - a_{n-1}}{b_{n} - b_{n-1}} = A.
\end{equation}
\end{theo}

\noindent 5. 举例说明Stolz定理的逆命题不成立.

\begin{proof}
取
\begin{align}
    a_n &= n^2 + n \cdot (-1)^n \\
    b_n &= n^2
\end{align}
那么$\{ a_n \}, \{ b_n \}$符合Stolz条件,并且
\begin{equation}
    \lim_{n \to \infty} \frac{a_n}{b_n} = 1
\end{equation}
但是
\begin{equation}
    \lim_{n \to \infty} \frac{a_{n} - a_{n-1}}{b_{n} - b_{n-1}} = +\infty
\end{equation}
这就证明了,在Stolz定理中:$\displaystyle\lim_{n \to \infty} \frac{a_n - a_{n-1}}{b_n - b_{n-1}} = A$不是$\displaystyle\lim_{n \to \infty} \frac{a_n}{b_n} = A$的必要条件.
\end{proof}

\section*{练习题2.1}
\noindent 1. 设$A = \{ a, b, c, d \}$.证明有唯一的映射:$f: A \to A$满足下列条件:
\begin{enumerate}
    \item $f(a) = b, f(c) = d$;
    \item $( f \circ f ) (x) = x$对一切$x \in A$成立.
\end{enumerate}

\begin{proof}
任取$A$到$A$的两个映射:$f: A \to A, \, g: A \to A$,设$f, g$都满足条件(1)、(2),我们去验证$f$与$g$相等:
\begin{align}
    &f(a) = b, \, g(a) = b \, \implies f(a) = g(a) \\
    &f(b) = f(f(a)) = (f \circ f)(a) = a, \, g(b) = g(g(a)) = (g \circ g)(a) = a \, \implies f(b) = g(b) \\
    &f(c) = d, \, g(c) = d \, \implies f(c) = g(c) \\
    &f(d) = f(f(c)) = (f \circ f)(c) = c, \, g(d) = g(g(c)) = (g \circ g)(c) = c \, \implies f(d) = g(d)
\end{align}
从而$\forall x \in A, f(x) = g(x)$,从而$f = g$,从而这样的映射是唯一的.
\end{proof}

\noindent 2. 设映射$f$满足$(f \circ f)(a) = a$,求$f^n (a)$.
\begin{proof}
当$n=2$时
\begin{equation}
    f^2 (a) = (f \circ f)(a) = a
\end{equation}
由此可知当$n=2$时$f^2 (a) = a$成立,现假设对某个$k \in \nat$,当$n = 2(k-1)$时$f^n (a) = a$成立,于是
\begin{align}
f^{2k} (a) &= (f \circ f^{2k-1}) (a) = (f \circ (f \circ f^{2(k-1)}))(a) = ((f \circ f) \circ f^{2(k-1)})(a) = (f^2 \circ f^{2(k-1)})(a) \\
&= f^2 (f^{2(k-1)}(a)) = f^2 (a) = a
\end{align}
这说明对于$n = 2k$命题也成立,那么根据数学归纳法原理,对任意正偶数$n$,$f^n (a) = a$成立.
\end{proof}

\noindent 解:对于$n=1$,有$f^1 (a) = f(a)$,对于任意正偶数$n$,有$f^n (a) = a$,对任意大于等于$3$的正奇数$n$,$n$可被表示为$n=2k+1, k \in \nat$,于是有
\begin{equation}
    f^{n}(a) = \begin{cases}
        a, & n\text{为正偶数}; \\
        f(a), & n\text{为正奇数}.
    \end{cases}
\end{equation}

\noindent 3. 定义映射$D: \real \to \{ 0, 1 \}$如下:
\begin{equation}
    D(x) = \begin{cases}
        1, & \text{当}x\text{为有理数时}, \\
        0, & \text{当}x\text{为无理数时}.
    \end{cases}
\end{equation}
\begin{enumerate}
    \item 求复合映射$D \circ D$;
    \item 求$D^{-1}(\{0\}), D^{-1}(\{1\}), D^{-1}(\{0,1\})$.
\end{enumerate}

\noindent (1) 解:任取$x \in \real$,如果$x$为有理数,那么
\begin{equation}
    D(x) = 1, \quad (x\text{为有理数})
\end{equation}
那么由于$1$是有理数,所以
\begin{equation}
    D(D(x)) = D(1) = 1, \quad (x\text{为有理数})
\end{equation}
如果$x$为无理数,那么
\begin{equation}
    D(x) = 0, \quad (x\text{为无理数})
\end{equation}
那么,由于$0$为有理数,所以
\begin{equation}
    D(D(x)) = D(0) = 1, \quad (x\text{为无理数})
\end{equation}
所以无论$x$是有理数还是无理数,都有
\begin{equation}
    D(D(x)) = (D \circ D)(x) = 1, \quad \forall x \in \real
\end{equation}
所以$D \circ D$可写为$D \circ D: \real \to \{ 1 \}$.

\noindent (2) 解:$\real = \mathbb{Q} \bigcup (\real \setminus \mathbb{Q})$并且$\rational \bigcap (\real \setminus \rational) = \emptyset$,$\forall x \in \rational, D(x) = 1$,$\forall x \in \real \setminus \rational, D(x) = 0$,所以$D^{-1}(\{0\}) = \irrational$.同理,$D^{-1}(\{1\}) = \mathbb{Q}$.$D^{-1}(\{0, 1\}) = \real$.

\noindent 4. 设$A$是由$n$个元素组成的集合.若映射$f: A \to A$是单射,则称$f$是$A$的一个排列.
\begin{enumerate}
    \item 证明:$f(A) = A$. 
    \item 证明:$f^{-1}$存在.
    \item $A$共有多少个排列?
\end{enumerate}

\noindent (1)
\begin{proof}
由于$f: A \to A$是集合$A$到自身的映射,所以$f(A) \subset A$.

\noindent 对任意$x \in A$都应有$x \in f(A)$,否则,假设存在$x_0 \in A$但是$x_0 \not\in f(A)$,那么$| f(A) | < |A| = n$,但是由于$f$是单射并且$|A| = n$,所以$|f(A)| \geq n$,矛盾.所以$A \subset f(A)$.因为$f(A) \subset A$以及$A \subset f(A)$,所以$f(A) = A$.
\end{proof}

\noindent (2)
\begin{proof}
任取$x_0 \in A$,让$x_{1} = f(x_0), x_{m+1} = f(x_m)$,让$m$取遍全体正整数,我们得到一个数列$\{ x_m \}$,它是
\begin{equation}
    \{ x_m \} = \{ f(x_0), f(f(x_0)), f(f(f(x_0))), \cdots \}
\end{equation}
采用反证法,假设对任意$m \in \nat$,都有$x_m \neq x_0$.那么任取$\{ x_m \}$中的两项$x_i, x_j, i \neq j$,设$ i < j$,事实上,$x_i, x_j$可写成$x_i = f^{i}(x_0), x_j = f^j (x_0)$,如果$x_i = x_j$,那么就有$f^{i}(x_0) = f^{j}(x_0)$,由于$f$是单射,所以就有$f^{i-1}(x_0) = f^{j-1}(x_0)$,反复应用$f$的单射性质,最终会得到$f^{0}(x_0) = f^{j-i}(x_0)$也就是$x_0 = f^{j-i}(x_0)$,但根据假设不可能有$x_0 = f^{j-i}(x_0)$成立,所以也不可能有$x_i = x_j$成立,所以$\{ x_m \}$中任意两项都是两两不等的,从而$\{ x_m \}$有无穷多项,但是,由于$x_{m+1} = f(x_m), x_1 = f(x_0), \forall m \in \nat$,可知$x_m \in f(A) = A, \forall m \in \nat$,也就是说$|A|$有无穷多个元素,矛盾.

\noindent 这就说明,对每一个$x \in A$,都存在$m \in \nat$,使得$f^m (x_0) = x_0$,这就证明了$f^{-1}$是存在的.
\end{proof}

\noindent 点评:从直觉收到启发,我们发现在$\{ x_n \}$的前$n$项中,应该必然有一项与$x_0$相等,否则经过无数次将$f$反复应用于$x_0$将会``创造''出无穷多个不同的元素,这与$A$是有限集矛盾.

\noindent 或者我们还可以这样理解:想象$A$中的每一个元素是荷塘上的一片荷叶,想象这样一只青蛙,它最开始在编号为$x_0$的荷叶上,青蛙按照$f$的指示从编号为$x_0$的荷叶跳到编号为$f(x_0)$的荷叶,又从编号为$f(x_0)$的荷叶跳到编号为$f(f(x_0))$的荷叶,也就是说第$t$次跳跃就是从编号为$f^{t-1}(x_0)$的荷叶跳到编号为$f^{t}(x_0)$的荷叶,从直觉上理解,我们一定有信心认为,青蛙在经过有限次这样的跳跃后必定会回到最初的编号为$x_0$的荷叶上,否则就说明荷塘中有无穷多片荷叶,这是荒谬的.

\noindent 在群论中,设$S_n$是$n$元置换群,容易证明它是循环群,因此它有生成元,$f$与自身的复合可以看做是$S_n$中任意一个元素自己与自己做加法运算,譬如说,任取$a_0 \in S_n$,我们可以这样定义$f$:
\begin{align}
    f: \{1,2,3,\cdots,n\} &\to \{ 1,2,3,\cdots, n\} \\
    x &\mapsto a_0 x
\end{align}
于是$f \circ f$等价于$a \oplus a$,其中$\oplus$是群$S_n$的二元代数运算.这样就将代数结构与排列映射联系了起来.

\noindent (3) 确定了$f$所描述的具体的对应法则,也就知道了$A$有多少个排列.将$A$中的$n$个元素分别记为$x_1, x_2, \cdots, x_n$,为了构造出一种可能的$f$,我们要依次确定$x_1, x_2, \cdots, x_m$的映象,任取其中一个,记做$x_{i_1}$,那么有$n$个元素可作为$x_{i_1}$的映象,从剩下的$n-1$个没有确定映象的元素中再任取一个,记为$x_{i_2}$,有$n-1$个元素可作为$x_{i_2}$的映象,类似地,有$n-2$个元素可作为$x_{i_3}$的映项,这样进行下去,有$2$个元素可作为$x_{i_{n-1}}$的映象,有$1$个元素可作为$x_{i_n}$的映象,按照计数原理,确定$x_{i_1}, x_{i_2}, \cdots, x_{i_n}$这$n$个元素的映象的方式总共有$n \times (n-1) \times \cdots 2 \times 1 = n!$种.因此所有可能的$f$只有$n!$种.因此$A$有$n!$个排列.

\noindent 5. 设$A$是由$n$个元素组成的集合.若映射$f$满足$f(a) = a$对一切$a \in G$成立,则称$f$为$A$的恒等排列.求证:当$n \geq 2$时,存在非恒等排列$f$,使得$f \circ f$为恒等排列.

\begin{proof}
当$n=2$时,设
\begin{equation}
    A = \{ x_1 ,x_2 \}
\end{equation}
其中$x_1, x_2$是集合$A$的全部元素并且$x_1 \neq x_2$.我们令:
\begin{equation}
    f(x) = \begin{cases}
        x_2, & x = x_1; \\
        x_1, & x = x_2. 
    \end{cases}
\end{equation}
那么$f(x_1) = x_2$,显然它是非恒等映射,并且$f(f(x_1)) = f(x_2) = x_1$,以及$f(f(x_2)) = f(x_1) = x_2$,所以$f \circ f$是恒等映射,也就是说当$n=2$时命题成立.当$n \geq 3$时,设
\begin{equation}
    A = \{ x_1, x_2, \cdots , x_n \}
\end{equation}
其中$x_1, x_2, \cdots, x_n$是$A$的全部元素并且它们两两不相等.令
\begin{equation}
    f(x) = \begin{cases}
        x_2, & x = x_1; \\
        x_1, & x = x_2; \\
        x, & \text{otherwise.}
    \end{cases}
\end{equation}
因为$f(x_1) = x_2$,所以$f$是非恒等映射,但是对于$x = x_1$或者$x = x_2$,可验证$f \circ f$是恒等映射,对于$x \neq x_1$并且$x \neq x_2$,由于$f(x) = x$,所以$f \circ f(x) = x$,所以$f \circ f$对于任意$x \in A$都是恒等映射.于是对每一个$n \in \nat, n \geq 2$都可构造出这样的非恒等映射$f$使得$f \circ f$为恒等映射.命题得证.
\end{proof}

\section*{练习题2.2}
\noindent 1. 在平面直角坐标系中,两坐标均为有理数的点$(x,y)$称为{\bfseries{有理点}}.试证:平面上全体有理点所成的集合是一可数集.

\begin{proof}
按箭头顺序,可将$\nat \times \nat$中所有元组
    \begin{equation}
        \begin{tikzcd}
            (1,1) & (1,2) & (1,3) & (1,4) & \cdots \\
            (2,1) \arrow[ur] & (2,2) \arrow[ur] & (2,3) \arrow[ur] & (2,4) & \cdots \\
            (3,1) \arrow[ur] & (3,2) \arrow[ur] & (3,3) \arrow[ur] & (3,4) & \cdots \\
            (4,1) \arrow[ur] & (4,2) \arrow[ur] & (4,3) \arrow[ur] & (4,4) & \cdots \\
            \vdots & \vdots & \vdots & \vdots & \ddots
        \end{tikzcd}
    \end{equation}
写成一行:
\begin{equation}
    (1,1), \; (2,1), \; (1,2), \; (3,1), \; (3,2), \; (2,3), \; (1,4), \; (5,1), \; (4,2), \; (3,3), \; (2,4), \; \cdots
\end{equation}
这说明$\nat \times \nat$是可数的,那么就存在一个双射$f: \nat \to \nat \times \nat$,不妨记为$f(n) = (x_n, y_n)$,并且$x_n, y_n \in \nat$.由Thm 2.2.3:全体有理数构成的集合$\rational$是可数的,于是就存在一个双射$g: \nat \to \rational$,令
\begin{align}
    p: \nat \times \nat &\longrightarrow \rational \times \rational \\
    (n_1, n_2) &\longmapsto (g(n_1), g(n_2))
\end{align}
再令:
\begin{align}
    q: \nat &\longrightarrow \rational \times \rational \\
    n &\longmapsto p(f(n)) = (p \circ f) (n)
\end{align}
就得到$\nat$到全体有理数对的一个双射$q$,这说明$\rational \times \rational$是可数的.
\end{proof}

\noindent 2. 如果复数$x$满足多项式方程
\begin{equation}
    a_0 x^n + a_1 x^{n-1} + a_2 x^{n-2} + \cdots + a_{n-1} x + a_{n} = 0,
\end{equation}
其中$a_0 \neq 0$,$a_1, \cdots, a_n$都是整数,那么$x$称为{\bfseries{代数数}}.试证:代数数全体是可数集.

\begin{proof}
每一个一个代数数都可由它的$n+1$个系数所唯一确定,我们只要证明这$n+1$个系数元组可数就行了.采用数学归纳法,对多项式的次数$n$做归纳:当$n=1$时代数数方程为:
\begin{equation}
    a_0 x + a_1 = 0
\end{equation}
令
\begin{align}
    E_1 &= \{ (1,0), (1,1), (1,2), \cdots \} \\
    E_2 &= \{ (2,0), (2,1), (2,2), \cdots \} \\
    \vdots \\
    E_m &= \{ (m,0), (m,1), (m,2), \cdots \} \\
    m &= 1,2,3,4,5,\cdots
\end{align}
则$\nat \times \nnat = \displaystyle\bigcup_{m=1}^{\infty} E_m$,依Thm 2.2.2,$\nat \times \nnat$是可数的,于是全体$n=1$多项式所确定的代数数可数.假设对某个$k \in \nat$,当$n=k$时命题成立,也就是说全体$k$次多项式确定的代数数可数,这就是说,$k$个自然数元组集$\underbrace{\nat \times \nnat \times \cdots \times \nnat}_{k+1\text{个}}$可数,那么,我们令$A = \underbrace{\nat \times \nnat \times \cdots \times \nnat}_{k+1\text{个}}$,再令
\begin{align}
    D_1 &= \{ (x_1, x_2, \cdots, x_{k+1}, 0) : (x_1,x_2,\cdots,x_{k+1}) \in A \} \\
    D_2 &= \{ (x_1, x_2, \cdots, x_{k+1}, 1) : (x_1,x_2,\cdots,x_{k+1}) \in A \} \\
    \vdots \\
    D_m &= \{ (x_1, x_2, \cdots, x_{k+1}, m-1) : (x_1,x_2,\cdots,x_{k+1}) \in A \} \\
    m &= 1,2,3,\cdots
\end{align}
由归纳假设$D_m, m=1,2,3,\cdots$都可数,那么,依Thm 2.2.2,$\underbrace{\nat \times \nnat \times \cdots \times \nnat}_{k+2\text{个}}$可数,也就是说,$k+1$阶代数数多项式
\begin{equation}
    a_0 x^{k+1} + a_1 x^{k} + \cdots + a_{k} x + a_{k+1} = 0
\end{equation}
的系数元组可数,从而$k+1$阶代数数多项式确定的代数数可数,从而依归纳原理,全体代数数可数.
\end{proof}

\noindent 3. 设$A$是数轴上长度不为零的、互不相交的区间所成的集合(注意:集合$A$的元素是区间).试证:$A$是至多可数的.
\begin{proof}
如果$A$中的唯一元素为整个数轴,或者只用有限多个实数$x_1,x_2,\cdots, x_n \in \real$将整个数轴划分为有限多个区间作为$A$的构成
\begin{equation}
    A = \{ (-\infty, x_1), [x_1, x_2), [x_2, x_3), \cdots, [x_{n-1}, x_n), [x_n, +\infty) \}
\end{equation}
那么$A$是有限集.如果用可数多个实数$y_1, y_2, y_3, \cdots \in \real$将整个数轴划分为可数多个部分去构成$A$:
\begin{equation}
    A = \{ (-\infty, y_1), [y_3, y_4), [y_4, y_5), \cdots, [y_2, +\infty) \}
\end{equation}
那么如果我们令
\begin{equation}
    f(n) = \begin{cases}
        (-\infty, y_1), & n = 1, \\
        (y_2, +\infty), & n = 2, \\
        [y_{n}, y_{n+1}), & \text{otherwise.}
    \end{cases}
\end{equation}
就得到了$\nat$到$A$的一个双射$f: \nat \longrightarrow A$,从而$A$是至多可数的.
\end{proof}

\noindent 4. 设$S = \{ (x_1, x_2, \cdots) : x_i = 0 \, \text{或} \, 1, i = 1,2,3,\cdots \}$.求证:$S$与区间$[0,1]$有相等的势.

\begin{proof}
任取$x \in S$,记$x$为$x = (x_1,x_2,x_3,\cdots)$,令
\begin{align}
    (a_1, b_1) &= \begin{cases}
        (0, 1/2), & x_1 = 0 \\
        (1/2, 1), & x_1 = 1
    \end{cases} \\
    (a_2, b_2) &= \begin{cases}
        (a_1, \displaystyle\frac{a_1+b_1}{2}), & x_2 = 0 \\
        (\displaystyle\frac{a_1+b_1}{2}, b_1), & x_2 = 1
    \end{cases} \\
    \vdots \\
    (a_m, b_m) &= \begin{cases}
        (a_{m-1}, \displaystyle\frac{a_{m-1}+b_{m-1}}{2}), & x_m = 0 \\
        (\displaystyle\frac{a_{m-1}+b_{m-1}}{2}, b_{m-1}), & x_m = 1 \\
    \end{cases} \\
    I_m &= [a_m, b_m] \\
    m &= 2,3,4,\cdots 
\end{align}
我们得一列闭区间套$\{ I_m: m \in \nat \}$满足$I_{m+1} \subset I_m, \forall m \in \nat$,并且$|I_m| \to 0, (m \to \infty)$,依闭区间套定理,存在唯一实数$x^\star \in [0, 1]$,使得$x^\star \in \displaystyle\bigcap_{m=1}^\infty I_m$,这样我们就得到了$S$到$[0,1]$的一个双射:
\begin{align}
    f: S &\longrightarrow [0,1] \\
    x &\longmapsto x^\star
\end{align}
从而$S$与$[0,1]$等势.
\end{proof}

\section*{练习题2.3}
\noindent 1. 求下列函数的定义域:

\begin{table}[H]
    \centering
    \begin{tabularx}{\textwidth} {  >{\raggedright\arraybackslash}X >{\raggedright\arraybackslash}X  }
        (1)~$f(x)=\displaystyle\sqrt{1-x^2}$; & (2)~$f(x)=\displaystyle\sqrt[3]{\displaystyle\frac{1+x}{1-x}}$; \\ [1em]
        (3)~$f(x)=\displaystyle\frac{x+1}{x^2 + x -2}$; & (4)~$f(x) = \ln \, \displaystyle\frac{1+\sin \, x}{1-\cos \, x}$.
      \end{tabularx}
\end{table}
\noindent (1) 解:设定义域为$D$:
\begin{align}
    &\mathrel{\phantom{\implies}} \forall x \in D, \exists y \in \real, y = \displaystyle\sqrt{1-x^2} \\
    &\implies 1 - x^2 \geq 0 \\
    &\implies 0 \leq x^2 \leq 1 \implies -1 \leq x \leq 1 \\
    &\implies D = [-1, 1].
\end{align}
\noindent (2) 解:设定义域为$D$:
\begin{align}
    &\mathrel{\phantom{\implies}} \forall x \in D, \exists y \in \real, y = \displaystyle\sqrt[3]{\displaystyle\frac{1+x}{1-x}} \\
    &\implies 1 - x \neq 0 \implies x \neq 1 \implies D = \real \setminus \{ 1 \}.
\end{align}
\noindent (3) 解:设定义域为$D$:
\begin{align}
    &\mathrel{\phantom{\implies}} \forall x \in D, \exists y \in \real, y = \displaystyle\frac{x+1}{x^2 +x - 2} \\
    &\implies x^2 + x - 2 \neq 0 \implies (x+2)(x-1) \neq 0 \implies x \neq 1 \; \land \; x \neq -2 \\
    &\implies D = \real \setminus \{ 1, 2 \}.
\end{align}
\noindent (4) 解:设定义域为$D$:
\begin{align}
    &\mathrel{\phantom{\implies}} \forall x \in \real, \exists y \in \real, y = \ln \, \frac{1+\sin \, x}{1+ \cos \, x} \\
    &\implies \frac{1+\sin \, x}{1+\cos \, x} > 0 \implies (1+\sin \, x)(1+\cos \, x) > 0  \\
    &\implies 1 + \cos \, x \neq 0 \implies \cos \, x \neq -1 \implies x \not\in \{x : \cos \, x = -1, x \in \real \} \\
    &\implies x \not\in \{ 2\pi n + \frac{3\pi}{2} : n \in \integer \} \implies D = \real \setminus \{ 2 \pi n + \frac{3 \pi}{2} : n \in \integer \}
\end{align}

\noindent 2. 给定函数$f: \real \longrightarrow \real$,如果$x \in R$使$f(x)=x$,则称$x$为$f$的一个不动点.若$f \circ  f$有唯一的不动点,求证:$f$也有唯一的不动点.

\begin{proof}
设$x_0$是$f \circ f$的不动点,依题意,这个$x_0$是唯一的仅此一个.如果$x_0$是$f \circ f$的不动点,那么
\begin{equation}
    f(f(f(x_0))) = f(x_0)
\end{equation}
这说明$f(x_0)$是$f \circ f$的不动点,而作为$f \circ f$的不动点的$x_0$又是唯一的,所以$f(x_0) = x_0$,所以$f$有唯一的不动点$x_0$.命题得证.
\end{proof}

\noindent 3. 设$f: \real \longrightarrow \real$.若$f \circ f$有且仅有两个不动点$a, b\,(a \neq b)$,求证只有一下两种情况:
\begin{enumerate}
    \item $a, b$都是$f$的不动点;
    \item $f(a) = b, \, f(b) = a$.
\end{enumerate}

\begin{proof}
设$a, b (a \neq b)$是$f \circ f$仅有的为数不多的两个不动点.依题2的证明结论,如果$x$是$f \circ f$的不动点那么$f(x)$必定也是$f \circ f$的不动点,因为$a$是$f \circ f$的不动点,所以$f(a)$是$f \circ f$的不动点,而$f \circ f$的全部不动点构成一个有限集合
\begin{equation}
    \{ a, b \}
\end{equation}
所以
\begin{equation}
    f(a) \in \{a, b \}
\end{equation}
如果说$f(a) = a$,那么同理可推出$f(b)=b$,那么$a, b$都是$f$的不动点,否则如果$f(a) = b$那么可同理推出$f(b)=a$.
\end{proof}

\noindent 4. 设函数$f: \real \longrightarrow \real$,且每一个实数都是$f \circ f$的不动点.试问:
\begin{enumerate}
    \item 有几个这样的函数?
    \item 若$f$在$\real$上递增,有几个这样的函数?
\end{enumerate}

\noindent (1) 解:4个,分别是:
\begin{align}
    f_1(x) &= x \\
    f_2(x) &= -x \\
    f_3(x) &= \begin{cases}
        \displaystyle\frac{1}{x}, & x \neq 0 \\
        0, & x = 0
    \end{cases} \\
    f_4(x) &= \begin{cases}
        \displaystyle-\frac{1}{x}, & x \neq 0 \\
        0, & x = 0
    \end{cases}
\end{align}

\noindent (2) 解:1个,是
\begin{equation}
    f(x) = x.
\end{equation}

\noindent 5. 求下列函数$f$的$n$次复合$f^n$:

\begin{enumerate}
    \item $f(x) = \displaystyle\frac{x}{\displaystyle\sqrt{1+x^2}}$;
    \item $f(x) = \displaystyle\frac{x}{1+bx}$.
\end{enumerate}

\noindent (1) 解:
当$n = 0$时,$f^0 (x) = I (x) = x = \displaystyle\frac{x}{\displaystyle\sqrt{1+0 x^2}}$;

\noindent 当$n=1$时,$f^1 (x) = f(x) = \displaystyle\frac{x}{\displaystyle\sqrt{1+x^2}}=\displaystyle\frac{x}{\displaystyle\sqrt{1+1 x^2}}$;

\noindent 当$n=2$时,$f^2 (x) = f(f(x)) = \displaystyle\frac{\displaystyle\frac{x}{\displaystyle\sqrt{1+x^2}}}{\displaystyle\sqrt{1+\left(\displaystyle\frac{x}{\displaystyle\sqrt{1+x^2}}\right)^2}} = \displaystyle\frac{x}{\displaystyle\sqrt{1+2x^2}}$;

\noindent 采用数学归纳法来证明:假设对于某个非负整数$k$有:
\begin{equation}
    f^k (x) = \frac{x}{\sqrt{1+kx^2}}
\end{equation}
成立,那么
\begin{equation}
    f^{k+1}(x) = f(f^k(x)) = \frac{\displaystyle\frac{x}{\sqrt{1+kx^2}}}{\sqrt{1+\left(\displaystyle\frac{x}{\sqrt{1+kx^2}}\right)^2}} = \frac{x}{\sqrt{1+(k+1)x^2}}
\end{equation}
依数学归纳法原理,对任意非负整数$n$,都有$f^n (x) = \displaystyle\frac{x}{\displaystyle\sqrt{1+nx^2}}$成立.

\noindent (2) 解:当$n=0$时:
\begin{equation}
    f^0 (x) = I(x) = x = \frac{x}{1 + 0bx}
\end{equation}
当$n=1$时:
\begin{equation}
    f^1 (x) = f(x) = \frac{x}{1+bx} = \frac{x}{1+1bx}
\end{equation}
当$n=2$时:
\begin{equation}
    f^2 (x) = f(f(x)) = \frac{\displaystyle\frac{x}{1+bx}}{1+\displaystyle\frac{bx}{1+bx}} = \displaystyle\frac{x}{1+2bx}
\end{equation}
猜测对一般的$n \in \nnat$都有$f^n (x) = \displaystyle\frac{x}{1+nbx}$成立,采用数学归纳法来证明,假设对于某个$k \in \nnat$,当$n=k$时命题是成立的,那么当$n=k+1$时:
\begin{equation}
    f^{k+1}(x) = f(f^k (x)) = \displaystyle\frac{\displaystyle\frac{x}{1+kbx}}{1+\displaystyle\frac{bx}{1+kbx}} = \displaystyle\frac{x}{1+(k+1)bx}
\end{equation}
依数学归纳法原理,对任意$n \in \nnat$命题都成立.

\noindent 6. 设$f: \real \longrightarrow \real$满足方程
\begin{equation}
    f(x+y)=f(x)+f(y) \quad (x,y \in \real)
\end{equation}
试证:对一切有理数$x$,有$f(x) = xf(1)$.

\begin{proof}
设$x$是一个非零有理数,那么存在$p, q \in \integer, p, q \neq 0$使得$x = \displaystyle\frac{p}{q}$,于是
\begin{equation}
    f(x) = f(\displaystyle\frac{p}{q}) = f(\underbrace{\displaystyle\frac{1}{q}+\cdots+\displaystyle\frac{1}{q}}_{p\text{个}}) = \underbrace{f(\displaystyle\frac{1}{q})+\cdots+f(\displaystyle\frac{1}{q})}_{p\text{个}} = p f(\displaystyle\frac{1}{q})
    \label{eq:rational1}
\end{equation}
让$q$个$f(\displaystyle\frac{1}{q})$相加,得到
\begin{equation}
    \underbrace{f(\displaystyle\frac{1}{q})+\cdots+f(\displaystyle\frac{1}{q})}_{q\text{个}} = f(\underbrace{\displaystyle\frac{1}{q}+\cdots+\displaystyle\frac{1}{q}}_{q\text{个}})=f(q \cdot \displaystyle\frac{1}{q}) = f(1)
\end{equation}
从而得$\displaystyle\frac{1}{q}=\displaystyle\frac{1}{q}f(1)$,于是代入式(\ref{eq:rational1})得
\begin{equation}
    f(x) = p f(\displaystyle\frac{1}{q}) = p \cdot \displaystyle\frac{1}{q}f(1) = \displaystyle\frac{p}{q}f(1)=xf(1)
\end{equation}
以上是当$x \neq 0$的情形,如果$x=0$,那么依题意
\begin{equation}
    f(0 + 0) = f(0) + f(0)
\end{equation}
从而
\begin{equation}
    f(0) = 0 = 0 f(1)
\end{equation}
于是对一切$x \in \rational$命题成立.
\end{proof}

\noindent 7. 设函数$f: \real \longrightarrow \real$,$l$为一正数,如果$f(x+l)=f(x)$对一切$x$成立,则称$f$是周期为$l$的{\bfseries{周期函数}}.如果$f$以任何正数为周期,求证:$f$为常值函数.
\begin{proof}
采用反证法,假设$f$不是常值函数,那么就存在$x_1, x_2 \in \real, x_1 < x_2$使得$f(x_1) \neq f(x_2)$,令$d = x_2 - x_1$,那么就有$f(x_1)\neq f(x_2)=f(x_1 + d)$,这与周期函数的定义不符,矛盾.
\end{proof}

\noindent 8. 试证$\sin \, \left(x^2\right), \sin \, x + \cos \, \sqrt{2}x$均不是周期函数.
\begin{proof}
采用反证法,设存在$l > 0$,使得$\sin \, \left(x+l\right)^2=\sin \, \left(x\right)^2$对一切$x \in \real$成立.那么对任意$x \in \real$,有
\begin{align}
    \sin \, \left(x+l\right)^2 = \sin \, \left(x\right)^2 &\implies \sin \, \left(x+l\right)^2-\sin \, \left(x\right)^2 = 0 \\
    &\implies \sin \, \left(x^2 + 2xl + l^2\right) - \sin \, \left(x^2\right) = 0 \\
    &\implies 2xl+l^2 = 2\pi n, \, n \in \integer \\
    &\implies x = \frac{\pi n}{l} - \frac{l}{2}
\end{align}
可是当$x \not\in \{ \displaystyle\frac{\pi n}{l} - \displaystyle\frac{l}{2} : n \in \integer \}$时,会有$2xl+l^2 \neq 2\pi n$,于是$\sin \, \left(x^2+2xl+l^2\right)^2 \neq \sin \left( x \right)^2$,矛盾.
\end{proof}
\begin{proof}
采用反证法,设存在$l > 0$,使得$\sin \, x + \cos \, \sqrt{2} x$成为一周期为$l$的周期函数.那么,对任意$x \in \real$都有$\sin \, (x + l) + \cos \, (\sqrt{2}(x+l)) = \sin \, x + \cos \, \sqrt{2}x$成立,也就是
\begin{align}
    &\mathrel{\phantom{\implies}} \sin \, (x+l) + \cos \, (\sqrt{2}x + \sqrt{2}l) - \sin \, x - \cos \, \sqrt{2} x  = 0 \\
    &\implies \sin \, x (\cos \, l - 1) + \sin \, l \cos \, x + \cos \, \sqrt{2} x ( \cos \, \sqrt{2} l - 1) - \sin \, \sqrt{2} x \sin \, \sqrt{2} l = 0 \label{eq:irrelavant}
\end{align}
考虑函数列$\sin \, x, \cos \, x, \cos \, \sqrt{2}x , \sin \, \sqrt{2} x$的Wronskian行列式:
\begin{equation}
    W(x) = \begin{vmatrix}
        \sin \, x & \cos \, x & \cos \, \sqrt{2} x & \sin \, \sqrt{2} x \\
        \cos \, x & -\sin \, x & -\sqrt{2} \sin \, \sqrt{2} x & \sqrt{2} \cos \, \sqrt{2} x \\
        -\sin \, x & -\cos \, x & -2 \cos \sqrt{2} x & -2 \sin \, \sqrt{2} x \\
        -\cos \, x & \sin \, x & 2\sqrt{2} \sin \, \sqrt{2} x & -2\sqrt{2} \cos \, \sqrt{2} x 
    \end{vmatrix}
\end{equation}
取$x_0 = 0$,去计算$W(x_0)$:
\begin{align}
    &W(x_0) = \begin{vmatrix}
        0 & 1 & 1 & 0  \\
        1 & 0 & 0 & \sqrt{2} \\
        0 & -1 & -2 & 0 \\
        -1 & 0 & 0 & -2\sqrt{2}
    \end{vmatrix} \xrightarrow{[3] + 2 \times [1]} \begin{vmatrix}
        0 & 1 & 1 & 0 \\
        1 & 0 & 0 & \sqrt{2} \\
        0 & 1 & 0 & 0 \\
        -1 & 0 & 0 & -2\sqrt{2}
    \end{vmatrix} \xrightarrow{[4]+[2]} \begin{vmatrix}
            0 & 1 & 1 & 0 \\
            1 & 0 & 0 & \sqrt{2} \\
            0 & 1 & 0 & 0 \\
            0 & 0 & 0 & -\sqrt{2}
    \end{vmatrix} \\
    &\xrightarrow{[1] + (-1) \times [3]} \begin{vmatrix}
        0 & 0 & 1 & 0 \\
        1 & 0 & 0 & \sqrt{2} \\
        0 & 1 & 0 & 0 \\
        0 & 0 & 0 & -\sqrt{2}
    \end{vmatrix} \xrightarrow{[1] \longleftrightarrow [2]} \begin{vmatrix}
        1 & 0 & 0 & \sqrt{2} \\
        0 & 0 & 1 & 0 \\
        0 & 1 & 0 & 0 \\
        0 & 0 & 0 & -\sqrt{2}
    \end{vmatrix} \xrightarrow{[2] \longleftrightarrow [3]} \begin{vmatrix}
            1 & 0 & 0 & \sqrt{2} \\
            0 & 1 & 0 & 0 \\
            0 & 0 & 1 & 0 \\
            0 & 0 & 0 & -\sqrt{2}
    \end{vmatrix} \\
    &\xrightarrow{[1] + [4]} \begin{vmatrix}
        1 & 0 & 0 & 0 \\
            0 & 1 & 0 & 0 \\
            0 & 0 & 1 & 0 \\
            0 & 0 & 0 & -\sqrt{2}
    \end{vmatrix} = -\sqrt{2} \neq 0
\end{align}
因此函数列$\sin \, x, \cos \, x, \cos \, \sqrt{2}x , \sin \, \sqrt{2} x$线性无关,因此从式(\ref{eq:irrelavant})右端等于$0$可推出
\begin{equation}
    \begin{cases}
        (\cos \, l) - 1 &= 0 \\
        \sin \, l &= 0 \\
        (\cos \, \sqrt{2} l) - 1 &= 0 \\
        \sin \, \sqrt{2} l &= 0
    \end{cases}
\end{equation}
这说明$l = 0$,从而与$l > 0$矛盾,从而不存在这样的$l > 0$使得题设函数成为一周期函数.
\end{proof}

\noindent 9. 设函数$f: (-a, a) \longrightarrow \real$.如果对任何$x \in (-a, a)$,有$f(x) = f(-x)$,则称$f$为偶函数;若$f(x) = -f(-x)$,则称$f$为奇函数.求证:$(-a, a)$上的任何函数均可表示为一个奇函数和一个偶函数之和.

\begin{proof}
令
\begin{align}
    p: (-a, a) \;\; &\longrightarrow \;\; \real \\
    x \;\; &\longmapsto \;\; \displaystyle\frac{f(x)+f(-x)}{2}
\end{align}
再令
\begin{align}
    q: (-a, a) \;\; &\longrightarrow \;\; \real \\
    x \;\; &\longmapsto \;\; \displaystyle\frac{f(x)-f(-x)}{2}
\end{align}
则
\begin{equation}
    p(-x) = \displaystyle\frac{f(-x)+f(-(-x))}{2} = \displaystyle\frac{f(-x)+f(x)}{2} = \displaystyle\frac{f(x)+f(-x)}{2} = p(x)
\end{equation}
这说明$p$在$(-a, a)$上成为一偶函数.又有
\begin{equation}
    q(-x) = \displaystyle\frac{f(-x)-f(-(-x))}{2} = \displaystyle\frac{f(-x)-f(x)}{2} = -\displaystyle\frac{f(x)-f(-x)}{2} = -q(x)
\end{equation}
这说明$q$在$(-a, a)$上成为一奇函数.并且
\begin{equation}
    p(x)+q(x) = \displaystyle\frac{f(x)+f(-x)}{2} + \displaystyle\frac{f(x)-f(-x)}{2} = f(x)
\end{equation}
这就将函数$f$表示成了一个偶函数$p$和一个奇函数$q$的和.
\end{proof}

\noindent 10. 函数
\begin{equation}
    \cosh \, x = \frac{\expe^x + \expe^{-x}}{2}, \quad \sinh \, x = \frac{\expe^{x} - \expe^{-x}}{2}
\end{equation}
分别称为{\bfseries{双曲正弦}}和{\bfseries{双曲余弦}}.证明:
\begin{enumerate}
    \item $\cosh$是偶函数,$\sinh$是奇函数;
    \item $(\cosh \, x)^2 - (\sinh \, x)^2 = 1 \; (x \in \real)$.
\end{enumerate}

\begin{proof}
计算得
\begin{equation}
    \cosh \, (-x) = \frac{\expe^{-x} + \expe^{-(-x)}}{2} = \frac{\expe^{-x}+\expe^{x}}{2} = \frac{\expe^x + \expe^{-x}}{2} = \cosh \, x
\end{equation}
这说明$\cosh$是偶函数.又有
\begin{equation}
    \sinh \, (-x) = \frac{\expe^{-x} - \expe^{-(-x)}}{2} = \frac{\expe^{-x}-\expe^{x}}{2} = - \frac{\expe^{x} - \expe^{-x}}{2} = - \sinh \, x
\end{equation}
这说明$\sinh$是奇函数.
\end{proof}

\begin{proof}
    \begin{align}
        \text{左边} &= (\cosh \, x)^2 - (\sinh \, x)^2 = (\cosh \, x + \sinh \, x)(\cosh \, x - \sinh \, x) \\
        &= \expe^x \cdot \expe^{-x} = \expe^0 = 1 = \text{右边}
    \end{align}
\end{proof}

\section*{练习题2.4}

\noindent 1. 用$\epsilon$-$\delta$语言表述$f(x_0 -) = 1$.

\noindent 1. 答:设函数$f$在$x_0 - r, x_0$有定义,$r$为一常数且$r > 0$,设$l$是一个给定的常实数.若对任意$\epsilon > 0$,都存在$\delta \in (0, r)$,使得对所有$x \in (x_0-\delta, x_0)$,都有
\begin{equation}
    |f(x) - l| < \epsilon
\end{equation}
就说$f$在$x_0$处的左极限存在,写为$f(x_0 -)$.

\noindent 2. 求证:$\displaystyle\lim_{x \to x_0} f(x)$存在当且仅当$f(x_0 -) = f(x_0 +)$为有限数.

\begin{proof}
必要性.设函数$f$在$x_0$处的极限存在.采用反证法,假设结论不成立,也就是说假设$f(x_0 -)$或者$f(x_0 +)$为非正常数$\infty$,不失一般性,假设$f(x_0 -)$为$+\infty$.

\noindent 那么对任意$M > 0$,都存在$\delta_1 > 0$,使得只要$0 < x_0 - x < \delta_1$,就有
\begin{equation}
    f(x) > M + l
\end{equation}
取$\epsilon_0 = M$,那么对任意$\delta > 0$,都存在$x$,满足$|x - x_0| < \min \{ \delta_1, \delta \} \leq \delta $并且
\begin{equation}
    |f(x) - l| > \epsilon_0 = M
\end{equation}
这说明函数$f$在$x_0$处的极限不存在,矛盾.于是必要性得证.

\noindent 充分性.设$f(x_0 -) = f(x_0 +) = c$,$c \in \real$为一非非正常数.由于$f(x_0 -)$存在,所以,对任意$\epsilon > 0$,存在$\delta_1 > 0$,使得只要$0 < x_0 - x < \delta_1$,就有
\begin{equation}
    |f(x) - c| < \epsilon
\end{equation}
由于$f(x_0 +)$存在,所以对任意$\epsilon > 0$,存在$\delta_2 > 0$,使得只要$0 < x - x_0 < \delta_2$,就有
\begin{equation}
    |f(x) - c| < \epsilon
\end{equation}
那么只需取$\delta = \min \{ \delta_1, \delta_2 \}$,当$|x - x_0| < \delta$时,必有$0 < x_0 - x < \delta_1$或者$0 < x - x_0 < \delta_2$其中之一是满足的,所以
\begin{equation}
    |f(x) - c| < \epsilon
\end{equation}
依定义函数$f$在$x_0$处的极限存在并且等于$c = f(x_0 -) = f(x_0 +)$.充分性得证.
\end{proof}

\noindent 3. 设$\displaystyle\lim_{x \to x_0} f(x) = A$.用$\epsilon$-$\delta$语言证明:
\begin{table}[H]
    \centering
    \begin{tabularx}{0.8\textwidth} {  >{\raggedright\arraybackslash}X >{\raggedright\arraybackslash}X  }
       (1)~$\displaystyle\lim_{x \to x_0} |f(x)| = |A|$; & (2)~$\displaystyle\lim_{x \to x_0} f^2(x)=A^2$; \\ [1em]
       (3)~$\displaystyle\lim_{x \to x_0} \sqrt{f(x)} = \sqrt{A} \, (A > 0)$; & (4)~$\displaystyle\lim_{x \to x_0} \sqrt[3]{f(x)} = \sqrt[3]{A}$.
      \end{tabularx}
\end{table}

\noindent (1) 
\begin{proof}
如果$A > 0$,那么$|A|=A$,并且在$x_0$的一个足够小的邻域内有$f(x) > 0$,也就是$f(x) = |f(x)|$,从而
\begin{equation}
    \lim_{x \to x_0} |f(x)| = \lim_{x \to x_0} f(x) = A = |A|
\end{equation}
如果$A = 0$,那么依题意有$\displaystyle\lim_{x \to x_0} f(x) = 0$,那么对任意$\epsilon > 0$,存在$\delta > 0$,当$|x - x_0| < \delta$时有
\begin{equation}
    |f(x) - 0 | < \epsilon
\end{equation}
也就是
\begin{equation}
    |f(x)| < \epsilon
\end{equation}
也就是
\begin{equation}
    |f(x)| - 0 < \epsilon
\end{equation}
也就是
\begin{equation}
    ||f(x)| - 0| < \epsilon
\end{equation}
那么依定义有$\displaystyle\lim_{x \to x_0} |f(x)| = 0$.又因为$A = 0$,所以$|A| = 0$,所以$\displaystyle\lim_{x \to x_0} |f(x)| = |A|$.

\noindent 如果$A < 0$,那么在$x_0$的一个足够小去芯邻域$U_0(x_0, \delta_0), \, \delta_0 > 0$内有$f(x) < 0$,于是有$|f(x)|=-f(x)$,并且$|A|=-a$,从而对任意$\epsilon > 0$,我们有
\begin{equation}
    |f(x)| - |A| = -f(x) - (-A) = A - f(x) \leq |A - f(x)| = |f(x) - A| 
\end{equation}
而因为$\displaystyle\lim_{x \to x_0} f(x) = A$,所以存在$\delta > 0$,使得只要$|x-x_0|<\min\{\delta_0, \delta \} \leq \delta$就有
\begin{equation}
    |f(x)|-|A| \leq |f(x) - A| < \epsilon
\end{equation}
从而依定义有$\displaystyle\lim_{x \to x_0} |f(x)| = |A|$.
\end{proof}

\noindent (2)
\begin{proof}
设$M>0$为一有限数,并且$M>|A|$,那么对于足够小的$\epsilon$将会有
\begin{equation}
0 < |A| - \epsilon < |f(x)| < |A| + \epsilon < M
\end{equation}
由题(1)结论,对任意$\epsilon > 0$,存在$\delta > 0$,使得当$|x - x_0| < \delta$时,有
\begin{equation}
    ||f(x)|-|A||<\frac{\epsilon}{2M}
\end{equation}
注意到
\begin{align}
    ||f(x)||f(x)|-|A||A|| &= ||f(x)||f(x)| - |A||f(x)| + |A||f(x)| - |A||A|| \\
    &\leq ||f(x)||f(x)| - |A||f(x)|| + ||f(x)||A|-|A||A|| \label{ieq:times}
\end{align}
应用$|A| < M, |f(x)| < M$,不等式(\ref{ieq:times})变为
\begin{align}
    ||f(x)||f(x)|-|A||A|| &\leq ||f(x)||f(x)| - |A||f(x)|| + ||f(x)||A|-|A||A|| \\
    &< M||f(x)|-|A|| + |A|||f(x)|-|A|| \\
    &<2M||f(x)|-|A|| < \epsilon
\end{align}
这就证明了$\displaystyle\lim_{x \to x_0} |f(x)|^2 = |A|^2$也就是$\displaystyle\lim_{x \to x_0} (f(x))^2 = A^2$.
\end{proof}

\noindent (3)
\begin{proof}
利用公式
\begin{equation}
    (\sqrt{f(x)}-\sqrt{A})(\sqrt{f(x)}+\sqrt{A})=f(x)-A
\end{equation}
得
\begin{equation}
    \sqrt{f(x)}-\sqrt{A} = \frac{f(x)-A}{\sqrt{f(x)}+\sqrt{A}}
\end{equation}
由于$\displaystyle\lim_{x \to x_0} f(x) = A$,所以函数$f$在$x_0$的足够小的邻域内是有界的,因而表达式$\sqrt{f(x)}+\sqrt{A}$也是有界的,不妨设这个下界为$M_1 \, (M_1 > 0)$.因为$\displaystyle\lim_{x \to x_0} f(x) = A$,所以对任意$\epsilon > 0$,存在$\delta > 0$,使得当$|x - x_0| < \delta$时有
\begin{equation}
    |f(x)-A|<M_1\epsilon
\end{equation}
所以
\begin{equation}
    |\sqrt{f(x)}-\sqrt{A}|=\bigg\lvert \frac{f(x)-A}{\sqrt{f(x)}+\sqrt{A}} \bigg\rvert < \frac{|f(x)-A|}{M_1} < \epsilon
\end{equation}
这就证明了$\displaystyle\lim_{x \to x_0} \sqrt{f(x)} = \sqrt{A}$.
\end{proof}

\noindent(4)
\begin{proof}
利用公式
\begin{equation}
    f(x) - A = ((f(x))^\frac{1}{3})^3 - (A^\frac{1}{3})^3 = ((f(x))^\frac{1}{3}-A^\frac{1}{3})((f(x))^{\frac{2}{3}} + (f(x))^\frac{1}{3} A^\frac{1}{3} + A^\frac{2}{3})
\end{equation}
得
\begin{equation}
    (f(x))^\frac{1}{3}-A^\frac{1}{3} = \frac{f(x) - A}{(f(x))^{\frac{2}{3}} + (f(x))^\frac{1}{3} A^\frac{1}{3} + A^\frac{2}{3}}
\end{equation}
由于$\displaystyle\lim_{x \to x_0} f(x) = A$,所以函数$f$在$x_0$足够小的邻域内是有界的,因而表达式$(f(x))^{\frac{2}{3}} + (f(x))^\frac{1}{3} A^\frac{1}{3} + A^\frac{2}{3}$也是有界的,不妨设该表达式的下界为$M_1 \, (M_1 > 0)$.又因为$\displaystyle\lim_{x \to x_0} f(x) = A$,所以对任意$\epsilon > 0$,都存在$\delta > 0$,使得$|x - x_0| < \delta$时有
\begin{equation}
    |f(x) - A| < M_1 \epsilon
\end{equation}
所以
\begin{equation}
    |(f(x))^\frac{1}{3}-A^\frac{1}{3}| = \bigg\lvert \frac{f(x) - A}{(f(x))^{\frac{2}{3}} + (f(x))^\frac{1}{3} A^\frac{1}{3} + A^\frac{2}{3}} \bigg\rvert < \frac{|f(x)-A|}{M_1} < \epsilon
\end{equation}
这就说明$\displaystyle\lim_{x \to x_0} \sqrt[3]{f(x)} = \sqrt[3]{A}$.
\end{proof}

\noindent 4. 用$\epsilon$-$\delta$语言证明:
\begin{table}[H]
    \centering
    \begin{tabularx}{0.8\textwidth} {  >{\raggedright\arraybackslash}X >{\raggedright\arraybackslash}X  }
       (1)~$\displaystyle\lim_{x \to 2} x^3 = 8$; & (2)~$\displaystyle\lim_{x \to 3} \frac{x-3}{x^2-9} = \frac{1}{6}$; \\ [1em]
       (3)~$\displaystyle\lim_{x \to 1} \displaystyle\frac{x^4-1}{x-1} = 4$; & (4)~$\displaystyle\lim_{x \to 0} \sqrt{1+2x} = 1$; \\ [1em]
       (5)~$\displaystyle\lim_{x \to 1^{+}} \displaystyle\frac{x-1}{\sqrt{x^2-1}} = 0$.
      \end{tabularx}
\end{table}

\noindent (1)
\begin{proof}
因为
\begin{equation}
    x^3 - 8 = x^3 - 2^3 = (x-2)(x^2 + 2x + 2^2)
\end{equation}
并且因为$\displaystyle\lim_{x \to 2} x = 2$,所以在$2$的足够小的邻域内函数$x \longmapsto x$有界,所以函数$x \longmapsto x^2+2x+2^2$也有界,不妨设函数$x \longmapsto x^2+2x+2^2$在$2$的足够小的邻域内的上界是$M \, (M > 0)$.并且注意到$\displaystyle\lim_{x \to 2} x = 2$,所以,对任意$\epsilon > 0$,存在$\delta > 0$,使得当$|x - 2| < \delta$的时候,有
\begin{equation}
    |x - 2| < \frac{\epsilon}{M}
\end{equation}
所以
\begin{equation}
    |x^3-8|=|(x-2)(x^2+2x+2^2)|=|x-2||x^2+2x+2^2|<\frac{\epsilon}{M}M=\epsilon
\end{equation}
所以$\displaystyle\lim_{x \to 2} x^3 = 8$.
\end{proof} 

\noindent (2)
由于$x \to 3$不必使得$x = 3$,所以$x-3 \neq 0$,所以
\begin{equation}
    \lim_{x \to 3} \frac{x-3}{x^2 - 9} = \lim_{x \to 3} \frac{1}{x+3}
\end{equation}
易证$\displaystyle\lim_{x \to 3} 6(x+3)$存在,于是函数$x \longmapsto 6(x+3)$在$3$的足够小的去芯邻域$B(\check{3}, \delta_0)$内有界$(\delta_0 > 0)$,设$B(\check{3}, \delta_0)$内函数$x \longmapsto 6(x+3)$的下界是$M \, (M > 0)$,于是当$0<|x-3|<\delta_0$,有
\begin{equation}
    \bigg\lvert \frac{1}{x+3}-\frac{1}{6} \bigg\rvert= \bigg\lvert \frac{3-x}{6(x+3)} \bigg\rvert<\frac{|3-x|}{M}
\end{equation}
故对任意$\epsilon > 0$,我们只需取$\delta = \epsilon M$,那么当$0<|x-3|<\min\{\delta, \delta_0 \}$,有
\begin{equation}
    \bigg\lvert \frac{x-3}{x^2-9} - \frac{1}{6} \bigg\rvert = \bigg\lvert \frac{1}{x+3} - \frac{1}{6} \bigg\rvert < \frac{|3-x|}{M} < \frac{\epsilon M}{M} = \epsilon
\end{equation}
这就证明了$\displaystyle\lim_{x \to 3} \displaystyle\frac{x-3}{x^2-9} = \displaystyle\frac{1}{6}$.

\noindent (3)
\begin{proof}
当$0 < |x-1|$时,$x-1 \neq 0$,于是
\begin{equation}
    \lim_{x \to 1} \frac{x^4 - 1}{x-1} = \lim_{x \to 1} \left( 1 + x + x^2 + x^3 \right)
\end{equation}
由于对任意$n \in \nat$,$\displaystyle\lim_{x \to 1} x^n$存在并且等于$1$,所以,对任意$\epsilon > 0$,存在$\delta > 0$,当$0 < |x - 1| < \delta$时
\begin{align}
    |x-1|< \frac{1}{3}\epsilon, \quad |x^2-1|<\frac{1}{3}\epsilon, \quad |x^3-1|<\frac{1}{3}\epsilon
\end{align}
同时成立.于是
\begin{align}
    &|1+x+x^2+x^3-4| = |x - 1 + x^2 - 1 + x^3 - 1| < |x-1| + |x^2-1| + |x^3 - 1| \\
    &< \frac{1}{3}\epsilon + \frac{1}{3}\epsilon + \frac{1}{3}\epsilon = \epsilon
\end{align}
这说明$\displaystyle\lim_{x \to 1} \left(1 + x + x^2 + x^3\right) = 4$也就是$\displaystyle\lim_{x \to 1} \displaystyle\frac{x^4-1}{x-1} = 4$.
\end{proof}

\noindent (4)
\begin{proof}
注意到
\begin{equation}
    (\sqrt{1+2x} - 1)(\sqrt{1+2x} + 1) = 2x
\end{equation}
所以,只需取$\delta = \min \{ \displaystyle\frac{1}{2}, \frac{\epsilon}{4} \}$,那么当$0 < |x-0| < \delta$时,就有
\begin{equation}
    |\sqrt{1+2x} - 1| = |\frac{2x}{\sqrt{1+2x} + 1}| \leq |2x| = 2|x| < 2\delta \leq \frac{\epsilon}{2} < \epsilon
\end{equation}
这就证明了$\displaystyle\lim_{x \to 0} \sqrt{1+2x} = 1$.
\end{proof}

\noindent (5)
\begin{proof}
变换得
\begin{equation}
    \lim_{x \to 1^+} \frac{x-1}{\sqrt{x^2-1}} = \lim_{x \to 1^+} \frac{x-1}{\sqrt{x-1}\sqrt{x+1}} = \lim_{x \to 1^+} \frac{\sqrt{x-1}}{\sqrt{x+1}}
\end{equation}
只需取$\delta = \displaystyle\frac{\epsilon^2}{4}$,那么当$0 < x - 1 < \delta$时有
\begin{equation}
    \bigg\lvert \frac{\sqrt{x-1}}{\sqrt{x+1}} \bigg\rvert = \frac{|\sqrt{x-1}|}{|\sqrt{x+1}|} < \frac{|\sqrt{x-1}|}{1} = |\sqrt{x-1}| = \sqrt{x-1} < \sqrt{\delta} = \sqrt{\displaystyle\frac{\epsilon^2}{4}} = \frac{\epsilon}{2} < \epsilon
\end{equation}
这就证明了$\displaystyle\lim_{x \to 1^+} \displaystyle\frac{x-1}{\sqrt{x^2-1}} = 0$.
\end{proof}

\noindent 5. 设
\begin{equation}
    f(x) = \begin{cases}
        x^2, & x \geq 2, \\
        -ax, & x < 2
    \end{cases}
\end{equation}
\begin{enumerate}
    \item 求$f(2+)$与$f(2-)$;
    \item 若$\displaystyle\lim_{x \to 2} f(x)$存在,$a$应取何值?
\end{enumerate}

\noindent (1)
\begin{proof}
    对任意$\epsilon > 0$,取$\delta = \min \{ 4, \displaystyle\frac{\epsilon}{16} \}$,那么对所有$x \, (0 < x - 2 < \delta)$就都有
    \begin{align}
        |x^2 - 4| &= |x-2||x+2| = |x-2||x-2 + 2 +2| \leq |x-2||x-2| + 4|x-2| \\
        &< 4|x-2| + 4|x-2| = 8|x-2| \leq 8 \delta = 8 \cdot \frac{\epsilon}{16} = \frac{\epsilon}{2} < \epsilon
    \end{align}
这就证明了$\displaystyle\lim_{x \to 2+} x^2 = 4$.
\end{proof}

\begin{proof}
设$|a| \neq 0$,则对任意$\epsilon > 0$,取$\delta = \displaystyle\frac{\epsilon}{2|a|}$,那么当$0 < 2 - x < \delta$时就有
\begin{align}
|-ax - (-2a)| = |-ax + 2a| = |a||2-x| \leq |a|\delta = |a| \cdot \frac{\epsilon}{2 |a|} = \frac{\epsilon}{2} < \epsilon
\end{align}
这就证明了$\displaystyle\lim_{x \to 2^-} -ax = -2a$.($|a| \neq 0$)
\end{proof}

\noindent 当$|a|=0$时,$f(x) = -ax = 0 \, (x < 2)$,对任意$\epsilon > 0$,取$\delta = 1$,那么当$0 < 2 - x < \delta$时有
\begin{align}
    |-ax - 0| = |0 - 0| = |0| = 0 < \epsilon
\end{align}
这说明如果$|a|=0$,那么$\displaystyle\lim_{x \to 2^-} -ax = 0$.

\noindent (2) 解:应有$\displaystyle \lim_{x \to 2^+} f(x) = \displaystyle\lim_{x \to 2^-} f(x)$.如果$|a| = 0$,那么根据上述论证,将会有$\displaystyle\lim_{x \to 2^+} f(x) = 4 \neq \displaystyle\lim_{x \to 2^-} f(x) = 0$,故$|a|\neq 0$.

\noindent 当$|a| \neq 0$时,由$\displaystyle\lim_{x \to 2^+} f(x) = \displaystyle\lim_{x \to 2^-} f(x)$得
\begin{equation}
    4 = -2a
\end{equation}
解出$a = -2$.

\noindent 6. 设$\displaystyle\lim_{x \to x_0} f(x) > a$.求证:当$x$足够靠近$x_0$但$x \neq x_0$时,$f(x) > a$.

\begin{proof}
设$\displaystyle\lim_{x \to x_0} f(x) = A$,取$\epsilon_0 = \displaystyle\frac{A-a}{2}$,此时有$\epsilon_0 > 0$.由于$\displaystyle\lim_{x \to x_0} f(x) = A$,故对于$\epsilon_0 > 0$,存在$\delta > 0$,使得使得对所有$x \, (0 < |x-x_0| \leq \delta)$都有
\begin{equation}
    |f(x)-A|<\epsilon_0
\end{equation}
将绝对值展开也就是
\begin{equation}
    A-\epsilon_0 < f(x) < A+\epsilon
\end{equation}
由于$\epsilon_0=\displaystyle\frac{A-a}{2}$,所以$a = A - 2\epsilon_0$,而$A-2\epsilon_0 < A - \epsilon_0 < f(x)$,所以$a < f(x)$.

\noindent 这就证明了当$x$在$x_0$足够小的邻域内取值时恒有$f(x) > a$.
\end{proof}

\noindent 7. 设$f(x_0 -) < f(x_0 +)$.求证:存在$\delta > 0$,使得当
\begin{equation}
    x \in (x_0 - \delta, x_0), \quad y \in (x_0, x_0 + \delta)
\end{equation}
时,有$f(x) < f(y)$.

\begin{proof}
设$A = f(x_0 -) = \displaystyle\lim_{x \to x_0^-} f(x)$,设$B = f(x_0 +) = \displaystyle\lim_{x \to x_0^+} f(x)$.令$d = B - A$,由于$f(x_0 +) > f(x_0 -)$,所以$B > A$,所以$d > 0$.取$\epsilon_0 = \displaystyle\frac{d}{3}$,那么由于$\displaystyle\lim_{x \to x_0^-} f(x) = A$,所以,存在$\delta_1 > 0$,使得当$0 < x_0 - x < \delta_1$时,有
\begin{align}
    |f(x) - A|<\epsilon_0
\end{align}
也就是
\begin{align}
    A - \epsilon_0 < f(x) < A + \epsilon_0
\end{align}
又由于$\displaystyle\lim_{x \to x_0^+} f(x) = B$,所以,存在$\delta_2 > 0$,使得对任意的$y \, (0 < y - x_0 < \delta_2)$,有
\begin{equation}
    B - \epsilon_0 < f(y) < A + \epsilon_0
\end{equation}
由于$\epsilon_0 = \displaystyle\frac{B-A}{3}$,所以
\begin{align}
    A + \epsilon = A + \frac{B-A}{3} < B - \frac{B-A}{3} = B-\epsilon_0
\end{align}
这说明$f(x) < f(y)$.
\end{proof}

\noindent 8. 设$f$在$(-\infty, x_0)$上是递增的,并且存在一个数列$\{ x_n \}$满足$x_n < x_0 \, (n = 1,2,\cdots), x_n \to x_0, \, (n \to \infty)$,且使得
\begin{equation}
    \lim_{x \to \infty} f(x_n) = A.
\end{equation}
求证:$f(x_0 -) = A$.

\begin{proof}
由于$\displaystyle\lim_{n \to \infty} f(x_n) = A$,所以数列$\{ f(x_n) \}$是一个柯西列,也就是说,对任意给定的$\epsilon > 0$,都存在$N_0(\epsilon) \in \nat$,使得对所有$p \in \nat$,都有
\begin{equation}
    |f(x_{N_0+p}) - f(x_{N_0})|<\epsilon
    \label{eq:ptoinfinity}
\end{equation}
让式(\ref{eq:ptoinfinity})中的$p$趋于无穷,我们得到
\begin{equation}
    |f(x_{N_0}) - f(x_0)|<\epsilon
\end{equation}
因为极限的唯一性,得到$f(x_0)=A$.又因为$x_{N_0}<x_0$以及$f$的单调性,得
\begin{equation}
    0 < f(x_0) - f(x_{N_0}) < \epsilon
\end{equation}
故只要取$\delta = x_0 - x_{N_0}$,那么当$0 < x_0 - x < \delta$的时候就有$x_{N_0}<x<x_0$,再利用函数$f$的单调性就可推出
\begin{equation}
    |f(x)-f(x_0)|<\epsilon
\end{equation}
这就证明了$\displaystyle\lim_{x \to x_0^-} f(x) = A$.
\end{proof}

\noindent 9. 用肯定的语气表达``当$x \to x_0$时,$f(x)$不收敛于$l$''.

\noindent 答:存在$\epsilon_0 > 0$,对任意$\delta > 0$,都存在$x$满足$|x-x_0|<\delta$并且$|f(x)-x_0|\geq\epsilon_0$.

\noindent 10. 对任何$n \in \nat$,$A_n \subset [0,1]$是有限集,且$A_i \bigcap A_j = \emptyset \; (i \neq j, \, i,j \in \nat)$.定义函数
\begin{equation}
    f(x) = \begin{cases}
        \displaystyle\frac{1}{n}, & x \in A_n , \\
        0, & x \in [0,1], \, x \not\in A_n.
    \end{cases}
\end{equation}
对任意的$x_0 \in [0,1]$,求极限$\displaystyle\lim_{x \to x_0} f(x)$.

\begin{proof}
对任意$\epsilon > 0$和$n \in \nat$,存在足够大的正整数$q_0$使得$\displaystyle\frac{1}{q_0}$,由于只有当$x \in \displaystyle\bigcap_{n=1}^{q_0} A_n$时才有$f(x) \geq \displaystyle\frac{1}{q_0}$,所以这意味着,只有有限多个$x$的取值能够使得$f(x) \geq \displaystyle\frac{1}{q_0}$,于是又存在足够大的正整数$q_1$,满足$q_1 > q_0$,并且对任意$x \in B(\check{x_0}, \displaystyle\frac{1}{q_1})$且$x \in A_n$都有$f(x) < \displaystyle\frac{1}{q_0}$,而对于$x \in B(\check{x_0}, \displaystyle\frac{1}{q_1})$依定义有$f(x) = 0$,从而,取$\delta = \displaystyle\frac{1}{q_1}$,则对于任何实数$x \in B(\check{x_0}, \delta)$,都有$0 \leq |f(x) - 0| < \epsilon$.于是这就证明了$\displaystyle\lim_{x \to x_0} f(x) = 0 \, (\forall x \in [0,1])$.
\end{proof}

\noindent 11. 计算下列极限
\begin{table}[H]
    \centering
    \begin{tabularx}{\textwidth} {  >{\raggedright\arraybackslash}X >{\raggedright\arraybackslash}X  }
       (1)~$\displaystyle\lim_{x \to 2}\displaystyle\frac{1+x-x^3}{1+x^2}$; & (2)~$\displaystyle\lim_{x \to 1}\displaystyle\frac{x^2-2x+1}{x^2-x}$; \\ [1em]
       (3)~$\displaystyle\lim_{x \to 1}\displaystyle\frac{x^m-1}{x-1}$; & (4)~$\displaystyle\lim_{x \to 1}\displaystyle\frac{x^m-1}{x^n-1}$; \\ [1em]
       (5)~$\displaystyle\lim_{x \to 0}\displaystyle\frac{\sqrt{1+x}-1}{x}$; & (6)~$\displaystyle\lim_{x \to 0}\displaystyle\frac{\sqrt{1+x}-\sqrt{1-x}}{x}$; \\ [1em]
       (7)~$\displaystyle\lim_{x \to 0}\displaystyle\frac{(1+x)^{1/m}-1}{x}$; & (8)~$\displaystyle\lim_{x \to 1}\displaystyle\frac{x+x^2+\cdots+x^m-m}{x-1}$.
      \end{tabularx}
\end{table}

\noindent (1) 解:
\begin{align}
    \text{原式} &= \displaystyle\frac{\displaystyle\lim_{x \to 2} 1+x-x^3}{\displaystyle\lim_{x \to 2} 1+x^2} = \displaystyle\frac{-5}{5} = -1.
\end{align}

\noindent (2) 解:
\begin{align}
    \text{原式} &= \lim_{x \to 1} \displaystyle\frac{(x-1)^2}{x(x-1)} = \lim_{x \to 1} \displaystyle\frac{x-1}{x} = \displaystyle\frac{\displaystyle\lim_{x \to 1} x-1}{\displaystyle\lim_{x \to 1} x} = \frac{0}{1} = 0.
\end{align}

\noindent (3) 解:当$m=0$时
\begin{equation}
    \frac{x^m-1}{x-1} = \frac{1-1}{x-1} = 0 \quad (x \neq 1)
\end{equation}
此时$\displaystyle\lim_{x \to 1} \displaystyle\frac{x^m-1}{x-1}=0$.

\noindent 当$m$取正整数时
\begin{align}
    \lim_{x \to 1}\displaystyle\frac{x^m-1}{x-1} = \displaystyle\lim_{x \to 1} 1+x+\cdots+x^{m-1} = \sum_{i=0}^{m-1} \lim_{x \to 1} x^i = \sum_{i=0}^{m-1} 1 = m.
\end{align}

\noindent 当$m$取负整数时
\begin{align}
    \lim_{x \to 1}\displaystyle\frac{x^m-1}{x-1} &= \lim_{x \to 1} \displaystyle\frac{\displaystyle\frac{1}{x^{-m}}-1}{x-1} = - \lim_{x \to 1} \displaystyle\frac{x^{-m}-1}{x^{-m}(x-1)} \\
    &=- \lim_{x \to 1} \displaystyle\frac{1}{x^{-m}} \frac{x^{-m}-1}{x-1} = -\left( \lim_{x \to 1} \displaystyle\frac{1}{x^{-m}} \right) \left( \lim_{x \to 1} \displaystyle \frac{x^{-m}-1}{x-1} \right) \\
    &= - 1 \cdot (-m)  = m .
\end{align}

\noindent (4) 解:当$n=0$时,$x^n-1 = 1 - 1 = 0$,表达式$\displaystyle\frac{x^m-1}{x^n-1}$无意义.假设$n , m \in \integer, n \neq 0$.
\begin{align}
    \lim_{x \to 1} \frac{x^m-1}{x^n-1} = \frac{\displaystyle\lim_{x \to 1} x^m-1}{\displaystyle\lim_{x \to 1} x^n-1} = \frac{m}{n}.
\end{align}

\noindent (5) 解:

\begin{align}
    \lim_{x \to 0} \frac{\sqrt{1+x}-1}{x} &= \lim_{x \to 0} \frac{\sqrt{1+x} - 1}{(\sqrt{1+x} - 1)(\sqrt{1+x}+1)} = \lim_{x \to 0} \frac{1}{\sqrt{1+x}+1} \\
    &= \frac{\displaystyle\lim_{x \to 0} 1}{\displaystyle\lim_{x \to 0} \sqrt{1+x} + 1} = \frac{1}{2}.
\end{align}

\noindent (6) 解:

\begin{align}
    \lim_{x \to 0} \frac{\sqrt{1+x}-\sqrt{1-x}}{x} &= \lim_{x \to 0} \frac{\sqrt{1+x}-1}{x} - \lim_{x \to 0} \frac{\sqrt{1-x}-1}{x} \\
    &= \lim_{x \to 0} \frac{1}{\sqrt{1+x}+1} - \lim_{x \to 0} - \frac{1}{\sqrt{1-x}+1} \\
    &= \frac{1}{2} - (-\frac{1}{2}) = 1.
\end{align}

\noindent (7) 解:当$m$取正整数时

\begin{align}
    \lim_{x \to 0} \frac{(1+x)^{1/m}-1}{x} = \lim_{x \to 0} \frac{(1+x)^{1/m}-1}{1+x-1} = \lim_{x \to 0} \frac{(1+x)^{1/m}-1}{(1+x)^{m/m}-1}
\end{align}
令$q = 1+x$,得到
\begin{align}
    \text{原式} &= \lim_{q \to 1} \frac{q-1}{q^m-1} = \lim_{q \to 1} \frac{q-1}{(q-1)(q^{m-1}+\cdots+1)} = \lim_{q \to 1} \frac{1}{q^{m-1}+\cdots+1} = \frac{1}{m}
\end{align}

\noindent 当$m$取负整数时

\begin{align}
    \text{原式} &= \lim_{x \to 0} \frac{(1+x)^{-\displaystyle\frac{1}{-m}}-1}{1+x-1} = \lim_{x \to 0} \frac{1 - (1+x)^{\displaystyle\frac{1}{-m}}}{(1+x)^{\displaystyle\frac{1}{-m}} (1+x-1)} \\
    &= - \lim_{x \to 0} \frac{(1+x)^{\displaystyle\frac{1}{-m}}-1}{(1+x)-1} \lim_{x \to 0} \frac{1}{(1+x)^{\displaystyle\frac{1}{-m}}} = - \left( - \frac{1}{m} \right) \cdot 1 = \frac{1}{m}.
\end{align}

\noindent (8) 解:设$m$为正整数

\begin{align}
    \text{原式} &= \lim_{x \to 1} \frac{x-1 + x^2-1 + \cdots + x^m - 1}{x-1} \\
    &= \lim_{x \to 1} \frac{x-1}{x-1} + \lim_{x \to 1} \frac{x^2-1}{x-1} + \cdots + \lim_{x \to 1} \frac{x^m-1}{x-1} \\
    &= 1 + 2 + \cdots + m = \frac{m(m+1)}{2}.
\end{align}

\noindent 12. 求下列极限:
\begin{table}[H]
    \centering
    \begin{tabularx}{\textwidth} {  >{\raggedright\arraybackslash}X >{\raggedright\arraybackslash}X  }
       (1)~$\displaystyle\lim_{x \to 0} \displaystyle\frac{\sin \, ax}{\sin \, bx} \; (b \neq 0)$; & (2)~$\displaystyle\lim_{x \to 0}\displaystyle\frac{x^2}{1-\cos \, x}$; \\ [1em]
       (3)~$\displaystyle\lim_{x \to 0} \frac{\sin \, \sin \, x}{x}$; & (4)~$\displaystyle\lim_{x \to 0} \displaystyle\frac{\tan \, x}{x}$; \\ [1em]
       (5)~$\displaystyle\lim_{h \to 0} \sin \, \left(x+h\right)$; & (6)~$\displaystyle\lim_{h \to 0} \displaystyle\frac{\sin \, \left(x+h\right)-\sin \, x}{h}$; \\ [1em] 
       (7)~$\displaystyle\lim_{x \to 0} \displaystyle\frac{1-\cos \, x \cdot \cos \, 2x \cdots \cos \, nx}{x^2}$; & (8)~$\displaystyle\lim_{n \to \infty} \cos \, \displaystyle\frac{x}{2} \, \cdot \cos \, \displaystyle\frac{x}{4} \cdots \cos \, \displaystyle\frac{x}{2^n}$.
      \end{tabularx}
\end{table}

\noindent (1) 解:
\begin{align}
    \text{原式} &= \lim_{x \to 0} \frac{\sin \, ax}{\sin \, bx} = \lim_{x \to 0} \frac{\sin \, ax}{ax} \cdot \frac{\displaystyle\frac{a}{b} \cdot bx}{\sin \, bx} = \lim_{x \to 0}\frac{\sin \, ax}{ax} \cdot \frac{a}{b} \lim_{x \to 0} \frac{bx}{\sin \, bx} \\
    &= 1 \cdot \frac{a}{b} \cdot 1 = \frac{a}{b}.
\end{align}

\noindent (2) 解:
\begin{align}
    \text{原式} &= \lim_{x \to 0} \frac{x^2}{\displaystyle\frac{1-(\cos \, x)^2}{1+\cos \, x}} = \lim_{x \to 0} \frac{x^2 (1+\cos \, x)}{(\sin \, x)^2} = \lim_{x \to 0} \left(\frac{x}{\sin \, x}\right)^2 \cdot \lim_{x \to 0} 1 + \cos \, x \\
    &= \lim_{x \to 0} \frac{x}{\sin \, x} \cdot \lim_{x \to 0} \frac{x}{\sin \, x} \cdot \lim_{x \to 0} 1+\cos \, x= 1 \cdot 1 \cdot 2 = 2.
\end{align}

\noindent (3) 解:
\begin{align}
    \text{原式} &= \lim_{x \to 0} \frac{\sin \, \sin \, x}{x} = \lim_{x \to 0} \frac{\sin \, \sin \, x}{\sin \, x} \cdot \frac{\sin \, x}{x} = \lim_{x \to 0} \frac{\sin \, \sin \, x}{\sin \, x} \cdot \lim_{x \to 0} \frac{\sin \, x}{x} \\
    &= 1 \cdot 1 = 1.
\end{align}

\noindent (4) 解:
\begin{align}
    \text{原式} &= \lim_{x \to 0} \frac{\tan \, x}{x} = \lim_{x \to 0} \frac{\sin \, x}{x \cdot \cos \, x} = \lim_{x \to 0} \frac{\sin \, x}{x} \cdot \lim_{x \to 0} \frac{1}{\cos \, x} = 1 \cdot 1 = 1.
\end{align}

\noindent (5) 解:
\begin{align}
    \text{原式} &= \lim_{h \to 0} \sin \, x \cdot \cos \, h + \sin \, h \cdot \cos \, x \\
    &= \lim_{h \to 0} \sin \, x \cdot \cos \, h + \lim_{h \to 0} \sin \, h \cdot \cos \, x \\
    &= \sin \, x \lim_{h \to 0} \cos \, h + \cos \, x \lim_{h \to 0} \sin \, h \\
    &= \sin \, x + \cos \, x \cdot 0 = \sin \, x.
\end{align}

\noindent (6) 解:
\begin{align}
    \text{原式} &= \lim_{h \to 0} \frac{\sin \, x \left(\cos \, h - 1\right) + \sin \, h \cdot \cos \, x}{h} \\
    &= \lim_{h \to 0} \frac{\sin \, x \cdot \left(-2 (\sin \, \displaystyle\frac{h}{2} )^2 \right) + \sin \, h \cdot \cos \, x}{h} \\
    &= \lim_{h \to 0} \sin \, x \cdot \frac{-2 \left(\sin \, \displaystyle\frac{h}{2}\right)^2}{h} + \lim_{h \to 0} \frac{\sin \, h \cdot \cos \, x}{h} \\
    &= - \sin \, x \lim_{h \to 0} \frac{\sin \, \displaystyle\frac{h}{2}}{\displaystyle\frac{h}{2}} \cdot \left(\sin \, \displaystyle\frac{h}{2}\right) + \cos \, x \lim_{h \to 0} \frac{\sin \, h}{h} \\
    &= - \sin \, x \cdot \lim_{h \to 0} \displaystyle\frac{\sin \, \displaystyle\frac{h}{2}}{\displaystyle\frac{h}{2}} \cdot \lim_{h \to 0} \sin \, \displaystyle\frac{h}{2} + \cos \, x \cdot \lim_{h \to 0} \displaystyle\frac{\sin \, h}{h} \\
    &= - \sin \, x \cdot 1 \cdot 0 + \cos \, x \cdot 1 = \cos \, x.
\end{align}

\noindent (7) 解:当$n=1$时,由已知结论,极限为$\displaystyle\frac{1}{2}$.当$n=2$,极限为
\begin{align}
\text{原式} &= \lim_{x \to 0} \displaystyle\frac{1-\cos \, x \cos \, 2x}{x^2} = \lim_{x \to 0} \frac{1 - \cos \, x + (1 - \cos \, x \cos \, 2x) - (1 - \cos \, x)}{x^2} \\
&= \lim_{x \to 0} \frac{1 - \cos \, x + \cos \, x (1 - \cos \, 2x)}{x^2} = \lim_{x \to 0} \frac{1 - \cos \, x}{x^2} + \lim_{x \to 0} \cos \, x \cdot \lim_{x \to 0} \frac{1-\cos \, 2x}{x^2} \\
&= \frac{1}{2} + 1 \cdot \lim_{x \to 0} \frac{1-\cos \, 2x}{x^2} = \frac{1}{2} + \lim_{x \to 0} \frac{1-\cos \, 2x}{x^2} = \frac{1}{2} + \lim_{x \to 0} \left( 4 \cdot \frac{1-\cos \, 2x}{4x^2} \right) \\
&= \frac{1}{2} + 4 \cdot \lim_{x \to 0} \cdot \frac{1-\cos \, 2x}{4x^2} = \frac{1}{2} + 4 \cdot \lim_{x \to 0} \cdot \frac{1-\cos \, 2x}{(2x)^2} = \frac{1}{2} + 4 \cdot \frac{1}{2} = \frac{5}{2}
\end{align}

\noindent 当$n=3$,极限为
\begin{align}
    \text{原式} &= \lim_{x \to 0} \frac{1-\cos \, x \cos \, 2x \cos \, 3x}{x^2} \\
    &= \lim_{x \to 0} \frac{1-\cos \, x \cos \, 2x + (1-\cos \, x \cos \, 2x \cos \, 3x) - (1-\cos \, x \cos \, 2x)}{x^2} \\
    &= \lim_{x \to 0} \frac{1-\cos \, x \cos \, 2x + \cos \, x \cos \, 2x(1-\cos \, 3x)}{x^2} \\
    &= \lim_{x \to 0} \frac{1-\cos \, x \cos \, 2x}{x^2} + \lim_{x \to 0} \cos \, x \cos \, 2x \cdot \lim_{x \to 0} \frac{1-\cos \, 3x}{x^2} \\
    &= \frac{5}{2} + 1 \cdot \lim_{x \to 0} \left( 9 \cdot \frac{1-\cos \, 3x}{9 x^2} \right) = \frac{5}{2} + 9 \cdot \lim_{x \to 0} \frac{1-\cos \, 3x}{(3x)^2} \\
    &= \frac{5}{2} + 9 \cdot \frac{1}{2} = 7
\end{align}
我们注意到这样的规律,记$L(n) = \displaystyle\lim_{x \to 0} \displaystyle\frac{1-\cos \, x \cos \, 2x \cdots \cos \, nx}{x^2}, \; (n \in \nat)$,则有
\begin{align}
L(1) &= \frac{1}{2} \\
L(2) &= \frac{1}{2} + 2^2 \cdot \frac{1}{2} = \frac{5}{2} \\
L(3) &= \frac{1}{2} + 2^2 \cdot \frac{1}{2} + 3^2 \cdot \frac{1}{2} = 7
\end{align}
由此受到启发:猜测对于一般的$n \in \nat$都有$L(n) = \displaystyle\frac{1}{2} + 2^2 \cdot \displaystyle\frac{1}{2} + 3^2 \cdot \displaystyle\frac{1}{2} + \cdots + n^2 \cdot \displaystyle\frac{1}{2}$成立.

\begin{proof}
采用数学归纳法,假设对于某一个$k \in \nat$,当$n=k$时,$L(n) = \displaystyle\frac{1}{2} \sum_{i=1}^n i^2$成立.那么当$n$取$n=k+1$时
\begin{align}
    L(k+1) &= \lim_{x \to 0} \frac{1-\cos \, x \cos \, 2x \cdots \cos \, (k+1)x}{x^2} \\
    &= \lim_{x \to 0} \frac{1-\cos \, x \cos \, 2x \cdots \cos \, kx + \left(\displaystyle\prod_{t=1}^{k} \cos \, tx \right)\left(1-\cos \, (k+1)x\right)}{x^2} \\
    &= L(k) + \left(\lim_{x \to 0} \displaystyle\prod_{t=1}^{k} \cos \, tx \right) \cdot \lim_{x \to 0} \frac{1-\cos \, (k+1)x}{x^2} \\
    &= L(k) + 1 \cdot \lim_{x \to 0} \frac{1-\cos \, (k+1)x}{x^2} \\
    &= L(k) + (k+1)^2 \lim_{x \to 0} \frac{1-\cos \, (k+1)x}{((k+1)x)^2} = L(k) + (k+1)^2 \cdot \frac{1}{2} = \frac{1}{2} \sum_{i=1}^{k+1} i^2.
\end{align}
那么根据数学归纳法原理,$L(n) = \displaystyle\frac{1}{2}\displaystyle\sum_{i=1}^n i^2$对一切$n \in \nat$成立.
\end{proof}

\noindent (8) 解:反复对$\sin \, x$应用正弦二倍角公式,我们发现:
\begin{align}
    \sin \, x & = 2 \cos \, \frac{x}{2} \sin \, \frac{x}{2} \\
    &= 4 \cos \, \frac{x}{2} \cos \, \frac{x}{4} \sin \, \frac{x}{4} \\
    &= 8 \cos \, \frac{x}{2} \cos \, \frac{x}{4} \cos \, \frac{x}{8} \sin \, \frac{x}{8} \\
    &\cdots
\end{align}
由此受到启发,猜测对一般的$n \in \nat$,有$\sin \, x = 2^n \sin \, \displaystyle\frac{x}{2^n} \displaystyle\prod_{i=1}^n \cos \, \displaystyle\frac{x}{2^i}$成立.
\begin{proof}
当$n=1$时,依正弦二倍角公式命题成立.现假设对某个$k \in \nat$,命题成立,那么
\begin{align}
    \sin \, x &= 2^{k} \sin \, \displaystyle\frac{x}{2^k} \prod_{i=1}^k \cos \, \frac{x}{2^i} \\
    &= 2^{k} \cdot 2 \sin \, \left( 2 \cdot \frac{x}{2^{k+1}} \right) \prod_{i=1}^k \cos \, \frac{x}{2^i} \\
    &= 2^{k+1} \sin \, \frac{x}{2^{k+1}} \cos \, \frac{x}{2^{k+1}} \prod_{i=1}^{k} \cos \, \frac{x}{2^i} \\
    &= 2^{k+1} \sin \, \frac{x}{2^{k+1}} \prod_{i=1}^{k+1} \cos \, \frac{x}{2^{i}}
\end{align}
那么依数学归纳法原理,对任意$n \in \nat$命题都成立.
\end{proof}

\noindent 利用上述结论,得到
\begin{align}
    &\mathrel{\phantom{=}} \lim_{n \to \infty} \cos \, \frac{x}{2} \cos \, \frac{x}{4} \cdots \cos \, \frac{x}{2^n} \\
    &= \lim_{n \to \infty} \prod_{i=1}^n \cos \, \frac{x}{2^i} = \lim_{n \to \infty} \frac{\sin \, x}{2^{n} \sin \, \displaystyle\frac{x}{2^n}} = \lim_{n \to \infty} \frac{\sin \, x}{x} \cdot \frac{\displaystyle\frac{x}{2^n}}{\sin \, \displaystyle\frac{x}{2^n}} \\
    &= \frac{\sin \, x}{x} \cdot \lim_{n \to \infty} \displaystyle \frac{1}{\displaystyle\frac{\sin \, \displaystyle\frac{x}{2^n}}{\displaystyle\frac{x}{2^n}}} = \frac{\sin \, x}{x} \cdot \frac{\displaystyle\lim_{n \to \infty} 1}{\displaystyle\lim_{n \to \infty} \displaystyle\frac{\sin \, \displaystyle\frac{x}{2^n}}{\displaystyle\frac{x}{2^n}}} = \displaystyle\frac{\sin \, x}{x} \cdot \displaystyle\frac{1}{1} = \frac{\sin \, x}{x}.
\end{align}
\end{document}