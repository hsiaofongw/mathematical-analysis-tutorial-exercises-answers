% 练习题1.7
\exercise
1. 对任意给定的$\epsilon > 0$,存在$N \in \mathbb{N}^\star$,当$n > N$时,有
\begin{equation}
    |a_n - a_N|<\epsilon
\end{equation}
问$\{a_n\}$是不是基本列?

思路:在实数轴上,所有$a_N$附件的项都落在以它为中心的半径不超过$\epsilon$的一个范围内,那么,假设$a_n, a_m$也是在这个范围内,$a_n$与$a_m$的距离只会更加近,因为这个$\epsilon$可以任意小,从而这个以$a_N$为中心的范围可以是任意地``拥挤''.由此我们猜测$\{a_n\}$是基本列,去证$|a_m - a_n|<\epsilon$.

\begin{proof}
设$m, n>N$,那么由题设,
\begin{equation}
    |a_m - a_N|<\epsilon, \; |a_n - a_N| <\epsilon
\end{equation}
也就是
\begin{align}
    a_N - \epsilon < a_m < a_N + \epsilon \\
    a_N - \epsilon < a_n < a_N + \epsilon 
\end{align}
从而
\begin{equation}
    |a_m - a_n| < 2\epsilon
\end{equation}
由于$2\epsilon$可以是任意的正数,所以按定义,$\{a_n\}$是基本列.
\end{proof}

2. (1)数列$\{a_n\}$满足
\begin{equation}
    |a_{n+p}-a_n|\leq\frac{p}{n}
\end{equation}
且对一切$n,p\in\mathbb{N}^\star$成立,问$\{ a_n\}$是不是基本列?

(2) 当$|a_{n+p} - a_n| \leq p/n^2$时,上述结论又如何?

思路:对于题(1),当$n\to\infty$时$\displaystyle\frac{p}{n}$显然趋于$0$,但我们从直觉上能够感觉到这个趋于$0$的速度还不够快,由此猜想$\{a_n\}$应该不是基本列,可以去举反例.
\begin{proof}
(1) 考虑
\begin{equation}
    a_n = 1 + \frac{1}{2} + \frac{1}{3} + \cdots + \frac{1}{n} 
\end{equation}
于是
\begin{align}
    a_{n+p} - a_n &= \frac{1}{n+1} + \frac{1}{n+2} + \cdots + \frac{1}{n+p}\\
    &\leq \frac{1}{n+1} + \frac{1}{n+1} + \cdots + \frac{1}{n+1} \\
    &= \frac{p}{n+1} < \frac{p}{n}
\end{align}
数列$\{a_n\}$满足$|a_{n+p}-a_n|\leq \displaystyle\frac{p}{n}, \, (n,p\in\mathbb{N}^\star)$,但是$\{a_n\}$显然不是基本列.

(2) 由题设,有
\begin{align}
    |a_{n+p} - a_n| &\leq |a_{n+1} - a_n| + |a_{n+2} - a_{n+1}| + \cdots + |a_{n+p} - a_{n+p-1}| \\
    &\leq \frac{1}{n^2} + \frac{1}{(n+1)^2} + \cdots + \frac{1}{(n+p-1)^2}
\end{align}
令
\begin{equation}
    b_n = 1 + \frac{1}{2^2} + \frac{1}{3^2} + \cdots + \frac{1}{n^2} = \sum_{i=1}^n \frac{1}{i^2}
\end{equation}
于是
\begin{equation}
    |a_{n+p} - a_n| \leq b_{n+p-1} - b_n = |b_{n+p-1} - b_n| < |b_{n+p} - b_n|
\end{equation}
显然,数列$\{b_n\}$是收敛的,因此数列$\{ b_n \}$是基本列,因此,对任意$\epsilon > 0$,存在$N \in \mathbb{N}^\star$,使得当$n > N$时,
\begin{equation}
    |a_{n+p} - a_n| < |b_{n+p} - b_n| < \epsilon
\end{equation}
对任意$n, p \in \mathbb{N}^\star$成立.所以,由定义,$\{ a_n \}$是基本列.
\end{proof}

3. 证明下列数列收敛:

\begin{table}[H]
    \centering
    \begin{tabularx}{\textwidth} { >{\raggedright\arraybackslash}X }
    (1) $\displaystyle a_n = 1-\frac{1}{2^2}+\frac{1}{3^2}-\cdots+(-1)^{n-1}\frac{1}{n^2}\, (n\in\mathbb{N}^\star)$; \\ [0.7em]
    (2) $\displaystyle b_n = a_0 + a_1 q + \cdots + a_n q^n \, (n \in \mathbb{N}^\star)$,其中$\{a_0,a_1,\cdots\}$为一有界数列,$|q|<1$; \\ [0.7em]
    (3) $a_n = \displaystyle \sin x + \frac{\sin 2x}{2^2} + \cdots + \frac{\sin nx}{n^2}\, (n \in \mathbb{N}^\star, x \in \mathbb{R})$; \\ [0.7em]
    (4) $a_n = \displaystyle \frac{\sin 2x}{2(2+\sin 2x)} + \frac{\sin 3x}{3(3+\sin 3x)} + \cdots + \frac{\sin nx}{n(n+\sin nx)} \, (n\in\mathbb{N}^\star, x \in \mathbb{R})$.
    \end{tabularx}
\end{table}

\begin{proof}
(1). 将$a_n$写成
\begin{equation}
    a_n = \sum_{i=1}^n (-1)^{i-1} \frac{1}{i^2}
\end{equation}
于是
\begin{align}
    |a_{n+p}-a_n| = \Bigg\lvert\sum_{i=n+1}^{n+p} (-1)^{i-1}\frac{1}{i^2} \Bigg\rvert < \Bigg\lvert \sum_{i=n+1}^{n+p} \frac{1}{i^2} \Bigg\rvert
\end{align}
令
\begin{equation}
    b_n = \sum_{i=1}^n \frac{1}{i^2}
\end{equation}
于是
\begin{equation}
    |a_{n+p}-a_n|<b_{n+p} - b_n =|b_{n+p}-b_n|
\end{equation}
并且数列$\{ b_n \}$是基本列,从而对任意$\epsilon > 0$,存在$N \in \mathbb{N}^\star$,使得当$n > N$时,
\begin{equation}
    |a_{n+p}-a_n|<|b_{n+p}-b_n|<\epsilon
\end{equation}
对任意$n,p \in \mathbb{N}^\star$都成立.于是,按定义,数列$\{a_n\}$是基本列,从而数列$\{a_n\}$收敛.
\end{proof}

\begin{proof}
(2). 设$l$是$\{a_n\}$的下界,$u$是$\{a_n\}$的上界,令$a = \max\{ |l|, |u| \}$:
\begin{align}
    |b_{n+p} - b_n| &= \Bigg\lvert \sum_{i=0}^{n+p} a_i q^i - \sum_{i=0}^{n} a_i q^i \Bigg\rvert \\
    &= \Bigg\lvert \sum_{i=n+1}^{n+p} a_i q^i \Bigg\rvert < a |q|^{n+1} \sum_{i=0}^{p-1} |q|^i = |q|^{n+1} \frac{a(1-|q|^p)}{1-|q|} \\
    &< |q|^{n+1} \frac{a}{1-|q|}
\end{align}
那么,对任意$\epsilon > 0$,我们可以取
\begin{equation}
    N = \lceil \log_{|q|} \left(\frac{\epsilon(1-|q|)}{a}\right) - 1\rceil
\end{equation}
则当$n > N$时
\begin{align}
    |b_{n+p}-b_n| &< |q|^{n+1}\frac{a}{1-|q|} < |q|^{N+1} \frac{a}{1-|q|} \\
    &< \frac{\epsilon(1-|q|)}{a} \frac{a}{1-|q|} = \epsilon
\end{align}
对任意$n, p\in\mathbb{N}^\star$都成立,于是由定义,$\{b_n\}$是一个基本列,于是$\{b_n\}$收敛.
\end{proof}
\begin{proof}
(3). 设$n,p\in\mathbb{N}^\star$:
\begin{align}
    |a_{n+p}-a_n| &= \Bigg\lvert \sum_{i=1}^{n+p} \frac{\sin ix}{i^2} - \sum_{i=1}^n \frac{\sin ix}{i^2} \Bigg\rvert \\
    &= \Bigg\lvert \sum_{i=n+1}^{n+p} \frac{\sin ix}{i^2} \Bigg\rvert \\
    &< \sum_{i=n+1}^{n+p} \frac{1}{i^2}
\end{align}
令
\begin{equation}
    b_n = \sum_{i=1}^n \frac{1}{i^2}
\end{equation}
易知数列$\{b_n\}$是收敛数量,故$\{b_n\}$是基本列,于是,对任意正数$\epsilon > 0$,存在$N \in \mathbb{N}^\star$,当$n>N$时,
\begin{equation}
    |b_{n+p}-b_{n}|<\epsilon
\end{equation}
对任意$n,p\in\mathbb{N}^\star$成立,又由于
\begin{equation}
    |a_{n+p}-a_n|<\sum_{i=n+1}^{n+p} \frac{1}{i^2}= |b_{n+p}-b_n|
\end{equation}
所以同样也有
\begin{equation}
    |a_{n+p}-a_n|<|b_{n+p}-b_n|<\epsilon
\end{equation}
对任意$n,p\in\mathbb{N}^\star$成立.所以由定义$\{a_n\}$是基本列,所以$\{a_n\}$收敛.
\end{proof}
\begin{proof}
(4). 将$a_n$写作
\begin{equation}
    a_n = \sum_{i=2}^n \frac{\sin ix}{i(i+\sin ix)}
\end{equation}
于是,对于$n,p\in\mathbb{N}^\star, n \geq 2$,有
\begin{align}
    |a_{n+p}-a_n| &= \Bigg\lvert \sum_{i=2}^{n+p}\frac{\sin ix}{i(i+\sin ix)} - \sum_{i=2}^n \frac{\sin ix}{i(i+\sin ix)} \Bigg\rvert \\
    &= \Bigg\lvert \sum_{i=n+1}^{n+p} \frac{\sin ix}{i(i+\sin ix)} \Bigg\rvert = \Bigg\lvert \sum_{i=n+1}^{n+p} \frac{\sin ix}{i^2 + i \sin ix} \Bigg\rvert \\
    &< \sum_{i=n+1}^{n+p} \bigg\lvert \frac{\sin ix}{i^2 + i \sin ix} \bigg\rvert < \sum_{i=n+1}^{n+p}\frac{1}{i^2-i}  = \sum_{i=n+1}^{n+p} \frac{1}{i(i-1)} \\
    &< \sum_{i=n+1}^{n+p}\frac{1}{(i-1)^2}
\end{align}
令
\begin{equation}
    b_n = 1+\frac{1}{2^2}+\cdots+\frac{1}{n^2}=\sum_{i=1}^n\frac{1}{i^2}
\end{equation}
则
\begin{equation}
    |a_{n+p}-a_n|<\sum_{i=n+1}^{n+p} = |b_{n+p-1}-b_{n}|
\end{equation}
由于数列$\{b_n\}$收敛,所以$\{b_n\}$是基本列,所以,对任意$\epsilon>0$,存在$N\in\mathbb{N}^\star$,使得当$n>N$时,
\begin{equation}
    |a_{n+p}-a_n|<|b_{n+p-1}-b_n|<\epsilon
\end{equation}
对任意$p\in\mathbb{N}^\star$都成立,从而按照定义,$\{a_n\}$是基本列,于是$\{a_n\}$收敛.
\end{proof}

4. 设数列
\begin{equation}
    \{|a_2-a_1|+|a_3-a_2|+\cdots+|a_n-a_{n-1}|\}
\end{equation}
有界,求证$\{a_n\}$收敛.
\begin{proof}
令
\begin{equation}
    b_n = |a_2-a_1|+|a_3-a_2|+\cdots+|a_n-a_{n-1}| =\sum_{i=2}^n |a_{i}-a_{i-1}|
\end{equation}
于是
\begin{align}
    b_{n+1}-b_n &= \sum_{i=2}^{n+1} |a_i - a_{i-1}| -\sum_{i=2}^{n} |a_i-a_{i-1}| \\
    &= |a_{n+1}-a_{n}| > 0
\end{align}
从而,数列$\{b_n\}$,也就是数列$\{|a_2-a_1|+\cdots+|a_{n}-a_{n-1}|\}$是单调递增的,又由题设知数列$\{b_n\}$有界,所以数列$\{b_n\}$收敛.

现在我们来证明数列$\{a_n\}$是一个基本列:
\begin{align}
    |a_{n+p}-a_n| &\leq |a_{n+1}-a_n|+|a_{n+2}-a_{n+1}|+\cdots+|a_{n+p}-a_{n+p-1}| \\
    &= b_{n+p} - b_{n-1} = b_{n+p} - b_n + b_{n} - b_{n-1} = |b_{n+p}-b_n|+|b_n-b_{n-1}|
\end{align}
由于数列$\{b_n\}$收敛,所以$\{b_n\}$是基本列,所以,对任意$\epsilon > 0$,存在$N\in\mathbb{N}^\star$,使得当$n > N$时
\begin{equation}
    |b_{n+p}-b_n|<\epsilon, \; |b_n-b_{n-1}|<\epsilon
\end{equation}
对任意$p \in \mathbb{N}^\star$都成立,而又因为$|a_{n+p}-a_n|<|b_{n+p} - b_n|+|b_{n}-b_{n-1}|$,所以
\begin{equation}
    |a_{n+p}-a_n|<|b_{n+p}-b_n|+|b_n-b_{n-1}|<\epsilon+\epsilon=2\epsilon
\end{equation}
对任意$p\in\mathbb{N}^\star$都成立,所以,由定义知$\{a_n\}$是基本列,所以$\{a_n\}$收敛.
\end{proof}
5. 用精确语言表述``数列$\{a_n\}$不是基本列''.

答:

一个数列$\{ a_n \}$不是基本列,如果存在$\epsilon_0 > 0$,使得对任意$N \in \nat$,存在$n, m > N$,使得$|a_m - a_n| \geq \epsilon_0$.

6. 设$a_n \in [a,b]\,(n\in\mathbb{N}^\star)$.证明:如果$\{a_n\}$发散,则$\{a_n\}$必有两个子列收敛于不同的数.
\begin{proof}
任取$\{ a_n \}$的两个收敛子列,分别记为$\{ a_{i_n} \}$和$\{ a_{j_n} \}$,由于$\{ a_n \}$发散,所以$\{ a_n \}$不是基本列,那么就存在$\epsilon_0 > 0$,使得对任意$N \in \nat$,都存在$n, m > N$,使得$|a_n - a_m| \geq \epsilon_0$,于是就存在$i_n, j_m > N$,使得$|a_{i_n} - a_{j_m}| \geq \epsilon_0$,这就说明了$\{ a_{i_n} \}$和$\{ a_{j_m} \}$分别收敛到不同的极限.
\end{proof}
