% 练习题1.6
\exercise

1. 求下列极限
\begin{table}[H]
    \centering
    \begin{tabularx}{0.8\textwidth} {  >{\raggedright\arraybackslash}X >{\raggedright\arraybackslash}X  }
       (1)$\displaystyle \lim_{n\to\infty} \left(1+\frac{1}{n-2}\right)^n$; &  (2)$\displaystyle \lim_{n \to \infty} \left(1-\frac{1}{n+3}\right)^n$; \\ [1.5em]
       (3)$\displaystyle \lim_{n \to \infty} \left(\frac{1+n}{2+n}\right)^n$; & (4) $\displaystyle \lim_{n \to \infty} \left(1+\frac{3}{n}\right)^n$; \\ [1.5em]
       (5)$\displaystyle \lim_{n \to \infty} \left(1+\frac{1}{2n^2}\right)^{4n^2}$.
      \end{tabularx}
\end{table}
(1). 解:令$m = n-2$,则
\begin{align}
    \lim_{n \to \infty} \left(1+\frac{1}{n-2}\right)^n &= \lim_{m \to \infty} \left(1+ \frac{1}{m}\right)^{m+2} \\
    &= \lim_{m \to \infty} \left(1+\frac{1}{m}\right)^m \left(1+\frac{1}{m}\right)^2 \\
    &= \lim_{m \to \infty} \left(1+\frac{1}{m}\right)^m \lim_{m \to \infty} \left(1+\frac{1}{m}\right)^2 \\
    &= \mathrm{e} \cdot 1 \\
    &= \mathrm{e}
\end{align}
(2). 解:令$m=n+3$,则
\begin{align}
    \lim_{n \to \infty} \left(1-\frac{1}{n+3}\right)^n &= \lim_{m \to \infty} \left(1-\frac{1}{m}\right)^{m+2} \\
    &= \lim_{m \to \infty} \left(1-\frac{1}{m}\right)^m \lim_{m \to \infty} \left(1-\frac{1}{m}\right)^2 \\
    &= \lim_{m \to \infty} \left(1-\frac{1}{m}\right)^m \\
    &= \mathrm{e}^{-1}
\end{align}
(3). 解:
\begin{align}
    \lim_{n \to \infty} \left(\frac{1+n}{2+n}\right)^n &= \lim_{n \to \infty} \left(1 - \frac{1}{2+n}\right)^n
\end{align}
令$m = 2+n$,那么
\begin{align}
    \text{原式} &= \lim_{m \to \infty} \left(1-\frac{1}{m}\right)^{m-2} \\
    &= \lim_{m\to\infty} \frac{\left(1-\frac{1}{m}\right)^m}{\left(1-\frac{1}{m}\right)^2} \\
    &= \frac{\displaystyle\lim_{n \to \infty}\left(1-\frac{1}{m}\right)^m}{\displaystyle\lim_{m\to\infty} \left(1-\frac{1}{m}\right)^2} \\
    &= \frac{\mathrm{e}^{-1}}{1} \\
    &= \mathrm{e}^{-1}
\end{align}
(4). 解:令$m=3n$,则
\begin{align}
    \lim_{n\to\infty}\left(1+\frac{3}{n}\right)^n &= \lim_{m\to\infty}\left(1+\frac{3}{3m}\right)^{3m} \\
    &= \lim_{m\to\infty}\left(1+\frac{1}{m}\right)^{3m} \\
    &= \lim_{m\to\infty}\left(1+\frac{1}{m}\right)^{m}\left(1+\frac{1}{m}\right)^{m}\left(1+\frac{1}{m}\right)^{m} \\
    &= \lim_{m\to\infty}\left(1+\frac{1}{m}\right)^{m} \lim_{m\to\infty}\left(1+\frac{1}{m}\right)^{m}\lim_{m\to\infty}\left(1+\frac{1}{m}\right)^{m} \\
    &= \mathrm{e}\cdot\mathrm{e}\cdot\mathrm{e}\\
    &=\mathrm{e}^3
\end{align}
(5). 解:令$m=2n^2$,则
\begin{align}
    \lim_{n\to\infty} \left(1+\frac{1}{2n^2}\right)^{4n^2} &= \lim_{m\to\infty}\left(1+\frac{1}{m}\right)^{2m} \\
    &= \lim_{m\to\infty}\left(1+\frac{1}{m}\right)^{m} \left(1+\frac{1}{m}\right)^{m} \\
    &= \lim_{m\to\infty}\left(1+\frac{1}{m}\right)^{m} \lim_{m\to\infty}\left(1+\frac{1}{m}\right)^{m} \\
    &=\mathrm{e} \cdot \mathrm{e} \\
    &=\mathrm{e}^2
\end{align}

2. 设$k \in \mathbb{N}^\star$,求证:$\displaystyle \lim_{n \to \infty} (1+\frac{k}{n})^n = e^k$.
\begin{proof}
采用数学归纳法.当$k=1$时:
\begin{equation}
    \lim_{n\to \infty}(1+\frac{k}{n})^n = \lim_{n \to \infty}(1+\frac{1}{n})^n = e = e^1 = e^k
\end{equation}
因此当$k=1$时命题成立.现在假设当$k = m-1, \, (m-1\in \mathbb{N}^\star)$时命题成立.那么有
\begin{align}
    &\phantom{=} \lim_{n \to \infty} (1+\frac{m}{n})^n \\
    (\text{令} \; n = m t) \quad & = \lim_{m \to \infty} (1 + \frac{1}{t})^{m t} \\
    &= \lim_{t \to \infty} (1+\frac{1}{t})^{(m-1) t} (1+\frac{1}{t})^t \\
    &= \left( \lim_{t \to \infty} (1+\frac{1}{t})^{(m-1) t} \right) \cdot \left( \lim_{t \to \infty} (1+\frac{1}{t})^t \right)\\ 
    (\text{令} \; t = \frac{s}{m-1}) \quad &= \left( \lim_{s \to \infty} (1 + \frac{m-1}{s})^s \right) \cdot e \\
    (\text{由归纳假设} ) \quad &= \mathrm{e}^{m-1} \cdot \mathrm{e} \\
    &= \mathrm{e}^m
\end{align}
这里证明了对于$k = m$命题成立.从而根据数学归纳法原理,对任意$k \in \mathbb{N}^\star$命题都成立.
\end{proof}

3. 求证:$\displaystyle \{ (1+\frac{1}{n})^n \}$是严格递增数列.
\begin{proof}
由算式---几何平均不等式:
\begin{equation}
(1+\frac{1}{n})^n = (\frac{n+1}{n})^n = 1 \cdot \prod_{i=1}^{n} \frac{n+1}{n} < \left( \frac{1 + n(\frac{n+1}{n})}{n+1} \right)^{n+1} = \left( 1 + \frac{1}{n+1} \right)^{n+1}
\end{equation}
从而$\displaystyle \{ (1+\frac{1}{n})^n\}$严格递增.
\end{proof}
4. 求证:$\displaystyle \{ (1+\frac{1}{n})^{n+1}\}$是严格递减数列.
\begin{proof}
令$\displaystyle q_n = (1 + \frac{1}{n})^{n+1}$,原命题等价于$\{ \displaystyle \frac{1}{q_n}\}$是严格递增数列.由算式---几何平均不等式
\begin{equation}
    \frac{1}{q_n} = \left( \frac{n}{n+1} \right)^{n+1} = 1 \cdot \prod_{i = 1}^{n+1} \frac{n}{n+1} < \left( \frac{1 + (n+1) \frac{n}{n+1}}{n+2} \right)^{n+2} = \left( \frac{1+n}{n+2}\right)^{n+2} = \frac{1}{q_{n+1}}
\end{equation}
从而数列$\{ \displaystyle \frac{1}{q_n} \}$严格单调递增,因此数列$\{ q_n \}$也就是$\{ \displaystyle (1 + \frac{1}{n})^{n+1}\}$严格单调递减.
\end{proof}
5. 证明不等式:
\begin{equation}
    \left( 1+\frac{1}{n} \right)^n < \mathrm{e} < \left( 1 + \frac{1}{n} \right)^{n+1}.
\end{equation}
\begin{proof}
令$\displaystyle p_n = \left( 1+\frac{1}{n} \right)^n, \, q_n = \left( 1+\frac{1}{n} \right)^{n+1}$.先证明不等式的左半部分.

采用数学归纳法,当$n=1$时:
\begin{equation}
    p_1 = (1+\frac{1}{1})^1 = 2 < e
\end{equation}
显然成立.假设当$n=m-1, \, (m-1 \in \mathbb{N}^\star)$时命题成立,我们去证明当$p_{m-1} < \mathrm{e}$时有$p_m < \mathrm{e}$,为此,采用反证法,假设对某个$m$,当$p_{m-1} < \mathrm{e}$时有$p_m > \mathrm{e}$,记$\epsilon_0 = \min \{ |p_{m-1} - \mathrm{e}|, |p_m - \mathrm{e}| \}$,显然$\epsilon_0 > 0$,由数列$\{ p_n \}$的严格单调性,对任意$n \in \mathbb{N}^\star$都有$|p_n - \mathrm{e}| \geq \epsilon_0 > 0$,但是由于数列$\{ p_n \}$的极限是$\mathrm{e}$,所以由极限的定义,对任意$\epsilon > 0$,存在$N_0 \in \mathbb{N}^\star$,使得当$n > N_0$时有$|p_n - e| < \epsilon$,由此推出矛盾,故对于$p_m > e$的假设是错误的.又根据数列$\{ p_n \}$的严格单调性,$p_m \neq e$,因此只能是$p_m < e$. 从而当$p_{m-1} < e$成立时$p_m < e$也成立,根据数学归纳法原理,对任意$n \in \mathbb{N}^\star$,$p_n < e$都成立.

利用数列$\{ q_n \}$的严格单调性质以及$\{ q_n \}$的极限也是$\mathrm{e}$的事实可类似地证明$e < q_n, \, (n \in \mathbb{N}^\star)$.
\end{proof}
6.利用对数函数$\ln \, x$的严格递增性质,证明:
\begin{equation}
    \frac{1}{n+1} < \ln \, \left( 1+\frac{1}{n} \right) < \frac{1}{n}
\end{equation}
对一切$n \in \mathbb{N}^\star$成立.
\begin{proof}
先证不等式的左半部分.要证$\displaystyle \frac{1}{n+1} < \ln \, \left( 1+\frac{1}{n} \right)$等价于去证$\displaystyle 1 < (n+1) \ln \, \left( 1+\frac{1}{n} \right)$,利用对数函数$\ln \, x$的严格单调递增性质,也就是等价于$\ln \, \mathrm{e} < \ln \, \left(\left( 1 + \frac{1}{n} \right)^{n+1} \right)$,由题5得到的结论:$\displaystyle \mathrm{e} < \left( 1+\frac{1}{n} \right)^{n+1}$(对任意$n \in \mathbb{N}^\star$),以及对数函数$\ln \, x$的严格单调性质,可知$\ln \, \mathrm{e} < \ln \, \left( \left( 1 + \frac{1}{n} \right)^{n+1} \right)$成立,因此$\displaystyle 1 < (n+1) \ln \, \left( 1+\frac{1}{n} \right)$成立,因此$\displaystyle \frac{1}{n+1} < \ln \, \left( 1 + \frac{1}{n} \right)$成立,即,要证的不等式的左半部分成立.

再用类似的方法来证明不等式的右半部分.由题5得到的结论:$\displaystyle \left( 1 + \frac{1}{n} \right)^n < \mathrm{e}$(对任意$n \in \mathbb{N}^\star$),对这个不等式两端取对数并且利用对数函数$\ln \, x$的单调递增性质,得到$\displaystyle n \ln \, \left( 1 + \frac{1}{n} \right) < \ln \, \mathrm{e} = 1$,再对这个不等式两端除以$n$即得$\displaystyle \ln \, \left( 1 + \frac{1}{n}\right) < \frac{1}{n}$.于是命题得证.
\end{proof}
7. 设$n \in \mathbb{N}^\star$且$k = 1,2,\cdots$. 证明不等式:
\begin{equation}
    \frac{k}{n+k} < \ln \, \left( 1+\frac{k}{n} \right) < \frac{k}{n}
\end{equation}
思路:回顾题4、题5以及题6的证明过程,我们受到启发:要证$\displaystyle \frac{k}{n+k} < \ln \, \left( 1 + \frac{k}{n}\right)$成立就要去证$\displaystyle \ln \, \left( \mathrm{e}^k \right) < (n+k) \ln \, \left( 1 + \frac{k}{n}\right)$成立,相应地就要去证$\displaystyle \mathrm{e}^k < \left(1+\frac{k}{n}\right)^{n+k}$并且要证数列$\displaystyle \{ \left(1+\frac{k}{n}\right)^{n+k} \}$严格单调递减,而对于数列$\displaystyle \{ \left( 1 + \frac{k}{n} \right)^{n+k}\}$的单调递减性的证明可以参考和借鉴题4的方法.
\begin{proof}
先证明数列$\displaystyle \{ \left(1+\frac{k}{n}\right)^{n+k}\}$是严格单调递减的.令$\displaystyle p_n =  \left(1+\frac{k}{n}\right)^{n+k}$,这等价于证明数列$\displaystyle \{ \frac{1}{p_n} \}$是严格单调递减的.利用算式---几何均值不等式,我们有:
\begin{equation}
    \frac{1}{p_n} =  \left(\frac{n}{n+k}\right)^{n+k} = 1 \cdot \prod_{i=1}^{n+k} \left(\frac{n}{n+k}\right) <(\frac{1+(n+k) \frac{n}{n+k}}{n+k+1})^{n+k+1} = \frac{1}{p_{n+1}}
\end{equation}
这就证明了数列$\displaystyle \{ \frac{1}{p_n}\}$是严格单调递减的,从而数列$\{p_n \}$也就是$\{ \displaystyle \left(1+\frac{k}{n}\right)^{n+k}\}$是严格单调递增的.又根据
\begin{equation}
    \lim_{n \to \infty} \left(1+\frac{k}{n}\right)^{n+k} = \mathrm{e}^k
\end{equation}
以及数列$\{ \displaystyle \left( 1 + \frac{k}{n} \right)^{n+k}\}$的严格单调递减性,以及$\mathrm{e}^k < p_1 = \displaystyle \left(1+\frac{k}{n}\right)^{n+k}$的事实,可证明
\begin{equation}
    \mathrm{e}^k < \left( 1 + \frac{k}{n}\right)^{n+k}
    \label{eq:ek}
\end{equation}
对每一个$k, n\in \mathrm{N}^{\star}$都成立.对式(\ref{eq:ek})两边取对数,再根据对数函数$\ln \, x$的严格单调递增性,得到
\begin{equation}
    k < (n+k) \ln \, \left(1+\frac{k}{n}\right)
\end{equation}
也就是
\begin{equation}
    \frac{k}{n+k} < \ln \, \left(1+\frac{k}{n}\right)
\end{equation}
从而要证明的不等式的左半部分成立.采用类似的方法可类似地证明不等式的右半部分成立.
\end{proof}
8. 对$n \in \mathbb{N}^\star$,求证:
\begin{equation}
    \frac{1}{2} + \frac{1}{3} + \cdots + \frac{1}{n} < \ln \, \left( n+1 \right) < 1 + \frac{1}{2} + \cdots + \frac{1}{n}.
\end{equation}
\begin{proof}
利用题6的结论:$\displaystyle \frac{1}{n+1} < \ln \, \left( 1 + \frac{1}{n} \right) = \ln \, (n+1) - \ln \, n$,我们有
\begin{equation}
\sum_{i = 1}^{n} \frac{1}{i + 1} < \sum_{i=1}^n \ln \, \left( i+1 \right) - \ln \, i = \sum_{i=1}^n \ln \, \left( i + 1\right) - \sum_{i=1}^n \ln \, i = \sum_{i=1}^{n+1} \ln \, i - \sum_{i=1}^{n} \ln \, i = \ln \, \left( n+1 \right)
\end{equation}
于是左半部分得证.再利用题6的结论:$\displaystyle \ln \, \left( n+1 \right) - \ln \, n = \ln \, \left( 1 + \frac{1}{n} \right) < \frac{1}{n}$,可以得到
\begin{equation}
\sum_{i=1}^n \ln \, \left( i+1 \right) - \ln \, i = \sum_{i=1}^n \ln \, \left( i+1 \right) - \sum_{i=1}^n \ln \, i = \sum_{i=1}^{n+1} \ln \, i - \sum_{i=1}^n \ln \, i = \ln \, \left(n+1 \right) < \sum_{i=1}^n \frac{1}{i}
\end{equation}
于是右半部分得证.
\end{proof}
9. 令
\begin{equation}
    x_n = 1 + \frac{1}{2} + \cdots + \frac{1}{n} - \ln \, \left( n+1 \right) \quad (n \in \mathbb{N}^\star).
\end{equation}
证明:$\displaystyle \lim_{n \to \infty} x_n $存在,此极限记为$\gamma$,叫做Euler(欧拉,1707\textasciitilde 1783)常数.
\begin{proof}
先证单调性.
\begin{align}
    x_{n+1} &= 1 + \frac{1}{2} + \cdots + \frac{1}{n} + \frac{1}{n+1} - \ln \, \left(n+2\right) \\
    &< 1 + \frac{1}{2} + \cdots + \frac{1}{n} + \ln \, \left( 1 + \frac{1}{n} \right) - \ln \, \left( n+2\right) \\
    &= 1 + \frac{1}{2} + \cdots + \frac{1}{n} + \ln \, \left( n+1 \right) - \ln \, n - \ln \, \left( n+2 \right) \\
    &< 1 + \frac{1}{2} + \cdots + \frac{1}{n} + \ln \, \left( n+1 \right) \\
    &= x_n \quad (n \in \mathbb{N}^\star)
\end{align}
再证有界性.利用题8的结论,有
\begin{equation}
    0 < 1 + \frac{1}{2} + \cdots + \frac{1}{n} - \ln \left( n + 1\right) 
\end{equation}
从而数列$\{ x_n \}$既单调又有界,所以数列$\{ x_n \}$的极限存在.
\end{proof}
10. 利用第9题,证明:
\begin{equation}
    1+\frac{1}{2} + \frac{1}{3} + \cdots + \frac{1}{n} = \ln \, n + \gamma + \epsilon_n,
\end{equation}
其中$\displaystyle \lim_{n \to \infty} \epsilon_n = 0$.
\begin{proof}
令$\displaystyle p_n = 1 + \frac{1}{2} + \cdots + \frac{1}{n} - \ln \, n$,我们先来证明数列$\{ p_n \}$的极限存在.先证单调性:
\begin{align}
    p_{n+1} &= 1 + \frac{1}{2} + \cdots + \frac{1}{n} + \frac{1}{n+1} - \ln \, \left( n + 1 \right) \\
    &< 1 + \frac{1}{2} + \cdots + \frac{1}{n} + \ln \, \left( 1 + \frac{1}{n} \right) - \ln \, \left(n+1\right) \\
    &= 1 + \frac{1}{2} + \cdots + \frac{1}{n} - \ln \, n \\
    &= p_n \quad (n \in \mathbb{N}^\star)
\end{align}
再证有界性.利用题8的结论:$\displaystyle \ln \, n < \ln \, \left( n+1 \right) < 1 + \frac{1}{2} + \cdots +\frac{1}{n}$,于是
\begin{equation}
    p_n = 1 + \frac{1}{2} +\cdots + \frac{1}{n} - \ln \, n > 0
\end{equation}
从而数列$\{ p_n \}$有界,因此数列$\{ p_n \}$的极限存在,即$ \displaystyle \lim_{n \to \infty} p_n$存在.令$\displaystyle x_n = 1 + \frac{1}{2} + \cdots + \frac{1}{n} - \ln \, \left( n + 1\right)$,我们来证$\displaystyle \lim_{n \to \infty} p_n$的确等于$\displaystyle \lim_{n \to \infty} x_n$:
\begin{align}
    \lim_{n \to \infty} p_n &= \lim_{n \to \infty} \left( x_n + \ln \, \left( n+1 \right) - \ln \, n \right) \\
    &= \lim_{n \to \infty} x_n + \lim_{n \to \infty} \left( \ln \, \left(n+1\right) - \ln \, n \right) \\
    &= \lim_{n \to \infty} x_n + 0 \\
    &= \lim_{n \to \infty} x_n
\end{align}
其中$\displaystyle \lim_{n \to \infty} \left( \ln \, \left(n+1\right) - \ln \, n\right) = 0$的条件是从题6的结论:
\begin{equation}
\frac{1}{n+1} < \ln \, \left(1 + \frac{1}{n}\right) = \ln \, \left(n+1\right) - \ln \, n < \frac{1}{n}
\end{equation}
得到的.令$\displaystyle \epsilon_n = 1 + \frac{1}{2} +\cdots + \frac{1}{n} - \ln \, n - \gamma$,令$\displaystyle \gamma = \lim_{n \to \infty} x_n$,我们要证$\displaystyle \lim_{n \to \infty} \epsilon_n = 0$:
\begin{align}
    \lim_{n \to \infty} \epsilon_n &= \lim_{n \to \infty} \left(1 + \frac{1}{2} + \cdots + \frac{1}{n} - \ln \, n - \gamma \right) \\
    &= \lim_{n \to \infty} \left( p_n - \gamma \right) \\
    &= \lim_{n\to \infty} p_n - \lim_{n\to \infty} \gamma \\
    &= \left( \lim_{n\to \infty} p_n \right) - \gamma \\
    &= \lim_{n\to \infty} p_n - \lim_{n \to \infty} x_n \\ 
    &= 0
\end{align}
命题得证.
\end{proof}
11. 证明不等式:
\begin{equation}
    \left(\frac{n+1}{\mathrm{e}}\right)^n < n! < \mathrm{e} \left(\frac{n+1}{\mathrm{e}}\right)^{n+1}
\end{equation}
\begin{proof}
先证不等式左半部分.先将要证的不等式改写成便于证明的等价形式:
\begin{align}
    \left( \frac{n+1}{\mathrm{e}}\right)^n < n! &\iff n \left(\ln \, \left(n+1\right) - 1\right) < \sum_{i=1}^n \ln \, i \\
    &\iff \sum_{i=1}^n \left(\ln \, \left(n+1\right) - \ln \, i\right) < n \label{ieq:1}
\end{align}
我们现采用数学归纳法证明不等式(\ref{ieq:1})成立,显然当$n=1$时不等式(\ref{ieq:1})成立,假设当$n = m-1$时($m-1 \in \mathbb{N}^\star$)不等式(\ref{ieq:1})成立.那么当$n=m$时:
\begin{align}
    \sum_{i=1}^m \left(\ln\,\left(m+1\right) - \ln \, i\right) &= \sum_{i=1}^{m-1} \left(\ln \, m - \ln \, i\right) + \sum_{i=1}^m \left(\ln\,\left(m+1\right) - \ln \, i\right) - \sum_{i=1}^{m-1} \left(\ln \, m - \ln \, i\right) \\
    &= \left(\sum_{i=1}^{m-1} \ln \, m - \ln \, i \right) + m \left( \ln \, \left(m+1\right) - \ln \, m \right) \\
    &< m-1 + m \cdot \frac{1}{m} \\
    &= 1
\end{align}
由此可见当$n=m$时不等式(\ref{ieq:1})也成立,根据数学归纳法原理,不等式(\ref{ieq:1})对每一个$n\in\mathbb{N}^\star$都成立,从而不等式(\ref{ieq:1})的等价形式
\begin{equation}
\left(\frac{n+1}{\mathrm{e}}\right)^n < n!
\end{equation}
对每一个$n\in\mathbb{N}^\star$都成立.

现在再去证明不等式右半部分,先将它改写成便于证明的等价形式:
\begin{align}
    n! < \mathrm{e}\left(\frac{n+1}{\mathrm{e}}\right)^{n+1} &\iff \sum_{i=1}^n \ln \, i < (n+1)\ln \, \left(n+1\right) - n \\
    &\iff n < \left( \sum_{i=1}^n \ln \, \left(n+1\right) - \ln \, i \right) + \ln \, \left(n+1\right)
\end{align}
令
\begin{equation}
    p_n = \left( \sum_{i=1}^n \ln \, \left(n+1\right) - \ln \, i \right) + \ln \, \left(n+1\right)
\end{equation}
我们现在采用数学归纳法证明$n < p_n$对每一个$n\in\mathbb{N}^\star$成立.当$n=1$时:
\begin{align}
    p_n = p_1 = 2 \ln \, 2 = 2 \left( \ln \, 2 - \ln \, 1\right) = 2 \ln \, \left( 1 + \frac{1}{1} \right) > 2 \cdot \frac{1}{1+1} = 1 = n
\end{align}
从而当$n=1$时$n < p_n$成立,现在假设:当$n=m-1, \, (m-1 \in \mathbb{N}^\star)$时,$n < p_n$成立.于是
\begin{align}
    p_m &= p_{m-1} + p_{m} - p_{m-1} \\
    &= p_{m-1} + (m+1) \left( \ln \, \left(m+1\right) - \ln \, m \right) \\
    &= p_{m-1} + (m+1) \ln \, \left(1 + \frac{1}{m}\right) \\
    &> m-1 + (m+1) \frac{1}{m+1} \\
    &= m
\end{align}
从而对于$n=m$,$n < p_n$也成立,根据数学归纳法原理,$n < p_n$对每一个$n \in \mathbb{N}^\star$都成立,也就是
\begin{equation}
    n < \left( \sum_{i=1}^n \ln \, \left(n+1\right) - \ln \, i \right) + \ln \, \left(n+1\right)
\end{equation}
以及它的等价形式
\begin{equation}
    n! < \mathrm{e}\left(\frac{n+1}{\mathrm{e}}\right)^{n+1}
\end{equation}
对每一个$n \in \mathbb{N}^\star$都成立.
\end{proof}
12. 证明:利用题11的结论:
\begin{align}
    &\phantom{\implies} \left(\frac{n+1}{\mathrm{e}}\right)^n < n! < \left(\frac{n+1}{\mathrm{e}}\right)^{n+1} \\
    &\implies \frac{1}{\mathrm{e}^n} < \frac{n!}{(n+1)^n} < \frac{n+1}{\mathrm{e}^n} \\
    &\implies \frac{1}{\mathrm{e}} < \frac{(n!)^\frac{1}{n}}{n+1} < \frac{(n+1)^\frac{1}{n}}{\mathrm{e}} \\
    &\implies \lim_{n\to\infty} \frac{1}{\mathrm{e}} < \lim_{n \to \infty} \frac{(n!)^\frac{1}{n}}{n+1} < \lim_{n\to\infty} \frac{(n+1)^\frac{1}{n}}{\mathrm{e}} \\
    &\implies \lim_{n\to\infty} \frac{(n!)^\frac{1}{n}}{n+1} = \frac{1}{\mathrm{e}}
\end{align}
再次利用题11的结论:
\begin{align}
    &\phantom{\implies} \left(\frac{n+1}{\mathrm{e}}\right)^n < n! < \left(\frac{n+1}{\mathrm{e}}\right)^{n+1} \\
    &\implies \frac{1}{\mathrm{e}^n} < \frac{n!}{(n+1)^n} < \frac{n+1}{\mathrm{e}^n} \\
    &\implies \frac{1}{\mathrm{e}} < \frac{(n!)^\frac{1}{n}}{n+1} < \frac{(n+1)^\frac{1}{n}}{\mathrm{e}} \\
    &\implies \frac{1}{n \mathrm{e}} < \frac{(n!)^\frac{1}{n}}{n(n+1)} < \frac{(n+1)^\frac{1}{n}}{\mathrm{e}} \\
    &\implies \lim_{n\to\infty} \frac{1}{n \mathrm{e}} = \lim_{n\to\infty} \frac{(n!)^\frac{1}{n}}{n(n+1)} = \lim_{n\to\infty} \frac{(n+1)^\frac{1}{n}}{\mathrm{e}} = 0 \\
    &\implies \lim_{n\to\infty} \left(\frac{(n!)^\frac{1}{n}}{n} - \frac{(n!)^\frac{1}{n}}{n+1}\right) = 0 \\
    &\implies \lim_{n\to\infty} \frac{(n!)^\frac{1}{n}}{n} = \lim_{n\to\infty} \frac{(n!)^\frac{1}{n}}{n+1} + \lim_{n\to\infty} \left(\frac{(n!)^\frac{1}{n}}{n} - \frac{(n!)^\frac{1}{n}}{n+1}\right) = \frac{1}{\mathrm{e}} + 0 = \frac{1}{\mathrm{e}}
\end{align}
这就证明了数列$\displaystyle \{\frac{(n!)^\frac{1}{n}}{n}\}$的极限是$\displaystyle \frac{1}{\mathrm{e}}$.
