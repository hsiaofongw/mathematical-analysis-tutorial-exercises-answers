% 练习题1.10
\exercise

1. 求$\displaystyle\lim_{n\to \infty} \inf a_n$和$\displaystyle\lim_{n \to \infty} \sup a_n$,设:
\begin{table}[H]
    \centering
    \begin{tabularx}{0.8\textwidth} {  >{\raggedright\arraybackslash}X >{\raggedright\arraybackslash}X  }
       (1) ~ $a_n = \displaystyle \frac{(-1)^n}{n} + \frac{1+(-1)^n}{2}$; & (2)~$a_n=\displaystyle n^{(-1)^n}$; \\ [1em]
       (3) ~ $a_n = \arctan \, \left( n^{(-1)^n} \right)$;
      \end{tabularx}
\end{table}

(1) 解:利用上、下极限的等价定义:
\begin{align}
    a^\star = \lim_{n \to \infty} \sup_{k \geq n} \, \{ a_k \}, \; a_\star = \lim_{n \to \infty} \inf_{k \geq n} \, \{a_k\}
\end{align}
我们首先求$\displaystyle\sup_{k \geq n} \, \{a_k\}$,它自身可以看成是一个数列:
\begin{equation}
    \sup_{k \geq n} \, a_k = \begin{cases}
        a_n, & n\text{为偶数} \\
        a_{n+1}, & n\text{为奇数} 
    \end{cases}
\end{equation}
这样我们就得到
\begin{equation}
    \{ \sup_{k \geq n} \, a_k \} = \{ a_2, a_2, a_4, a_4, a_6, a_6, \cdots \}
\end{equation}
它可以看成是$\{ a_n \}$所有偶数项组成的子列,再求$\displaystyle \inf_{k \geq n} \, \{ a_k \}$,它等于
\begin{equation}
    \inf_{k \geq n} \, a_k = \begin{cases}
        a_n, & n\text{为奇数} \\
        a_{n+1}, & n\text{为偶数} 
    \end{cases}
\end{equation}
这样我们就得到
\begin{equation}
    \{ \inf_{k \geq n} \, a_k \} = \{ a_1, a_1, a_3, a_3, a_5, a_5, \cdots \}
\end{equation}
它可以看成是$\{ a_n \}$所有奇数项组成的子列.于是$\displaystyle \lim_{n \to \infty} \sup_{k \geq n} \, \{ a_k \}$就等于$\{ a_{2n} \}$的极限,就等于
\begin{equation}
    \lim_{n \to \infty} a_{2n} = \lim_{n \to \infty} \left( \frac{1}{n} + 1 \right) = \left(\lim_{n\to\infty} \frac{1}{n}\right) + 1= 0 + 1 = 1
\end{equation}
而$\displaystyle \lim_{n \to \infty} \inf_{k \geq n} \, \{ a_k \}$等于$\{ a_{2n-1} \}$的极限,就等于
\begin{equation}
    \lim_{n \to \infty} a_{2n - 1} = \lim_{n \to \infty} -\frac{1}{n} = - \left(\lim_{n\to\infty} \frac{1}{n}\right) = -\left( 0 \right) = -0 = 0.
\end{equation}

(2)解:首先对$a_n$的幂次进行计算,也就是,对$(-1)^n$进行计算,我们发现:
\begin{equation}
    (-1)^n = \begin{cases}
        -1, & n\text{为奇数} \\
        1, & n\text{为偶数}
    \end{cases}
\end{equation}
这样一来就得到
\begin{equation}
    a_n = \begin{cases}
        n^{-1}, & n\text{为奇数} \\
        n, & n\text{为偶数}
    \end{cases}
\end{equation}
进而有
\begin{equation}
    \sup_{k \geq n} \, \{ a_k \} = n, \; \inf_{k \geq n} \, \{ a_k \} = n^{-1}
\end{equation}
从而得
\begin{align}
    &a^\star = \lim_{n\to\infty} \sup_{k \geq n} \, \{ a_k \} = \lim_{n \to \infty} n = +\infty \\
    &a_\star = \lim_{n\to\infty} \inf_{k \geq n} \, \{ a_k \} = \lim_{n \to \infty} n^{-1} = \lim_{n \to \infty} \frac{1}{n} = 0.
\end{align}

(3) 解:$\arctan$是定义在$(-\infty, +\infty)$上的严格单调递增函数,利用这个性质可以得到
\begin{align}
    &\ulimit \{ \arctan \, \left(n^{(-1)^n}\right) \} = \arctan \, \left( \ulimit \{ n^{(-1)^n}\}\right) = \arctan \, +\infty = \frac{\pi}{2} \\
    &\llimit \{ \arctan \, \left(n^{(-1)^n}\right) \} = \arctan \, \left( \llimit \{ n^{(-1)^n}\}\right) = \arctan \, 0 = 0
\end{align}

2. 试证下面诸式当两边有意义时成立:

(1) 若$\displaystyle\lim_{n\to\infty}b_n=b$,则
\begin{align}
    &\llimit \left( a_n + b_n \right) = \llimit a_n + b, \\
    &\ulimit \left( a_n + b_n \right) = \ulimit a_n + b;
\end{align}

(1)
\begin{proof}
由例3的结论:
\begin{align}
    &\llimit \left( a_n + b_n \right) \leq \llimit a_n + \ulimit b_n = \llimit a_n + b \\
    &\llimit \left( a_n + b_n \right) \geq \llimit a_n + \llimit b_n = \llimit a_n + b 
\end{align}
所以这只能是
\begin{equation}
    \llimit \left( a_n + b_n \right) = \llimit a_n + b.
\end{equation}
再次利用例3的结论:
\begin{align}
    &\ulimit \left( b_n + a_n \right) \leq \ulimit b_n + \ulimit a_n = b + \ulimit a_n \\
    &\ulimit \left( b_n + a_n \right) \geq \llimit b_n + \ulimit a_n = b + \ulimit a_n
\end{align}
所以这只能是
\begin{equation}
    \ulimit \left( a_n + b_n \right) = \ulimit a_n + b.
\end{equation}
\end{proof}