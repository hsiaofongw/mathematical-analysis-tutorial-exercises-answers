% 练习题1.8
\exercise
2. 求数列$\{ (1+1/n)^n : n \in \mathbb{N}^\star \}$和$\{(1+1/n)^{n+1}:n\in\mathbb{N}^\star\}$的下确界和上确界.

思路:注意到$\{ (1+1/n)^n : n \in \mathbb{N}^\star \}$是严格单调递增数列并且极限是$\mathrm{e}$,所以我们去证它的上确界是$\mathrm{e}$,而数列$\{(1+1/n)^{n+1}:n\in\mathbb{N}^\star\}$的极限也是$\mathrm{e}$并且是严格单调递减数列,我们去证它的下确界是$\mathrm{e}$.
\begin{proof}
令$a_n = (1+1/n)^n$,易证,数列$\{ a_n \}$是严格单调递增的,并且极限是$\mathrm{e}$,从而$\mathrm{e} < a_n$对所有$n \in \mathbb{N}^\star$成立,也就是对任意$x \in \{a_n:n\in\mathbb{N}^\star\}$都有$\mathrm{e}<x$;由于$\{a_n\}$的极限是$\mathrm{e}$,所以,对任意$\epsilon >0$,都存在$N \in \mathbb{N}^\star$,使得当$n>N$时,有
\begin{equation}
    |a_n - \mathrm{e}| = \mathrm{e}-a_n < \epsilon
\end{equation}
令$x_\epsilon = a_n$,该不等式可以写成$x_\epsilon > e - \epsilon$,并且$x_\epsilon = a_n \in \{a_n : n \in \mathbb{N}^\star \}$,因此$\mathrm{e}$是集合$\{ a_n : n \in \mathbb{N}^\star \}$也就是集合$\{ (1+1/n)^n : n \in \mathbb{N}^\star \}$的上确界.易知$\{(1+1/n)^n:n\in\mathbb{N}\}$的下确界是$a_1$也就是$3/2$.

再来证$\mathrm{e}$是集合$\{(1+1/n)^{n+1}:n\in\mathbb{N}\}$的下确界.令$b_n=(1+1/n)^{n+1}$,易证数列$\{b_n\}$是严格单调递减的并且极限是$\mathrm{e}$,从而对任意$n \in \mathbb{N}^\star$都有$b_n > \displaystyle\lim_{n\to\infty} b_n = \mathrm{e}$,也就是对任意$ x \in \{ b_n : n \in \mathbb{N}^\star \}$都有$x > \mathrm{e}$.

由$\displaystyle\lim_{n\to\infty}b_n = e$知,对任意$\epsilon > 0$,存在$N \in \mathbb{N}^\star$,使得当$n > N$时,有
\begin{equation}
    |b_n - \mathrm{e}| = b_n - \mathrm{e} < \epsilon 
\end{equation}
令$y_\epsilon = b_n$,该不等式可以写成$y_\epsilon = b_n < \mathrm{e} + \epsilon$(对任意$n\in\mathbb{N}^\star$),从而$\mathrm{e}$正是集合$\{b_n:n\in\mathbb{N}^\star\}$的下确界.

显然$\{b_n:n\in\mathbb{N}^\star\}$的上确界是$b_1$,也就是$\displaystyle \left(\frac{3}{2}\right)^2$.
\end{proof}

3. 求数列$\{n^{1/n}:n\in\mathbb{N}^\star\}$的下确界和上确界.

解:先求上确界.通过计算可知$1^{1/1} < 2^{1/2} <3^{1/3}$,令$a_n = n^{1/n}$,则$a_n$的增减性等价于$\ln \, a_n$的增减性,为了判断$\ln \, a_n$的增减性,我们对数列$\{ \ln \, a_n \}$的任意相邻两项做差:
\begin{align}
    \ln \, a_{n+1} - \ln \, a_n &= \frac{1}{n+1} \ln \, \left(n+1\right) - \frac{1}{n} \ln \, n\\
    &= \frac{1}{n(n+1)} \left( n \ln \, \left(n+1\right) - (n+1) \ln \, n\right)
\end{align}
并且当$n \geq 3$时,有
\begin{align}
    n \ln \, \left(n+1\right) - (n+1) \ln \, n &= n \left(\ln \, \left(n+1\right) - \ln \, n \right) - \ln \, n \\
    &< n \cdot \frac{1}{n} - \ln \, n = 1 - \ln \, n \leq 1 - \ln 3 < 0
\end{align}
即$\ln \, a_{n+1} - \ln \, a_n < 0$,由此可知,当$n \geq 3$时,有$a_{n+1} < a_n$.因此,对任意$n \in \mathbb{N}^\star$,都有$a_3 \geq a_n$,并且,对任意$\epsilon > 0$,可以取$x_\epsilon = a_3$,这时有
\begin{equation}
    a_3 - x_\epsilon = a_3 - a_3 = 0 < \epsilon
\end{equation}
也就是$x_\epsilon > a_3 - \epsilon$
所以$a_3$也就是$\displaystyle 3^\frac{1}{3}$,是$\{a_n:n\in\mathbb{N}^\star\}$的上确界.

再来证$1$是$\{n^\frac{1}{n}\}$的下确界,由于$\displaystyle\lim_{n\to\infty}n^\frac{1}{n}=1$,所以,对任意$\epsilon>0$,存在$N \in \mathbb{N}^\star$,使得当$n > N$时,有
\begin{equation}
    |n^\frac{1}{n} - 1| = n^\frac{1}{n} - 1 < \epsilon
\end{equation}
令$y_\epsilon = n^\frac{1}{n}$,则对所有$n > N$,都有$y_\epsilon = n^\frac{1}{n} < 1 + \epsilon$成立,并且显然$y_\epsilon \in \{ n^\frac{1}{n}:n \in \mathbb{N}^\star \}$,因此,由定义,$1$是$\{\displaystyle n^\frac{1}{n}:n\in\mathbb{N}^\star\}$的下确界.

4. 设在数列$\{ a_n:n\in\mathbb{N}^\star\}$中,既没有最小值,也没有最大值.求证:数列$\{a_n\}$发散.
\begin{proof}
如果数列$\{ a_n \}$是无界的,譬如说,不存在上界,那么我们可以从中找出一个趋于正无穷的子列,于是$\{a_n \}$发散.

否则根据确界原理,$\{a_n\}$有上确界和下确界,分别记为$u$和$l$,由于$u$是$\{ a_n \}$的上确界,所以对任意$\epsilon > 0$,存在$i_1 \in \nat$,使得$u - a_{i_1} < \epsilon$,又因为$\{ a_n \}$是不存在最大值的,于是就存在$i_2 \in \nat$,使得$a_{i_1} < a_{i_2} < u$,类似地,我们可以找出一系列的下标$i_1, i_2, i_3, i_4, \cdots \in \nat$,它们满足
\begin{equation}
    a_{i_1} < a_{i_2} < a_{i_3} < a_{i_4} < \cdots < u
\end{equation}
并且对以上这两个不等式做放缩可得
\begin{equation}
    0 < \cdots < u - a_{i_4} < u - a_{i_3} < u - a_{i_2} < u - a_{i_1} < \epsilon
\end{equation}
这样一来,我们就找到了一个趋于$u$的子列,它就是$\{ a_{i_n} \}$.

由于$l$是下确界,所以对任意$\epsilon > 0$,都存在$j_1 \in \nat$,使得$a_{j_1} - l < \epsilon$,由于$\{ a_n \}$不存在最小值,所以一定存在$j_2 \in \nat$使得
\begin{equation}
    a_{j_1} > a_{j_2} > l
\end{equation}
类似地,总能找出一系列的下标$j_1, j_2, j_3, \cdots$,满足
\begin{equation}
    a_{j_1} > a_{j_2} > a_{j_3} > \cdots > l
\end{equation}
对以上这两个不等式做放缩,得到
\begin{equation}
    \epsilon > a_{j_1} - l > a_{j_2} - l > a_{j_3} - l > \cdots > 0
\end{equation}
这样我们就找到了一个趋于$l$的子列,它就是$\{ a_{j_m} \}$.如果$u = l$,也就是$\{ a_n \}$的上确界等于下确界,那么将可以推出$\{ a_n \}$是常数列,这时每一个$a_n$都可以当成最大值(最小值),这与题设矛盾,所以$u \neq l$,所以$\{ a_{i_n} \}$和$\{ a_{j_m} \}$是$\{ a_n \}$的两个趋向不同极限的子列,所以$\{ a_n \}$发散.
\end{proof}