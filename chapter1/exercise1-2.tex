% 练习题1.2
\exercise

5. 用精确语言表达``数列$\{a_n\}$不以$a$为极限''这一陈述.

答:存在$\epsilon_0 > 0$,使得对任意$N \in \nat$,都存在$n > N$并且$|a_n - a| \geq \epsilon_0$.

顺便一提函数极限的否定定义:存在$\epsilon_0 > 0$,使得对任意$\delta > 0$,都存在$|x - x_0| < \delta$并且$|f(x_0) - y_0| \geq \epsilon_0$.

7. 设$a,b,c$是三个给定的实数,令$a_0 = a, b_0 = b, c_0 = c$,并归纳地定义
\begin{equation}
    \begin{cases}
        a_n = \displaystyle\frac{b_{n-1}+c_{n-1}}{2} \\
        b_n = \displaystyle\frac{a_{n-1}+c_{n-1}}{2} \\
        c_n = \displaystyle\frac{a_{n-1}+b_{n-1}}{2}
    \end{cases}, \quad n = 1,2,3,\cdots
\end{equation}
求证:
\begin{equation}
    \lim_{n\to\infty} a_n = \lim_{n\to\infty} b_n = \lim_{n\to\infty} c_n = \frac{1}{3} \left( a_0+b_0+c_0 \right).
\end{equation}

\begin{proof}
    对每一个$n \in \nat$,将$r_n,s_n,t_n$分别定义为$a_n,b_n,c_n$中最小的、次小的和最大的数.$a_n,b_n,c_n$收敛到同一个数等价于它们之中的最小数和最大数都收敛到同一个数,也就是说,我们要去证明$r_n,t_n$都收敛到同一个数.

由
\begin{align}
    &\mathrel{\phantom{\implies}} s_n \leq t_n \\
    &\implies r_n + s_n \leq r_n + t_n \\
    &\implies \frac{r_n+s_n}{2} \leq \frac{r_n+t_n}{2} 
\end{align}
以及
\begin{align}
    &\mathrel{\phantom{\implies}} r_n \leq s_n \\
    &\implies r_n + t_n \leq s_n + t_n \\
    &\implies \frac{r_n+t_n}{2} \leq \frac{s_n+t_n}{2}
\end{align}
得到
\begin{equation}
    \frac{r_n+s_n}{2} \leq \frac{r_n+t_n}{2} \leq \frac{s_n+t_n}{2}
\end{equation}
因此
\begin{equation}
    r_{n+1} = \frac{r_n+s_n}{2}, \; s_{n+1} = \frac{r_n+t_n}{2}, \; t_{n+1}=\frac{s_n+t_n}{2}
\end{equation}
于是
\begin{equation}
    t_{n+1} - r_{n+1} = \frac{s_n+t_n}{2} - \frac{r_n+s_n}{2} = \frac{t_n - r_n}{2}
\end{equation}
并且由此得
\begin{equation}
t_n - r_n = \frac{t_{n-1}-r_{n-1}}{2} = \frac{t_{n-2}-r_{n-2}}{4} = \cdots = \frac{1}{2^{n}}\left(t_0 - s_0\right)
\end{equation}
于是对任意$\epsilon > 0$,我们可以取$N = \lceil \log_2 \left(t_0-s_0\right) - \log_2 \epsilon \rceil$,于是当$n > N$的时候就有
\begin{equation}
    |t_n-r_n|=\frac{1}{2^n}(t_0-s_0) < \frac{1}{2^N}(t_0-s_0) \leq \epsilon
\end{equation}
为了把$r_n$也就是$t_n$的共同极限找出来,我们要寻找一个始终位于$r_n$和$t_n$中间的常数,为此,我们要去证明$\displaystyle\frac{r_0+s_0+t_0}{3}$就是这个数.首先,根据
\begin{equation}
    r_{n+1} = \frac{r_n+s_n}{2}, \; s_{n+1} = \frac{r_n+t_n}{2}, \; t_{n+1}=\frac{s_n+t_n}{2} \; (\forall n \in \integer)
\end{equation}
我们得
\begin{equation}
    r_{n+1}+s_{n+1}+t_{n+1} = \frac{2r_n+2s_n+2t_n}{2} = r_n + s_n + t_n \; (\forall n \in \integer)
\end{equation}
这说明
\begin{equation}
    r_0+s_0+t_0=r_n+s_n+t_n, \; (\forall n \in \integer)
\end{equation}
于是放缩得
\begin{equation}
    r_n \leq \frac{r_0+s_0+t_0}{3} \leq t_n, \; (\forall n \in \integer)
\end{equation}
这说明,在数轴上,不管$n$取哪一个整数,$\displaystyle\frac{r_0+s_0+t_0}{3}$这个数,总是插在$r_n$与$t_n$之间,又根据$r_n,s_n,t_n$的定义,有$r_n \leq \min \{a_n,b_n,c_n\}$以及$t_n \geq \max \{a_n, b_n, c_n \}$,于是,依几何事实,有
\begin{align}
    |a_n - \frac{r_0+s_0+t_0}{3}| < |r_n - t_n| < \epsilon, \\
    |b_n - \frac{r_0+s_0+t_0}{3}| < |r_n - t_n| < \epsilon, \\
    |c_n - \frac{r_0+s_0+t_0}{3}| < |r_n - t_n| < \epsilon
\end{align}
又由于$\displaystyle\frac{r_0+s_0+t_0}{3}=\frac{a_0+b_0+c_0}{3}$,所以
\begin{equation}
    \lim_{n\to\infty} a_n = \lim_{n\to\infty} b_n = \lim_{n\to\infty} c_n = \frac{a_0+b_0+c_0}{3}
\end{equation}
于是命题得证.
\end{proof}