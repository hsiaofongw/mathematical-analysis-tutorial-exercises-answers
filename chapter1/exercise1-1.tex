% 练习题1.1
\exercise

1. 设$a$为有理数,$b$为无理数.求证:$a+b$与$a-b$都是无理数;当$a\neq 0$时,$ab$与$b/a$也是无理数.
\begin{proof}
采用反证法.设$a+b$是有理数,那么就存在互质的整数$p,q$,使得
\begin{equation}
    a+b = \frac{p}{q}
\end{equation}
由于$a$都是有理数,所以存在$p_1,q_1 \in \mathbb{Z}$使得$a = \displaystyle \frac{p_1}{q_1}$,其中$p_1,q_1$互质.于是
\begin{equation}
    b = a+b - a = \frac{p}{q} - \frac{p_1}{q_1} = \frac{p q_1 - q p_1}{q q_1}
\end{equation}
这说明$b$是有理数,矛盾.于是$a+b$是无理数.

再证$a-b$是无理数,采用反证法,先假设$a-b$是有理数,即,存在互质的整数$p_2,q_2$使得$\displaystyle a-b=\frac{p_2}{q_2}$,那么
\begin{equation}
    b = a - (a-b) = \frac{p_1}{q_1} - \frac{p_2}{q_2} = \frac{p_1 q_2 - q_1 p_2}{q_1 q_2}
\end{equation}
这说明$b$是有理数,矛盾.于是$a-b$是无理数.
\end{proof}

2. 证明:两个不同的有理数之间有无限多个有理数,也有无限多个无理数.
\begin{proof}
设$\displaystyle a_1 = \frac{p_1}{q_1}, b_1 = \frac{p_2}{q_2}$是有理数,且$a_1 \neq b_1$,令
\begin{equation}
    c_1 = \frac{a_1+b_1}{2} = \frac{1}{2} \left( \frac{p_1}{q_1} + \frac{p_2}{q_2} \right) = \frac{p_1 q_2 + p_2 q_1}{2 q_1 q_2}
\end{equation}
则显然$c_1$是无理数,且$a_1 < c_1 < b_1$,对任意的$n\geq 2, n \in \mathbb{N}$,我们令
\begin{equation}
    c_{n-1} = \frac{a_{n-1}+b_{n-1}}{2}, \; a_n = c_{n-1}, \; b_n = b_{n-1}
\end{equation}
由于有理数集关于加法运算封闭,所以按此方法可计算出无穷多个有理数$c_n$满足$a_1 < c_1 < c_2 < \cdots < b$,这就证明了两个不同的有理数之间有无穷多个有理数.

设$t_1$是一个无理数,那么我们可以找到两个整数$n_1, m_1$,满足$n_1 < m_1$,并且使得
\begin{equation}
    n_1 < t_1 < m_1
    \label{ieq:n1t1m1}
\end{equation}
在$0$到$b_1 - a_1$之间必定存在$N_1 \in \mathbb{N}^\star$,使得
\begin{equation}
    0 < \frac{m_1}{N_1} - \frac{n_1}{N_1} = \frac{m_1-n_1}{N_1} < b_1 - a_1
\end{equation}
再结合不等式(\ref{ieq:n1t1m1})也就是
\begin{equation}
    a_1 < \frac{t_1}{N_1} - \frac{n_1}{N_1} + a_1 < b_1
\end{equation}
令
\begin{equation}
    s_1 = \frac{t_1}{N_1} - \frac{n_1}{N_1} + a_1
\end{equation}
显然,由于$a_1$是有理数,而$\displaystyle \frac{t_1}{N_1} - \frac{n_1}{N_1}$是无理数,所以$s_1$是无理数并且介于$a_1$到$b_1$之间.对任意$n \in \mathbb{N}^\star$,令
\begin{equation}
    s_{n+1} = \frac{s_n+b_1}{2}, 
\end{equation}
则可构造出无穷多个无理数$s_{n}$,并且满足$a_1 < s_1 < s_2 < \cdots < b_1$,这就证明了任意两个有理数之间有无限多个无理数.
\end{proof}

16. 设$n=2,3,\cdots, \, x > -1$且$x\neq 0$.求证:$(1+x)^n > 1+nx$.
\begin{proof}
当$n=2$时,
\begin{equation}
    (1+x)^2 = 1 + 2x + x^2 > 1 + 2x + 0 = 1 + 2x
\end{equation}
于是对于$n=2$命题成立.假设对某个$k \in \mathbb{N}^\star, \, k \geq 2$,当$n=k$时命题成立,则
\begin{align}
    (1+x)^{k+1} &= (1+x)^k (1+x) > (1+kx)(1+x) \\
    &= 1 + (k+1)x + kx^2 > 1+(k+1)x
\end{align}
这说明如果对$n=k$命题成立,那么对$n=k+1$命题也成立.根据数学归纳法原理,命题对一切$n\in\mathbb{N}^\star$都成立.
\end{proof}